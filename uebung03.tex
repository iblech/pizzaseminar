\documentclass[a4paper,ngerman]{scrartcl}

%\usepackage{ucs}
\usepackage[utf8]{inputenc}

\usepackage[ngerman]{babel}

\usepackage{amsmath,amsthm,amssymb,amscd,color,graphicx}
\usepackage{mathabx}

%\usepackage[small,nohug]{diagrams}
%\diagramstyle[labelstyle=\scriptstyle]

\usepackage[protrusion=true,expansion=true]{microtype}

\usepackage{lmodern}
\usepackage{tabto}

\usepackage[natbib=true,style=numeric]{biblatex}
\usepackage[babel]{csquotes}
\bibliography{lit}

\usepackage[all]{xy}

%\usepackage{hyperref}

\setlength\parskip{\medskipamount}
\setlength\parindent{0pt}

\theoremstyle{definition}
\newtheorem{defn}{Definition}
\newtheorem{bsp}[defn]{Beispiel}

\theoremstyle{plain}

\newtheorem{prop}[defn]{Proposition}
\newtheorem{ueberlegung}[defn]{Überlegung}
\newtheorem{lemma}[defn]{Lemma}
\newtheorem{kor}[defn]{Korollar}
\newtheorem{hilfsaussage}[defn]{Hilfsaussage}
\newtheorem{satz}[defn]{Satz}

\theoremstyle{remark}
\newtheorem{bem}[defn]{Bemerkung}

\clubpenalty=10000
\widowpenalty=10000
\displaywidowpenalty=10000

\newcommand{\lra}{\longrightarrow}
\newcommand{\lhra}{\ensuremath{\lhook\joinrel\relbar\joinrel\rightarrow}}
\newcommand{\thlra}{\relbar\joinrel\twoheadrightarrow}
\newcommand{\xra}[1]{\xrightarrow{#1}}

\newcommand{\A}{\mathcal{A}}
\newcommand{\Z}{\mathbb{Z}}
\newcommand{\NN}{\mathbb{N}}
\newcommand{\Q}{\mathbb{Q}}
\newcommand{\R}{\mathbb{R}}
\newcommand{\C}{\mathcal{C}}
\newcommand{\D}{\mathcal{D}}
\newcommand{\RP}{\mathbb{R}\mathrm{P}}
\newcommand{\Hom}{\mathrm{Hom}}
\newcommand{\Set}{\mathrm{Set}}
\newcommand{\Grp}{\mathrm{Grp}}
\newcommand{\Vect}{\mathrm{Vect}}
\newcommand{\Spur}[1]{\operatorname{Spur}#1}
\newcommand{\rank}[1]{\operatorname{rank}#1}
\newcommand{\sgn}[1]{\operatorname{sgn}#1}
\newcommand{\id}{\mathrm{id}}
\newcommand{\Aut}[1]{\operatorname{Aut}(#1)}
\newcommand{\GL}[1]{\operatorname{GL}(#1)}
\newcommand{\ORTH}[1]{\operatorname{O}(#1)}
\newcommand{\freist}{\underline{\ \ }}
\newcommand{\op}{\mathrm{op}}
\DeclareMathOperator{\rk}{rk}
\DeclareMathOperator{\Spec}{Spec}
\DeclareMathOperator{\Bild}{im}
\DeclareMathOperator{\Kern}{ker}
\DeclareMathOperator{\Int}{int}
\DeclareMathOperator{\Ob}{Ob}
\newcommand{\Zzwei}{\Z_2}

\newcommand{\XXX}[1]{\textcolor{red}{#1}}

\renewcommand*\theenumi{\alph{enumi}}
\renewcommand{\labelenumi}{\theenumi)}

\pagestyle{empty}

%\newarrow{Equals}=====

\usepackage{geometry}
\geometry{tmargin=2cm,bmargin=2cm,lmargin=2.8cm,rmargin=2.8cm}

\begin{document}

\vspace*{-4em}
\begin{flushright}Universität Augsburg \\ 13. März 2013\end{flushright}

\begin{center}\Large \textbf{Pizzaseminar zur Kategorientheorie} \\
3. Übungsblatt
\end{center}
\vspace{1.5em}

\newbox{\mybox}
\setbox\mybox=\hbox{\textbf{Aufgabe 1:}}

\begin{list}{}{\labelwidth0em \leftmargin0em \itemindent0.5em \itemsep 2.0em}
\item[\textbf{Aufgabe 1:}]
"`Funktoren bewahren Isomorphie"':
Sei~$F : \C \to \D$ ein Funktor zwischen Kategorien~$\C$, $\D$ und seien~$X$
und~$Y$ Objekte von~$\C$. Zeige:
\[ X \cong Y \quad\Longrightarrow\quad F(X) \cong F(Y). \]
Zeige ferner, dass die Umkehrung ebenfalls gilt, wenn~$F$ voll und treu ist (siehe
Aufgabe~3).

\item[\textbf{Aufgabe 2:}]
Sei~$f:P \to Q$ eine \emph{monotone} Abbildung zwischen Quasiordnungen~$P$,
$Q$, d.\,h. für alle~$x,y \in P$ gilt
\[ \text{$x \preceq y$ in $P$} \ \Longrightarrow\ \text{$f(x) \sqsubseteq f(y)$
in~$Q$}. \]
Überlege, wie man daraus einen Funktor $BP \to BQ$ der zugehörigen Kategorien
aus Aufgabe~3 von Blatt~2 basteln kann. Wieso sind die Funktoraxiome erfüllt?

\item[\textbf{Aufgabe 3:}]
Ein Funktor~$F : \C \to \D$ heißt\ldots
\begin{enumerate}
\item \emph{treu}, wenn für je zwei parallele Morphismen~$f,g$ in~$\C$ (d.\,h.
Morphismen mit gleicher Quelle und gleichem Ziel) gilt:
$F(f) = F(g) \quad\Longrightarrow\quad f = g.$
\item \emph{voll}, wenn es für alle Objekte~$X,Y \in \C$ und jeden
Morphismus~$h : F(X) \to F(Y)$ in~$\D$ einen Morphismus $f:X \to Y$ in~$\C$
mit $F(f) = h$ gibt.
\item \emph{wesentlich surjektiv}, wenn es für jedes Objekt~$Z \in \Ob \D$ ein
Objekt~$X \in \Ob \C$ mit
$F(X) \cong Z$ gibt.
\end{enumerate}
Untersuche einen Funktor deiner Wahl daraufhin, ob er treu, voll oder
wesentlich surjektiv ist.

\item[\textbf{Aufgabe 4:}]
Sei~$\C$ eine Kategorie und~$A \in \Ob \C$ ein Objekt. Wir nehmen an, dass wir
für jedes Objekt~$X \in \Ob \C$ ein bestimmtes Produkt~$A \times X \in \Ob \C$
gegeben haben. Überlege, wie man die unvollständige Zuordnungsvorschrift
\[ \begin{array}{@{}rrcl@{}}
  F{:} & \C &\longrightarrow& \C \\
  & X &\longmapsto& A \times X
\end{array} \]
zu einer Funktordefinition ausweiten kann. Wie kann man~$F$ auf Morphismen
definieren? Wieso sind die Funktoraxiome erfüllt?

\small
\item[\textbf{Projektaufgabe:}]
Sei~$\C$ eine Kategorie und~$A \in \Ob \C$ ein Objekt. Der \emph{kovariante
Hom-Funktor zu~$A$} ist definiert als
\[ \begin{array}{@{}rrcl@{}}
  \widecheck A{:} & \C &\longrightarrow& \Set \\
  & X &\longmapsto& \Hom_\C(A,X) \\
  & (f: X \to Y) &\longmapsto& f_\star,
\end{array} \]
wobei~$f_\star$ die Abbildung
\[ \begin{array}{@{}rrcl@{}}
  f_\star{:} & \Hom_\C(A,X) &\longrightarrow& \Hom_\C(A,Y) \\
  & g &\longmapsto& f \circ g
\end{array} \]
ist. Zeige als Hinführung auf den Yoneda-Vortrag, dass~$\widecheck A$ tatsächlich
ein Funktor ist.
\end{list}

\end{document}

\item[\textbf{Aufgabe 2:}]
Sei~$F:\C\to\C$ ein Endofunktor auf einer Kategorie~$\C$. Wir definieren
folgende Kategorie der \emph{$F$-Algebren}:
\begin{align*}
  \text{Objekte: } & \text{Diagramme der Form $F(A) \to A$ in~$\C$} \\
  \text{Morphismen: } &
    \Hom(F(A) \xra{\alpha} A, F(B) \xra{\beta} B) := \left\{
      f:A \to B \,\middle|\,
\vcenter{\xymatrix{
  F(A) \ar[r]^\alpha \ar[d]_{F(f)} & A \ar[d]^f \\
  F(B) \ar[r]_\beta & B
}} \right\}
\end{align*}
\begin{enumerate}
\item Zeige das \emph{Theorem von Lambek}: Ist~$F(A) \xra{\alpha} A$ eine
initiale~$F$-Algebra, so ist~$\alpha$ ein Isomorphismus. [Die Umkehrung gilt
nicht.]
\item Sei speziell~$\C := \Set$ und~$F$ der Funktor
\[ F(X) := 1 \coprod X \]
mit~$1 := \{ \star \}$. Zeige, dass die Abbildung
\[\begin{array}{@{}rcl@{}}
  F(\NN) &\longrightarrow& \NN \\
  \star &\longmapsto& 0 \\
  n &\longmapsto& n+1
\end{array}\]
eine initiale~$F$-Algebra ist. Damit hast du die induktive Struktur der
natürlichen Zahlen charakterisiert!
\end{enumerate}
\end{list}

% Gruppenzentrum kein Funktor
% Bifunktoren
% Split-Epis bleiben unter beliebigen Funktoren erhalten
% Datenbanken?

\end{document}
