\documentclass[a4paper,english,12pt]{scrartcl}

\usepackage[utf8]{inputenc}

\usepackage[english]{babel}

\usepackage{amsmath,amsthm,amssymb,stmaryrd,color,graphicx}
\usepackage{array}
\usepackage[all]{xy}

\usepackage[protrusion=true,expansion=true]{microtype}

\usepackage{lmodern}
\usepackage{tabto}

\usepackage{hyperref}

\theoremstyle{definition}
\newtheorem{defn}{Definition}[section]
\newtheorem{ex}[defn]{Example}

\theoremstyle{plain}

\newtheorem{prop}[defn]{Proposition}
\newtheorem{lemma}[defn]{Lemma}
\newtheorem{thm}[defn]{Theorem}
\newtheorem{cor}[defn]{Corollary}

\theoremstyle{remark}
\newtheorem{rem}[defn]{Remark}

\clubpenalty=10000
\widowpenalty=10000
\displaywidowpenalty=10000

\newcommand{\CC}{\mathbb{C}}
\newcommand{\NN}{\mathbb{N}}
\newcommand{\ZZ}{\mathbb{Z}}
\newcommand{\FF}{\mathbb{F}}
\newcommand{\C}{\mathcal{C}}
\newcommand{\E}{\mathcal{E}}
\newcommand{\F}{\mathcal{F}}
\newcommand{\G}{\mathcal{G}}
\newcommand{\id}{\mathrm{id}}
\newcommand{\op}{\mathrm{op}}
\newcommand{\xra}[1]{\xrightarrow{#1}}
\newcommand{\Mod}{\mathrm{Mod}}
\newcommand{\Set}{\mathrm{Set}}
\newcommand{\Cat}{\mathrm{Cat}}
\newcommand{\Vect}{\mathrm{Vect}}
\newcommand{\VB}{\mathrm{VB}}
\newcommand{\Id}{\mathrm{Id}}
\newcommand{\QCoh}{\mathrm{QCoh}}
\newcommand{\pt}{\mathrm{pt}}
\newcommand{\Ob}{\operatorname{Ob}}
\newcommand{\Hom}{\mathrm{Hom}}
\newcommand{\ul}[1]{\underline{#1}}

\begin{document}

\title{Fibered categories}
\author{Ingo Blechschmidt}
\date{October 11, 2014}
\maketitle

\begin{center}\begin{minipage}{0.8\textwidth}
Fibered and indexed categories formalize objects and morphisms
which can be pulled back along morphisms of a base category. One reason they
are important is that stacks are fibered categories with certain extra
properties.\medskip

These are informal notes prepared for the October 2014 meeting of the \emph{Kleine
Bayerische AG} at the University of Regensburg. The notes summarize part of
chapter~3 of Angelo Vistoli's
\href{http://homepage.sns.it/vistoli/descent.pdf}{\emph{Notes on Grothendieck
topologies, fibered categories and descent theory}}.
\end{minipage}\end{center}
\vspace{1em}

\tableofcontents

\section{Indexed categories}

\begin{defn}Let~$\C$ be a category. A \emph{$\C$-indexed category} is a
pseudofunctor~$F : \C^\op \to \Cat$.\end{defn}

The notion of a \emph{pseudofunctor} is a slight weakening of the notion of an
ordinary functor. It is best explained by looking at a example. By~$\Cat$ we
mean the (2-)category of categories, functors, and natural transformations. We
ignore all set-theoretical issues.

\begin{ex}\label{ex:vb}Let~$\C$ be a category of spaces (topological spaces,
manifolds, schemes, schemes over a fixed scheme, \ldots). Then there is
a~$\C$-indexed category of vector bundles:
\[ \begin{array}{@{}rcl@{}}
  X &\longmapsto& \VB(X) = \text{category of vector bundles on~$X$} \\
  (f : X \to Y) &\longmapsto&
    (f^* : \VB(Y) \to \VB(X)) = \text{pullback along~$f$}
\end{array} \]
\end{ex}

Pullback of vector bundles is not associative on the nose; for maps~$X \xra{f}
Y \xra{g} Z$ it is not the case that~$(g \circ f)^* = f^* \circ g^*$ as
functors~$\VB(Z) \to \VB(X)$. In fact, recall from general category theory that
it does not make sense to compare functors on equality, since this involves
comparing objects on equality. A pseudofunctor does not need to satisfy the
functor axioms on the nose, it suffices that they are satisfied up to coherent
natural isomorphisms. We give the precise definition below.

\begin{ex}Let~$\C$ be a category of schemes. Then there is a~$\C$-indexed
category of quasicoherent sheaves of modules, defined by~$X \mapsto \QCoh(X)$.
\end{ex}

\begin{ex}There is a~$\mathrm{Ring}$-indexed category of modules, given by~$R
\mapsto \Mod(R)$ (the category of~$R$-modules) and the restriction of scalars
functors.
\end{ex}

\begin{ex}\label{ex:self-indexing}Let~$\C$ be a category with pullbacks. There
is a canonical~$\C$-indexed category~$\CC$, the \emph{self-indexing of~$\C$}:
\[ \begin{array}{@{}rcl@{}}
  X &\longmapsto& \C/X \\
  (f : X \to Y) &\longmapsto&
    (f^* : \C/Y \to \C/X)
\end{array} \]
As usual,~$\C/X$ is the slice category of morphisms to~$X$; its objects are
morphisms~$T \to X$, where~$X \in \C$ is arbitrary, and its morphisms are
commutative triangles. The functor~$f^*$ is defined by sending an object~$(T
\to Y)$ to some chosen pullback~$(T \times_Y X \to X)$.\end{ex}

\begin{rem}A morphism~$T \xra{\pi} X$ in the category of sets can be thought of as
an~$X$-indexed family of sets, namely the fibers~$\pi^{-1}[\{x\}]$, where~$x$
ranges over~$X$. By analogy, we visualize a morphism~$T \to X$ in an arbitrary
category~$\C$ as an~$X$-indexed family of objects in~$\C$. Thus~$\C/X$ can be
thought of as the category of~$X$-indexed families of objects in~$\C$, and the
functors~$f^*$ \emph{reindex families}.
\end{rem}

\begin{ex}\label{ex:presheaf}Any presheaf~$F : \C^\op \to \Set$ (that is, an ordinary functor)
gives rise to a~$\C$-indexed category by postcomposing with the embedding~$\Set
\to \Cat$, which associates to any set its induced discrete category. In
particular, any object~$A \in \C$ gives rise to a \emph{representable}~$\C$-indexed
category~$\ul{A}$ given by~$T \mapsto \Hom_\C(T,A)$.
\end{ex}

Stacks are special kinds of indexed categories, so indexed categories should
be geometric in some sense. Where is it? Where are the topological spaces, the
points? Let~$\C$ be a category of spaces and~$F$ a~$\C$-indexed category. We
can infuse geometric content by imagining, for any space~$A \in \C$, the
category~$F(A)$ to be \emph{the category of maps~$A \to F$}. So even though~$F$
does not define topological structure directly, we can \emph{probe} it by test
spaces. This is the general idea of Grothendieck's \emph{functor of points}
philosophy (see
\url{http://ncatlab.org/nlab/show/motivation+for+sheaves,+cohomology+and+higher+stacks}
for a beautiful exposition).

How good is this idea? We expect to be able to precompose morphisms~$f : A \to F$
by morphisms~$g : B \to A$ in~$\C$ to obtain morphisms~$f \circ g : B \to F$.
This is indeed possible:~$f \circ g$ can be realized as~$g^*(f)$.

But the resulting notion fails in general to be local: If~$A = \bigcup_i U_i$
is covered by open subsets, we expect to be able to define maps~$A \to F$ by
glueing compatible maps~$U_i \to F$. This is not possible with an arbitrary
indexed category -- the notion of a topology does not enter the definition in
any way. A stack is an indexed category where this kind of pathology does not
arise.

\begin{ex}The \emph{points} of the indexed category~$\VB$ of vector bundles
(example~\ref{ex:vb}) are by definition (and by analogy with the classical
situation) the maps~$\pt \to \VB$. So for any natural number~$n$, there is a point
of~$\VB$ (corresponding to a vector space of dimension~$n$). These points
have lots of automorphisms, namely the invertible~$(n \times n)$-matrices.
\end{ex}

\begin{rem}In speaking of morphisms from a test space~$A$ to an indexed
category~$F$ we commit a type error. This can be fixed by replacing~$A$ with
its induced indexed category~$\ul{A}$. A~2-categorical Yoneda lemma then gives
a canonical equivalence~$\Hom(\ul{A},F) \simeq F(A)$.
\end{rem}

We are now ready to appreciate a precise definition of a pseudofunctor.

\begin{defn}Let~$\C$ be a category. A \emph{pseudofunctor}~$F : \C^\op \to
\Cat$ consists of
\begin{itemize}
\item a category~$F(X)$ for each object~$X$ of~$\C$,
\item a functor~$F(f) =: f^* : F(Y) \to F(X)$ for each morphism~$X \xra{f} Y$ in~$\C$,
\item a natural isomorphism~$\eta_X : (\id_X)^* \Rightarrow \Id_{F(X)}$ for each
object~$X$ in~$\C$, and
\item a natural isomorphism~$\alpha_{f,g} : f^* \circ g^* \Rightarrow (g \circ
f)^*$ for each diagram~$X \xra{f} Y \xra{g} Z$ in~$\C$
\end{itemize}
such that the following \emph{coherence conditions} hold:
\begin{itemize}
\item For any~$X \xra{f} Y$ in~$\C$, $\alpha_{\id_X,f} = \eta_X f^* : \id_X^* \circ f^*
\Rightarrow f^*$.
\item For any~$X \xra{f} Y$ in~$\C$, $\alpha_{f,\id_Y} = f^* \eta_Y : f^* \circ
\id_Y^* \Rightarrow f^*$.
\item For any $X \xra{f} Y \xra{g} Z \xra{h} W$ in~$\C$, the diagram commutes.
\[ \text{XXX diagram} \]
\end{itemize}
\end{defn}

The coherence conditions are needed for the following reason:
When dealing with vector bundles or similar objects which can be pulled back,
we are used to identify iterated pullbacks with pullbacks along compositions
(``$f^* \circ g^* = (g \circ f)^*$'').
This is justified only because between any
two pullback expressions, e.\,g.\@
\[ (f^* \circ (h \circ g)^* \circ p^* \circ q^*)(E) \quad\text{and}\quad
  ((g \circ f)^* \circ (q \circ p \circ h)^*)(E), \]
we have \emph{exactly one} agreed-upon canonical isomorphism (of course, there
are lots of other non-canonical isomorphisms as well). In a general indexed
category, we can construct these canonical isomorphisms out of the
given~$\eta_X$ and~$\alpha_{f,g}$; the coherence conditions ensure that -- no
matter in which way we do this -- we always obtain the same canonical
isomorphism.

\begin{rem}Indexed categories are also important in categorical logic and topos
theory. For instance, any geometric morphism~$\F \xra{f} \E$ of toposes induces
an~$\E$-indexed category~$\FF$, given by~$X \mapsto \F/f^*X$. From the point of
view of the internal language of~$\E$, an~$\E$-indexed category looks like an
ordinary category. This is useful for doing \emph{relative category theory}.
\end{rem}


\section{Fibered categories}

Recall that an~$I$-indexed family~$(X_i)_{i \in I}$ of sets can equivalently be
given by
\begin{itemize}
\item a map~$I \to \Ob\Set$, $i \mapsto X_i$, or
\item a map~$\coprod_{i \in I} X_i \xra{\pi} I$. (Recover~$X_i$ as the fiber
of~$i$ under~$\pi$.)
\end{itemize}
Indexed categories are like the first approach, while fibered categories are
like the second approach.

\begin{defn}Let~$\C$ be a category. A \emph{category fibered over~$\C$} is a
category~$\F$ together with a functor~$\F \xra{p} \C$ such that ``pullbacks
exist'': For any morphism~$Y \xra{f} X$ in~$\C$ and any object~$E$ in the fiber
category~$\F(X)$ (the subcategory of~$\F$ consisting of those objects which map
to~$X$ and those morphisms which map to~$\id_X$ under~$p$), there should
exist an object~$E'$ of~$\F(X)$ and a \emph{cartesian} morphism~$E' \xra{\phi}
E$ with~$p(\phi) = f$.
\[ \xymatrixcolsep{3pc}\xymatrixrowsep{3pc}\xymatrix{
  E' \ar@{-}[d] \ar^\phi[r] & E \ar@{-}[d] \\
  Y \ar[r]_f & X
} \]
\end{defn}

The notion of a cartesian morphism is modeled after the universal property of
usual pullbacks: A morphisms~$\phi : E' \to E$ with~$p(\phi) = f$ like in the
definition is cartesian if and only if, for any morphism~$g : Z \to Y$, for any
object~$E''$ in~$\F(Z)$, and any morphism~$\psi : E'' \to E$ such that~$p(\psi)
= f \circ g$, there exists a unique morphism~$\alpha : E'' \to E'$ such
that~$p(\alpha) = g$ and~$\phi \circ \alpha = \psi$.
\[ \text{XXX diagram} \]

\begin{rem}The definition of a fibered category as stated is \emph{evil} in the
technical sense that it requires certain equalities between objects (namely,
$p(E') = Y$). Since important examples of categories which ought to be fibered
categories actually fulfill these evil equalities, this defect is tolerable in
practice. But for some purposes, a non-evil variant is necessary; see for
instance
\url{http://ncatlab.org/nlab/show/Grothendieck+fibration#StreetFibration}.\end{rem}

\begin{ex}The incarnation of example~\ref{ex:vb} as a fibered category is the
category~$\VB$ of vector bundles on arbitrary base spaces. The projection
functor~$\VB \to \C$ associates to any vector bundle its basespace.\end{ex}

\begin{defn}An \emph{indexed functor}~$\F \to \G$ between fibered categories
over~$\C$ is an ordinary functor~$\F \to \G$ which commutes with the projection
functors to~$\C$ and which carries cartesian morphisms to cartesian
morphisms.\end{defn}


\section{Equivalence between indexed and fibered categories}

\begin{prop}There is a pseudoequivalence between the 2-category of indexed
categories, indexed functors, and indexed natural transformations; and the
2-category of fibered categories, fibered functors, and fibered natural
transformations.\end{prop}

\paragraph{From fibered categories to indexed categories.}
If suitable choice principles are available, any fibered category~$\F \to \C$
defines a~$\C$-indexed category~$F : \C^\op \to \Cat$ by setting~$F(X) :=
\F(X)$. For morphisms~$f : Y \to X$ in~$\C$, the pullback functors~$f^* : F(X)
\to F(Y)$ are defined by associating to any object~$E$ a chosen pullback~$E'$
like in the definition of a fibered category. The action of~$f^*$ on morphisms
is given by the universal property of cartesian morphisms.

One can check that this construction gives indeed rise to a pseudofunctor. In
fact, the proof is very similar to the proof that the self-indexing~$X \mapsto
\C/X$ is a pseudofunctor.

\paragraph{From indexed categories to fibered categories.}
In the opposite direction, any~$\C$-indexed category~$F : \C^\op \to \Cat$
induces a fibered category~$\int F \to \C$ via the \emph{Grothendieck
construction}. Mimicking the case of families of sets, the objects of~$\int F$
are pairs~$(X \in \C, E \in F(X))$. A morphism~$(X,E) \to (Y,F)$ is a pair~$(f
: X \to Y, \alpha : E \to f^*(F))$. The projection functor~$\int F \to \C$
sends an object~$(X,E)$ to~$X$ and a morphism~$(f,\alpha)$ to~$f$.

One can check that this category over~$\C$ is indeed a fibered category. The
cartesian morphisms~$\phi : (Y,E') \to (X,E)$ required by the definition are
the morphisms~$(f : Y \to X, \id : f^*(E) \to f^*(E)$.

\begin{ex}Let~$\C$ be a category with pullbacks. Then~$\int \CC$ (see
example~\ref{ex:self-indexing}) is equivalent to the \emph{category of morphisms
in~$\C$}. This has arbitrary morphisms~$Y \to X$ as objects and commutative
squares as morphisms. The equivalence is given by sending an object~$(X \in \C,
(Y \xra{f} X))$ to~$f$.
\end{ex}

\begin{ex}Let~$\C$ be an arbitrary category and~$A$ a fixed object of~$\C$.
Then~$\int \ul{A}$ (see example~\ref{ex:presheaf}) is
equivalent to the slice category~$\C/A$, by sending an object~$(X \in \C, X
\xra{f} A)$ to~$f$.
\end{ex}


\section{Special types of indexed and fibered categories}

\begin{defn}A \emph{category fibered in groupoids (sets) [equivalence relations]}
is a fibered category~$\F$ such that the fiber categories~$\F(X)$ are all
groupoids (discrete categories) [categories equivalent to discrete
categories].\end{defn}

Equivalently, the associated pseudofunctor~$\C^\op \to \Cat$ factors
over~$\mathrm{Grpd} \hookrightarrow \Cat$ ($\Set \hookrightarrow
\Cat$) [$\mathrm{EqRel} \hookrightarrow \Cat$].

\begin{prop}Let~$\F \xra{p} \C$ be a functor. This is \ldots
\begin{itemize}
\item a category fibered in groupoids if and only if any morphism
in~$\F$ is cartesian and for any morphism~$Y \xra{f} X$ in~$\C$ and any
object~$E$ in~$\F(X)$, there exists a morphism~$E' \xra{\phi} E$ in~$\F$ such
that~$p(\phi) = f$.
\item a category fibered in sets if and only if
for any morphism~$Y \xra{f} X$ in~$\C$ and any
object~$E$ in~$\F(X)$, there exists a unique morphism~$E' \xra{\phi} E$ in~$\F$ such
that~$p(\phi) = f$.
\end{itemize}
\end{prop}


\section{An important example: quotient stacks}

For this example, we fix a category~$\C$ of spaces. We will be so sketchy that
it does not matter which category~$\C$ actually is. The source for this section
is the article \href{http://arxiv.org/abs/math/9911199}{\emph{Algebraic stacks}
of Tomás Gómez}.

Let~$G$ be a group acting on a space~$X$. Assume that the action is
\emph{free}, i.\,e.\@ for any~$g \in G$ and any~$x \in X$, $g \cdot x = x$
implies~$g = 1$. Then recall that there is a well-defined projection~$X \to
X/G$; that this map is a~$G$-principal bundle; and that this map is the
\emph{universal}~$G$-principal bundle equipped with a~$G$-equivariant map
to~$X$: For any~$G$-principal bundle~$P \to Y$ equipped with a~$G$-equivariant
map~$P \to X$, there is a unique map~$Y \to X/G$ such that the diagram
\[ \xymatrixcolsep{3pc}\xymatrixrowsep{3pc}\xymatrix{
  P \ar[r] \ar[d] & X \ar[d] \\
  Y \ar[r] & X/G
} \]
is a pullback diagram. Thus the space~$X/G$ has the pleasant property that
maps~$Y \to X/G$ correspond to~$G$-principal bundles~$P \to Y$ equipped
with~$G$-equivariant maps~$Y \to X$. (If such a bundle is given, construct~$Y
\to X/G$ by selecting, for any~$y \in Y$, a preimage under~$P \to Y$, and
applying~$P \to X \to X/G$ to this preimage. One can check that the resulting
element in~$X/G$ is independent of the choice.)

Unfortunately, if the action of~$G$ on~$X$ is not free,~$X/G$ fails to exist as
an object of~$\C$ or fails to have this universal property. But there is always
a quotient \emph{stack}~$[X/G]$ (at least an indexed category) with this
property, tautologically defined by sending a test object~$Y$ of~$\C$ to the
category of~$G$-principal bundles~$P \to Y$ equipped with~$G$-equivariant
maps~$Y \to X$.

Indeed, one can canonically define a map~$\ul{X} \to [X/G]$ and check that
is the universal bundle.

\begin{prop}The stack~$[X/G]$ is representable, i.\,e.\@ of the form~$\ul{A}$
for some space~$A$ in~$\C$, if and only if the action of~$G$ on~$X$ is
free.\end{prop}
\begin{proof}[Proof (sketch)]
For the ``only if'' direction, check that the morphism~$X \to A$ corresponding
to the universal bundle~$\ul{X} \to \ul{A}$ is a~$G$-principal bundle
isomorphic to~$X \to X/G$.
\end{proof}

\begin{ex}Let~$X = \pt$. Then the quotient stack~$[X/G]$ is commonly
named~``$BG$''.\end{ex}

\begin{rem}The category of points of~$[X/G]$, i.\,e.\@~$[X/G](\pt)$,
is equivalent to the \emph{action groupoid} (\emph{weak quotient})~$X/\!/G$.
This groupoid has as objects the elements of~$X$, and the set of morphisms~$x
\to x'$ is the set of group elements~$g$ such that~$g \cdot x = x'$.
The action groupoid is the ``right replacement'' for the ill-behaved space of
orbits. A very accessible exposition is in \emph{Higher-Dimensional Algebra
VII: Groupoidification} by John Baez, Alexander Hoffnung and Christopher
Walker: \url{http://arxiv.org/abs/0908.4305}.
\end{rem}

\begin{rem}I believe that in some cases, the category of coherent sheaves
on~$[X/G]$ can simply be described as the category of~$G$-equivariant sheaves
on~$X$.
\end{rem}

\end{document}

Relevance in logic, topos theory, ...
