\documentclass[a4paper,english,12pt]{scrartcl}

\usepackage[utf8]{inputenc}

\usepackage[english]{babel}

\usepackage{amsmath,amsthm,amssymb,stmaryrd,color,graphicx}
\usepackage{array}
\usepackage[all]{xy}

\usepackage[protrusion=true,expansion=true]{microtype}

\usepackage{lmodern}
\usepackage{tabto}

\usepackage{hyperref}

\theoremstyle{definition}
\newtheorem{defn}{Definition}[section]
\newtheorem{ex}[defn]{Example}

\theoremstyle{plain}

\newtheorem{prop}[defn]{Proposition}
\newtheorem{lemma}[defn]{Lemma}
\newtheorem{thm}[defn]{Theorem}
\newtheorem{cor}[defn]{Corollary}

\theoremstyle{remark}
\newtheorem{rem}[defn]{Remark}

\clubpenalty=10000
\widowpenalty=10000
\displaywidowpenalty=10000

%\setlength\parskip{\medskipamount}
%\setlength\parindent{0pt}

\newcommand{\NN}{\mathbb{N}}
\newcommand{\ZZ}{\mathbb{Z}}
\renewcommand{\aa}{\mathfrak{a}}
\newcommand{\bb}{\mathfrak{b}}
\newcommand{\pp}{\mathfrak{p}}
\renewcommand{\O}{\mathcal{O}}
\newcommand{\C}{\mathcal{C}}
\newcommand{\D}{\mathcal{D}}
\newcommand{\id}{\mathrm{id}}
\newcommand{\op}{\mathrm{op}}
\newcommand{\xra}[1]{\xrightarrow{#1}}
\DeclareMathOperator{\Spec}{Spec}
\renewcommand{\_}{\mathpunct{.}\,}
\newcommand{\?}{\,{:}\,}

\newcommand{\hilight}[2]{\begin{center}\framebox{#2}\par#1\end{center}}

\newcommand{\speak}[1]{\ulcorner\text{#1}\urcorner}

\setcounter{tocdepth}{2}

\newlength{\aufgabenskip}
\setlength{\aufgabenskip}{1.5em}
\newcounter{aufgabennummer}
\newenvironment{aufgabe}[1]{
  \addtocounter{aufgabennummer}{1}
  \textbf{Aufgabe \theaufgabennummer{}.} \emph{#1} \par
}{\vspace{\aufgabenskip}}

\newenvironment{indentblock}{%
  \list{}{\leftmargin\leftmargin}%
  \item\relax
}{%
  \endlist
}

\newcommand{\Mod}{\mathrm{Mod}}
\newcommand{\Set}{\mathrm{Set}}
\newcommand{\Cat}{\mathrm{Cat}}
\newcommand{\Vect}{\mathrm{Vect}}
\newcommand{\VB}{\mathrm{VB}}
\newcommand{\Id}{\mathrm{Id}}
\newcommand{\QCoh}{\mathrm{QCoh}}
\newcommand{\pt}{\mathrm{pt}}
\newcommand{\Hom}{\underline{\mathrm{Hom}}}
\renewcommand{\hom}{\mathrm{Hom}}
\newcommand{\End}{\mathrm{End}}
\newcommand{\ev}{\mathrm{ev}}
\newcommand{\freist}{\underline{\ \ }}
\newcommand{\Rep}{\mathrm{Rep}}
\newcommand{\Cob}{\mathrm{Cob}}
\newcommand{\tr}{\operatorname{tr}}
\newcommand{\rk}{\operatorname{rk}}
\newcommand{\im}{\operatorname{im}}
\newcommand{\ul}[1]{\underline{#1}}

\begin{document}

\title{Fibered categories}
\author{Ingo Blechschmidt}
\date{October 11, 2014}
\maketitle

\begin{center}\begin{minipage}{0.8\textwidth}
Fibered and indexed categories formalize the notion of objects and morphisms
which can be pulled back along morphisms of a base category. One reason they
are important is that stacks are fibered categories with certain extra
properties.\medskip

These are informal notes prepared for the October 2014 meeting of the \emph{Kleine
Bayerische AG} at TU München. The notes summarize part of chapter~3 of
Angelo Vistoli's \href{http://homepage.sns.it/vistoli/descent.pdf}{\emph{Notes
on Grothendieck topologies, fibered categories and descent theory}}.
\end{minipage}\end{center}
\vspace{1em}

\tableofcontents

\section{Indexed categories}

\begin{defn}Let~$\C$ be a category. A \emph{$\C$-indexed category} is a
pseudofunctor~$F : \C^\op \to \Cat$.\end{defn}

The notion of a \emph{pseudofunctor} is a slight weakening of the notion of an
ordinary functor. It is best explained by looking at a example. By~$\Cat$ we
mean the (2-)category of categories, functors, and natural transformations. We
ignore all set-theoretical issues.

\begin{ex}\label{ex:vb}Let~$\C$ be a category of spaces (topological spaces,
manifolds, schemes, schemes over a fixed scheme, \ldots). Then there is
a~$\C$-indexed category of vector bundles:
\[ \begin{array}{@{}rcl@{}}
  X &\longmapsto& \VB(X) = \text{category of vector bundles on~$X$} \\
  (f : X \to Y) &\longmapsto&
    (f^* : \VB(Y) \to \VB(X)) = \text{pullback along~$f$}
\end{array} \]
\end{ex}

Pullback of vector bundles is not associative on the nose; for maps~$X \xra{f}
Y \xra{g} Z$ it is not the case that~$(g \circ f)^* = f^* \circ g^*$ as
functors~$\VB(Z) \to \VB(X)$. In fact, recall from general category theory that
it does not make sense to compare functors on equality, since this involves
comparing objects on equality. A pseudofunctor does not need to satisfy the
functor axioms on the nose, it suffices that they are satisfied up to coherent
natural isomorphisms. We give the precise definition below.

\begin{ex}Let~$\C$ be a category of schemes. Then there is a~$\C$-indexed
category of quasicoherent sheaves of modules, defined by~$X \mapsto \QCoh(X)$.
\end{ex}

\begin{ex}There is a~$\mathrm{Ring}$-indexed category of modules, given by~$R
\mapsto \Mod(R)$ (the category of~$R$-modules) and the restriction of scalars
functors.
\end{ex}

\begin{ex}Let~$\C$ be a category with pullbacks. There is a
canonical~$\C$-indexed category, the \emph{self-indexing of~$\C$}:
\[ \begin{array}{@{}rcl@{}}
  X &\longmapsto& \C/X \\
  (f : X \to Y) &\longmapsto&
    (f^* : \C/Y \to \C/X)
\end{array} \]
As usual,~$\C/X$ is the slice category of morphisms to~$X$; its objects are
morphisms~$T \to X$, where~$X \in \C$ is arbitrary, and its morphisms are
commutative triangles. The functor~$f^*$ is defined by sending an object~$(T
\to Y)$ to some chosen pullback~$(T \times_Y X \to X)$.\end{ex}

\begin{rem}A morphism~$T \xra{\pi} X$ in the category of sets can be thought of as
an~$X$-indexed family of sets, namely the fibers~$\pi^{-1}[\{x\}]$, where~$x$
ranges over~$X$. By analogy, we visualize a morphism~$T \to X$ in an arbitrary
category~$\C$ as an~$X$-indexed family of objects in~$\C$. Thus~$\C/X$ can be
thought of as the category of~$X$-indexed families of objects in~$\C$, and the
functors~$f^*$ \emph{reindex families}.
\end{rem}

\begin{ex}Any presheaf~$F : \C^\op \to \Set$ (that is, an ordinary functor)
gives rise to a~$\C$-indexed category by postcomposing with the embedding~$\Set
\to \Cat$, which associates to any set its induced discrete category. In
particular, any object~$A \in \C$ gives rise to a \emph{representable}~$\C$-indexed
category~$\ul{A}$ given by~$T \mapsto \Hom_\C(T,A)$.
\end{ex}

Stacks are special kinds of indexed categories, so indexed categories should
be geometric in some sense. Where is it? Where are the topological spaces, the
points? Let~$\C$ be a category of spaces and~$F$ a~$\C$-indexed category. We
can infuse geometric content by imagining, for any space~$A \in \C$, the
category~$F(A)$ to be \emph{the category of maps~$A \to F$}. So even though~$F$
does not define topological structure directly, we can \emph{probe} it by test
spaces. This is the general idea of Grothendieck's \emph{functor of points}
philosophy (see XXXnLab).

How good is this idea? We expect to be able to precompose morphisms~$f : A \to F$
by morphisms~$g : B \to A$ in~$\C$ to obtain morphisms~$f \circ g : B \to F$.
This is indeed possible --~$f \circ g$ can be realized as~$g^*(f)$.

But the resulting notion fails in general to be local: If~$A = \bigcup_i U_i$
is covered by open subsets, we expect to be able to define maps~$A \to F$ by
glueing compatible maps~$U_i \to F$. This is not possible with an arbitrary
indexed category -- the notion of a topology does not enter the definition in
any way. A stack is a fibered category where this kind of pathology does not
arise.

\begin{ex}The \emph{points} of the fibered category~$\VB$ of vector bundles
(example~\ref{ex:vb}) are by definition (and by analogy with the classical
situation) the maps~$\pt \to \VB$. So for any natural number~$n$, there is a point
of~$\VB$ (corresponding to a vector space of dimension~$n$). These points
have lots of automorphisms, namely the invertible~$(n \times n)$-matrices.
\end{ex}

\begin{rem}In speaking of morphisms from a test space~$A$ to a fibered
category~$F$ we commit a type error. This can be fixed by replacing~$A$ with
its induced fibered category~$\ul{A}$. A~2-categorical Yoneda lemma then gives
a canonical equivalence~$\Hom(\ul{A},F) \simeq F(A)$.
\end{rem}

We are now ready to appreciate a precise definition of a pseudofunctor.

\begin{defn}Let~$\C$ be a category. A \emph{pseudofunctor}~$F : \C^\op \to
\Cat$ consists of
\begin{itemize}
\item a category~$F(X)$ for each object~$X$ of~$\C$,
\item a functor~$F(f) =: f^* : F(Y) \to F(X)$ for each morphism~$X \xra{f} Y$ in~$\C$,
\item a natural isomorphism~$\eta_X : (\id_X)^* \Rightarrow \Id_{F(X)}$ for each
object~$X$ in~$\C$, and
\item a natural isomorphism~$\alpha_{f,g} : f^* \circ g^* \Rightarrow (g \circ
f)^*$ for each diagram~$X \xra{f} Y \xra{g} Z$ in~$\C$
\end{itemize}
such that the following \emph{coherence conditions} hold:
\begin{itemize}
\item For any~$X \xra{f} Y$ in~$\C$, $\alpha_{\id_X,f} = \eta_X f^* : \id_X^* \circ f^*
\Rightarrow f^*$.
\item For any~$X \xra{f} Y$ in~$\C$, $\alpha_{f,\id_Y} = f^* \eta_Y : f^* \circ
\id_Y^* \Rightarrow f^*$.
\item For any $X \xra{f} Y \xra{g} Z \xra{h} W$ in~$\C$, the diagram commutes.

XXX
\end{itemize}
\end{defn}

The coherence conditions are needed for the following reason: XXX

\end{document}

Relevance in logic, topos theory, ...
