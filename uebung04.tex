\documentclass[a4paper,ngerman]{scrartcl}

%\usepackage{ucs}
\usepackage[utf8]{inputenc}

\usepackage[ngerman]{babel}

\usepackage{amsmath,amsthm,amssymb,amscd,color,graphicx}

%\usepackage[small,nohug]{diagrams}
%\diagramstyle[labelstyle=\scriptstyle]

\usepackage[protrusion=true,expansion=true]{microtype}

\usepackage{lmodern}
\usepackage{tabto}

\usepackage[natbib=true,style=numeric]{biblatex}
\usepackage[babel]{csquotes}
\bibliography{lit}

\usepackage[all]{xy}

%\usepackage{hyperref}

\setlength\parskip{\medskipamount}
\setlength\parindent{0pt}

\theoremstyle{definition}
\newtheorem{defn}{Definition}
\newtheorem{bsp}[defn]{Beispiel}

\theoremstyle{plain}

\newtheorem{prop}[defn]{Proposition}
\newtheorem{ueberlegung}[defn]{Überlegung}
\newtheorem{lemma}[defn]{Lemma}
\newtheorem{kor}[defn]{Korollar}
\newtheorem{hilfsaussage}[defn]{Hilfsaussage}
\newtheorem{satz}[defn]{Satz}

\theoremstyle{remark}
\newtheorem{bem}[defn]{Bemerkung}

\clubpenalty=10000
\widowpenalty=10000
\displaywidowpenalty=10000

\newcommand{\lra}{\longrightarrow}
\newcommand{\lhra}{\ensuremath{\lhook\joinrel\relbar\joinrel\rightarrow}}
\newcommand{\thlra}{\relbar\joinrel\twoheadrightarrow}
\newcommand{\xra}[1]{\xrightarrow{#1}}

\newcommand{\A}{\mathcal{A}}
\newcommand{\Z}{\mathbb{Z}}
\newcommand{\NN}{\mathbb{N}}
\newcommand{\Q}{\mathbb{Q}}
\newcommand{\R}{\mathbb{R}}
\newcommand{\C}{\mathcal{C}}
\newcommand{\D}{\mathcal{D}}
\newcommand{\E}{\mathcal{E}}
\newcommand{\RP}{\mathbb{R}\mathrm{P}}
\newcommand{\Hom}{\mathrm{Hom}}
\newcommand{\Set}{\mathrm{Set}}
\newcommand{\Grp}{\mathrm{Grp}}
\newcommand{\Vect}{\mathrm{Vect}}
\newcommand{\Spur}[1]{\operatorname{Spur}#1}
\newcommand{\rank}[1]{\operatorname{rank}#1}
\newcommand{\sgn}[1]{\operatorname{sgn}#1}
\newcommand{\id}{\mathrm{id}}
\newcommand{\Id}{\mathrm{Id}}
\newcommand{\Aut}[1]{\operatorname{Aut}(#1)}
\newcommand{\GL}[1]{\operatorname{GL}(#1)}
\newcommand{\ORTH}[1]{\operatorname{O}(#1)}
\newcommand{\freist}{\underline{\ \ }}
\newcommand{\op}{\mathrm{op}}
\DeclareMathOperator{\rk}{rk}
\DeclareMathOperator{\Spec}{Spec}
\DeclareMathOperator{\Bild}{im}
\DeclareMathOperator{\Kern}{ker}
\DeclareMathOperator{\Int}{int}
\DeclareMathOperator{\Ob}{Ob}
\newcommand{\Zzwei}{\Z_2}

\newcommand{\XXX}[1]{\textcolor{red}{#1}}

\renewcommand*\theenumi{\alph{enumi}}
\renewcommand{\labelenumi}{\theenumi)}

\pagestyle{empty}

%\newarrow{Equals}=====

\usepackage{geometry}
\geometry{tmargin=2cm,bmargin=2cm,lmargin=3cm,rmargin=3cm}

\begin{document}

\vspace*{-4em}
\begin{flushright}Universität Augsburg \\ 20. März 2013\end{flushright}

\begin{center}\Large \textbf{Pizzaseminar zur Kategorientheorie} \\
4. Übungsblatt
\end{center}
\vspace{1.5em}

\newbox{\mybox}
\setbox\mybox=\hbox{\textbf{Aufgabe 1:}}

\begin{list}{}{\labelwidth0em \leftmargin0em \itemindent0.5em \itemsep 2.0em}
\item[\textbf{Aufgabe 1:}]
Sei~$\Id_\Set : \Set \to \Set$ der Identitätsfunktor auf~$\Set$, $P : \Set \to
\Set$ der (kovariante) Potenzmengenfunktor und~$K : \Set
\to \Set$ der Funktor
\[ \begin{array}{@{}rrcl@{}}
  & X &\longmapsto& X \times X \\
  & f &\longmapsto& f \times f := ((a,b) \mapsto (f(a),f(b))).
\end{array} \]
\begin{enumerate}
\item
Zeige: Es gibt nur eine einzige natürliche Transformation $\eta : \Id_\Set
\Rightarrow \Id_\Set$, nämlich
\[ \eta_X : X \to X,\ x \mapsto x. \]
\item
Zeige: Es gibt nur eine einzige natürliche Transformation $\omega : \Id_\Set
\Rightarrow K$, nämlich
\[ \omega_X : X \to X \times X,\ x \mapsto (x,x). \]
\emph{Tipp für a) und b):} Betrachte geeignete Abbildungen $1 \to X$, $\star
\mapsto x$.
\item
Zeige: Es gibt keine natürliche Transformation $P \Rightarrow \Id_\Set$, wohl
aber eine in die andere Richtung.
\item Wir nehmen an, dass wir für jede nichtleere Menge~$X$ ein bestimmtes
Element~$a_X \in X$ gegeben haben. Zeige:
Die Setzung
$\tau_X : X \to X,\ x \mapsto a_X$
definiert \emph{nicht} eine natürliche Transformation~$\Id_\C \Rightarrow \Id_\C$,
wobei~$\C$ die Kategorie der nichtleeren Mengen und beliebigen Abbildungen
bezeichnet.
\item
Welche natürlichen Transformationen $\Id_\C \Rightarrow \Id_\C$ gibt es,
wenn~$\C$ die Kategorie der reellen Vektorräume bezeichnet?
\end{enumerate}

\item[\textbf{Aufgabe 2:}]\mbox{}
\begin{enumerate}
\item
Sei~$F : \C \to \D$ eine Äquivalenz von Kategorien, mit Quasi-Inversem~$G : \D
\to \C$. Sei~$X$ ein Objekt von~$\C$. Zeige:
$\text{$X$ initial in~$\C$} \quad\Longleftrightarrow\quad
  \text{$F(X)$ initial in~$\D$.}$
\item
Seien nun~$X$ und~$Y$ Objekte einer Kategorie~$\E$. Zeige, dass die Kategorie der
Möchtegern-Produkte von~$X$ und~$Y$ äquivalent zur Kategorie der
Möchtegern-Produkte von~$Y$ und~$X$ ist. Welche bekannte Aussage folgt daher
mit a)?
\end{enumerate}

\item[\textbf{Aufgabe 3:}]
Sei~$\C$ die Kategorie mit
\begin{align*}
  \Ob \C &:= \{ \R^n \,|\, n \geq 0 \}, \\
  \Hom_\C(\R^n, \R^m) &:= \R^{m \times n},
\end{align*}
wobei die Morphismenverkettung durch die Matrixmultiplikation gegeben ist.
Zeige: Die Kategorie~$\C$ ist äquivalent zur Kategorie der
endlich-dimensionalen~$\R$-Vektorräume.

\emph{Tipp:} Wähle für jeden endlich-dimensionalen Vektorraum~$V$ einen Iso
$\eta_V : \R^{\dim V} \to V$.

\item[\textbf{Projektaufgabe:}]
Sei~$\varphi : A \to B$ ein Morphismus in einer lokal kleinen Kategorie~$\C$.
Bastele daraus eine natürliche Transformation der Hom-Funktoren
\[ \Hom_\C(\freist, A) \Longrightarrow \Hom_\C(\freist, B). \]
\end{list}

\end{document}
