\documentclass{pizzablatt}

\geometry{tmargin=2cm,bmargin=2cm,lmargin=2.8cm,rmargin=2.8cm}

\begin{document}

\maketitle{5}{3. April 2013}

\setlength{\aufgabenskip}{1.5em}

\begin{aufgabe}{Spitze schon im Diagramm}
Sei~$F:\D \to \C$ ein Diagramm in einer Kategorie~$\C$. Besitze~$\D$ ein
terminales Objekt~$T$.
Zeige per Hand oder mit dem Kriterium aus Aufgabe~4, dass dann
schon~$F(T)$ selbst zu einem Kolimes von~$F$ wird.
\end{aufgabe}

\begin{aufgabe}{Polynome und Potenzreihen}
Sei~$K$ ein Körper und sei~$K[X]_n$ der Vektorraum der Polynome
vom Grad~$\leq n$ mit Koeffizienten in~$K$. Bearbeite eine der folgenden Teilaufgaben:
\begin{enumerate}
\item Zeige, dass der Vektorraum~$K[X]$ zu einem Kolimes des Diagramms
\[ K[X]_0 \lhra K[X]_1 \lhra K[X]_2 \lhra \cdots \]
wird. Die Morphismen sind jeweils die Inklusionsabbildungen.
\item Zeige, dass der Vektorraum~$K\llbracket X \rrbracket := \{
\sum_{i=0}^\infty a_i X^i \,|\,
a_0, a_1, \ldots \in K \}$ der formalen
Potenzreihen in~$K$ zu einem Limes des Diagramms
\[ \cdots \lthra K[X]_2 \lthra K[X]_1 \lthra K[X]_0 \]
wird. Die Morphismen schneiden jeweils den höchsten Koeffizienten ab.
\end{enumerate}
\end{aufgabe}

\begin{aufgabe}{Monomorphe natürliche Transformationen}
\begin{enumerate}
\item Sei~$f:X \to Y$ ein Morphismus einer Kategorie. Zeige, dass~$f$ genau
dann ein Monomorphismus ist, wenn das Diagramm
\[ \xymatrix{
  \ar @{} [dr] |{\begin{array}{l}\lrcorner\ \ \ \ \ \ \\\\\end{array}}
  X \ar[r]^\id \ar[d]_\id & X \ar[d]^f \\
  X \ar[r]_f & Y
} \]
ein Faserproduktdiagramm ist (Definition im Skript).
\item Sei~$\eta : F \Rightarrow G$ eine natürliche Transformation zwischen
Funktoren $F, G : \C \to \D$. Besitze~$\D$ alle Faserprodukte. Zeige: $\eta$
ist genau dann ein Monomorphismus in~$\Funct(\C,\D)$, wenn alle
Komponenten~$\eta_X$ Monomorphismen in~$\D$ sind.

\emph{Tipp:} Limiten in Funktorkategorien berechnet man objektweise, siehe
Skript.
\end{enumerate}
\end{aufgabe}

\small
\begin{aufgabe*}{Kofinale Unterdiagramme}
In der Analysis gibt es folgende Mottos: \emph{Das Weglassen endlich vieler Folgeglieder
ändert nicht das Konvergenzverhalten. Teilfolgen konvergenter Folgen
konvergieren ebenfalls, und zwar gegen denselben Grenzwert.} Diese Mottos
wollen wir auf (Ko-)Limiten in der Kategorientheorie übertragen.

Sei dazu~$H : \D_0 \to \D$ ein \emph{kofinaler} Funktor (Definition im Skript)
und $F : \D \to \C$ ein~$\D$-förmiges Diagramm in einer Kategorie~$\C$.
\begin{enumerate}
\item
Zeige: Die Kategorie der Kokegel von~$F$ ist äquivalent zur Kategorie der
Kokegel von~$F \circ H$.
\item Was folgt daher über das Verhältnis der Kolimiten von~$F$ und~$F \circ
H$?
\end{enumerate}
\end{aufgabe*}

\end{document}
