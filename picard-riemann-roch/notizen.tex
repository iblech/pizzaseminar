\documentclass[a4paper,ngerman,12pt]{scrartcl}

\usepackage[utf8]{inputenc}

\usepackage[ngerman]{babel}

\usepackage{amsmath,amsthm,amssymb,stmaryrd,color,graphicx}
\usepackage{array}
\usepackage[all]{xy}

\usepackage[protrusion=true,expansion=true]{microtype}

\usepackage{lmodern}
\usepackage{tabto}

\usepackage{hyperref}

\theoremstyle{definition}
\newtheorem{defn}{Definition}[section]
\newtheorem{ex}[defn]{Example}

\theoremstyle{plain}

\newtheorem{prop}[defn]{Proposition}
\newtheorem{lemma}[defn]{Lemma}
\newtheorem{thm}[defn]{Theorem}
\newtheorem{cor}[defn]{Corollary}

\theoremstyle{remark}
\newtheorem{rem}[defn]{Remark}

\clubpenalty=10000
\widowpenalty=10000
\displaywidowpenalty=10000

\newcommand{\CC}{\mathbb{C}}
\newcommand{\NN}{\mathbb{N}}
\newcommand{\ZZ}{\mathbb{Z}}
\newcommand{\QQ}{\mathbb{Q}}
\newcommand{\FF}{\mathbb{F}}
\newcommand{\PP}{\mathbb{P}}
\newcommand{\C}{\mathcal{C}}
\newcommand{\E}{\mathcal{E}}
\newcommand{\F}{\mathcal{F}}
\newcommand{\G}{\mathcal{G}}
\renewcommand{\O}{\mathcal{O}}
\newcommand{\id}{\mathrm{id}}
\newcommand{\op}{\mathrm{op}}
\newcommand{\xra}[1]{\xrightarrow{#1}}
\newcommand{\Mod}{\mathrm{Mod}}
\newcommand{\Set}{\mathrm{Set}}
\newcommand{\Cat}{\mathrm{Cat}}
\newcommand{\Vect}{\mathrm{Vect}}
\newcommand{\VB}{\mathrm{VB}}
\newcommand{\Id}{\mathrm{Id}}
\newcommand{\Coh}{\mathrm{Coh}}
\newcommand{\GL}{\mathrm{GL}}
\newcommand{\pt}{\mathrm{pt}}
\newcommand{\Ob}{\operatorname{Ob}}
\newcommand{\rank}{\operatorname{rank}}
\newcommand{\Hom}{\mathrm{Hom}}
\newcommand{\ul}[1]{\underline{#1}}
\newcommand{\placeholder}{\underline{\ \ }}
\newcommand{\lra}{\longrightarrow}

\begin{document}

\title{Die Picard-Gruppe und \\ der Satz von Riemann--Roch}
\author{Ingo Blechschmidt}
\date{16. Oktober 2014}
\maketitle

\begin{center}\begin{minipage}{0.8\textwidth}
Das sind informale Notizen zum ersten Kapitel von Arnaud Beauvilles Buch
\emph{Complex Algebraic Surfaces}.
\end{minipage}\end{center}
\vspace{1em}

\tableofcontents


\section{Garbenkohomologie}

Sei~$X$ ein Raum (Schema über~$\CC$ oder komplexe Mannigfaltigkeit). Dann
untersucht man (algebraische bzw. holomorphe) Vektorbündel auf~$X$; äquivalent
dazu ist die Kategorie der lokal freien~$\O_X$-Modulgarben.\footnote{Die
Äquivalenz wird wie folgt vermittelt: Einem Vektorbündel~$E \to X$ ordnet man
die Garbe seiner algebraischen bzw. holomorphen Schnitte zu. Da man Schnitte
addieren und mit regulären bzw. holomorphen Funktionen multiplizieren kann,
wird die erhaltene Garbe zu einer Garbe von~$\O_X$-Moduln. Die lokale
Trivialität von~$E$ übersetzt sich in die lokale Freiheit der zugehörigen
Garbe. Fasern des Vektorbündels entsprechen Fasern der Garbe (das sind die
Halme, tensoriert mit dem Restklassenkörper an der jeweiligen Stelle).}
Diese ist aber keine abelsche Kategorie und daher kein geeigneter Kontext, um
Funktoren abzuleiten.

Man behilft sich mit der größeren Kategorie~$\Coh(X)$ der
\emph{kohärenten~$\O_X$-Modulgarben}. Das ist eine volle Unterkategorie der
Kategorie aller~$\O_X$-Modulgarben, welche abelsch ist. Im Fall, dass~$X$ ein
lokal noethersches Schema oder eine komplexe Mannigfaltigkeit ist, umfasst
diese Kategorie die Kategorie der lokal freien Garben und ist die kleinste
volle Unterkategorie mit dieser Eigenschaft. Kohärenz ist eine
Endlichkeitseigenschaft, die im Allgemeinen stärker ist, als von endlichem Typ
zu sein.\footnote{Eine~$\O_X$-Modulgarbe~$\E$ heißt genau dann \emph{kohärent},
wenn sie von endlichem Typ ist und wenn für jede offene Teilmenge~$U \subseteq
X$ der Kern eines jeden~$\O_X|_U$-linearen Morphismus~$(\O_X|_U)^n \to \E|_U$
ebenfalls von endlichem Typ ist. Die Kohärenz von~$\O_X$ für den Fall einer
komplexen Mannigfaltigkeit ist die Aussage des Kohärenzsatzes von Oka (XXX Quelle).} Die
Faserdimension einer kohärenten Garbe kann -- anders als bei lokal freien
Garben -- von Punkt zu Punkt variieren.

Von zentraler Bedeutung ist der Funktor~$\Gamma(X,\placeholder) : \Coh(X) \to
\Vect(\CC)$, welcher nur linksexakt, aber in allen interessanten Fällen nicht
exakt ist und daher abgeleitet werden muss. Damit definiert man
\emph{Garbenkohomologie}: Die~$n$-te Kohomologie einer kohärenten Garbe~$\E$
ist $H^n(X, \E) := R^n \Gamma(X,\placeholder)(\E)$.\footnote{Das ist etwas
gemogelt. Der Kategorie~$\Coh(X)$ mangelt es im Allgemeinen an genügend
Injektiven, weswegen man zu größeren Kategorien übergeht.}

Gewöhnliche (singuläre) Kohomologie erhält man aus dieser Definition zurück,
wenn man für~$\E$ konstante Garben verwendet.\footnote{Eine Garbe heißt genau
dann \emph{konstant}, wenn sie die Garbifizierung einer konstanten Prägarbe ist
-- einer solchen, die jeder offenen Teilmenge des Raums dieselbe Menge~$A$
zuordnet. Explizit ist die Menge der~$U$-Schnitte einer solchen konstanten Garbe
die Menge der stetigen Funktionen~$U \to A$, wobei~$A$ mit der diskreten
Topologie versehen wird. Konstante Garben sind nicht kohärent -- sie tragen
nicht einmal eine~$\O_X$-Modulstruktur -- durch Übergang von~$\Coh(X)$ zur
Kategorie aller Garben abelscher Gruppen kann man aber auch für solche Garben
Kohomologie erklären.} Im klassischen Fall spielt die Wahl der Koeffizienten
($\ZZ$, $\QQ$, \ldots) dank des universellen Koeffiziententheorems keine große
Rolle; das ist bei Garbenkohomologie nicht so. Garbenkohomologie kann man sich
\emph{nicht} über "`Zykel modulo Ränder"' anschaulich vorstellen, geometrische
Vorstellung ist bedingt aber durch \emph{Čech-Methoden} gegeben (XXX Quelle).

Garbenkohomologie hat drei für uns sehr wesentliche Eigenschaften:
\begin{enumerate}
\item Falls~$X$ eigentlich über dem Punkt bzw. kompakt ist, so sind
die~$H^n(X,\E)$ \emph{endlich-dimensionale} Vektorräume\footnote{Dass das
nicht immer so ist, zeigt schon das Beispiel~$H^0(\CC, \O_\CC)$, der
unendlich-dimensionale Raum der holomorphen Funktionen auf~$\CC$.} und für~$i > \dim X$
verschwindet die~$i$-ten Kohomologie. (XXX Quelle)
\item Ist~$0 \to \E' \to \E \to \E'' \to 0$ eine kurze exakte Sequenz von
kohärenten Garben, so erhält man eine lange exakte Sequenz in Kohomologie:
\[ \cdots \lra H^n(X,\E') \lra H^n(X,\E) \lra H^n(X,\E'') \lra H^{n+1}(X,\E)
\lra \cdots \]
\item Die \emph{Eulercharakteristik} einer kohärenten Garbe~$\E$, definiert
als die alternierende Summe~$\chi(\E) := \sum_{n=0}^\infty \dim_\CC H^n(X,\E)
\in \ZZ$, ist additiv in kurzen exakten Sequenzen. Deshalb hängt~$\chi(\E)$ nur
von der \emph{Klasse von~$\E$ in der K-Theorie ab}, also nur von
\emph{diskreten Invarianten} von~$\E$.\footnote{Außerdem ist die
Eulercharakteristik auf flachen Familien von kohärenten Garben konstant (XXX
Quelle). Die K-Theorie von~$X$ ist die abelsche Gruppe
formaler~$\ZZ$-Linearkombinationen von Isomorphieklassen von kohärenten Garben
auf~$X$ modulo den Relationen~$\E = \E' + \E''$ für jede kurze exakte
Sequenz~$0 \to \E' \to \E \to \E'' \to 0$.  Die K-Theorie von~$\CC$ ist
isomorph zu~$\ZZ$ (mit Isomorphismus~$\E \mapsto \rank\E$), die von~$\PP^n$ ist
isomorph zu~$\ZZ[X]/(X+1)^{n+1}$. (XXX Quelle)}
\end{enumerate}

Ein Grund, Garbenkohomologie statt klassischer Kohomologie zu studieren, ist
schlichtweg der, dass höhere Kohomologie mit Werten in konstanten Garben auf
irreduziblen Schemata stets trivial ist. Das liegt daran, dass auf irreduziblen
topologischen Räumen konstante Garben stets welk (flabby) und daher azyklisch
bezüglich des globale-Schnitte-Funktors ist.

Ein weiterer Grund liegt darin, dass man oft an der Dimension des globalen
Schnittraums einer kohärenten Garbe~$\E$ interessiert ist, also an~$\dim_\CC
H^0(X,\E)$. Diese ist im Allgemeinen aber nicht leicht zu berechnen. Die
Eulercharakteristik dagegen, in der diese Dimension als ein Summand auftritt,
ist dank ihrer Stabilitätseigenschaften leichter zugänglich. Im Fall, dass~$\E$
das zu einem Divisor~$D$ assoziierte Geradenbündel ist (siehe Abschnitt~2), ist
die Frage nach der Dimension des globalen Schnittraums die sehr konkrete Frage
nach der Dimension des Raums meromorpher Funktionen mit durch~$D$ vorgegebenem
Null- und Polstellenverhalten.

Historisch war auch die \emph{Leray-Spektralsequenz} eine große Motivation,
Garbenkohomologie zu untersuchen. Diese liefert im Kontext einer stetigen
Abbildung~$f : X \to Y$ eine Möglichkeit, aus Kenntnis der (gewöhnlichen)
Kohomologie von~$Y$ und der Fasern~$f^{-1}[y]$ Rückschlüsse auf die Kohomologie
von~$X$ zu ziehen; dabei kommt unweigerlich Garbenkohomologie vor.\footnote{Nur
im Fall, dass~$f$ eine Faserung ist, kann man noch auf den Speziallfall der
Leray--Serre-Spek\-tral\-sequenz ausweichen, bei der man nur Kohomologie mit Werten
in lokalen Systemen benötigt.}

Als letzter Grund sei angeführt, dass Garbenkohomologie geometrische Objekte
klassiziferen kann. Etwa stehen die Elemente von~$H^1(X,\O_X^\times)$ auf
kanonische Art und Weise mit den Geradenbündeln auf~$X$ (bis auf Isomorphie) in
Eins-zu-Eins-Korrespondenz.\footnote{Geradenbündel kann man auch
als~$\GL_1(\O_X)$-Hauptfaserbündel ansehen. Allgemeiner
klassifiziert~$H^1(X,\GL_n(\O_X))$ Vektorbündel vom Rang~$n$
und~$H^1(X,\G)$~$G$-Hauptfaserbündel.}

\end{document}
