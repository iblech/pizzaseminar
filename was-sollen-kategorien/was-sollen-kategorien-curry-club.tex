\documentclass[12pt,utf8,notheorems,compress]{beamer}
\usepackage{etex}

\usepackage[ngerman]{babel}

\usepackage{amsmath,amssymb}
%\usepackage[framed,amsmath,thmmarks,hyperref]{ntheorem}

%\usepackage[small,nohug]{diagrams}
%\diagramstyle[labelstyle=\scriptstyle]

%\usepackage[protrusion=true,expansion=false]{microtype}

%\usepackage{lmodern}
\usepackage{tabto}
\usepackage{tikz}
\usepackage{array}
\usepackage[all]{xy}

%\usepackage[natbib=true,style=numeric]{biblatex}
%\usepackage[babel]{csquotes}
%\bibliography{lit}

%\usepackage{hyperref}

\setlength\parskip{\medskipamount}
\setlength\parindent{0pt}

%\theoremseparator{:}
\theoremstyle{plain}  %nonumberplain
%\newtheorem{beh}{Behauptung}
\newtheorem{proposition}{Proposition}
\newtheorem{corollary}{Korollar}
\newtheorem{theorem}{Satz}
\theoremstyle{definition}
\newtheorem{definition}{Definition}
%\newtheorem{kor}{Korollar}
%\newtheorem{satz}{Satz}
%\newtheorem{lemma}{Lemma}
%\newtheorem{hilfsaussage}{Hilfsaussage}
%\theorembodyfont{\normalfont}
\newtheorem{axiom}{Axiom}
%\newtheorem{defnprop}{Definition/Proposition}
%\newtheorem{bem}{Bemerkung}
%\newtheorem{bsp}{Beispiel}
%\theoremsymbol{\ensuremath{\openbox}}
%\newtheorem{proof}{Beweis}
%\newtheorem{defn}{Definition}

\newcommand{\lra}{\longrightarrow}
\newcommand{\lhra}{\ensuremath{\lhook\joinrel\relbar\joinrel\rightarrow}}
\newcommand{\thlra}{\relbar\joinrel\twoheadrightarrow}

\newcommand{\Z}{\mathbb{Z}}
\renewcommand{\C}{\mathcal{C}}
\newcommand{\N}{\mathbb{N}}
\newcommand{\R}{\mathbb{R}}
\newcommand{\Hom}{\mathrm{Hom}}
\newcommand{\id}{\mathrm{id}}
\newcommand{\Aut}[1]{\operatorname{Aut}(#1)}
\newcommand{\GL}[1]{\operatorname{GL}(#1)}
\newcommand{\freist}{\_{}\_{}}
\newcommand{\Set}{\mathrm{Set}}
\newcommand{\Hask}{\mathrm{Hask}}
\newcommand{\Grp}{\mathrm{Grp}}
\newcommand{\Vect}{\mathrm{Vect}}
\newcommand{\fst}{\mathrm{fst}}
\newcommand{\snd}{\mathrm{snd}}

\def\longleadsto{\mathrel{-}\joinrel\leadsto}
\DeclareMathOperator{\ggT}{ggT}
\DeclareMathOperator{\Ob}{Ob}
\newcommand{\op}{\mathrm{op}}

\title{Was sind und was sollen Kategorien?}
\author[Curry Club Augsburg]{%
  \includegraphics[scale=0.4]{relationen.png} \\\bigskip
  \footnotesize Ingo Blechschmidt}
\date{18. Juni 2015}

\usetheme{Warsaw}
\useoutertheme{split}
\usecolortheme{seahorse}
\usefonttheme{serif}
\usepackage{palatino}
\useinnertheme{rectangles}

\setbeamertemplate{navigation symbols}{}
%\setbeamertemplate{footline}{}
%\setbeamertemplate{headline}{}

\setbeamertemplate{title page}[default][colsep=-1bp,rounded=false,shadow=false,bg=white]
\setbeamertemplate{frametitle}[default][colsep=-2bp,rounded=false,shadow=false,center]
\setbeamertemplate{footline}{%
  \leavevmode%
  \hfill%
  \begin{beamercolorbox}[ht=2.25ex,dp=1ex,right]{}%
    \usebeamerfont{date in head/foot}
    \insertframenumber\,/\,\inserttotalframenumber\hspace*{1ex}
  \end{beamercolorbox}%
  \vskip0pt%
}


%\beamertemplateboldcenterframetitle
%\setbeamerfont{frametitle}{size={\Large}}

\newcommand*\oldmacro{}%
\let\oldmacro\insertshorttitle%
\renewcommand*\insertshorttitle{%
  \oldmacro\hfill\insertframenumber\,/\,\inserttotalframenumber\hfill}

\newenvironment{changemargin}[2]{%
  \begin{list}{}{%
    \setlength{\topsep}{0pt}%
    \setlength{\leftmargin}{#1}%
    \setlength{\rightmargin}{#2}%
    \setlength{\listparindent}{\parindent}%
    \setlength{\itemindent}{\parindent}%
    \setlength{\parsep}{\parskip}%
  }%
  \item[]}{\end{list}}

\newcommand{\slogan}[1]{%
  \begin{center}%
    \setlength{\fboxrule}{2pt}%
    \setlength{\fboxsep}{-3pt}%
    {\usebeamercolor[fg]{item}\fbox{\usebeamercolor[fg]{normal
    text}\parbox{0.9\textwidth}{\begin{center}#1\end{center}}}}%
  \end{center}%
}

\newcommand{\hil}[1]{{\usebeamercolor[fg]{item}{#1}}}

\begin{document}

\setbeameroption{show notes}
\setbeamertemplate{note page}[plain]

\frame{\titlepage}
\frame[t]{\frametitle{Gliederung}\tableofcontents}

\section[Motivation]{Motivation: Beispiele für kategorielles Verständnis}

\subsection{Produkte}
\frame[t]{\frametitle{Produkte in Kategorien I}
  \begin{itemize}
    \item Kartesisches Produkt von Mengen: $X \times Y$
    \item Kartesisches Produkt von Vektorräumen: $V \times W$
    \item Kartesisches Produkt von Gruppen: $G \times H$
    \item Minimum von Zahlen: $\min\{n,m\}$
    \item Größter gemeinsamer Teiler von Zahlen: $\ggT(n,m)$
    \item Paartyp in Programmiersprachen: \texttt{(a,b)}
    \item Mutterknoten zweier Knoten in einem Graph
  \end{itemize}

  \slogan{%
    All dies sind Spezialfälle des allgemeinen \\ \emph{kategoriellen Produkts}.
  }

  \begin{tikzpicture}[remember picture,overlay]  
    \node [xshift=-1cm,yshift=-8.5cm] at (current page.north east)
      {\includegraphics[scale=0.4]{produkt.png}};
  \end{tikzpicture}
}

\frame[t]{\frametitle{Produkte in Kategorien II}
  \begin{align*}
    X \times (Y \times Z) &\cong (X \times Y) \times Z \\
    U \times (V \times W) &\cong (U \times V) \times W \\
    \min\{m,\min\{n,p\}\} &= \min\{\min\{m,n\},p\} \\
    \ggT(m,\ggT(n,p)) &= \ggT(\ggT(m,n),p)
  \end{align*}

  \slogan{%
    All dies sind Spezialfälle der allgemeinen \\ \emph{Assoziativität} des kategoriellen Produkts.
  }

  \begin{tikzpicture}[remember picture,overlay]  
    \node [xshift=-1cm,yshift=-7.5cm] at (current page.north east)
      {\includegraphics[scale=0.4]{produkt.png}};
  \end{tikzpicture}
}

\note{
  \begin{itemize}
    \item Die Mengen~$X \times (Y \times Z)$ und~$(X \times Y) \times Z$ sind
    nicht im Wortlaut gleich. Sie sind aber \emph{isomorph}: Es gibt eine
    Abbildung~$f$ von links nach rechts, und diese Abbildung besitzt eine
    Umkehrabbildung~$g$ von rechts nach links, sodass~$g \circ f$ und~$f \circ
    g$ jeweils die Identitätsabbildungen sind.
    \item In Haskell-Notation lassen sich~$f$ und~$g$ wie folgt angeben:

    \texttt{f :: (X,(Y,Z)) -> ((X,Y),Z)} \\
    \texttt{f (x,(y,z)) = ((x,y),z)} \\\ \\
    \texttt{g :: ((X,Y),Z) -> (X,(Y,Z)) } \\
    \texttt{g ((x,y),z) = (x,(y,z))}
  \end{itemize}
}

\subsection{Isomorphismen}
\frame[t]{\frametitle{Isomorphismen in Kategorien}
  \begin{itemize}
    \item Zwei Mengen $X,Y$ \tabto{4.63cm} können gleichmächtig sein.
    \item Zwei Vektorräume $V,W$ \tabto{4.63cm} können isomorph sein.
    \item Zwei Gruppen $G,H$ \tabto{4.63cm} können isomorph sein.
    \item Zwei top. Räume $X,Y$ \tabto{4.63cm} können homöomorph sein.
    \item Zwei Zahlen $n,m$ \tabto{4.63cm} können gleich sein.
    \item Zwei Typen \texttt{a}, \texttt{b} \tabto{4.63cm} können sich verlustfrei ineinander umwandeln lassen.
  \end{itemize}

  \slogan{%
    All dies sind Spezialfälle des allgemeinen \\ \emph{kategoriellen
    Isomorphiekonzepts}.
  }

  \begin{tikzpicture}[remember picture,overlay]  
    \node [xshift=-1.1cm,yshift=-8.0cm] at (current page.north east)
      {\includegraphics[scale=0.4]{isomorphie.png}};
  \end{tikzpicture}
}

\subsection{Dualität}
\frame[t]{\frametitle{Dualität}
  \vspace{-0.5em}
  \only<1>{
    \begin{center}
      \setlength{\extrarowheight}{0.3em}
      \begin{tabular}{r|l}
        $f \circ g$ & $g \circ f$ \\
        $\leq$ & $\geq$ \\
        injektiv & surjektiv \\
        $\{\star\}$ & $\emptyset$ \\
        $\times$ & $\amalg$ \\
        ggT & kgV \\
        $\cap$ & $\cup$ \\
        Teilmenge & Faktormenge
      \end{tabular}
    \end{center}
  }
  \only<2>{
    \begin{center}
      \setlength{\extrarowheight}{0.3em}
      \begin{tabular}{r|l}
        \texttt{(a,b)} & \texttt{Either a b} \\
        Typ der Streams & Typ der endlichen Listen \\
        Monaden & Komonaden \\
        Rechts-Kan-Erweiterung & Links-Kan-Erweiterung
      \end{tabular}
    \end{center}
  }

  \slogan{%
    All dies sind Spezialfälle eines allgemeinen \\
    \emph{kategoriellen Dualitätsprinzips}.
  }

  \begin{tikzpicture}[remember picture,overlay]  
    \node [xshift=-1cm,yshift=-7.5cm] at (current page.north east)
      {\includegraphics[scale=0.4]{dualitaet.png}};
  \end{tikzpicture}
}

\note{
  \begin{itemize}
    \item Jedes allgemeine kategorielle Resultat über ein Konzept liefert
    automatisch auch ein Resultat für das zugehörige duale Konzept.
    \item Wenn man etwa einmal nachgewiesen hat, dass Produkte stets bis auf
    Isomorphie assoziativ sind -- das heißt
    \[ X \times (Y \times Z) \cong (X \times Y) \times Z, \]
    so folgt automatisch die duale Aussage für Koprodukte:
    \[ X \amalg (Y \amalg Z) \cong (X \amalg Y) \amalg Z. \]
  \end{itemize}
}

\section{Grundlagen}

\subsection[Def.]{Definition des Kategorienbegriffs}
\frame[t]{\frametitle{Kategorien}%
  \begin{tikzpicture}[remember picture,overlay]
    \node [xshift=-2cm,yshift=-2cm] at (current page.north east)
      {\includegraphics[scale=0.3]{kategorie.png}};
  \end{tikzpicture}%
  \small
  \textbf{Definition:} Eine Kategorie~$\C$ besteht aus
  \begin{enumerate}
    \item einer Klasse von \emph{Objekten} $\Ob \C$,
    \item zu je zwei Objekten $X,Y \in \Ob \C$ einer Klasse $\Hom_\C(X,Y)$ von
    \emph{Morphismen} zwischen ihnen und
    \item einer Kompositionsvorschrift:
    \begin{align*}
      \text{zu }\ & f \in \Hom_\C(X,Y) &
      \qquad\text{zu }\ & f : X \to Y \\
      \text{und }\ & g\in\Hom_\C(Y,Z) &
      \qquad\text{und }\ & g : Y \to Z \\
      \text{habe }\ & g\circ f\in\Hom_\C(X,Z), &
      \qquad\text{habe }\ & g\circ f : X \to Z,
    \end{align*}
  \end{enumerate}
  sodass
  \begin{enumerate}
    \item die Komposition $\circ$ assoziativ ist: $f \circ (g \circ h) = (f
    \circ g) \circ h$, und
    \item es zu jedem $X \in \Ob\C$ einen Morphismus $\id_X
    \in \Hom_\C(X,X)$ mit~$f \circ \id_X = f$ und~$\id_X \circ g = g$.
  \end{enumerate}
}

\note{
  \begin{itemize}
    \item Die Morphismen müssen nicht unbedingt Abbildungen
    sein. Die Schreibweise "`$f:X \to Y$"' missbraucht also Notation.
    \item Archetypisches Beispiel ist $\Set$, die Kategorie der Mengen und Abbildungen:
    \begin{align*}
      \Ob \Set &:= \{ M \,|\, \text{$M$ ist eine Menge} \} \\
      \Hom_\Set(X,Y) &:= \{ f:X \to Y \,|\, \text{$f$ ist eine Abbildung} \}
    \end{align*}
    \item Die meisten Teilgebiete der Mathematik studieren jeweils eine bestimmte
    Kategorie: Gruppentheoretiker beschäftigen sich etwa mit der Kategorie
    $\Grp$ der Gruppen und Gruppenhomomorphismen:
    \begin{align*}
      \Ob \Grp &:= \text{Klasse aller Gruppen} \\
      \Hom_\Grp(G,H) &:= \{ f:G \to H \,|\, \text{$f$ ist ein Gruppenhomo} \}
    \end{align*}
  \end{itemize}
}

\note{
  \begin{itemize}
    \item Es gibt aber auch wesentlich kleinere Kategorien. Etwa kann man aus
    jeder Partialordnung~$(P,\preceq)$ eine Kategorie~$\C$ basteln:
    \begin{align*}
      \Ob \C &:= P \\
      \Hom_\C(x,y) &:= \begin{cases}
        \text{einelementige Menge}, & \text{falls $x \preceq y$,} \\
        \text{leere Menge}, & \text{sonst}
      \end{cases}
    \end{align*}

    \item Auch sind gewisse endliche Kategorien bedeutsam, etwa die durch
    folgende Skizze gegebene:

    \[ \xymatrix{
      & \bullet \ar[d] \ar@(ur,ul) \\
      \bullet \ar[r] \ar@(ul,dl) & \bullet \ar@(dr,ur)
    } \]
  \end{itemize}
}

\note{
  Gleichungen zwischen Morphismen schreibt man gerne als kommutative
  Diagramme:
  \[ h = g \circ f
    \quad\quad\Longleftrightarrow:\quad\quad
    \vcenter{\xymatrix{
      X \ar[rr]^f \ar[ddr]_h & & Y \ar[ddl]^g \\
      & \% \\
      & Z
    }}
  \]
}


\frame[t]{\frametitle{Fundamentales Motto}
  \slogan{%
    Kategorientheorie stellt \emph{Beziehungen zwischen Objekten} statt
    etwaiger innerer Struktur in den Vordergrund.}

  \begin{center}
    \includegraphics[scale=0.3]{relationen.png}
  \end{center}
}

\subsection[Init. u. term. Obj.]{Initiale und terminale Objekte}
\frame[t]{\frametitle{Initiale und terminale Objekte}
  \textbf{Definition:} Ein Objekt~$X$ einer Kategorie~$\C$ heißt genau dann
  \begin{itemize}
    \item \emph{initial}, wenn
      \[ \forall Y \in \Ob \C{:}\ \exists! f : X \to Y. \]
    \item \emph{terminal}, wenn
      \[ \forall Y \in \Ob \C{:}\ \exists! f : Y \to X. \]
  \end{itemize}

  \vfill
  \only<1>{\textbf{Frage:} Was ist ein terminales Objekt in~$\Set$?}
  \only<2>{%
    In $\Set$: \tabto{2.15cm} $\emptyset$ initial, $\{\star\}$ terminal.

    In $\Hask$: \tabto{2.15cm} \texttt{Void} initial, \texttt{()} terminal.}
}

\subsection[Monos u. Epis]{Mono- und Epimorphismen}
\frame[t]{\frametitle{Mono- und Epimorphismen}
  \textbf{Definition:}
  Ein Morphismus $f:X \to Y$ einer Kategorie~$\C$ heißt genau dann
  \begin{itemize}
    \item \emph{Monomorphismus}, \tabto{3.35cm}wenn für alle Objekte~$A \in \Ob \C$ \\
    \tabto{3.35cm}und $p,q:A \to X$ gilt:
    \[ f \circ p = f \circ q \quad\Longrightarrow\quad p = q. \]
    \item \emph{Epimorphismus}, \tabto{3.35cm}wenn für alle Objekte~$A \in \Ob \C$ \\
    \tabto{3.35cm}und $p,q:Y \to A$ gilt:
    \[ p \circ f = q \circ f \quad\Longrightarrow\quad p = q. \]
  \end{itemize}

  \textbf{Beobachtung} in $\Set$:
  \begin{align*}
    \text{$f$ Mono} &\Longleftrightarrow \text{$f$ injektiv.} \\
    \text{$f$ Epi} &\Longleftrightarrow \text{$f$ surjektiv.}
  \end{align*}
}

\subsection[Duale Kat.]{Die duale Kategorie einer Kategorie}
\frame[t]{\frametitle{Duale Kategorie}%
  \begin{itemize}
    \item \textbf{Definition:} Zu jeder Kategorie~$\C$ gibt es eine zugehörige
    \emph{duale Kategorie} $\C^\op$:
    \begin{align*}
      \Ob \C^\op &:= \Ob \C \\
      \Hom_{\C^\op}(X,Y) &:= \Hom_\C(Y,X)
    \end{align*}
    \item \textbf{Beispiel:} $\quad$ $X$ in $\C^\op$ initial
    $\quad\Longleftrightarrow\quad$ $X$ in $\C$ terminal
    \item \textbf{Beispiel:} $\quad$ $f$ in $\C^\op$ Mono $\quad\Longleftrightarrow\quad$ $f$ in $\C$ Epi
    \item \textbf{Nichttriviale Frage:} Wie kann man in konkreten
    Fällen~$\C^\op$ explizit (inhaltlich) beschreiben?
  \end{itemize}

  \begin{center}
    \includegraphics[scale=0.3]{kategorie.png}\quad
    \includegraphics[scale=0.3]{kategorie-dual.png}
  \end{center}
}

\subsection[Produkte]{Produkte in Kategorien}
\frame[t,shrink]{\frametitle{Produkte in Kategorien}%
  \textbf{Definition:} Ein \emph{Produkt} zweier Objekte~$X$
  und~$Y$ in einer Kategorie ist ein
  \begin{itemize}
    \item Objekt~$P$
    \item zusammen mit Morphismen~$\fst : P \to X$, $\snd : P \to Y$
  \end{itemize}
  sodass
  \begin{itemize}
    \item für jedes Objekt~$Q$
    \item und Morphismen~$\fst' : Q \to X$,~$\snd' : Q \to Y$
  \end{itemize}
  genau ein Morphismus~$f : Q \to P$ existiert, sodass das Diagramm
  kommutiert.
  \[ \xymatrix{
    & Q \ar@/_/[ldd]_{\fst'} \ar@/^/[rdd]^{\snd'} \ar@{-->}[d]^f \\
    & P \ar[ld]^\fst \ar[rd]_\snd \\
    X && Y
  } \]
  \vspace{-2em}
}

\section{Anwendungen}
\frame[t]{\frametitle{Anwendungen}%
  \begin{itemize}
    \item Kategorientheorie liefert einen Leitfaden, \\ um richtige Definitionen
    zu formulieren.

    \item Triviales wird \emph{trivialerweise} trivial: \\
    \hil{Allgemeiner abstrakter Nonsens.}

    \item Konzeptionelle Vereinheitlichung: Viele Konstruktionen in der
    Mathematik und Informatik sind Spezialfälle von allgemeinen kategoriellen: \\
    \hil{Limiten, Kolimiten, adjungierte Funktoren}

    \item Forschungsprogramm der Kategorifizierung,
          um tiefere Gründe für Altbekanntes zu finden.
  \end{itemize}

  \begin{tikzpicture}[remember picture,overlay]
    \node [xshift=-1cm,yshift=-3cm] at (current page.north east)
      {\includegraphics[scale=0.3]{nonsens.png}};
  \end{tikzpicture}%
}

\end{document}

http://homepages.inf.ed.ac.uk/jcheney/presentations/ct4d1.pdf
