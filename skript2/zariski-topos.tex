\documentclass[a4paper,ngerman,12pt]{scrartcl}

\usepackage[utf8]{inputenc}

\usepackage[ngerman]{babel}
\addto\captionsngerman{\renewcommand\tablename{Tafel}}

\usepackage{amsmath,amsthm,amssymb,stmaryrd,color,graphicx}
\usepackage{setspace}
\usepackage{bussproofs}
\usepackage{array}
\usepackage{comment}

\usepackage[protrusion=true,expansion=true]{microtype}

\usepackage{lmodern}
\usepackage{tabto}

\usepackage[all]{xy}

\usepackage{tikz}
\usetikzlibrary{calc,shapes.callouts,shapes.arrows}
\newcommand{\hcancel}[5]{%
    \tikz[baseline=(tocancel.base)]{
        \node[inner sep=0pt,outer sep=0pt] (tocancel) {#1};
        \draw[red, line width=0.5mm] ($(tocancel.south west)+(#2,#3)$) -- ($(tocancel.north east)+(#4,#5)$);
    }%
}

\usepackage{hyperref}

\setlength\parskip{\medskipamount}
\setlength\parindent{0pt}

\theoremstyle{definition}
\newtheorem{defn}{Definition}[section]
\newtheorem{axiom}[defn]{Axiom}
\newtheorem{bsp}[defn]{Beispiel}

\theoremstyle{plain}

\newtheorem{prop}[defn]{Proposition}
\newtheorem{motto}[defn]{Motto}
\newtheorem{wunder}[defn]{Wunder}
\newtheorem{ueberlegung}[defn]{Überlegung}
\newtheorem{lemma}[defn]{Lemma}
\newtheorem{kor}[defn]{Korollar}
\newtheorem{hilfsaussage}[defn]{Hilfsaussage}
\newtheorem{satz}[defn]{Satz}

\theoremstyle{remark}
\newtheorem{bem}[defn]{Bemerkung}
\newtheorem{aufg}[defn]{Aufgabe}

\clubpenalty=10000
\widowpenalty=10000
\displaywidowpenalty=10000

\newcommand{\xra}[1]{\xrightarrow{#1}}
\newcommand{\lra}{\longrightarrow}
\newcommand{\lhra}{\ensuremath{\lhook\joinrel\relbar\joinrel\rightarrow}}
\newcommand{\thlra}{\relbar\joinrel\twoheadrightarrow}

\newcommand{\brak}[1]{\llbracket {#1} \rrbracket}

\newcommand{\ZZ}{\mathbb{Z}}
\newcommand{\QQ}{\mathbb{Q}}
\newcommand{\RR}{\mathbb{R}}
\newcommand{\CC}{\mathbb{C}}
\newcommand{\NN}{\mathbb{N}}
\newcommand{\PP}{\mathbb{P}}
\renewcommand{\aa}{\mathfrak{a}}
\newcommand{\bb}{\mathfrak{b}}
\newcommand{\pp}{\mathfrak{p}}
\newcommand{\mm}{\mathfrak{m}}
\newcommand{\I}{\mathcal{I}}
\newcommand{\J}{\mathcal{J}}
\newcommand{\C}{\mathcal{C}}
\newcommand{\D}{\mathcal{D}}
\newcommand{\E}{\mathcal{E}}
\newcommand{\F}{\mathcal{F}}
\newcommand{\G}{\mathcal{G}}
\newcommand{\U}{\mathcal{U}}
\renewcommand{\I}{\mathcal{I}}
\renewcommand{\P}{\mathcal{P}}
\renewcommand{\O}{\mathcal{O}}
\newcommand{\Hom}{\mathrm{Hom}}
\newcommand{\Ouv}{\mathrm{Ouv}}
\newcommand{\res}{\mathrm{res}}
\newcommand{\Rad}{\mathrm{Rad}}
\newcommand{\Sh}{\mathrm{Sh}}
\newcommand{\PSh}{\mathrm{PSh}}
\newcommand{\Bohr}{\mathrm{Bohr}}
\newcommand{\pt}{\mathrm{pt}}
\newcommand{\ev}{\mathrm{ev}}
\newcommand{\id}{\mathrm{id}}
\newcommand{\Id}{\mathrm{Id}}
\newcommand{\freist}{\underline{\ \ }}
\newcommand{\ul}[1]{\underline{#1}}
\newcommand{\csalgebra}{C\textsuperscript{*}\kern-.1ex-Algebra}
\newcommand{\csalgebren}{C\textsuperscript{*}\kern-.1ex-Alge\-bren}
\DeclareMathOperator{\colim}{colim}
\DeclareMathOperator{\Ob}{Ob}
\DeclareMathOperator{\ggT}{ggT}
\DeclareMathOperator{\im}{im}
\DeclareMathOperator{\Quot}{Quot}
\DeclareMathOperator{\Spec}{Spec}
\DeclareMathOperator{\interior}{int}
\newcommand{\op}{\mathrm{op}}
\newcommand{\Set}{\mathrm{Set}}
\newcommand{\Grp}{\mathrm{Grp}}
\newcommand{\Vect}[1]{{#1\text{-}\mathrm{Vect}}}
\newcommand{\AbGrp}{\mathrm{AbGrp}}
\newcommand{\Ring}{\mathrm{Ring}}
\newcommand{\Cat}{\mathrm{Cat}}
\newcommand{\Funct}{\mathrm{Funct}}
\newcommand{\Eins}{\mathbf{1}}
\newcommand{\Man}{\mathrm{Man}}
\newcommand{\Top}{\mathrm{Top}}
\newcommand{\seq}[1]{\mathrel{\vdash\!\!\!_{#1}}}
\renewcommand{\_}{\mathpunct{.}\,}
\newcommand{\?}{\,{:}\,}

\newcommand{\hilight}[2]{\begin{center}\framebox{#2}\par#1\end{center}}

\newcommand{\speak}[1]{\ulcorner\text{#1}\urcorner}

\renewcommand*\theenumi{\alph{enumi}}
\renewcommand{\labelenumi}{\theenumi)}

\newcommand\subsubsubsection[1]{\subsubsection*{#1}}
\definecolor{grey}{rgb}{0.7,0.7,0.7}

\setcounter{tocdepth}{2}

\newenvironment{indentblock}{%
  \list{}{\leftmargin\leftmargin}%
  \item\relax
}{%
  \endlist
}

\newlength{\aufgabenskip}
\setlength{\aufgabenskip}{1.5em}
\newcounter{aufgabennummer}
\newenvironment{aufgabe}[1]{
  \addtocounter{aufgabennummer}{1}
  \textbf{Aufgabe \theaufgabennummer{}.} \emph{#1} \par
}{\vspace{\aufgabenskip}}

\begin{document}

\title{Spiel und Spaß mit der internen Welt des kleinen Zariski-Topos}
%\author{Ingo Blechschmidt}
\date{13. Dezember 2013}
\maketitle

%\begin{center}\begin{minipage}{0.8\textwidth}
%Zu einem kommutativen Ring ist der \emph{kleine
%Zariski-Topos} ein alternatives Mathematik-Universum, in dem der Ring wie ein
%lokaler Ring aussieht.\end{minipage}\end{center}

\newcommand{\Ll}{:\Longleftrightarrow}
\hilight{Die Kripke-Joyal-Semantik des kleinen Zariski-Topos.}{%
  $\renewcommand{\arraystretch}{1.3}\begin{array}{@{}lcl@{}}
    R \models x = y \? \O &\Ll&
      \text{Für die gegebenen Elemente $x, y \in R$ gilt $x = y$.} \\
    R \models \top &\Ll&
      1 = 1 \in R. \text{ (Das ist stets erfüllt.)} \\
    R \models \bot &\Ll&
      1 = 0 \in R. \text{ (Das ist genau in Nullringen erfüllt.)} \\
    R \models \phi \wedge \psi &\Ll&
      \text{$R \models \phi$ und $R \models \psi$.} \\
    R \models \phi \vee \psi &\Ll&
      \hcancel{\text{$R \models \phi$ oder $R \models \psi$.}}{0pt}{3pt}{0pt}{-2pt} \\
    R \models \phi \vee \psi &\Ll&
      \text{Es gibt eine Zerlegung $\sum_i s_i = 1 \in R$ sodass} \\
    &&{\quad\quad} \text{für alle~$i$ jeweils $R[s_i^{-1}] \models \phi$
      oder $R[s_i^{-1}] \models \psi$.} \\
    R \models \phi \Rightarrow \psi &\Ll&
      \text{Für jedes~$s \in R$ gilt: Aus $R[s^{-1}] \models \phi$ folgt $R[s^{-1}] \models \psi$.} \\
    R \models \forall x\?\O\_ \phi &\Ll&
      \text{Für jedes~$s \in R$ und jedes $x \in R[s^{-1}]$
      gilt: $R[s^{-1}] \models \phi(x)$.} \\
    R \models \exists x\?\O\_ \phi &\Ll&
      \text{Es gibt eine Zerlegung $\sum_i s_i = 1 \in R$ und} \\
    &&{\quad\quad} \text{Elemente $x_i \in R[s_i^{-1}]$ sodass für alle $i$:
    $R[s_i^{-1}] \models \phi(x_i)$.}
  \end{array}$%
}

\tableofcontents


\section{Etwas formale Logik}

Im Folgenden wollen wir über mathematische Aussagen sprechen, in denen
\[ {=} \quad {\top} \quad {\bot} \quad {\wedge} \quad {\vee} \quad {\Rightarrow} \quad {\forall} \quad {\exists} \]
die einzigen vorkommenden logischen Symbole sind. Dabei steht~"`$\bot$"' für
eine ausgezeichnete falsche und~"`$\top$"' für eine ausgezeichnete Aussage.
Negation ist ebenfalls wichtig, muss aber nicht als primitiv
angenommen werden, da man sie über die Beziehung
\[ \neg\phi \quad:\equiv\quad (\phi \Rightarrow \bot) \]
durch~$\Rightarrow$ und~$\bot$ ausdrücken kann.
Wir schreiben die universelle und existenzielle Quantifikation nicht mit dem
Elementsymbol, sondern dem in der Typtheorie üblichen Doppelpunkt:
\[ \forall x\?X\_ \phi(x). \]
Gelegentlich werden wir die Abhängigkeit von~$\phi(x)$ von~$x$ in der Notation
unterdrücken und kurz nur~"`$\phi$"' schreiben. Ferner verwenden wir den
Quantor der eindeutigen Existenz, formal definiert als
\[ \exists! x\?X\_ \phi(x) \quad:\equiv\quad
  \bigl(\exists x\?X\_ \phi(x)\bigr) \ \wedge\ \bigl(\forall x\?X\_ \forall x'\?X\_
  \phi(x) \wedge \phi(x') \Rightarrow x = x'\bigr). \]
Da wir (gezwungenermaßen) in einem konstruktiven Kontext arbeiten werden,
können wir im Wesentlichen keine weiteren der aufgeführten primitiven Symbole
durch die anderen ausdrücken. Etwa können wir \emph{nicht} die Regel
\[ \phi \vee \psi \quad\Longleftrightarrow\quad \neg(\neg\phi \wedge \neg\psi) \]
verwenden, um Disjunktion auf Negation und Konjunktion zurückzuführen -- die
umgangssprachliche Interpretation der Aussage auf der linken Seite ist, dass
wir wissen, dass~$\phi$ gilt oder dass~$\psi$ gilt, während in der rechten
Aussage nur die Information steckt, dass es nicht sein kann, dass~$\phi$
und~$\psi$ beide falsch sind.


\section{Die Kripke-Joyal-Semantik des kleinen Zariski-Topos}

Zu jedem kommutativen Ring~$R$ gehört ein alternatives Mathematik-Universum,
der \emph{kleine Zariski-Topos zu~$R$}. Uns fehlen kategorientheoretische
Konzepte, um dieses Universum explizit anzugeben (obwohl das nicht intrinsisch
schwer ist). Wir können aber beschreiben, was es bedeuten soll, dass \emph{eine
Aussage~$\phi$ in dem kleinen Zariski-Topos zu~$R$ gilt}, in Formeln
ausgedrückt als
\[ R \models \phi. \]

\begin{defn}Die Bedeutung von Aussagen der internen Sprache des Zariski-Topos
soll durch Rekursion über den Aussageaufbau durch die auf der ersten Seite
angegebenen Übersetzungsregeln festgelegt sein.\end{defn}

Auf den ersten Blick erscheinen diese Regeln völlig
willkürlich. Tatsächlich aber sind sie fein aufeinander abgestimmt, schon
kleine Änderungen führen dazu, dass das gesamte System zusammenbricht. In
diesem Rahmen wollen wir sie schlichtweg als gegeben hinnehmen, man gewöhnt
sich schnell an sie.

In dem kleinen Zariski-Topos zu~$R$ gibt es ein Abbild des Rings~$R$, das
wir~"`$\O$"' schreiben. (Diese Bezeichnung hat nichts mit Ganzheitsringen zu tun.)


\section{Erste Gehversuche in der internen Welt}

\subsection{Interne Kommutativität}

Mit den Übersetzungsregeln an der Hand können wir beginnen, die interne Welt zu
erkunden. Folgende Beobachtung macht den Anfang:

\begin{prop}Der Ring~$\O$ des kleinen Zariski-Topos zu~$R$ ist wieder
kommutativ -- das heißt:
\[ R \models \forall x,y \? \O\_ x y = y x. \]
\end{prop}
\begin{proof}Die Doppelquantifikation in der Behauptung ist eine
Kurzschreibweise für
\[ R \models \forall x\?\O\_ \forall y\?\O\_ x y = y x. \]
Gemäß den Regeln bedeutet das:
\begin{indentblock}
Für alle~$s \in R$ und alle~$x \in R[s^{-1}]$ gilt:
\begin{indentblock}
Für alle~$t \in R$ und alle~$y \in R[s^{-1}][t^{-1}]$ gilt:
\begin{indentblock}
In~$R[s^{-1}][t^{-1}]$ gilt~$xy = yx$.
\end{indentblock}
\end{indentblock}
\end{indentblock}
Das erscheint vielleicht etwas verklausuliert, ist aber auch offensichtlich
wahr.
\end{proof}


\subsection{Interne Invertierbarkeit}

Das folgende Lemma ist ähnlich. Es besagt, dass es keinen Unterschied zwischen
Invertierbarkeit aus interner und externer (also üblicher) Sicht gibt.
\begin{prop}\label{interne-invertierbarkeit}%
Genau dann ist ein Ringelement~$f \in R$ invertierbar, wenn es als Element
von~$\O$ invertierbar ist, wenn also
\[ R \models \exists g\?\O\_ fg = 1. \]
\end{prop}
\begin{proof}
Die Übersetzung der internen Aussage lautet:
\begin{indentblock}
Es gibt eine Zerlegung~$1 = \sum_i s_i \in R$, sodass für jeden Index~$i$ ein
Element~$g_i \in R[s_i^{-1}]$ mit~$fg_i = 1$ in~$R[s_i^{-1}]$ existiert.
\end{indentblock}
Damit ist es nur noch eine Übungsaufgabe in elementarer Ringtheorie, die
behauptete Äquivalenz nachzuweisen.
\end{proof}


\subsection{Interne Lokalität}

Diese beiden Propositionen waren noch nicht besonders beeindruckend.
Die folgende Proposition ist dagegen beim ersten Kontakt völlig verblüffend und
illustriert gut den kuriosen Charakter der internen Welt. Dazu erinnern wir an
das Konzept des lokalen Rings:
\begin{defn}Ein Ring heißt genau dann \emph{lokal}, wenn, wann immer eine Summe
von Ringelementen invertierbar ist, schon mindestens ein Summand invertierbar
ist.\end{defn}
\begin{bsp}Die vertrauten Ringe~$\ZZ$ und~$K[X,Y]$ sind nicht lokal. Oft erhält
man lokale Ringe durch geeignete Lokalisierung: Jeder Körper ist lokal,
die Lokalisierung~$\ZZ_{(p)}$ nach einem Primideal~$(p)$ ist lokal, der Ring
$K[X,Y]_{(X-a,Y-b)} = \{ f/g \,|\, f,g \in K[X,Y], g(a,b) \neq 0 \}$ der in
beliebig kleinen Umgebungen von~$(a,b) \in K^2$ definierten rationalen
Funktionen ist lokal.
\end{bsp}

\begin{prop}\label{intern-lokal}%
Unabhängig davon, ob~$R$ lokal ist oder nicht, ist der Ring~$\O$
der internen Zariski-Welt lokal.\end{prop}
\begin{proof}
Wir zeigen, dass
\[ R \models \forall x,y\?\O\_\ \speak{$x+y$ inv.}\ \Longrightarrow\ 
  \speak{$x$ inv.} \,\vee\, \speak{$y$ inv.}. \]
Die Häkchen sollen andeuten, dass der entsprechende Teil nur umgangssprachlich
vorliegt und daher vom Leser formalisiert werden muss. Unter Verwendung der
vorhergehenden Proposition weisen und die
Übersetzungsregeln also an, folgende Behauptung zu zeigen:
\begin{indentblock}
Für alle~$s \in R$ und~$x \in R[s^{-1}]$ gilt:
\begin{indentblock}
Für alle~$t \in R[s^{-1}]$ und~$y \in R[s^{-1}][t^{-1}]$ gilt:
\begin{indentblock}
Für alle~$u \in R$ gilt: Falls~$x+y$ in~$R[s^{-1}][t^{-1}][u^{-1}] =: R'$
invertierbar ist, so gibt es eine Zerlegung der Eins von~$R'$, $1 = v_1 +
\cdots + v_n \in R'$, sodass in den
weiter lokalisierten Ringen~$R'[v_i^{-1}]$ jeweils~$x$ oder~$y$ invertierbar
ist.
\end{indentblock}
\end{indentblock}
\end{indentblock}
Der Nachweis dieser Behauptung ist eine Übungsaufgabe.
\end{proof}

Die interne Welt des kleinen Zariski-Topos ist also eine Möglichkeit, jeden
beliebigen Ring als einen lokalen Ring aufzufassen.


\section{Vereinfachungsregeln}

In den bisherigen Beispielen waren die übersetzten Aussagen recht
verschachtelt. Bevor wir fortfahren, wollen wir daher
Vereinfachungsregeln festhalten, die den praktischen Umgang mit internen
Aussagen angenehmer gestalten.

\begin{lemma}\label{vereinfachung}%
Für folgende Quantorenfiguren kann man die Regeln vereinfachen:
\[\renewcommand{\arraystretch}{1.3}\begin{array}{@{}lcl@{}}
  R \models \forall x\?\O\_ \forall y\?\O\_ \phi &\Ll&
    \text{Für alle~$s \in R$ und~$x,y \in R[s^{-1}]$ gilt~$R[s^{-1}] \models
    \phi(x,y)$.} \\
  R \models \forall x\?\O\_ \phi \Rightarrow \psi &\Ll&
    \text{Für alle~$s \in R$ und~$x \in R[s^{-1}]$ gilt:} \\
  &&{\quad\quad} \text{Aus $R[s^{-1}] \models \phi(x)$ folgt $R[s^{-1}] \models \psi(x)$.} \\
  R \models \exists x\?\O\_ \exists y\?\O\_ \phi &\Ll&
    \text{Es gibt eine Zerlegung~$1 = \sum_i s_i \in R$ und} \\
  &&{\quad\quad} \text{für jeden Index~$i$ Elemente~$x_i, y_i \in R[s_i^{-1}]$} \\
  &&{\quad\quad} \text{mit $R[s_i^{-1}] \models \phi(x_i,y_i)$.} \\
  R \models \exists!x \? \O\_ \phi &\Ll&
    \text{Für alle~$s \in R$ existiert genau ein~$x \in R[s^{-1}]$ mit} \\
  &&{\quad\quad} R[s^{-1}] \models \phi(x). \\
  R \models \forall x\?\O\_ \exists! y\?\O\_ \phi &\Ll&
    \text{Für alle~$s \in R$ und~$x \in R[s^{-1}]$ existiert} \\
  &&{\quad\quad} \text{genau ein~$y \in R[s^{-1}]$ mit $R[s^{-1}] \models
  \phi(x,y)$.}
\end{array}\]
\end{lemma}
\begin{proof}Sobald man die internen Aussagen übersetzt hat, muss man nur ein
paar allgemeine Fakten über die Lokalisierung von Ringen nachweisen. Das ist
nicht schwer, aber auch nicht besonders erhellend.\end{proof}


\section{Fundamentale Eigenschaften der internen Sprache}

\subsection{Erinnerung: Geometrische Vorstellung von Ringen}

Zum Ring~$R$ können wir uns einen geometrischen Raum~$\Spec R$ vorstellen; ein
Ringelement~$s \in R$ entspricht dann einer "`guten"' Funktion auf~$\Spec R$.
Den Ort, wo diese Funktion nicht verschwindet, wollen wir mit~"`$D(s)$"'
bezeichnen. Dieser Ort ist stets eine offene Menge.

\begin{bsp}Den Ring~$K[X,Y]$ stellen wir uns geometrisch als~$K^2$ vor. Zum
Element~$s := 2X+3Y$ gehört dann die Funktion~$(x,y) \mapsto 2x+3y$. Die
Menge~$D(s)$ ist das Komplement einer schrägen Gerade in~$K^2$.\end{bsp}

In diesem Bild können wir uns eine Zerlegung~$1 = \sum_i s_i \in R$ der Eins
als eine Über\-dec\-kung~$\bigcup_i D(s_i)$ von~$\Spec R$ vorstellen: Da die
Einsfunktion nirgendwo Null ist, können an keinem Punkt alle~$s_i$ zugleich
verschwinden.


\subsection{Lokalität der internen Sprache}

In einem gewissen Sinn gilt~$R \models \phi$ genau dann, wenn~$\phi$ auf ganz~$\Spec R$
gilt. Dagegen bedeutet~$R[s^{-1}] \models \phi$ nur, dass~$\phi$ auf~$D(s)$
gilt.

Die interne Welt des kleinen Zariski-Topos zu~$R$ fasst nun gewissermaßen die
\emph{lokalen Aspekte} von~$R$ -- solche, die genau dann ganz~$\Spec R$
betreffen, wenn sie auf den einzelnen Überdeckungsmengen einer offenen
Überdeckung gelten. Die folgende Proposition macht dieses Motto präzise:
\begin{prop}Sei~$1 = \sum_i s_i \in R$ eine Zerlegung der Eins und~$\phi$ eine
Aussage. Dann gilt genau dann~$R \models \phi$, wenn für alle~$i$
jeweils~$R[s_i^{-1}] \models \phi$ gilt.\end{prop}
\begin{proof}Induktion über den Aufbau von~$\phi$.\end{proof}

Eine Aussage~$\phi$ muss also nicht unbedingt im Wortlaut erfüllt sein, um in
der internen Welt des kleinen Zariski-Topos zu gelten. Es genügt, dass es
eine Zerlegung der Eins gibt, sodass sie in den jeweils lokalisierten Ringen gilt.
Die technische Verwaltung der Zerlegungen übernimmt dabei der
Übersetzungsapparat; mit der internen Welt ist es also möglich, \emph{lokal}
mit Ringen zu arbeiten, ohne manuell Zerlegungen einführen und mitschleppen
zu müssen.

\begin{bsp}\label{lokal-hauptideale}%
Sei~$R$ ein Prüferscher Bereich. Dann ist ein endlich erzeugtes
Ideal~$\aa$ zwar nicht unbedingt ein Hauptideal, aber \emph{lokal} ein
Hauptideal -- in dem Sinn, dass es eine Zerlegung~$1 = \sum_i s_i$ der Eins
gibt, sodass die erweiterten Ideale~$\aa[s_i^{-1}]$ jeweils in~$R[s_i^{-1}]$
Hauptideale sind. Der Ring~$\O$ der internen Welt spiegelt diese Eigenschaft
viel einfacher wieder: Er ist \emph{bézoutsch} -- jedes endlich
erzeugte Ideal ist selbst schon ein Hauptideal.\end{bsp}


\subsection{Verträglichkeit mit konstruktiver Logik}

Bisher haben wir die interne Welt des kleinen Zariski-Topos allein dadurch
erkundet, indem wir mit den Kripke-Joyal-Regeln die Rückübersetzung in unsere
gewohnte mathematische Sprache vorgenommen haben. Wenn das unsere einzige
Interaktionsmöglichkeit mit der internen Welt wäre, wäre das ganze Thema aber nicht
besonders spannend. Tatsächlich aber können wir in der internen Welt auch
\emph{mathematisch argumentieren} -- fast genau so, wie wir es gewohnt sind.

\begin{prop}\label{soundness}%
Wenn~$R \models \phi$ gilt und konstruktiv aus~$\phi$ eine weitere
Aussage~$\psi$ folgt, so gilt auch~$R \models \psi$.\end{prop}
\begin{proof}Wir müssen uns
zunächst überlegen, aus welchen grundlegenden Argumentationsschritten Beweise
aufgebaut sind. Dann müssen wir von jedem solchen Baustein nachweisen,
dass er in der internen Welt ebenfalls erfüllt ist. Etwa gibt es das logische
Prinzip
\begin{indentblock}\emph{Wenn $\phi \wedge \psi$ gilt, so gilt auch~$\phi$}.
\end{indentblock}
Dieses ist in der internen Sprache ebenfalls erfüllt, denn aus~$R \models \phi
\wedge \psi$ folgt nach den Übersetzungsregeln sofort~$R \models \phi$.

Die Hauptschwierigkeit eines präzisen Beweises der Proposition liegt darin,
eine über\-sicht\-li\-che Liste von Kernbeweisschritten derart zusammenzustellen,
dass jede denkbare Argumentation so formalisiert werden kann, dass sie nur diese
Bausteine verwendet. Danach ist es nur noch ein einfacher Induktionsbeweis über den
Aufbau formal niedergeschriebener konstruktiver Beweise.
\end{proof}

\begin{bsp}Man kann konstruktiv zeigen, dass jede Matrix über einem lokalen
Ring, die einen Rang besitzt, mittels Ähnlichkeitstransformationen auf eine
Diagonalgestalt gebracht werden kann. Daraus folgt \emph{ohne weiteres Zutun}
sofort, dass man jede Matrix (die einen Rang besitzt) über einem
\emph{beliebigen} Ring lokal auf eine Diagonalgestalt bringen kann -- in dem Sinn, dass es
eine Zerlegung der Eins gibt, sodass in den lokalisierten Ringen die Matrix
jeweils ähnlich zu einer Diagonalmatrix ist: Denn nach Proposition~\ref{intern-lokal}
ist der Ring~$\O$ der internen Welt ja stets lokal.\end{bsp}

\begin{bem}Ein direkter Beweis der Behauptung über die Diagonalisierbarkeit
über beliebigen Ringen ist natürlich ebenfalls möglich. Er erfordert jedoch
die Einführung, Verwaltung und Kombination mehrerer Zerlegungen der Eins. Diese
technischen Schritte fallen bei Arbeit in der internen Welt ersatzlos weg. Um
Zerlegungen der Eins muss man sich nur ein einziges Mal kümmern, nämlich im
Beweis der allgemeinen Proposition~\ref{soundness}.\end{bem}


\section{Der nichtklassische Charakter der internen Welt}

Proposition~\ref{soundness} erlaubt es, über den Ring~$\O$ der internen Welt
wie üblich mathematisch zu argumentieren -- solange man dabei nur konstruktive
Logik verwendet, also sich \emph{nicht} auf die sonst üblichen Axiome
\begin{itemize}
\item \emph{Prinzip vom ausgeschlossenen Dritten:} $\phi \vee \neg\phi$,
\item \emph{Prinzip der Doppelnegationselimination:} $\neg\neg\phi \Rightarrow
\phi$ und
\item das \emph{Auswahlaxiom}
\end{itemize}
beruft. Das ist anfangs ungewohnt, in der Praxis aber oftmals keine große
Einschränkung.

Die Beschränkung auf konstruktive Logik ist dabei keine Frage der Einstellung
-- die interne Welt erfüllt nun mal nicht die genannten klassischen Axiome. Das
wollen wir in diesem Abschnitt einsehen.

\begin{lemma}Sei~$f \in R$. Genau dann ist~$f$ in der internen Welt nicht
invertierbar, wenn~$f$ im gewöhnlichen Sinn nilpotent ist.\end{lemma}
\begin{proof}Die Aussage~$R \models \neg(\speak{$f$ inv.})$ lautet ausgeschrieben
\[ R \models \bigl(\exists g\?\O\_ fg = 1\bigr) \Rightarrow \bot \]
und unter Zuhilfenahme der Vereinfachungsregeln aus Lemma~\ref{vereinfachung}
und Lemma~\ref{interne-invertierbarkeit} übersetzt wie folgt:
\begin{indentblock}
Für alle~$s \in R$ gilt:
\begin{indentblock}
Sollte~$f$ in~$R[s^{-1}]$ invertierbar sein, so gilt~$1 = 0$ in~$R[s^{-1}]$
(das ist gleichbedeutend damit, dass~$s$ nilpotent ist).
\end{indentblock}
\end{indentblock}
Durch die Spezialisierung~$s := f$ erhalten wir daher sofort die Hinrichtung.
Wenn umgekehrt~$f$ nilpotent ist, enthalten die lokalisierten Ringe~$R[s^{-1}]$
ein invertierbares und trotzdem nilpotentes Element. Das geht nur, wenn sie
Nullringe sind.
\end{proof}

\begin{prop}In der internen Welt des kleinen Zariski-Topos eines Rings gilt im
Allgemeinen \emph{nicht}, dass jedes Element invertierbar oder nicht
invertierbar ist.\end{prop}
\begin{proof}Ein Gegenbeispiel liefert der Ring~$R := \ZZ$ mit dem Element~$f
:= 2$. Denn
\[ R \models \speak{$f$ inv.} \vee \neg(\speak{$f$ inv.}) \]
bedeutet, dass es eine Zerlegung~$1 = s_1 + \cdots + s_n$ der Eins
von~$\ZZ$ gibt, sodass in den lokalisierten Ringen~$\ZZ[s_i^{-1}]$ die Zahl~$2$
jeweils invertierbar oder nilpotent ist. Eine der Zahlen~$s_i$ muss ungerade
sein. In den Nennern der Elemente des zugehörigen lokalisierten
Rings~$\ZZ[s_i^{-1}]$ dürfen also gewisse ungerade Zahlen stehen (Potenzen
von~$s_i$). Daher ist die Zahl~$2$ dort weiterhin nicht invertierbar. Sie ist
aber auch nicht nilpotent.
\end{proof}


\section{Ausblick}

\emph{Auszuformulieren:}

\begin{itemize}
\item Diskussion anderer Objekte als~$\O$, insbesondere Objekte wie~$\O[X]$
und~$\O^{n \times m}$, die mühelos auch emuliert werden können
\item "`Kann man mit der Topossprache Aussagen
beweisen, die man ohne sie nicht beweisen könnte?"'
\item Geometrische Formeln, halmweise Charakterisierung
\item Explizite Beschreibung des kleinen Zariski-Topos
\item Anwendung auf Modulgarben über Schemata und Vektorbündel über Mannigfaltigkeiten
\item Diskussion des großen Zariski-Topos
\item Literaturverweise: Moerdijk, Johnstohne, Mulvey, Richman,
Coquand/Lombardi/Roy, Kock (universal projective geometry), Pizzaseminar
\end{itemize}


\appendix
\section{Aufgaben}

\begin{aufgabe}{Nichttrivialität des internen Rings}
Sei~$R$ ein beliebiger kommutativer Ring. Zeige, dass~$\O$ stets ein
nichttrivialer Ring ist, zeige also:
\[ R \models \neg(1 = 0). \]
\end{aufgabe}
\vspace{-1.5em}

\begin{aufgabe}{Die interne Welt des Nullrings}
Zeige, dass in der internen Welt des Nullrings jede beliebige Aussage gilt.
Zeige also, dass wenn~$R$ ein Ring mit~$1 = 0 \in R$ und~$\phi$ eine beliebige
Aussage ist, $R \models \phi$ gilt.

\emph{Tipp:} Man kann eine sehr einfache Zerlegung der Eins hinschreiben.
\end{aufgabe}

\begin{aufgabe}{Körpereigenschaften des internen Rings}
Ein Ring heißt genau dann \emph{reduziert}, wenn, wann immer ein Element
nilpotent ist, dieses schon gleich Null ist.
Sei~$R$ ein beliebiger kommutativer Ring.
\begin{enumerate}
\item Zeige, dass der interne Ring~$\O$ in folgendem Sinn beinahe ein Körper
ist:
\[ R \models \forall x\?\O\_ \neg(\speak{$x$ inv.}) \Rightarrow \speak{$x$
nilpotent}. \]
\vspace*{-1.8em}%
\item Zeige, dass~$R$ genau dann reduziert ist, wenn~$\O$ reduziert ist.
\item Sei~$R$ reduziert. Zeige, dass dann~$\O$ ein Körper in folgendem Sinn
ist:
\[ R \models \forall x\?\O\_ \neg(\speak{$x$ inv.}) \Rightarrow x = 0. \]
\vspace*{-1.8em}%
\item Zeige, dass die Umkehrung der Aussage in~c) ebenfalls gilt.
\end{enumerate}
\end{aufgabe}

\begin{aufgabe}{Diskretheit des internen Rings}
Ein Ring heißt genau dann \emph{diskret}, wenn jedes Element Null ist oder
nicht Null ist. Natürlich ist jeder Ring der gewöhnlichen mathematischen Welt
diskret.

Ein Ringelement~$f$ heißt genau dann \emph{pseudoregulär}, wenn,
wann immer ein Produkt von~$f$ mit einem weiteren Ringelement~$g$ Null ist, $g$
nilpotent ist.
\begin{enumerate}
\item Sei~$f$ ein Element eines kommutativen Rings~$R$. Zeige, dass~$f$ genau
dann in der internen Welt nicht Null ist (also~$R \models \neg(f = 0)$ gilt),
wenn~$f$ im gewöhnlichen Sinn pseudoregulär ist.
\item Zeige, dass der interne Ring~$\O$ im Allgemeinen nicht diskret ist.
\end{enumerate}
\end{aufgabe}

\begin{aufgabe}{Prüfersche Bereiche aus interner Sicht}
Beweise die Behauptung über die interne Welt in
Beispiel~\ref{lokal-hauptideale}.
\end{aufgabe}

\end{document}

Wieso konstruktiv?

Kann man damit Dinge beweisen, die man ohne es nicht kann?

Was passiert wirklich? 1.) beliebig kleine offene Teilmengen; 2.) Halme

alle Aussagen so formulierbar (abh. Typen)

halmweise geometrisch

über Matrizen reden

1 = 0 ==> bottom

schöne Illustration

\begin{defn}Ein Ringelement~$f$ heißt genau dann \emph{pseudoregulär}, wenn,
wann immer ein Produkt von~$f$ mit einem weiteren Ringelement~$g$ Null ist, $g$
nilpotent ist.\end{defn}

\begin{lemma}Sei~$f \in R$. Genau dann ist~$f$ in der internen Welt nicht Null,
wenn~$f$ im gewöhnlichen Sinn pseudoregulär ist.\end{lemma}
\begin{proof}Die Aussage~$R \models \neg(f = 0)$ -- ausgeschrieben $R \models (f
= 0 \Rightarrow \bot)$ -- bedeutet:
\begin{indentblock}
Für alle~$s \in R$ gilt:
\begin{indentblock}
Sollte~$f = 0$ in~$R[s^{-1}]$ gelten (also~$s^n f = 0$ für ein~$n \geq 0$), so
gilt~$1 = 0$ in~$R[s^{-1}]$ (also~$s^m = 0$ für ein~$m \geq 0$).
\end{indentblock}
\end{indentblock}
Damit ist es leicht, die behauptete Äquivalenz nachzuweisen.
\end{proof}
