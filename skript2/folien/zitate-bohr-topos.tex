\documentclass[12pt,a4paper,ngerman,landscape]{scrartcl}

\usepackage[utf8]{inputenc}
\usepackage[ngerman,english,french]{babel}

\usepackage[protrusion=true,expansion=true]{microtype}

\usepackage{hyperref}

\pagestyle{empty}

\setlength\parskip{\medskipamount}
\setlength\parindent{0pt}

\usepackage{geometry}
\geometry{tmargin=1.5cm,bmargin=0.5cm,lmargin=2.0cm,rmargin=2.0cm}

\usepackage{multicol}
\setlength{\columnsep}{2.0cm}

\begin{document}

\Large

\begin{multicols*}{2}
\selectlanguage{french}

Ces “nuages probabilistes”, remplaçant les rassurantes particules matérielles
d’antan, me rappellent étrangement les élusifs “voisinages ouverts” qui
peuplent les topos, tels des fantômes évanescents, pour entourer des “points”
imaginaires.

\hfill -- A. Grothendieck

\vfill

\selectlanguage{english}

However far the phenomena transcend the scope of classical physical
explanation, the account of all evidence must be expressed in
classical terms. [\ldots] The argument is simply that by the word {\it
experiment} we refer to a situation where we can tell others what we
have done and what we have learned and that, therefore, the account of
the experimental arrangements and of the results of the observations
must be expressed in unambiguous language with suitable application of
the terminology of classical physics.

\hfill -- N. Bohr

\vfill

\columnbreak

\selectlanguage{ngerman}

Diese "`Wahrscheinlichkeitswolken"', welche die beruhigenden materiellen
Partikel von früher ersetzen, erinnern mich irgendwie an die flüchtigen
"`offenen Umgebungen"' der Topoi -- wie dahinschwindende Phantome, um
die fiktiven "`Punkte"' zu umgeben.

\hfill -- A. Grothendieck

\vfill

\selectlanguage{ngerman}
\ \\
Inwieweit auch die Phänomene die Grenzen klassischer physikalischer
Erklärungen sprengen, die Darstellung aller Anhaltspunkte muss trotzdem
in klassischer Sprache ausgedrückt werden. [\ldots] Das Argument ist einfach:
Mit dem Wort {\it Experiment} drücken wir eine Situation aus, in der wir
anderen erzählen können, was wir getan und gelernt haben. Deshalb müssen
Versuchsanordnung und Beobachtungsresultate in
unmissverständlicher Sprache ausgedrückt werden -- unter passender
Anwendung der Terminologie klassischer Physik.

\hfill -- N. Bohr
\end{multicols*}

\end{document}
