\documentclass[a4paper,ngerman]{scrartcl}

\usepackage[utf8]{inputenc}

\usepackage[ngerman]{babel}
\addto\captionsngerman{\renewcommand\tablename{Tafel}}

\usepackage{amsmath,amsthm,amssymb,stmaryrd,color,graphicx}
\usepackage{bussproofs}
\usepackage{array}
\usepackage{comment}

\usepackage[protrusion=true,expansion=true]{microtype}

\usepackage{lmodern}
\usepackage{tabto}

\usepackage[natbib=true,style=numeric]{biblatex}
\usepackage[babel]{csquotes}
\bibliography{literatur}

\usepackage[all]{xy}

\usepackage{hyperref}

\setlength\parskip{\medskipamount}
\setlength\parindent{0pt}

\theoremstyle{definition}
\newtheorem{defn}{Definition}[section]
\newtheorem{axiom}[defn]{Axiom}
\newtheorem{bsp}[defn]{Beispiel}

\theoremstyle{plain}

\newtheorem{prop}[defn]{Proposition}
\newtheorem{motto}[defn]{Motto}
\newtheorem{ueberlegung}[defn]{Überlegung}
\newtheorem{lemma}[defn]{Lemma}
\newtheorem{kor}[defn]{Korollar}
\newtheorem{hilfsaussage}[defn]{Hilfsaussage}
\newtheorem{satz}[defn]{Satz}

\theoremstyle{remark}
\newtheorem{bem}[defn]{Bemerkung}
\newtheorem{aufg}[defn]{Aufgabe}

\clubpenalty=10000
\widowpenalty=10000
\displaywidowpenalty=10000

\newcommand{\xra}[1]{\xrightarrow{#1}}
\newcommand{\lra}{\longrightarrow}
\newcommand{\lhra}{\ensuremath{\lhook\joinrel\relbar\joinrel\rightarrow}}
\newcommand{\thlra}{\relbar\joinrel\twoheadrightarrow}

\newcommand{\ZZ}{\mathbb{Z}}
\newcommand{\QQ}{\mathbb{Q}}
\newcommand{\RR}{\mathbb{R}}
\newcommand{\NN}{\mathbb{N}}
\newcommand{\PP}{\mathbb{P}}
\newcommand{\I}{\mathcal{I}}
\newcommand{\J}{\mathcal{J}}
\newcommand{\C}{\mathcal{C}}
\newcommand{\D}{\mathcal{D}}
\newcommand{\E}{\mathcal{E}}
\renewcommand{\I}{\mathcal{I}}
\renewcommand{\P}{\mathcal{P}}
\renewcommand{\O}{\mathcal{O}}
\newcommand{\Hom}{\mathrm{Hom}}
\newcommand{\ev}{\mathrm{ev}}
\newcommand{\id}{\mathrm{id}}
\newcommand{\Id}{\mathrm{Id}}
\newcommand{\freist}{\underline{\ \ }}
\DeclareMathOperator{\colim}{colim}
\DeclareMathOperator{\Ob}{Ob}
\DeclareMathOperator{\ggT}{ggT}
\DeclareMathOperator{\im}{im}
\newcommand{\op}{\mathrm{op}}
\newcommand{\Set}{\mathrm{Set}}
\newcommand{\Grp}{\mathrm{Grp}}
\newcommand{\Vect}[1]{{#1\text{-}\mathrm{Vect}}}
\newcommand{\AbGrp}{\mathrm{AbGrp}}
\newcommand{\Ring}{\mathrm{Ring}}
\newcommand{\Cat}{\mathrm{Cat}}
\newcommand{\Funct}{\mathrm{Funct}}
\newcommand{\Eins}{\mathbf{1}}
\newcommand{\Man}{\mathrm{Man}}
\newcommand{\Top}{\mathrm{Top}}
\newcommand{\seq}[1]{\mathrel{\vdash\!\!\!_{#1}}}
\renewcommand{\_}{\mathpunct{.}\,}
\newcommand{\?}{\,{:}\,}

\newcommand{\XXX}[1]{\textcolor{red}{#1}}

\renewcommand*\theenumi{\alph{enumi}}
\renewcommand{\labelenumi}{\theenumi)}

\newcommand\subsubsubsection[1]{\subsubsection*{#1}}
\definecolor{grey}{rgb}{0.7,0.7,0.7}

\setcounter{tocdepth}{2}

\newenvironment{indentblock}{%
  \list{}{\leftmargin\leftmargin}%
  \item\relax
}{%
  \endlist
}

%\newarrow{Equals}=====

%\usepackage{geometry}
%\geometry{tmargin=2cm,bmargin=4cm,lmargin=3cm,rmargin=3cm}

\begin{document}

\vspace*{2em}%
\begin{center}%
  \vskip 1em
  {\LARGE Pizzaseminar zu konstruktiver Mathematik}
  \vskip 1.5em%
  {\large
   \lineskip .5em%
    \begin{tabular}[t]{c}%
      \today
    \end{tabular}\par}%
    \vskip 1em%
\end{center}\par
\par\vskip 1.5em

\begin{center}\emph{in Entstehung befindlich, nur grobe Zusammenfassung}
\end{center}

\tableofcontents

\section{Was ist konstruktive Mathematik?}

\begin{prop}Es gibt irrationale Zahlen~$x,y$, sodass~$x^y$ rational ist.
\end{prop}
\begin{proof}[Beweis 1] Die Zahl~$\sqrt{2}^{\sqrt{2}}$ ist rational oder nicht
rational. Setze im ersten Fall~$x := \sqrt{2}$, $y := \sqrt{2}$. Setze im
zweiten Fall~$x := \sqrt{2}^{\sqrt{2}}$, $y := \sqrt{2}$.
\end{proof}
\begin{proof}[Beweis 2] Setze~$x := \sqrt{2}$ und~$y := \log_{\sqrt{2}} 3$.
Dann
ist die Potenz~$x^y = 3$ sicher rational. Die Irrationalität von~$y$ lässt sich
sogar einfacher als die von~$\sqrt{2}$ beweisen:
Gelte~$y = p/q$ mit~$p, q \in \ZZ$ und~$q \neq 0$. Da~$y > 0$, können wir
sogar~$p, q \in \NN$ annehmen.
Dann folgt $3 = (\sqrt{2})^{p/q}$, also~$3^{2q} = 2^p$. Das ist ein
Widerspruch zum Satz über die eindeutige Primfaktorzerlegung, denn auf der linken
Seite kommt der Primfaktor~$3$ vor, auf der rechten aber nicht.
\end{proof}

Der erste Beweis war \emph{unkonstruktiv}: Einem interessierten Gegenüber kann
man immer noch nicht ein Zahlenpaar mit den gewünschten Eigenschaften nennen.
Der zweite Beweis dagegen war konstruktiv: Die Existenzbehauptung wurde durch
explizite Konstruktion eines Beispiels nachgewiesen.

Es stellt sich heraus, dass von den vielen Schlussregeln klassischer Logik genau
ein Axiom für die Zulässigkeit unkonstruktiver Argumente verantwortlich ist,
nämlich das \emph{Prinzip vom ausgeschlossenen Dritten}:
\begin{axiom}[vom ausgeschlossenen Dritten, LEM]Für jede Aussage~$\varphi$ gilt: $\varphi \vee
\neg\varphi$.\end{axiom}
Unter konstruktiver Mathematik im engeren Sinn, genauer
\emph{intuitionistischer Logik}, versteht man daher klassische Logik ohne LEM.
Das \emph{Prinzip der Doppelnegationselimination}, demnach man für jede
Aussage~$\varphi$ voraussetzen darf, dass~$\neg\neg\varphi \Rightarrow \varphi$
gilt, ist zu LEM äquivalent (Übungsaufgabe) und kann daher ebenfalls nicht
verwendet werden.

In konstruktiver Mathematik behauptet man \emph{nicht}, dass das
Prinzip vom ausgeschlossenen Dritten falsch wäre: Intuitionistische Logik ist
abwärtskompatibel zu klassischer Logik -- jede konstruktiv nachweisbare Aussage
gilt auch klassisch -- und manche konkrete Instanzen des Prinzips lassen sich
sogar konstruktiv nachweisen (siehe Proposition~\ref{natdiskret} für ein Beispiel).
Stattdessen verwendet man das Prinzip einfach
nur nicht. (Tatsächlich kann man leicht zeigen, dass es keine Gegenbeispiele
des Prinzips geben kann: Für jede Aussage~$\varphi$ gilt~$\neg(\neg\varphi
\wedge \neg\neg\varphi)$.)

\begin{bem}Manche Dozenten erzählen Erstsemestern folgende vereinfachte Version
der Wahrheit: Eine Aussage
erkennt man daran, dass sie einen eindeutigen Wahrheitswert hat. Diese
Charakterisierung mag bei klassischer Logik noch vertretbar sein, ist aber in
einem konstruktiven Kontext offensichtlich unsinnig. Stattdessen erkennt man
eine Aussage daran, dass sie rein von ihrer grammatikalischen Struktur her ein
Aussagesatz ist (und natürlich dass alle vorkommenden Begriffe eine klare
Bedeutung haben).\end{bem}

\begin{bem}In konstruktiver Mengenlehre muss man auf das Auswahlaxiom
verzichten, denn in Gegenwart des restlichen Axiome impliziert dieses das
Prinzip vom ausgeschlossenen Dritten.\end{bem}
% XXX: Beweis einfügen


\subsection{Widerspruchsbeweise vs. Beweise von Negationen}
\label{widerspruchvsnegation}

Ein übliches Gerücht über konstruktive Mathematik besagt, dass der Begriff
\emph{Widerspruch} konstruktiv generell verboten ist. Dem ist nicht so. Man
muss zwischen zwei für das klassische Auge sehr ähnlich aussehenden
Beweisfiguren unterscheiden:
\begin{enumerate}
\item[1.] "`Angenommen, es gilt~$\neg\varphi$. Dann \ldots, Widerspruch; also
gilt~$\neg(\neg\varphi)$ und somit~$\varphi$."'
\item[2.] "`Angenommen, es gilt~$\psi$. Dann \ldots, Widerspruch; also
gilt~$\neg\psi$."'
\end{enumerate}
Argumente der ersten Form sind tatsächlich Widerspruchsbeweise und daher
konstruktiv nicht pauschal zulässig -- wenn man nicht anderweitig für die
untersuchte Aussage~$\varphi$ begründen
kann, dass aus ihrer Doppelnegation schon sie selbst folgt, beweist ein
solches Argument nur die Gültigkeit von~$\neg\neg\varphi$; das ist konstruktiv
schwächer als~$\varphi$.

Argumente der zweiten Form sind dagegen konstruktiv völlig einwandfrei: Sie
sind Beweise negierter Aussagen und nicht Widerspruchsbeweise im eigentlichen
Sinn. Die Zulässigkeit erklärt sich direkt nach Definition:
Die Negation wird (übrigens auch in klassischer Logik) als
\[ \neg\psi :\equiv (\psi \Rightarrow \bot) \]
festgelegt. Dabei steht~"`$\bot$"' für \emph{Falschheit}, eine kanonische falsche
Aussage. Wer mag, kann~$1 = 0$ oder~$\lightning$ denken.

Hier ein konkretes Beispiel aus der Zahlentheorie, um den Unterschied zu
demonstrieren:
\begin{prop}Die Zahl~$\sqrt{2}$ ist nicht rational.\end{prop}
\begin{proof}[Beweis (nur klassisch zulässig)]
Angenommen, die Behauptung ist falsch, d.\,h. die Zahl~$\sqrt{2}$ ist nicht
nicht rational. Dann ist~$\sqrt{2}$ also rational. Somit gibt es ganze
Zahlen~$p$ und~$q$ mit~$\sqrt{2} = p / q$. Daraus folgt die Beziehung~$2q^2 =
p^2$, die einen Widerspruch zum Satz über die Eindeutigkeit der
Primfaktorzerlegung darstellt: Auf der linken Seite kommt der Primfaktor~$2$
ungerade oft, auf der rechten Seite aber gerade oft vor.
\end{proof}
\begin{proof}[Beweis (auch konstruktiv zulässig)]
Angenommen, die Zahl~$\sqrt{2}$ ist rational. Dann gibt es ganze Zahlen \ldots,
Widerspruch. (Der verwendete Satz über die Eindeutigkeit der
Primfaktorzerlegung lässt sich konstruktiv beweisen.)
\end{proof}


\subsection{Informale Bedeutung logischer Aussagen}

\subsubsection*{\ldots über Belege (die
Brouwer-Heyting-Kolmogorov-Interpretation)}

Die Ablehnung des Prinzips vom ausgeschlossenen Dritten erscheint uns durch
unsere klassische Ausbildung als völlig verrückt: \emph{Offensichtlich} ist
doch jede Aussage entweder wahr oder falsch! Die Verwunderung löst sich auf,
wenn man akzeptiert, dass konstruktive Mathematiker zwar dieselbe
\emph{logische Sprache} verwenden ($\wedge, \vee, \Rightarrow, \neg, \forall,
\exists$), aber eine andere Bedeutung im Sinn haben: Wenn eine konstruktive
Mathematikerin eine Aussage~$\varphi$ behauptet, meint sie, dass sie einen
\emph{expliziten Beleg} für~$\varphi$ hat.

Den Basisfall bilden dabei die sog. \emph{atomaren Aussagen}, von denen wir
intuitiv wissen, wie ein Beleg ihrer Gültigkeit aussehen sollte. Atomare Aussagen sind
solche, die nicht vermöge der logischen Operatoren $\wedge, \vee,
\Rightarrow$ und der Quantoren~$\forall, \exists$ aus weiteren Teilformeln
zusammengesetzt sind. In der Zahlentheorie sind atomare Aussagen etwa von der
Form
\[ n = m, \]
wobei~$n$ und~$m$ Terme für natürliche Zahlen sind; in der Mengenlehre sind
atomare Aussagen von der Form
\[ x \in M. \]

Für \emph{zusammengesetzte Aussagen} zeigt Tabelle~\ref{bhk}, was unter Belegen
jeweils zu verstehen ist. (An manchen Stellen steht dort~"`$x : X$"' -- das hat
einen Grund, aber momentan soll das einfach etwas seltsame Notation für~"`$x
\in X$"' sein.) Etwa ist ein Beleg für eine Aussage der Form
\[ \forall n \? \NN{:}\ \varphi(x) \Rightarrow \psi(x) \]
eine Vorschrift, wie man für jede natürliche Zahl~$n : \NN$ aus Belegen
für~$\varphi(x)$ Belege für~$\psi(x)$ erhalten kann. Dies soll tatsächlich nur
\emph{eine} Vorschrift sein (welche mit allen natürlichen Zahlen zurechtkommt),
nicht für jede natürliche Zahl jeweils eine. Das ist mit \emph{gleichmäßig} in
der Tabelle gemeint.

\begin{table}
  \centering
  \small
  \setlength{\extrarowheight}{0.3em}
  \begin{tabular}{@{}r|p{5.9cm}|p{6.5cm}}
    & {klassische Logik} & {intuitionistische Logik}
    \\\hline
    Aussage $\varphi$ & Die Aussage $\varphi$ gilt. & Wir haben Beleg für $\varphi$. \\
    $\bot$ & Es stimmt Falschheit. & Wir haben Beleg für Falschheit. \\
    $\varphi \wedge \psi$ & $\varphi$ und $\psi$ stimmen. & Wir haben Beleg für~$\varphi$ und für~$\psi$. \\
    $\varphi \vee \psi$ & $\varphi$ oder $\psi$ stimmt. & Wir haben Beleg für~$\varphi$ oder für~$\psi$. \\
    $\varphi \Rightarrow \psi$ & Sollte~$\varphi$ stimmen, dann auch~$\psi$. &
    Aus Belegen für~$\varphi$ können wir (gleichmäßig) Belege für~$\psi$ konstruieren. \\
    $\neg\varphi$ &
      $\varphi$ stimmt nicht. &
      Es kann keinen Beleg für~$\varphi$ geben. \\
    $\forall x\?X{:}\ \varphi(x)$ & Für alle $x : X$ stimmt jeweils~$\varphi(x).$ &
      Wir können (gleichmäßig) für alle~$x : X$ Belege für~$\varphi(x)$ konstruieren. \\
    $\exists x\?X{:}\ \varphi(x)$ & \raggedright Es gibt mindestens ein~$x : X$, für das~$\varphi(x)$
    stimmt. & {\raggedright
      Wir haben ein~$x : X$ zusammen mit Beleg für~$\varphi(x).$} \\
  \end{tabular}
  \caption{\label{bhk}Informale rekursive Definition des Belegbegriffs.}
\end{table}

\begin{bsp}
Unter dieser Interpretation meint das Prinzip vom ausgeschlossenen Dritten, dass wir für jede
Aussage Beleg für sie oder ihre Negation haben. Das ist aber offensichtlich
nicht der Fall.
\end{bsp}

\begin{bsp}
Die Interpretation von~$\neg\neg\varphi$ ist, dass es keinen Beleg
für~$\neg\varphi$ gibt. Daraus folgt natürlich noch nicht, dass wir tatsächlich
Beleg für~$\varphi$ haben; gewissermaßen ist eine solche Aussage~$\varphi$ nur
"`potenziell wahr"'.
\end{bsp}

\begin{bsp}Wenn wir wissen, dass sich unser Haustürschlüssel irgendwo in der
Wohnung befinden muss (da wir ihn letzte Nacht verwendet haben, um die Tür
aufzusperren), wir ihn momentan aber nicht finden, so können wir konstruktiv
nur die doppelt negierte Aussage
\[ \neg\neg (\exists x{:}\ \text{der Schlüssel befindet sich an Position~$x$})
\]
vertreten.\end{bsp}

\begin{bsp}[\cite{sigfpe:katemoss}]
Es war ein Video aufgetaucht, dass Kate Moss beim Konsumieren von Drogen zeigte,
und zwar entweder solche von einem Typ~A oder solche von einem Typ~B. Welcher
Typ aber tatsächlich vorlag, konnte nicht entschieden werden. Daher gab es für
keine der beiden Strafttaten einen Beleg, Kate Moss wurde daher nicht
strafrechtlich verfolgt.
\end{bsp}


\subsubsection*{\ldots über Berechenbarkeit}

Es gibt noch eine zweite Interpretation, die beim Verständnis konstruktiver
Mathematik sehr hilfreich ist:
\begin{motto}Eine Aussage gilt konstruktiv genau dann, wenn es ein
Computerprogramm gibt, welches sie in endlicher Zeit bezeugt.\end{motto}
Etwa ist mit dieser Interpretation klar, dass die formale Aussage
\[ \forall n \in \NN{:}\ \exists p \geq n{:}\ \text{$p$ ist eine Primzahl}, \]
eine Formulierung der Unendlichkeit der Primzahlen, auch konstruktiv
stimmt: Denn man kann leicht ein Computerprogramm angeben, das eine natürliche
Zahl~$n$ als Eingabe erwartet und dann, etwa über die Sieb-Methode von
Eratosthenes, eine Primzahl~$p \geq n$ produziert (zusammen mit einem Nachweis,
dass~$p$ tatsächlich prim ist).

\begin{bem}Das Motto kann man tatsächlich zu einem formalen Theorem
präzisieren, das ist Gegenstand der gefeierten
Curry--Howard-Korrespondenz.\end{bem}


\section{Beispiele}

\subsection{Diskretheit der natürlichen Zahlen}

Manche konkrete Instanzen des Prinzips vom ausgeschlossenen Dritten lassen sich
konstruktiv nachweisen:

\begin{prop}\label{natdiskret}Für beliebige natürlichen Zahlen~$x,y \in \NN$
gilt: $x = y \vee \neg(x = y)$.\end{prop}
\begin{proof}Das ist konstruktiv \emph{nicht} klar, aber beweisbar durch eine
Doppelinduktion.\end{proof}

Diese Eigenschaft wird auch als Diskretheit der Menge der natürlichen Zahlen
bezeichnet: Allgemein heißt eine Menge~$X$ genau dann \emph{diskret}, wenn für
alle~$x,y \in X$ die Aussage~$x = y \vee \neg(x = y)$ gilt. Klassisch ist jede
Menge diskret.

Die reellen Zahlen sind in diesem Sinne nicht diskret. Das macht
man sich am einfachsten über die algorithmische Interpretation klar: Es kann
kein Computerprogramm geben, dass \emph{in endlicher Zeit} zwei reelle Zahlen
auf Gleichheit testet. Denn in endlicher Zeit kann ein Programm nur endlich viele
Nachkommaziffern (besser: endlich viele rationale Approximationen) abfragen;
haben die beiden zu vergleichenden Zahlen dieselben Nachkommaziffern, so kann
sich das Programm aber in endlicher Zeit nie sicher sein, ob irgendwann doch noch
eine Abweichung auftreten wird.

Übrigens ist die Menge der algebraischen Zahlen durchaus diskret:
Man kann ein
Programm angeben, dass zwei algebraische Zahlen~$x,y$ zusammen mit \emph{Zeugen}
ihrer Algebraizität, also Polynomgleichungen mit rationalen Koeffizienten
und~$x$ bzw.~$y$ als Lösung, als Eingabe erwartet und dann entscheidet, ob~$x$
und~$y$ gleich sind oder nicht. Der Beweis ist nicht trivial, aber auch nicht
fürchterlich kompliziert; siehe etwa Proposition~1.6 in~\cite{nw:algebra}.


\subsection{Minima von Teilmengen der natürlichen Zahlen}

In klassischer Logik gilt folgendes Minimumsprinzip:
\begin{prop}[in klassischer Logik]Sei~$U \subseteq \NN$ eine bewohnte
Teilmenge. Dann enthält~$U$ ein kleinstes Element.\end{prop}
Dabei heißt eine Menge~$U$ \emph{bewohnt}, falls~$\exists u \in U$.
In konstruktiver Mathematik kann man diese Aussage nicht zeigen -- wegen der
Ab\-wärts\-kom\-pa\-ti\-bi\-li\-tät kann man zwar auch nicht ihr Gegenteil
nachweisen, aber man kann ein sog. \emph{brouwersches Gegenbeispiel}
anführen:
\begin{prop}Besitze jede bewohnte Teilmenge der natürlichen Zahlen ein Minimum.
Dann gilt das Prinzip vom ausgeschlossenen Dritten.\end{prop}
\begin{proof}Sei~$\varphi$ eine beliebige Aussage. Wir müssen zeigen,
dass~$\varphi$ oder~$\neg\varphi$ gilt. Dazu definieren wir die Teilmenge
\[ U := \{ n \in \NN \,|\, n = 1 \vee \varphi \}. \]
Die Zugehörigkeitsbedingung ist etwas komisch, da die Aussage~$\varphi$ ja
nicht von der frischen Variable~$n$ abhängt, aber völlig okay. Da~$U$
sicherlich bewohnt ist (durch~$1 \in U$), besitzt~$U$ nach Voraussetzung ein
Minimum~$z \in U$.

Wegen der diskutierten Diskretheit der natürlichen Zahlen gilt~$z = 0$ oder~$z
\neq 0$. Im ersten Fall folgt~$\varphi$ (denn~$0 \in U$ ist gleichbedeutend
mit~$0 = 1 \vee \varphi$, also mit~$\varphi$), im zweiten Fall folgt~$\neg\varphi$ (denn
wenn~$\varphi$ gälte, wäre~$U = \NN$ und somit~$z = 0$ im Widerspruch zu~$z
\neq 0$).\end{proof}

Wir können das Minimumsprinzip retten, wenn wir eine klassisch triviale
Zusatzbedingung stellen:
\begin{defn}Eine Teilmenge~$U \subseteq X$ heißt genau dann
\emph{herauslösbar}, wenn für alle~$x \in X$ gilt: $x \in U \vee \neg(x \in
U)$.\end{defn}
\begin{prop}Sei~$U \subseteq \NN$ eine bewohnte und herauslösbare Teilmenge.
Dann enthält~$U$ ein kleinstes Element.\end{prop}
\begin{proof}Da~$U$ bewohnt ist, liegt eine Zahl~$n$ in~$U$. Da ferner~$U$
diskret ist, gilt für jede Zahl~$0 \leq m \leq n$: $m \in U$ oder~$m \not\in
U$. Daher können wir diese Zahlen der Reihe nach durchgehen; die erste Zahl
mit~$m \in U$ ist das gesuchte Minimum.
\end{proof}
Weg mag, kann diesen Beweis auch präzisieren und einen formalen
Induktionsbeweis führen. Gut erkennbar ist, wie im Beweis ein expliziter
Algorithmus zur Findung des Minimums enthalten ist.

\begin{bem}Statt eine Zusatzbedingung einzuführen, kann man auch die Behauptung
abschwächen. Man kann nämlich mittels Induktion zeigen, dass jede
bewohnte Teilmenge der natürlichen Zahlen \emph{nicht nicht} ein Minimum
besitzt. Der algorithmische Inhalt eines Beweises dieser abgeschwächten Aussage
ist sehr interessant und wir werden noch lernen, wie man ihn deuten kann.\end{bem}


\subsection{Potenzmengen}

Klassisch ist die Potenzmenge der einelementigen Menge~$\{\star\}$ völlig
langweilig: Sie enthält genau zwei Elemente, nämlich die leere Teilmenge
und~$\{\star\}$ selbst. Konstruktiv lässt sich das nicht zeigen, die
Potenzmenge hat (potenziell!) viel mehr Struktur. Das ist Gegenstand einer
Übungsaufgabe.


\subsection{Die De Morganschen Gesetze}

In klassischer Logik verwendet man oft die De Morganschen Gesetze, manchmal
sogar implizit, um verschachtelte Aussagen zu vereinfachen. In konstruktiver
Mathematik lässt sich nur noch eines der beiden Gesetze in seiner vollen Form
nachweisen. Den Beweis der folgenden Proposition führen wir mit Absicht recht
ausführlich, damit man eine Imitationsgrundlage für die Bearbeitung des ersten
Übungsblatts hat. Es wird das Wort "`Widerspruch"' vorkommen, aber wir haben ja
schon in Abschnitt~\ref{widerspruchvsnegation} diskutiert, dass das nicht
automatisch unkonstruktiv ist.

\begin{prop}Für alle Aussagen~$\varphi$ und $\psi$ gilt
\begin{enumerate}
\item $\neg(\varphi \vee \psi) \quad\Longleftrightarrow\quad \neg\varphi \wedge
\neg\psi$,
\item $\neg(\varphi \wedge \psi) \quad\Longleftarrow\quad \neg\varphi \vee
\neg\psi$.
\end{enumerate}
\end{prop}
\begin{proof}\begin{enumerate}
\item "`$\Rightarrow$"': Wir müssen~$\neg\varphi$ und~$\neg\psi$ zeigen:
\begin{itemize}
\item Angenommen, es gilt doch~$\varphi$. Dann gilt auch~$\varphi \vee \psi$. Da
nach Voraussetzung~$\neg(\varphi \vee \psi)$, folgt ein Widerspruch.
\item Analog zeigt man~$\neg\psi$.
\end{itemize}

"`$\Leftarrow$"': Wir müssen zeigen, dass~$\neg(\varphi \vee \psi)$. Dazu
nehmen wir an, dass~$\varphi \vee \psi$ doch gilt, und streben einen Widerspruch an.
Dann gibt es zwei Fälle:
\begin{itemize}
\item Falls~$\varphi$ gilt: Aus der Voraussetzung~$\neg\varphi \wedge \neg\psi$
folgt insbesondere~$\neg\varphi$. Somit folgt ein Widerspruch.
\item Falls~$\psi$ gilt, folgt ein Widerspruch auf analoge Art und
Weise.
\end{itemize}

\item
Wir müssen zeigen, dass~$\neg(\varphi \wedge \psi)$. Dazu nehmen wir an, dass
doch~$\varphi \wedge \psi$ (also dass~$\varphi$ und dass~$\psi$), und streben
einen Widerspruch an. Nach Voraussetzung können wir zwei Fälle unterscheiden:
\begin{itemize}
\item Falls~$\neg\varphi$: Dann folgt ein Widerspruch zu~$\varphi$.
\item Falls~$\neg\psi$: Dann folgt ein Widerspruch zu~$\psi$.\qedhere
\end{itemize}
\end{enumerate}
\end{proof}

Die Hinrichtung in Regel~b) lässt sich konstruktiv nicht nachweisen. Im
Belegdenken ist das plausibel: Wenn wir lediglich wissen, dass es keinen Beleg
für~$\varphi \wedge \psi$ gibt, wissen wir noch nicht, ob es keinen Beleg
für~$\varphi$ oder keinen Beleg für~$\psi$ gibt. Tatsächlich ist die
Hinrichtung in Regel~b) äquivalent zu einer schwächeren Version des Prinzips
vom ausgeschlossenen Dritten:

\begin{prop}Folgende Prinzipien sind zueinander äquivalent:
\begin{enumerate}
\item[1.] LEM für negierte Aussagen: Für alle Aussagen~$\varphi$
gilt~$\neg\varphi \vee \neg\neg\varphi$.
\item[2.] Für alle Aussagen~$\varphi$ und $\psi$ gilt $\neg(\varphi \wedge \psi)
\Longrightarrow \neg\varphi \vee \neg\psi$.
\end{enumerate}
\end{prop}
Es ist besser, diese Proposition selbstständig zu beweisen als den folgenden
Beweis zu lesen. Denn wenn man nicht genau den Beweisvorgang mitverfolgt,
verirrt man sich leicht in den vielen Negationen.
\begin{proof}"`1. $\Rightarrow$ 2."': Seien~$\varphi$ und~$\psi$ beliebige
Aussagen. Gelte~$\neg(\varphi \wedge \psi)$. Nach
Voraussetzung gilt~$\neg\varphi$ oder~$\neg\neg\varphi$. Im ersten Fall sind
wir fertig. Im zweiten Fall folgt tatsächlich~$\neg\psi$: Denn
wenn~$\psi$ gälte, gälte auch~$\neg\varphi$ (denn wenn~$\varphi$, folgt ein
Widerspruch zu~$\neg(\varphi \wedge \psi)$), aber das wäre ein Widerspruch
zu~$\neg\neg\varphi$.

"`2. $\Rightarrow$ 1."': Sei~$\varphi$ eine beliebige Aussage. Da~$\neg(\varphi
\wedge \neg\varphi)$ (wieso?), folgt nach Voraussetzung $\neg\varphi \vee
\neg\neg\varphi$, das war zu zeigen.
\end{proof}


\section{Nutzen konstruktiver Mathematik}

\paragraph{Spaß.} Konstruktive Mathematik macht Spaß!

\paragraph{Philosophie.}
Konstruktive Logik ist philosophisch einfacher zu rechtfertigen als
klassische Logik.
% XXX: ausführen

\paragraph{Eleganzassistenz.}
Konstruktive Mathematik kann einen dabei unterstützen, Aussagen, Beweise und
ganze Theoriegebäude eleganter zu formulieren. Etwa hat man klassisch oft
\emph{Angst vor Spezialfällen} wie etwa der leeren Menge, einem
nulldimensionalen Vektorraum oder einer leeren Mannigfaltigkeit. Aussagen
formuliert dann nur für nichtleere Mengen, nichttriviale Vektorräume und so
weiter, obwohl diese Einschränkungen tatsächlich aber oftmals gar nicht
notwendig sind. In konstruktiver Mathematik wird man nun insofern darauf
aufmerksam gemacht, als dass der Nachweis, dass diese Einschränkungen in
bestimmten Fällen erfüllt sind, nicht mehr trivial ist, sondern Nachdenken
erfordert.

Ein anderes Beispiel liefert folgende Proposition, die oft als Übungsaufgabe in
einer Anfängervorlesung gestellt wird:
\begin{prop}Sei~$f : X \to Y$ eine Abbildung und~$f^{-1}[\freist] : \P(Y) \to
\P(X)$ die Urbildoperation (welche eine Teilmenge~$U \in \P(Y)$ auf~$\{ x \in X
\,|\, f(x) \in U \}$ schickt). Dann gilt: Genau dann ist~$f$ surjektiv,
wenn~$f^{-1}[\freist]$ injektiv ist.
\end{prop}
\begin{proof}[Beweis der Rückrichtung (umständlich, nur klassisch zulässig)]
Angenommen, die Abbildung~$f$ ist nicht surjektiv. Dann gibt es Element~$y \in
Y$, welches nicht im Bild von~$f$ liegt. Wenn wir die spezielle
Teilmenge~$\{y\} \in \P(Y)$ betrachten, sehen wir
\[ f^{-1}[\{y\}] = \emptyset = f^{-1}[\emptyset]. \]
Wegen der vorausgesetzten Injektivität folgt~$\{y\} = \emptyset$; das ist ein
Widerspruch.\end{proof}
\begin{proof}[Beweis der Rückrichtung (elegant, auch konstruktiv zulässig)]
Bezeichne~$\im f$ die Bildmenge von~$f$. Dann gilt
$f^{-1}[\im f] = f^{-1}[X]$
und damit~$\im f = X$, also ist~$f$ surjektiv.\end{proof}

\paragraph{Mentale Hygiene.} Arbeit in konstruktiver Logik ist gut für die
mentale Hygiene: Man lernt, genauer auf die Formulierung von Aussagen zu
achten, nicht unnötigerweise Verneinungen einzuführen und aufzupassen, an
welchen bestimmten Stellen klassische Axiome nötig sind. Bei passenden
Formulierungen ist das nämlich viel seltener, als man auf den ersten Blick
vielleicht vermutet.

\paragraph{Wertschätzung.} Klassische Mathematik kann man besser wertschätzen,
wenn man verstanden hat, wie anders sich konstruktive Mathematik anfühlt.
Die Frage, \emph{inwieweit genau} ein konstruktiver Beweis einer Aussage mehr
Inhalt als ein klassischer Beweis hat, kann in Einzelfällen sehr diffizil und
interessant sein. Wir werden zu diesem Thema noch einen mathematischen
Zaubertrick kennenlernen.

\paragraph{Programmextraktion.} Aus jedem konstruktiven Beweis einer Behauptung
kann man maschinell, ohne manuelles Zutun, ein Computerprogramm extrahieren,
welches die untersuchte Behauptung bezeugt (und bewiesenermaßen korrekt
arbeitet). Etwa ist in jedem konstruktiven Beweis der Behauptung
\begin{quote}Sei~$S$ eine endliche Menge von Primzahlen. Dann gibt es eine
weitere Primzahl, welche nicht in~$S$ liegt.\end{quote}
ein Algorithmus versteckt, welcher zu endlich vielen gegebenen Primzahlen
ganz konkret eine weitere Primzahl berechnet.

Solch maschinelle
Programmextraktion ist wichtig in der Informatik: Anstatt in einem ersten
Schritt ein Programm per Hand zu entwickeln und dann in einem zweiten
Schritt umständlich seine Korrektheit bezüglich einer vorgegebenen
Spezifikation zu zeigen, kann man auch direkt einen konstruktiven Beweis der
Erfüllbarkeit der Spezifikation führen und dann automatisch ein entsprechendes
Programm extrahieren lassen. In der akademischen Praxis wird dieses Vorgehen
tatsächlich angewendet.
% XXX: Literaturverweise!

\paragraph{Traummathematik.} Nur in einem konstruktiven Kontext ist die Arbeit
mit sog. \emph{Traum\-axio\-men}, wie etwa
\begin{quote}Jede Abbildung~$\RR \to \RR$ ist stetig.\end{quote}
oder
\begin{quote}Es gibt infinitesimale reelle Zahlen~$\varepsilon$
mit~$\varepsilon^2 = 0$, aber~$\varepsilon \neq 0$.\end{quote}
möglich: Denn in klassischer Logik sind diese Axiome offensichtlich schlichtweg
falsch. Sie sind aber durchaus interessant: Sie können die Arbeit
rechnerisch und konzeptionell vereinfachen (man muss nur einen Blick zu den
Physikern werfen), und es gibt Metatheoreme, die garantieren, dass Folgerungen
aus diesen Axiomen, welche nur mit konstruktiven Schlussregeln getroffen wurden
und eine bestimmte logische Form aufweisen, auch im üblichen klassischen Sinn
gelten.
% XXX: Stetigkeitsnachweise unnötig

\begin{bem}Hier ein kurzer Einschub, wieso das erstgenannte Traumaxiom
in einem konstruktiven Kontext zumindest nicht offensichtlich widersprüchlich
ist. Man könnte denken, dass die Signumsfunktion
\[ x \longmapsto \begin{cases}
  -1, & \text{falls $x < 0$,} \\
  \phantom{+}0, & \text{falls $x = 0$,} \\
  \phantom{+}1, & \text{falls $x > 0$}
\end{cases} \]
ein triviales Gegenbeispiel ist. Konstruktiv kann man aber nicht zeigen, dass
diese Funktion tatsächlich auf ganz~$\RR$ definiert ist: Die Definitionsmenge
ist nur
\[ \{ x \in \RR \ |\  x < 0 \,\vee\, x = 0 \,\vee\, x > 0 \}. \]
Andrej Bauer diskutiert dieses Beispiel in seinem Blog
ausführlicher~\cite{bauer:blog:stetigkeit}.
\end{bem}

\paragraph{Alternative Mathematik-Universen.} Wenn man ganz normal Mathematik
betreibt, arbeitet man tatsächlich \emph{intern im Topos der Mengen}. Es gibt
aber auch andere interessante Topoi; deren interne Sprache ist aber fast immer
nicht klassisch.
\begin{itemize}
\item Vielleicht hat man einen bestimmen topologischen Raum~$X$ besonders lieb
und möchte daher, dass alle untersuchten Objekte vom aktuellen Aufenthaltsort
auf dem Raum abhängen. Dann möchte man im \emph{Topos der Garben auf~$X$}
arbeiten.
\item Vielleicht ist man auch ein besonderer Freund einer bestimmten
Gruppe~$G$. Dann möchte man vielleicht, dass alle untersuchten Objekte
eine~$G$-Wirkung tragen und dass alle untersuchten Abbildungen~$G$-äquivariant
sind. Dann sollte man im \emph{Topos der~$G$-Mengen} arbeiten.
\item Vielleicht interessiert man sich sehr dafür, was Turingmaschinen
berechnen können. Dann kann man im \emph{effektiven Topos} arbeiten, der nur
solche Morphismen enthält, die durch Turingmaschinen algorithmisch gegeben
werden können.
\end{itemize}
Eine genauere Diskussion würde an dieser Stelle zu weit
führen. Es seien nur noch zwei Beispiele erwähnt, was mit der Topossichtweise
möglich ist:
\begin{itemize}
\item Aus dem recht einfach nachweisbaren Faktum konstruktiver linearer
Algebra, dass jeder endlich erzeugte Vektorraum nicht nicht eine endliche
Basis besitzt, folgt \emph{ohne weitere Arbeit} sofort folgende offensichtlich
kompliziertere Aussage, wenn man das Faktum intern im Garbentopos eines reduzierten
Schemas~$X$ interpretiert: Jeder~$\O_X$-Modul, der lokal von endlichem Typ ist,
ist auf einer dichten Teilmenge sogar lokal frei.
\item Zu quantenmechanischen Systemen kann man eine $C^*$-Algebra assoziieren.
Wichtiges Merkmal ist, dass diese in allen interessanten Fällen
\emph{nichtkommutativ} sein wird. Nun gibt es aber ein alternatives Universum,
den sog. \emph{Bohr-Topos}, aus dessen Sicht diese Algebra kommutativ ist; auf
diese Weise vereinfacht sich manches. (Was genau, werden wir noch gemeinsam
herausfinden.)
\end{itemize}


\section{Die Schlussregeln intuitionistischer Logik}

In den folgenden Abschnitten wollen wir \emph{Meta-Mathematik} betreiben: In
Abgrenzung von der sonst betriebenen Mathematik wollen wir nicht die üblichen
mathematischen Objekte wie Mengen, Vektorräume, Mannigfaltigkeiten untersuchen,
sondern \emph{Beweise}. Dazu müssen wir präzise festlegen, was unter einem
(intuitionistischen oder klassischen) Beweis zu verstehen ist.

\begin{defn}Ein \emph{Kontext} ist eine endliche Folge von
Variablendeklarationen der Form
\[ x_1 : A_1,\ \ldots,\ x_n : A_n. \]
Dabei sind die~$A_i$ \emph{Typen} der untersuchten formalen Systems.\end{defn}

Wir werden Kontexte oft kürzer als~$\vec x : \vec A$ notieren. Etwa ist die
Aussage
\[ n + n = n \quad\Longrightarrow\quad \forall m\?\NN{:}\ n + m = m \]
eine Aussage im Kontext~$n : \NN$. Die Variable~$m$ kommt nicht \emph{frei},
sondern durch den Allquantor \emph{gebunden} vor, und gehört daher nicht zum
Kontext. Wir vereinbaren, dass kollisionsfreie Umbenennung gebundener
Variablen die Aussage nicht verändert. Die anders geschribene Aussage
\[ n + n = n \quad\Longrightarrow\quad \forall u\?\NN{:}\ n + u = u \]
sehen wir also als dieselbe Aussage an. Wenn wir auch noch über die
Variable~$n$ quantifizieren, erhalten wir eine Aussage im \emph{leeren Kontext}:
\[ \forall n\?\NN{:}\ \Bigl[n + n = n \quad\Longrightarrow\quad \forall
u\?\NN{:}\ n + u = u\Bigr] \]

\begin{defn}Seien~$\varphi$ und~$\psi$ Aussagen in einem Kontext~$\vec x : \vec
A$. Genau dann ist~$\psi$ aus der Voraussetzung~$\varphi$ \emph{ableitbar}, in
Symbolen
\[ \varphi \seq{\vec x} \psi, \]
wenn es eine entsprechende endliche \emph{Ableitung} gibt, welche nur die
in Tafel~\ref{ableitungsregeln:int} aufgeführten Ableitungsregeln verwendet.
\end{defn}

% XXX: Termsubstitution, Beispiel, HA

\begin{table}
  \small
  \centering
  \textbf{Strukturelle Regeln} \\
  \vspace{-0.5em}
  \phantom{a}\hfill
  \AxiomC{$\phantom{\seq{\vec x}}$}\UnaryInfC{$\varphi \seq{\vec x} \varphi$}\DisplayProof\hfill
  \AxiomC{$\varphi \seq{\vec x} \psi$}\UnaryInfC{$\varphi[\vec s/\vec x]
  \seq{\vec y} \psi[\vec s/\vec x]$}\DisplayProof\hfill
  \AxiomC{$\varphi \seq{\vec x} \psi$}\AxiomC{$\psi \seq{\vec x}
  \chi$}\BinaryInfC{$\varphi \seq{\vec x} \chi$}\DisplayProof
  \phantom{a}\hfill
  \vspace{2.0em}

  \textbf{Regeln für Konjunktion} \\
  \vspace{-0.5em}
  \phantom{a}\hfill
  \AxiomC{$\phantom{\seq{\vec x}}$}\UnaryInfC{$\varphi \seq{\vec x} \top$}\DisplayProof\hfill
  \AxiomC{$\phantom{\seq{\vec x}}$}\UnaryInfC{$\varphi \wedge \psi \seq{\vec x} \varphi$}\DisplayProof\hfill
  \AxiomC{$\phantom{\seq{\vec x}}$}\UnaryInfC{$\varphi \wedge \psi \seq{\vec x} \psi$}\DisplayProof\hfill
  \AxiomC{$\varphi \seq{\vec x} \psi$}\AxiomC{$\varphi \seq{\vec x} \chi$}\BinaryInfC{$\varphi \seq{\vec x} \psi \wedge \chi$}\DisplayProof
  \phantom{a}\hfill
  \vspace{2em}

  \textbf{Regeln für Disjunktion} \\
  \vspace{-0.5em}
  \phantom{a}\hfill
  \AxiomC{$\phantom{\seq{\vec x}}$}\UnaryInfC{$\bot \seq{\vec x} \varphi$}\DisplayProof\hfill
  \AxiomC{$\phantom{\seq{\vec x}}$}\UnaryInfC{$\varphi \seq{\vec x} \varphi \vee \psi$}\DisplayProof\hfill
  \AxiomC{$\phantom{\seq{\vec x}}$}\UnaryInfC{$\psi \seq{\vec x} \varphi \vee \psi$}\DisplayProof\hfill
  \AxiomC{$\varphi \seq{\vec x} \chi$}\AxiomC{$\psi \seq{\vec x} \chi$}\BinaryInfC{$\varphi \vee \psi \seq{\vec x} \chi$}\DisplayProof
  \phantom{a}\hfill
  \vspace{2em}

  \textbf{Doppelregel für Implikation} \\
  \vspace{-0.5em}
  \phantom{a}\hfill
  \Axiom$\varphi \wedge \psi\ \fCenter\seq{\vec x} \chi$
  \doubleLine
  \UnaryInf$\varphi\ \fCenter\seq{\vec x} \psi \Rightarrow \chi$
  \DisplayProof
  \phantom{a}\hfill
  \vspace{2em}

  \textbf{Doppelregeln für Quantifikation} \\
  \vspace{-0.5em}
  \phantom{a}\hfill
  \Axiom$\varphi\ \fCenter\seq{\vec x, y} \psi$
  \doubleLine
  \UnaryInf$\exists y\?Y\_\! \varphi\ \fCenter\seq{\vec x} \psi$
  \DisplayProof
  {\tiny ($y$ keine Variable von~$\psi$)}
  \hfill
  \Axiom$\varphi\ \fCenter\seq{\vec x, y} \psi$
  \doubleLine
  \UnaryInf$\varphi\ \fCenter\seq{\vec x\phantom{, y}} \forall y\?Y\_\! \psi$
  \DisplayProof
  {\tiny ($y$ keine Variable von~$\varphi$)}
  \hfill\phantom{a}

  \caption{\label{ableitungsregeln:int}Die Schlussregeln intuitionistischer Logik.}
\end{table}

\begin{table}
  \small
  \centering
  \textbf{Regeln für Gleichheit} \\
  \vspace{-0.5em}
  \phantom{a}\hfill
  \AxiomC{$\phantom{\seq{\vec x}}$}
  \UnaryInfC{$\top \seq{x} x = x$}
  \DisplayProof
  \hfill
  \AxiomC{$\phantom{\seq{\vec x}}$}
  \UnaryInfC{$(\vec x = \vec y) \wedge \varphi \seq{\vec z} \varphi[\vec y/\vec x]$}
  \DisplayProof
  \hfill\phantom{a} \\[0.5em]
  (Dabei steht "`$\vec x = \vec y\,$"' für~$x_1 = y_1 \wedge \cdots \wedge x_n =
  y_n$.)
  \vspace{2em}

  \textbf{Prinzip vom ausgeschlossenen Dritten} \\
  \vspace{-0.5em}
  \phantom{a}\hfill
  \AxiomC{$\phantom{\seq{\vec x}}$}
  \UnaryInfC{$\top \seq{\vec x} \varphi \vee \neg\varphi$}
  \DisplayProof
  \hfill\phantom{a}

  \caption{\label{ableitungsregeln:weitere}Weitere Schlussregeln mancher
  formaler Systeme.}
\end{table}

\section{Beziehung zu klassischer Logik}

Doppelnegationsübersetzung, Continuation-Passing-Style Transformation,
historische Einordnung: Hilberts Programm, \ldots

\nocite{*}
\printbibliography

\end{document}

% Mögliche weitere Themen:
% * Lindenbaumzeugs
% * Suche auf unendlichen Datentypen
% * Andrejs Gems and Stones
% * Arbeit mit universellen Objekten und Elementen
%   (vielleicht erstmal universelle Ringelemente,
%   dann universelle Objektelemente (Scheibentopoi)
%   und zuletzt universelle Objekte (E[X]))
% * Mutter aller Monaden verstehen. Was ist die logische Interpretation?

% Im nächsten Vortrag erklären:
% * konkrete vs. abstrakte Aussagen
% * Hilberts Programm
% * verschiedene Stufen für Schlimmheit von LEM
% * "Int. Logik schwächer, aber feiner"
% * "Von konkreten Aussagen sollte es konkrete Beweise geben"
% * Proof mining auch für Beweise aus der Analysis relevant,
%   siehe etwa Kohlenbach, Applied Proof Theory.
% * Grund, wieso PA/HA inkonsistent sein KÖNNTE
% * Gelfand-Schneider-FUD wieder auflösen
%
% Dazu noch herausfinden:
% * historischer Umgang mit LEM
% * Theorem von Barr verstehen. Ist es wirklich nicht konstruktiv?
%   Bezug zu Friedmans Trick?
%
% Ideen für Übungsaufgaben:
% * Bei Coquand und Troelstra inspirieren lassen.

% Literatur:
% * http://math.stanford.edu/~feferman/papers/highlights.pdf
