\documentclass[a4paper,ngerman,12pt]{scrartcl}

\usepackage[utf8]{inputenc}

\usepackage[ngerman]{babel}
\addto\captionsngerman{\renewcommand\tablename{Tafel}}

\usepackage{amsmath,amsthm,amssymb,stmaryrd,color,graphicx}
\usepackage{setspace}
\usepackage{bussproofs}
\usepackage{array}
\usepackage{comment}

\usepackage[protrusion=true,expansion=true]{microtype}

\usepackage{lmodern}
\usepackage{tabto}

\usepackage[natbib=true,style=numeric]{biblatex}
\usepackage[babel]{csquotes}
\bibliography{literatur}

\usepackage[all]{xy}

\usepackage{hyperref}

\setlength\parskip{\medskipamount}
\setlength\parindent{0pt}

\theoremstyle{definition}
\newtheorem{defn}{Definition}[section]
\newtheorem{axiom}[defn]{Axiom}
\newtheorem{bsp}[defn]{Beispiel}

\theoremstyle{plain}

\newtheorem{prop}[defn]{Proposition}
\newtheorem{motto}[defn]{Motto}
\newtheorem{wunder}[defn]{Wunder}
\newtheorem{ueberlegung}[defn]{Überlegung}
\newtheorem{lemma}[defn]{Lemma}
\newtheorem{kor}[defn]{Korollar}
\newtheorem{hilfsaussage}[defn]{Hilfsaussage}
\newtheorem{satz}[defn]{Satz}

\theoremstyle{remark}
\newtheorem{bem}[defn]{Bemerkung}
\newtheorem{aufg}[defn]{Aufgabe}

\clubpenalty=10000
\widowpenalty=10000
\displaywidowpenalty=10000

\newcommand{\xra}[1]{\xrightarrow{#1}}
\newcommand{\lra}{\longrightarrow}
\newcommand{\lhra}{\ensuremath{\lhook\joinrel\relbar\joinrel\rightarrow}}
\newcommand{\thlra}{\relbar\joinrel\twoheadrightarrow}

\newcommand{\brak}[1]{\llbracket {#1} \rrbracket}

\newcommand{\ZZ}{\mathbb{Z}}
\newcommand{\QQ}{\mathbb{Q}}
\newcommand{\RR}{\mathbb{R}}
\newcommand{\NN}{\mathbb{N}}
\newcommand{\PP}{\mathbb{P}}
\renewcommand{\aa}{\mathfrak{a}}
\newcommand{\bb}{\mathfrak{b}}
\newcommand{\pp}{\mathfrak{p}}
\newcommand{\mm}{\mathfrak{m}}
\newcommand{\I}{\mathcal{I}}
\newcommand{\J}{\mathcal{J}}
\newcommand{\C}{\mathcal{C}}
\newcommand{\D}{\mathcal{D}}
\newcommand{\E}{\mathcal{E}}
\newcommand{\U}{\mathcal{U}}
\renewcommand{\I}{\mathcal{I}}
\renewcommand{\P}{\mathcal{P}}
\renewcommand{\O}{\mathcal{O}}
\newcommand{\Hom}{\mathrm{Hom}}
\newcommand{\Rad}{\mathrm{Rad}}
\newcommand{\ev}{\mathrm{ev}}
\newcommand{\id}{\mathrm{id}}
\newcommand{\Id}{\mathrm{Id}}
\newcommand{\freist}{\underline{\ \ }}
\DeclareMathOperator{\colim}{colim}
\DeclareMathOperator{\Ob}{Ob}
\DeclareMathOperator{\ggT}{ggT}
\DeclareMathOperator{\im}{im}
\newcommand{\op}{\mathrm{op}}
\newcommand{\Set}{\mathrm{Set}}
\newcommand{\Grp}{\mathrm{Grp}}
\newcommand{\Vect}[1]{{#1\text{-}\mathrm{Vect}}}
\newcommand{\AbGrp}{\mathrm{AbGrp}}
\newcommand{\Ring}{\mathrm{Ring}}
\newcommand{\Cat}{\mathrm{Cat}}
\newcommand{\Funct}{\mathrm{Funct}}
\newcommand{\Eins}{\mathbf{1}}
\newcommand{\Man}{\mathrm{Man}}
\newcommand{\Top}{\mathrm{Top}}
\newcommand{\seq}[1]{\mathrel{\vdash\!\!\!_{#1}}}
\renewcommand{\_}{\mathpunct{.}\,}
\newcommand{\?}{\,{:}\,}

\newcommand{\XXX}[1]{\textcolor{red}{#1}}

\renewcommand*\theenumi{\alph{enumi}}
\renewcommand{\labelenumi}{\theenumi)}

\newcommand\subsubsubsection[1]{\subsubsection*{#1}}
\definecolor{grey}{rgb}{0.7,0.7,0.7}

\setcounter{tocdepth}{2}

\newenvironment{indentblock}{%
  \list{}{\leftmargin\leftmargin}%
  \item\relax
}{%
  \endlist
}

\newcommand{\Alice}{\item[Alice]}
\newcommand{\Eve}{\item[Eve]}
\newenvironment{dialogue}{%
  \begin{list}{}{%
    \settowidth{\labelwidth}{\qquad\emph{Alice:}}
    \setlength{\labelsep}{0.3cm}
    \setlength{\leftmargin}{\labelwidth}
    \addtolength{\leftmargin}{\labelsep}
    \setlength{\rightmargin}{0pt}
    \setlength{\parsep}{0.5ex plus 0.2ex minus 0.1ex}
    \setlength{\itemsep}{0 ex plus 0.2ex}
    \renewcommand{\makelabel}[1]{\qquad\emph{##1:}\hfil}
    }
}{\end{list}}

%\newarrow{Equals}=====

%\usepackage{geometry}
%\geometry{tmargin=2cm,bmargin=4cm,lmargin=3cm,rmargin=3cm}

\begin{document}

\vspace*{2em}%
\begin{center}%
  \vskip 1em
  {\LARGE Pizzaseminar zu konstruktiver Mathematik}
  \vskip 1.5em%
  {\large
   \lineskip .5em%
    \begin{tabular}[t]{c}%
      \today
    \end{tabular}\par}%
    \vskip 1em%
\end{center}\par
\par\vskip 1.5em

\begin{center}\emph{in Entstehung befindlich, nur grobe Zusammenfassung}
\end{center}

\tableofcontents

\section{Was ist konstruktive Mathematik?}

\begin{prop}Es gibt irrationale Zahlen~$x,y$, sodass~$x^y$ rational ist.
\end{prop}
\begin{proof}[Beweis 1] Die Zahl~$\sqrt{2}^{\sqrt{2}}$ ist rational oder nicht
rational. Setze im ersten Fall~$x := \sqrt{2}$, $y := \sqrt{2}$. Setze im
zweiten Fall~$x := \sqrt{2}^{\sqrt{2}}$, $y := \sqrt{2}$.
\end{proof}
\begin{proof}[Beweis 2] Setze~$x := \sqrt{2}$ und~$y := \log_{\sqrt{2}} 3$.
Dann
ist die Potenz~$x^y = 3$ sicher rational. Die Irrationalität von~$y$ lässt sich
sogar einfacher als die von~$\sqrt{2}$ beweisen:
Gelte~$y = p/q$ mit~$p, q \in \ZZ$ und~$q \neq 0$. Da~$y > 0$, können wir
sogar~$p, q \in \NN$ annehmen.
Dann folgt $3 = (\sqrt{2})^{p/q}$, also~$3^{2q} = 2^p$. Das ist ein
Widerspruch zum Satz über die eindeutige Primfaktorzerlegung, denn auf der linken
Seite kommt der Primfaktor~$3$ vor, auf der rechten aber nicht.
\end{proof}

Der erste Beweis war \emph{unkonstruktiv}: Einem interessierten Gegenüber kann
man immer noch nicht ein Zahlenpaar mit den gewünschten Eigenschaften nennen.
Der zweite Beweis dagegen war konstruktiv: Die Existenzbehauptung wurde durch
explizite Konstruktion eines Beispiels nachgewiesen.

Es stellt sich heraus, dass von den vielen Schlussregeln klassischer Logik genau
ein Axiom für die Zulässigkeit unkonstruktiver Argumente verantwortlich ist,
nämlich das \emph{Prinzip vom ausgeschlossenen Dritten}:
\begin{axiom}[vom ausgeschlossenen Dritten, LEM]Für jede Aussage~$\varphi$ gilt: $\varphi \vee
\neg\varphi$.\end{axiom}
Unter konstruktiver Mathematik im engeren Sinn, genauer
\emph{intuitionistischer Logik}, versteht man daher klassische Logik ohne LEM.
Das \emph{Prinzip der Doppelnegationselimination}, demnach man für jede
Aussage~$\varphi$ voraussetzen darf, dass~$\neg\neg\varphi \Rightarrow \varphi$
gilt, ist zu LEM äquivalent (Übungsaufgabe) und kann daher ebenfalls nicht
verwendet werden.

In konstruktiver Mathematik behauptet man \emph{nicht}, dass das
Prinzip vom ausgeschlossenen Dritten falsch wäre: Intuitionistische Logik ist
abwärtskompatibel zu klassischer Logik -- jede konstruktiv nachweisbare Aussage
gilt auch klassisch -- und manche konkrete Instanzen des Prinzips lassen sich
sogar konstruktiv nachweisen (siehe Proposition~\ref{natdiskret} für ein Beispiel).
Stattdessen verwendet man das Prinzip einfach
nur nicht. (Tatsächlich kann man leicht zeigen, dass es keine Gegenbeispiele
des Prinzips geben kann: Für jede Aussage~$\varphi$ gilt~$\neg(\neg\varphi
\wedge \neg\neg\varphi)$.)

\begin{bem}Manche Dozenten erzählen Erstsemestern folgende vereinfachte Version
der Wahrheit: Eine Aussage
erkennt man daran, dass sie entweder wahr oder falsch ist. Diese
Charakterisierung mag bei klassischer Logik noch vertretbar sein, ist aber in
einem konstruktiven Kontext offensichtlich unsinnig. Stattdessen erkennt man
eine Aussage daran, dass sie rein von ihrer grammatikalischen Struktur her ein
Aussagesatz ist (und natürlich dass alle vorkommenden Begriffe eine klare
Bedeutung haben).\end{bem}

\begin{bem}In konstruktiver Mengenlehre muss man auf das Auswahlaxiom
verzichten, denn in Gegenwart des restlichen Axiome impliziert dieses das
Prinzip vom ausgeschlossenen Dritten.\end{bem}
% XXX: Beweis einfügen


\subsection{Widerspruchsbeweise vs. Beweise von Negationen}
\label{widerspruchvsnegation}

Ein übliches Gerücht über konstruktive Mathematik besagt, dass der Begriff
\emph{Widerspruch} konstruktiv generell verboten ist. Dem ist nicht so. Man
muss zwischen zwei für das klassische Auge sehr ähnlich aussehenden
Beweisfiguren unterscheiden:
\begin{enumerate}
\item[1.] "`Angenommen, es gilt~$\neg\varphi$. Dann \ldots, Widerspruch; also
gilt~$\neg(\neg\varphi)$ und somit~$\varphi$."'
\item[2.] "`Angenommen, es gilt~$\psi$. Dann \ldots, Widerspruch; also
gilt~$\neg\psi$."'
\end{enumerate}
Argumente der ersten Form sind tatsächlich Widerspruchsbeweise und daher
konstruktiv nicht pauschal zulässig -- wenn man nicht anderweitig für die
untersuchte Aussage~$\varphi$ begründen
kann, dass aus ihrer Doppelnegation schon sie selbst folgt, beweist ein
solches Argument nur die Gültigkeit von~$\neg\neg\varphi$; das ist konstruktiv
schwächer als~$\varphi$.

Argumente der zweiten Form sind dagegen konstruktiv völlig einwandfrei: Sie
sind Beweise negierter Aussagen und nicht Widerspruchsbeweise im eigentlichen
Sinn. Die Zulässigkeit erklärt sich direkt nach Definition:
Die Negation wird (übrigens auch in klassischer Logik) als
\[ \neg\psi :\equiv (\psi \Rightarrow \bot) \]
festgelegt. Dabei steht~"`$\bot$"' für \emph{Falschheit}, eine kanonische falsche
Aussage. Wer mag, kann~$1 = 0$ oder~$\lightning$ denken.

Hier ein konkretes Beispiel aus der Zahlentheorie, um den Unterschied zu
demonstrieren:
\begin{prop}Die Zahl~$\sqrt{2}$ ist nicht rational.\end{prop}
\begin{proof}[Beweis (nur klassisch zulässig)]
Angenommen, die Behauptung ist falsch, d.\,h. die Zahl~$\sqrt{2}$ ist nicht
nicht rational. Dann ist~$\sqrt{2}$ also rational. Somit gibt es ganze
Zahlen~$p$ und~$q$ mit~$\sqrt{2} = p / q$. Daraus folgt die Beziehung~$2q^2 =
p^2$, die einen Widerspruch zum Satz über die Eindeutigkeit der
Primfaktorzerlegung darstellt: Auf der linken Seite kommt der Primfaktor~$2$
ungerade oft, auf der rechten Seite aber gerade oft vor.
\end{proof}
\begin{proof}[Beweis (auch konstruktiv zulässig)]
Angenommen, die Zahl~$\sqrt{2}$ ist rational. Dann gibt es ganze Zahlen \ldots,
Widerspruch. (Der verwendete Satz über die Eindeutigkeit der
Primfaktorzerlegung lässt sich konstruktiv beweisen.)
\end{proof}


\subsection{Informale Bedeutung logischer Aussagen}

\subsubsection*{\ldots über Belege (die
Brouwer-Heyting-Kolmogorov-Interpretation)}

Die Ablehnung des Prinzips vom ausgeschlossenen Dritten erscheint uns durch
unsere klassische Ausbildung als völlig verrückt: \emph{Offensichtlich} ist
doch jede Aussage entweder wahr oder falsch! Die Verwunderung löst sich auf,
wenn man akzeptiert, dass konstruktive Mathematiker zwar dieselbe
\emph{logische Sprache} verwenden ($\wedge, \vee, \Rightarrow, \neg, \forall,
\exists$), aber eine andere Bedeutung im Sinn haben: Wenn eine konstruktive
Mathematikerin eine Aussage~$\varphi$ behauptet, meint sie, dass sie einen
\emph{expliziten Beleg} für~$\varphi$ hat.

Den Basisfall bilden dabei die sog. \emph{atomaren Aussagen}, von denen wir
intuitiv wissen, wie ein Beleg ihrer Gültigkeit aussehen sollte. Atomare Aussagen sind
solche, die nicht vermöge der logischen Operatoren $\wedge, \vee,
\Rightarrow$ und der Quantoren~$\forall, \exists$ aus weiteren Teilformeln
zusammengesetzt sind. In der Zahlentheorie sind atomare Aussagen etwa von der
Form
\[ n = m, \]
wobei~$n$ und~$m$ Terme für natürliche Zahlen sind; in der Mengenlehre sind
atomare Aussagen von der Form
\[ x \in M. \]

Für \emph{zusammengesetzte Aussagen} zeigt Tafel~\ref{bhk}, was unter Belegen
jeweils zu verstehen ist. (An manchen Stellen steht dort~"`$x : X$"' -- das hat
einen Grund, aber momentan soll das einfach etwas seltsame Notation für~"`$x
\in X$"' sein.) Etwa ist ein Beleg für eine Aussage der Form
\[ \forall n \? \NN{:}\ \varphi(x) \Rightarrow \psi(x) \]
eine Vorschrift, wie man für jede natürliche Zahl~$n : \NN$ aus Belegen
für~$\varphi(x)$ Belege für~$\psi(x)$ erhalten kann. Dies soll tatsächlich nur
\emph{eine} Vorschrift sein (welche mit allen natürlichen Zahlen zurechtkommt),
nicht für jede natürliche Zahl jeweils eine. Das ist mit \emph{gleichmäßig} in
der Tabelle gemeint.

\begin{table}
  \centering
  \small
  \setlength{\extrarowheight}{0.3em}
  \begin{tabular}{@{}r|p{5.9cm}|p{6.5cm}}
    & {klassische Logik} & {intuitionistische Logik}
    \\\hline
    Aussage $\varphi$ & Die Aussage $\varphi$ gilt. & Wir haben Beleg für $\varphi$. \\
    $\bot$ & Es stimmt Falschheit. & Wir haben Beleg für Falschheit. \\
    $\varphi \wedge \psi$ & $\varphi$ und $\psi$ stimmen. & Wir haben Beleg für~$\varphi$ und für~$\psi$. \\
    $\varphi \vee \psi$ & $\varphi$ oder $\psi$ stimmt. & Wir haben Beleg für~$\varphi$ oder für~$\psi$. \\
    $\varphi \Rightarrow \psi$ & Sollte~$\varphi$ stimmen, dann auch~$\psi$. &
    Aus Belegen für~$\varphi$ können wir (gleichmäßig) Belege für~$\psi$ konstruieren. \\
    $\neg\varphi$ &
      $\varphi$ stimmt nicht. &
      Es kann keinen Beleg für~$\varphi$ geben. \\
    $\forall x\?X{:}\ \varphi(x)$ & Für alle $x : X$ stimmt jeweils~$\varphi(x).$ &
      Wir können (gleichmäßig) für alle~$x : X$ Belege für~$\varphi(x)$ konstruieren. \\
    $\exists x\?X{:}\ \varphi(x)$ & \raggedright Es gibt mindestens ein~$x : X$, für das~$\varphi(x)$
    stimmt. & {\raggedright
      Wir haben ein~$x : X$ zusammen mit Beleg für~$\varphi(x).$} \\
  \end{tabular}
  \caption{\label{bhk}Informale rekursive Definition des Belegbegriffs.}
\end{table}

\begin{bsp}
Unter dieser Interpretation meint das Prinzip vom ausgeschlossenen Dritten, dass wir für jede
Aussage Beleg für sie oder ihre Negation haben. Das ist aber offensichtlich
nicht der Fall.
\end{bsp}

\begin{bsp}
Die Interpretation von~$\neg\neg\varphi$ ist, dass es keinen Beleg
für~$\neg\varphi$ gibt. Daraus folgt natürlich noch nicht, dass wir tatsächlich
Beleg für~$\varphi$ haben; gewissermaßen ist eine solche Aussage~$\varphi$ nur
"`potenziell wahr"'.
\end{bsp}

\begin{bsp}Wenn wir wissen, dass sich unser Haustürschlüssel irgendwo in der
Wohnung befinden muss (da wir ihn letzte Nacht verwendet haben, um die Tür
aufzusperren), wir ihn momentan aber nicht finden, so können wir konstruktiv
nur die doppelt negierte Aussage
\[ \neg\neg (\exists x{:}\ \text{der Schlüssel befindet sich an Position~$x$})
\]
vertreten.\end{bsp}

\begin{bsp}Wir stehen im Supermarkt und erinnern uns, dass wir unbedingt
gewisse Zutaten einkaufen müssen. Leider fällt uns momentan keine einzige der
Zutaten mehr ein. Dann können wir zwar die Aussage, dass die Menge der zu
besorgenden Zutaten nicht leer ist, vertreten, nicht jedoch die stärkere
Aussage, dass diese Menge ein Element enthält.\end{bsp}

\begin{bsp}[\cite{sigfpe:katemoss}]
Es war ein Video aufgetaucht, dass Kate Moss beim Konsumieren von Drogen zeigte,
und zwar entweder solche von einem Typ~A oder solche von einem Typ~B. Welcher
Typ aber tatsächlich vorlag, konnte nicht entschieden werden. Daher gab es für
keine der beiden Strafttaten einen Beleg, Kate Moss wurde daher nicht
strafrechtlich verfolgt.
\end{bsp}


\subsubsection*{\ldots über Berechenbarkeit}

Es gibt noch eine zweite Interpretation, die beim Verständnis konstruktiver
Mathematik sehr hilfreich ist:
\begin{motto}Eine Aussage gilt konstruktiv genau dann, wenn es ein
Computerprogramm gibt, welches sie in endlicher Zeit bezeugt.\end{motto}
Etwa ist mit dieser Interpretation klar, dass die formale Aussage
\[ \forall n \in \NN{:}\ \exists p \geq n{:}\ \text{$p$ ist eine Primzahl}, \]
eine Formulierung der Unendlichkeit der Primzahlen, auch konstruktiv
stimmt: Denn man kann leicht ein Computerprogramm angeben, das eine natürliche
Zahl~$n$ als Eingabe erwartet und dann, etwa über die Sieb-Methode von
Eratosthenes, eine Primzahl~$p \geq n$ produziert (zusammen mit einem Nachweis,
dass~$p$ tatsächlich prim ist).

\begin{bem}Das Motto kann man tatsächlich zu einem formalen Theorem
präzisieren, das ist Gegenstand der gefeierten
Curry--Howard-Korrespondenz.\end{bem}


\section{Beispiele}

\subsection{Diskretheit der natürlichen Zahlen}

Manche konkrete Instanzen des Prinzips vom ausgeschlossenen Dritten lassen sich
konstruktiv nachweisen:

\begin{prop}\label{natdiskret}Für beliebige natürlichen Zahlen~$x,y \in \NN$
gilt: $x = y \vee \neg(x = y)$.\end{prop}
\begin{proof}Das ist konstruktiv \emph{nicht} klar, aber beweisbar durch eine
Doppelinduktion.\end{proof}

Diese Eigenschaft wird auch als Diskretheit der Menge der natürlichen Zahlen
bezeichnet: Allgemein heißt eine Menge~$X$ genau dann \emph{diskret}, wenn für
alle~$x,y \in X$ die Aussage~$x = y \vee \neg(x = y)$ gilt. Klassisch ist jede
Menge diskret.

Die reellen Zahlen sind in diesem Sinne nicht diskret. Das macht
man sich am einfachsten über die algorithmische Interpretation klar: Es kann
kein Computerprogramm geben, dass \emph{in endlicher Zeit} zwei reelle Zahlen
auf Gleichheit testet. Denn in endlicher Zeit kann ein Programm nur endlich viele
Nachkommaziffern (besser: endlich viele rationale Approximationen) abfragen;
haben die beiden zu vergleichenden Zahlen dieselben Nachkommaziffern, so kann
sich das Programm aber in endlicher Zeit nie sicher sein, ob irgendwann doch noch
eine Abweichung auftreten wird.

Übrigens ist die Menge der algebraischen Zahlen durchaus diskret:
Man kann ein
Programm angeben, dass zwei algebraische Zahlen~$x,y$ zusammen mit \emph{Zeugen}
ihrer Algebraizität, also Polynomgleichungen mit rationalen Koeffizienten
und~$x$ bzw.~$y$ als Lösung, als Eingabe erwartet und dann entscheidet, ob~$x$
und~$y$ gleich sind oder nicht. Der Beweis ist nicht trivial, aber auch nicht
fürchterlich kompliziert; siehe etwa Proposition~1.6 in~\cite{nw:algebra}.


\subsection{Minima von Teilmengen der natürlichen Zahlen}

In klassischer Logik gilt folgendes Minimumsprinzip:
\begin{prop}[in klassischer Logik]Sei~$U \subseteq \NN$ eine bewohnte
Teilmenge. Dann enthält~$U$ ein kleinstes Element.\end{prop}
Dabei heißt eine Menge~$U$ \emph{bewohnt}, falls~$\exists u \in U$.
In konstruktiver Mathematik kann man diese Aussage nicht zeigen -- wegen der
Ab\-wärts\-kom\-pa\-ti\-bi\-li\-tät kann man zwar auch nicht ihr Gegenteil
nachweisen, aber man kann ein sog. \emph{brouwersches Gegenbeispiel}
anführen:
\begin{prop}Besitze jede bewohnte Teilmenge der natürlichen Zahlen ein Minimum.
Dann gilt das Prinzip vom ausgeschlossenen Dritten.\end{prop}
\begin{proof}Sei~$\varphi$ eine beliebige Aussage. Wir müssen zeigen,
dass~$\varphi$ oder~$\neg\varphi$ gilt. Dazu definieren wir die Teilmenge
\[ U := \{ n \in \NN \,|\, n = 1 \vee \varphi \}. \]
Die Zugehörigkeitsbedingung ist etwas komisch, da die Aussage~$\varphi$ ja
nicht von der frischen Variable~$n$ abhängt, aber völlig okay. Da~$U$
sicherlich bewohnt ist (durch~$1 \in U$), besitzt~$U$ nach Voraussetzung ein
Minimum~$z \in U$.

Wegen der diskutierten Diskretheit der natürlichen Zahlen gilt~$z = 0$ oder~$z
\neq 0$. Im ersten Fall folgt~$\varphi$ (denn~$0 \in U$ ist gleichbedeutend
mit~$0 = 1 \vee \varphi$, also mit~$\varphi$), im zweiten Fall folgt~$\neg\varphi$ (denn
wenn~$\varphi$ gälte, wäre~$U = \NN$ und somit~$z = 0$ im Widerspruch zu~$z
\neq 0$).\end{proof}

Wir können das Minimumsprinzip retten, wenn wir eine klassisch triviale
Zusatzbedingung stellen:
\begin{defn}Eine Teilmenge~$U \subseteq X$ heißt genau dann
\emph{herauslösbar}, wenn für alle~$x \in X$ gilt: $x \in U \vee \neg(x \in
U)$.\end{defn}
\begin{prop}Sei~$U \subseteq \NN$ eine bewohnte und herauslösbare Teilmenge.
Dann enthält~$U$ ein kleinstes Element.\end{prop}
\begin{proof}Da~$U$ bewohnt ist, liegt eine Zahl~$n$ in~$U$. Da ferner~$U$
diskret ist, gilt für jede Zahl~$0 \leq m \leq n$: $m \in U$ oder~$m \not\in
U$. Daher können wir diese Zahlen der Reihe nach durchgehen; die erste Zahl
mit~$m \in U$ ist das gesuchte Minimum.
\end{proof}
Weg mag, kann diesen Beweis auch präzisieren und einen formalen
Induktionsbeweis führen. Gut erkennbar ist, wie im Beweis ein expliziter
Algorithmus zur Findung des Minimums enthalten ist.

\begin{bem}Statt eine Zusatzbedingung einzuführen, kann man auch die Behauptung
abschwächen. Man kann nämlich mittels Induktion zeigen, dass jede
bewohnte Teilmenge der natürlichen Zahlen \emph{nicht nicht} ein Minimum
besitzt. Der algorithmische Inhalt eines Beweises dieser abgeschwächten Aussage
ist sehr interessant und wir werden noch lernen, wie man ihn deuten kann.\end{bem}


\subsection{Potenzmengen}

Klassisch ist die Potenzmenge der einelementigen Menge~$\{\star\}$ völlig
langweilig: Sie enthält genau zwei Elemente, nämlich die leere Teilmenge
und~$\{\star\}$ selbst. Konstruktiv lässt sich das nicht zeigen, die
Potenzmenge hat (potenziell!) viel mehr Struktur. Das ist Gegenstand einer
Übungsaufgabe.


\subsection{Die De Morganschen Gesetze}

In klassischer Logik verwendet man oft die De Morganschen Gesetze, manchmal
sogar implizit, um verschachtelte Aussagen zu vereinfachen. In konstruktiver
Mathematik lässt sich nur noch eines der beiden Gesetze in seiner vollen Form
nachweisen. Den Beweis der folgenden Proposition führen wir mit Absicht recht
ausführlich, damit man eine Imitationsgrundlage für die Bearbeitung des ersten
Übungsblatts hat. Es wird das Wort "`Widerspruch"' vorkommen, aber wir haben ja
schon in Abschnitt~\ref{widerspruchvsnegation} diskutiert, dass das nicht
automatisch unkonstruktiv ist.

\begin{prop}Für alle Aussagen~$\varphi$ und $\psi$ gilt
\begin{enumerate}
\item $\neg(\varphi \vee \psi) \quad\Longleftrightarrow\quad \neg\varphi \wedge
\neg\psi$,
\item $\neg(\varphi \wedge \psi) \quad\Longleftarrow\quad \neg\varphi \vee
\neg\psi$.
\end{enumerate}
\end{prop}
\begin{proof}\begin{enumerate}
\item "`$\Rightarrow$"': Wir müssen~$\neg\varphi$ und~$\neg\psi$ zeigen:
\begin{itemize}
\item Angenommen, es gilt doch~$\varphi$. Dann gilt auch~$\varphi \vee \psi$. Da
nach Voraussetzung~$\neg(\varphi \vee \psi)$, folgt ein Widerspruch.
\item Analog zeigt man~$\neg\psi$.
\end{itemize}

"`$\Leftarrow$"': Wir müssen zeigen, dass~$\neg(\varphi \vee \psi)$. Dazu
nehmen wir an, dass~$\varphi \vee \psi$ doch gilt, und streben einen Widerspruch an.
Dann gibt es zwei Fälle:
\begin{itemize}
\item Falls~$\varphi$ gilt: Aus der Voraussetzung~$\neg\varphi \wedge \neg\psi$
folgt insbesondere~$\neg\varphi$. Somit folgt ein Widerspruch.
\item Falls~$\psi$ gilt, folgt ein Widerspruch auf analoge Art und
Weise.
\end{itemize}

\item
Wir müssen zeigen, dass~$\neg(\varphi \wedge \psi)$. Dazu nehmen wir an, dass
doch~$\varphi \wedge \psi$ (also dass~$\varphi$ und dass~$\psi$), und streben
einen Widerspruch an. Nach Voraussetzung können wir zwei Fälle unterscheiden:
\begin{itemize}
\item Falls~$\neg\varphi$: Dann folgt ein Widerspruch zu~$\varphi$.
\item Falls~$\neg\psi$: Dann folgt ein Widerspruch zu~$\psi$.\qedhere
\end{itemize}
\end{enumerate}
\end{proof}

Die Hinrichtung in Regel~b) lässt sich konstruktiv nicht nachweisen. Im
Belegdenken ist das plausibel: Wenn wir lediglich wissen, dass es keinen Beleg
für~$\varphi \wedge \psi$ gibt, wissen wir noch nicht, ob es keinen Beleg
für~$\varphi$ oder keinen Beleg für~$\psi$ gibt. Tatsächlich ist die
Hinrichtung in Regel~b) äquivalent zu einer schwächeren Version des Prinzips
vom ausgeschlossenen Dritten:

\begin{prop}Folgende Prinzipien sind zueinander äquivalent:
\begin{enumerate}
\item[1.] LEM für negierte Aussagen: Für alle Aussagen~$\varphi$
gilt~$\neg\varphi \vee \neg\neg\varphi$.
\item[2.] Für alle Aussagen~$\varphi$ und $\psi$ gilt $\neg(\varphi \wedge \psi)
\Longrightarrow \neg\varphi \vee \neg\psi$.
\end{enumerate}
\end{prop}
Es ist besser, diese Proposition selbstständig zu beweisen als den folgenden
Beweis zu lesen. Denn wenn man nicht genau den Beweisvorgang mitverfolgt,
verirrt man sich leicht in den vielen Negationen.
\begin{proof}"`1. $\Rightarrow$ 2."': Seien~$\varphi$ und~$\psi$ beliebige
Aussagen. Gelte~$\neg(\varphi \wedge \psi)$. Nach
Voraussetzung gilt~$\neg\varphi$ oder~$\neg\neg\varphi$. Im ersten Fall sind
wir fertig. Im zweiten Fall folgt tatsächlich~$\neg\psi$: Denn
wenn~$\psi$ gälte, gälte auch~$\neg\varphi$ (denn wenn~$\varphi$, folgt ein
Widerspruch zu~$\neg(\varphi \wedge \psi)$), aber das wäre ein Widerspruch
zu~$\neg\neg\varphi$.

"`2. $\Rightarrow$ 1."': Sei~$\varphi$ eine beliebige Aussage. Da~$\neg(\varphi
\wedge \neg\varphi)$ (wieso?), folgt nach Voraussetzung $\neg\varphi \vee
\neg\neg\varphi$, das war zu zeigen.
\end{proof}


\section{Nutzen konstruktiver Mathematik}

\paragraph{Spaß.} Konstruktive Mathematik macht Spaß!

\paragraph{Philosophie.}
Konstruktive Logik ist philosophisch einfacher zu rechtfertigen als
klassische Logik.
% XXX: ausführen

\paragraph{Eleganzassistenz.}
Konstruktive Mathematik kann einen dabei unterstützen, Aussagen, Beweise und
ganze Theoriegebäude eleganter zu formulieren. Etwa hat man klassisch oft
\emph{Angst vor Spezialfällen} wie etwa der leeren Menge, einem
nulldimensionalen Vektorraum oder einer leeren Mannigfaltigkeit. Aussagen
formuliert dann nur für nichtleere Mengen, nichttriviale Vektorräume und so
weiter, obwohl diese Einschränkungen tatsächlich aber oftmals gar nicht
notwendig sind. In konstruktiver Mathematik wird man nun insofern darauf
aufmerksam gemacht, als dass der Nachweis, dass diese Einschränkungen in
bestimmten Fällen erfüllt sind, nicht mehr trivial ist, sondern Nachdenken
erfordert.

Ein anderes Beispiel liefert folgende Proposition, die oft als Übungsaufgabe in
einer Anfängervorlesung gestellt wird:
\begin{prop}Sei~$f : X \to Y$ eine Abbildung und~$f^{-1}[\freist] : \P(Y) \to
\P(X)$ die Urbildoperation (welche eine Teilmenge~$U \in \P(Y)$ auf~$\{ x \in X
\,|\, f(x) \in U \}$ schickt). Dann gilt: Genau dann ist~$f$ surjektiv,
wenn~$f^{-1}[\freist]$ injektiv ist.
\end{prop}
\begin{proof}[Beweis der Rückrichtung (umständlich, nur klassisch zulässig)]
Angenommen, die Abbildung~$f$ ist nicht surjektiv. Dann gibt es Element~$y \in
Y$, welches nicht im Bild von~$f$ liegt. Wenn wir die spezielle
Teilmenge~$\{y\} \in \P(Y)$ betrachten, sehen wir
\[ f^{-1}[\{y\}] = \emptyset = f^{-1}[\emptyset]. \]
Wegen der vorausgesetzten Injektivität folgt~$\{y\} = \emptyset$; das ist ein
Widerspruch.\end{proof}
\begin{proof}[Beweis der Rückrichtung (elegant, auch konstruktiv zulässig)]
Bezeichne~$\im f$ die Bildmenge von~$f$. Dann gilt
$f^{-1}[\im f] = f^{-1}[X]$
und damit~$\im f = X$, also ist~$f$ surjektiv.\end{proof}

\paragraph{Mentale Hygiene.} Arbeit in konstruktiver Logik ist gut für die
mentale Hygiene: Man lernt, genauer auf die Formulierung von Aussagen zu
achten, nicht unnötigerweise Verneinungen einzuführen und aufzupassen, an
welchen bestimmten Stellen klassische Axiome nötig sind. Bei passenden
Formulierungen ist das nämlich viel seltener, als man auf den ersten Blick
vielleicht vermutet.

\paragraph{Wertschätzung.} Klassische Mathematik kann man besser wertschätzen,
wenn man verstanden hat, wie anders sich konstruktive Mathematik anfühlt.
Die Frage, \emph{inwieweit genau} ein konstruktiver Beweis einer Aussage mehr
Inhalt als ein klassischer Beweis hat, kann in Einzelfällen sehr diffizil und
interessant sein. Wir werden zu diesem Thema noch einen mathematischen
Zaubertrick kennenlernen.

\paragraph{Programmextraktion.} Aus jedem konstruktiven Beweis einer Behauptung
kann man maschinell, ohne manuelles Zutun, ein Computerprogramm extrahieren,
welches die untersuchte Behauptung bezeugt (und bewiesenermaßen korrekt
arbeitet). Etwa ist in jedem konstruktiven Beweis der Behauptung
\begin{quote}Sei~$S$ eine endliche Menge von Primzahlen. Dann gibt es eine
weitere Primzahl, welche nicht in~$S$ liegt.\end{quote}
ein Algorithmus versteckt, welcher zu endlich vielen gegebenen Primzahlen
ganz konkret eine weitere Primzahl berechnet.

Solch maschinelle
Programmextraktion ist wichtig in der Informatik: Anstatt in einem ersten
Schritt ein Programm per Hand zu entwickeln und dann in einem zweiten
Schritt umständlich seine Korrektheit bezüglich einer vorgegebenen
Spezifikation zu zeigen, kann man auch direkt einen konstruktiven Beweis der
Erfüllbarkeit der Spezifikation führen und dann automatisch ein entsprechendes
Programm extrahieren lassen. In der akademischen Praxis wird dieses Vorgehen
tatsächlich angewendet.
% XXX: Literaturverweise!

Auch kann Programmextraktion didaktisch sinnvoll sein: Um etwa eine allgemeine
Existenzbehauptung zu verstehen, ist es hilfreich, sie in einem konkreten
Beispielfall durchzudenken. Einen konstruktiven Beweis der Existenzbehauptung
kann man dafür stets Schritt für Schritt durchgehen und so am Ende das behauptete
Objekt erhalten.

\paragraph{Traummathematik.} Nur in einem konstruktiven Kontext ist die Arbeit
mit sog. \emph{Traum\-axio\-men}, wie etwa
\begin{quote}Jede Abbildung~$\RR \to \RR$ ist stetig.\end{quote}
oder
\begin{quote}Es gibt infinitesimale reelle Zahlen~$\varepsilon$
mit~$\varepsilon^2 = 0$, aber~$\varepsilon \neq 0$.\end{quote}
möglich: Denn in klassischer Logik sind diese Axiome offensichtlich schlichtweg
falsch. Sie sind aber durchaus interessant: Sie können die Arbeit
rechnerisch und konzeptionell vereinfachen (man muss nur einen Blick zu den
Physikern werfen), und es gibt Metatheoreme, die garantieren, dass Folgerungen
aus diesen Axiomen, welche nur mit konstruktiven Schlussregeln getroffen wurden
und eine bestimmte logische Form aufweisen, auch im üblichen klassischen Sinn
gelten.
% XXX: Stetigkeitsnachweise unnötig

\begin{bem}Hier ein kurzer Einschub, wieso das erstgenannte Traumaxiom
in einem konstruktiven Kontext zumindest nicht offensichtlich widersprüchlich
ist. Man könnte denken, dass die Signumsfunktion
\[ x \longmapsto \begin{cases}
  -1, & \text{falls $x < 0$,} \\
  \phantom{+}0, & \text{falls $x = 0$,} \\
  \phantom{+}1, & \text{falls $x > 0$}
\end{cases} \]
ein triviales Gegenbeispiel ist. Konstruktiv kann man aber nicht zeigen, dass
diese Funktion tatsächlich auf ganz~$\RR$ definiert ist: Die Definitionsmenge
ist nur
\[ \{ x \in \RR \ |\  x < 0 \,\vee\, x = 0 \,\vee\, x > 0 \}. \]
Andrej Bauer diskutiert dieses Beispiel in seinem Blog
ausführlicher~\cite{bauer:blog:stetigkeit}.
\end{bem}

\paragraph{Alternative Mathematik-Universen.} Wenn man ganz normal Mathematik
betreibt, arbeitet man tatsächlich \emph{intern im Topos der Mengen}. Es gibt
aber auch andere interessante Topoi; deren interne Sprache ist aber fast immer
nicht klassisch.
\begin{itemize}
\item Vielleicht hat man einen bestimmen topologischen Raum~$X$ besonders lieb
und möchte daher, dass alle untersuchten Objekte vom aktuellen Aufenthaltsort
auf dem Raum abhängen. Dann möchte man im \emph{Topos der Garben auf~$X$}
arbeiten.
\item Vielleicht ist man auch ein besonderer Freund einer bestimmten
Gruppe~$G$. Dann möchte man vielleicht, dass alle untersuchten Objekte
eine~$G$-Wirkung tragen und dass alle untersuchten Abbildungen~$G$-äquivariant
sind. Dann sollte man im \emph{Topos der~$G$-Mengen} arbeiten.
\item Vielleicht interessiert man sich sehr dafür, was Turingmaschinen
berechnen können. Dann kann man im \emph{effektiven Topos} arbeiten, der nur
solche Morphismen enthält, die durch Turingmaschinen algorithmisch gegeben
werden können.
\end{itemize}
Eine genauere Diskussion würde an dieser Stelle zu weit
führen. Es seien nur noch zwei Beispiele dafür erwähnt, was mit der Topossichtweise
möglich ist:
\begin{itemize}
\item Aus dem recht einfach nachweisbaren Faktum konstruktiver linearer
Algebra, dass jeder endlich erzeugte Vektorraum \emph{nicht nicht} eine endliche
Basis besitzt, folgt \emph{ohne weitere Arbeit} sofort folgende offensichtlich
kompliziertere Aussage, wenn man das Faktum intern im Garbentopos eines reduzierten
Schemas~$X$ interpretiert: Jeder~$\O_X$-Modul, der lokal von endlichem Typ ist,
ist auf einer dichten Teilmenge sogar lokal frei.
\item Zu quantenmechanischen Systemen kann man eine $C^*$-Algebra assoziieren.
Wichtiges Merkmal ist, dass diese in allen interessanten Fällen
\emph{nichtkommutativ} sein wird. Nun gibt es aber ein alternatives Universum,
den sog. \emph{Bohr-Topos}, aus dessen Sicht diese Algebra kommutativ ist; auf
diese Weise vereinfacht sich manches. (Was genau, werden wir noch gemeinsam
herausfinden.)
\end{itemize}
Siehe~\cite{leinster:topos} für eine informale Einführung in Topostheorie
und~\cite{moerdijk:maclane:sheaves} für ein Lehrbuch, das auch die interne
Sprache ausführlich diskutiert. Als Referenzen sind~\cite{johnstone:elephant}
für Topostheorie und~\cite{lambek:scott:hocatlogic} speziell für kategorielle
Logik geeignet.


\section{Die Schlussregeln intuitionistischer Logik}

In den folgenden Abschnitten wollen wir \emph{Meta-Mathematik} betreiben: In
Abgrenzung von der sonst betriebenen Mathematik wollen wir nicht die üblichen
mathematischen Objekte wie Mengen, Vektorräume, Mannigfaltigkeiten untersuchen,
sondern \emph{Beweise}. Dazu müssen wir präzise festlegen, was unter einem
(intuitionistischen oder klassischen) Beweis zu verstehen ist.


\subsection{Formale logische Sprache}

\subsubsection*{Variablenkontexte}

\begin{defn}Ein \emph{Kontext} ist eine endliche Folge von
Variablendeklarationen der Form
\[ x_1 : A_1,\ \ldots,\ x_n : A_n. \]
Dabei sind die~$A_i$ \emph{Typen} der untersuchten formalen Systems.\end{defn}

Wir werden Kontexte oft kürzer als~$\vec x : \vec A$ notieren. Etwa ist die
Aussage
\[ n = m \]
eine Aussage im Kontext~$n : \NN, m : \NN$. Dagegen ist die Aussage
\[ \forall m\?\NN{:}\ n = m \]
lediglich eine Aussage im reduzierten Kontext~$n : \NN$: Die Variable~$m$ kommt
nicht mehr \emph{frei}, sondern nur noch \emph{gebunden} vor. Wir vereinbaren,
dass die kollisionsfreie Umbenennung gebundener Variablen die Aussage nicht
verändert. Die anders geschriebene Aussage
\[ \forall u\?\NN{:}\ n = u \]
sehen wir also als dieselbe Aussage an.
Wenn wir auch noch über die
Variable~$n$ quantifizieren, erhalten wir eine Aussage im \emph{leeren Kontext}:
\[ \forall n\?\NN{:}\ \forall u\?\NN{:}\ n = u. \]


\subsubsection*{Substitution von Variablen}

Ist~$\varphi$ eine Aussage im Kontext~$x_1,\ldots,x_n$. Sind dann
Terme~$s_1,\ldots,s_n$ (in einem neuen Kontext~$y_1,\ldots,y_m$) gegeben, so
kann man die~$x_i$ \emph{simultan durch die~$s_i$ ersetzen}. Als Ergebnis
erhält man eine Formel im Kontext~$y_1,\ldots,y_m$, die
man~"`$\varphi[s_1/x_1,\ldots,x_n/x_n]$"' oder kürzer~"`$\varphi[\vec s/\vec
x]$"' schreibt.

Bei der Substitution muss man Variablenkollisionen verhindern. Etwa gilt für
die Aussage
\[ \varphi :\equiv (\forall n\?\NN{:}\ n = m) \]
im Kontext~$m : \NN$, dass
\[ \varphi[n^2/m] \equiv (\forall \tilde n\?\NN{:}\ \tilde n = n^2). \]


\subsection{Sequenzen}

\begin{defn}Eine \emph{Sequenz} in einem Kontext~$\vec x : \vec A$ ist ein
Ausdruck der Form
\[ \varphi \seq{\vec x} \psi, \]
wobei~$\varphi$ und~$\psi$ Aussagen in diesem Kontext sind. Aussprache:
\emph{Aus der Voraussetzung~$\varphi$ ist die Aussage~$\psi$ ableitbar.}
\end{defn}

Welche Aussagen aus welchen Voraussetzungen ableitbar sind, entscheiden die
\emph{Ableitungsregeln} des untersuchten formalen Systems. Auf diese kommen wir
gleich, wollen aber vorher einen rein formalen Aspekt genauer beleuchten.


\subsubsection*{Sequenzen vs. Implikationen}

Wenn man das erste Mal mit der Definition einer Sequenz konfrontiert wird,
fragt man sich vielleicht, was der Unterschied zwischen
\[ \text{$\varphi \seq{\vec x} \psi$} \qquad\text{und}\qquad
  \text{$\top \seq{\vec x} (\varphi \Rightarrow \psi)$} \]
ist. Tatsächlich ist es bei Kenntnis der Ableitungsregeln für Implikation und
Konjunktion eine leichte Übungsaufgabe, die Äquivalenz der beiden Urteile zu
zeigen. Die Interpretation ist aber eine völlig andere:
\begin{itemize}
\item Die erste Sequenz besagt, dass unter der Globalvoraussetzung~$\varphi$
die Aussage~$\psi$ ableitbar ist. Eine typische Übungsaufgabe nach diesem
Muster sieht wie folgt aus:
\begin{quote}Sei~$n$ eine Primzahl~$\geq 3$. Zeige, dass~$n + 1$ keine Primzahl
ist.\end{quote}
\item Die zweite Sequenz besagt, dass unter keiner besonderen Voraussetzung
(zur Verfügung stehen also nur die gegebenen Ableitungsregeln) die
hypothetische Implikation~$\varphi \Rightarrow \psi$ folgt. Eine
Beispielformulierung für diese Art ist folgende:
\begin{quote}Zeige, dass wenn~$n$ eine Primzahl~$\geq 3$ ist, die Zahl~$n + 1$
keine Primzahl ist.\end{quote}
\end{itemize}
Der Unterschied ist subtil, aber sprachlich durchaus vorhanden.

\begin{bem}Logiker untersuchen auch formale Systeme, die deutlich weniger
sprachliche Mittel haben als klassische oder intuitionistische Logik -- etwa
solche, in denen Implikation als Junktor nicht vorkommt. Das antike System der
\emph{Syllogismen} (siehe Abbildung~\ref{penguin-logic})
ist ein Beispiel. Dann ist das Sequenzkonzept unverzichtbar.
\end{bem}
\begin{figure}
  \centering
  \includegraphics[scale=0.5]{penguin-logic}
  \caption{\label{penguin-logic}Ein Beispiel für einen (ungültigen)
  Syllogismus (Randy Glasbergen, verwendet ohne Erlaubnis).}
\end{figure}


\subsection{Ableitungen}

\begin{defn}Seien~$\varphi$ und~$\psi$ Aussagen in einem Kontext~$\vec x : \vec
A$. Genau dann ist~$\psi$ aus der Voraussetzung~$\varphi$ \emph{ableitbar}, in
Symbolen
$\varphi \seq{\vec x} \psi$,
wenn es eine entsprechende endliche \emph{Ableitung} gibt, welche nur die
in Tafel~\ref{ableitungsregeln:int} aufgeführten Ableitungsregeln verwendet.
\end{defn}

Aus dem Kontext muss hervorgehen, ob man eine Sequenz nur als solche
diskutieren möchte oder ob man ihre Ableitbarkeit unterstellt. Außerdem muss
man sich an die Notation der Ableitungsregeln gewöhnen, drei Beispiele seien im
Folgenden genauer erklärt.

\begin{table}
  \small
  \centering
  \textbf{Strukturelle Regeln} \\
  \vspace{-0.5em}
  \phantom{a}\hfill
  \AxiomC{$\phantom{\seq{\vec x}}$}\UnaryInfC{$\varphi \seq{\vec x} \varphi$}\DisplayProof\hfill
  \AxiomC{$\varphi \seq{\vec x} \psi$}\UnaryInfC{$\varphi[\vec s/\vec x]
  \seq{\vec y} \psi[\vec s/\vec x]$}\DisplayProof\hfill
  \AxiomC{$\varphi \seq{\vec x} \psi$}\AxiomC{$\psi \seq{\vec x}
  \chi$}\BinaryInfC{$\varphi \seq{\vec x} \chi$}\DisplayProof
  \phantom{a}\hfill
  \vspace{2.0em}

  \textbf{Regeln für Konjunktion} \\
  \vspace{-0.5em}
  \phantom{a}\hfill
  \AxiomC{$\phantom{\seq{\vec x}}$}\UnaryInfC{$\varphi \seq{\vec x} \top$}\DisplayProof\hfill
  \AxiomC{$\phantom{\seq{\vec x}}$}\UnaryInfC{$\varphi \wedge \psi \seq{\vec x} \varphi$}\DisplayProof\hfill
  \AxiomC{$\phantom{\seq{\vec x}}$}\UnaryInfC{$\varphi \wedge \psi \seq{\vec x} \psi$}\DisplayProof\hfill
  \AxiomC{$\varphi \seq{\vec x} \psi$}\AxiomC{$\varphi \seq{\vec x} \chi$}\BinaryInfC{$\varphi \seq{\vec x} \psi \wedge \chi$}\DisplayProof
  \phantom{a}\hfill
  \vspace{2em}

  \textbf{Regeln für Disjunktion} \\
  \vspace{-0.5em}
  \phantom{a}\hfill
  \AxiomC{$\phantom{\seq{\vec x}}$}\UnaryInfC{$\bot \seq{\vec x} \varphi$}\DisplayProof\hfill
  \AxiomC{$\phantom{\seq{\vec x}}$}\UnaryInfC{$\varphi \seq{\vec x} \varphi \vee \psi$}\DisplayProof\hfill
  \AxiomC{$\phantom{\seq{\vec x}}$}\UnaryInfC{$\psi \seq{\vec x} \varphi \vee \psi$}\DisplayProof\hfill
  \AxiomC{$\varphi \seq{\vec x} \chi$}\AxiomC{$\psi \seq{\vec x} \chi$}\BinaryInfC{$\varphi \vee \psi \seq{\vec x} \chi$}\DisplayProof
  \phantom{a}\hfill
  \vspace{2em}

  \textbf{Doppelregel für Implikation} \\
  \vspace{-0.5em}
  \phantom{a}\hfill
  \Axiom$\varphi \wedge \psi\ \fCenter\seq{\vec x} \chi$
  \doubleLine
  \UnaryInf$\varphi\ \fCenter\seq{\vec x} \psi \Rightarrow \chi$
  \DisplayProof
  \phantom{a}\hfill
  \vspace{2em}

  \textbf{Doppelregeln für Quantifikation} \\
  \vspace{-0.5em}
  \phantom{a}\hfill
  \Axiom$\varphi\ \fCenter\seq{\vec x, y} \psi$
  \doubleLine
  \UnaryInf$\exists y\?Y\_\! \varphi\ \fCenter\seq{\vec x} \psi$
  \DisplayProof
  {\tiny ($y$ keine Variable von~$\psi$)}
  \hfill
  \Axiom$\varphi\ \fCenter\seq{\vec x, y} \psi$
  \doubleLine
  \UnaryInf$\varphi\ \fCenter\seq{\vec x\phantom{, y}} \forall y\?Y\_\! \psi$
  \DisplayProof
  {\tiny ($y$ keine Variable von~$\varphi$)}
  \hfill\phantom{a}

  \caption{\label{ableitungsregeln:int}Die Schlussregeln intuitionistischer Logik.}
\end{table}

\begin{table}
  \small
  \centering
  \textbf{Regeln für Gleichheit} \\
  \vspace{-0.5em}
  \phantom{a}\hfill
  \AxiomC{$\phantom{\seq{\vec x}}$}
  \UnaryInfC{$\top \seq{x} x = x$}
  \DisplayProof
  \hfill
  \AxiomC{$\phantom{\seq{\vec x}}$}
  \UnaryInfC{$(\vec x = \vec y) \wedge \varphi \seq{\vec z} \varphi[\vec y/\vec x]$}
  \DisplayProof
  \hfill\phantom{a} \\[0.5em]
  (Dabei steht "`$\vec x = \vec y\,$"' für~$x_1 = y_1 \wedge \cdots \wedge x_n =
  y_n$.)
  \vspace{2em}

  \textbf{Prinzip vom ausgeschlossenen Dritten} \\
  \vspace{-0.5em}
  \phantom{a}\hfill
  \AxiomC{$\phantom{\seq{\vec x}}$}
  \UnaryInfC{$\top \seq{\vec x} \varphi \vee \neg\varphi$}
  \DisplayProof
  \hfill\phantom{a}

  \caption{\label{ableitungsregeln:weitere}Weitere Schlussregeln mancher
  formaler Systeme.}
\end{table}


\subsubsection*{Die Schnittregel}

Oberhalb des horizontalen Strichs in der sog. \emph{Schnittregel}
\begin{prooftree}
  \AxiomC{$\varphi \seq{\vec x} \psi$}
  \AxiomC{$\psi \seq{\vec x} \chi$}
  \BinaryInfC{$\varphi \seq{\vec x} \chi$}
\end{prooftree}
sind, nur durch horizontalen Freiraum getrennt, die Voraussetzungen der Regel
aufgeführt. Unterhalb des Strichs steht dann das Urteil, das man aus diesen
Voraussetzungen ziehen darf. Die Schnittregel besagt also: Ist in einem
Kontext~$\vec x$ aus~$\varphi$ die Aussage~$\psi$ ableitbar, und ist ferner
aus~$\psi$ die Aussage~$\chi$ ableitbar, so ist auch aus~$\varphi$
direkt~$\chi$ ableitbar. Die Schnittregel rechtfertigt also die Modularisierung
mathematischer Argumente in Lemmata.


\subsubsection*{Eine der Disjunktionsregeln}

Die Disjunktionsregel
\begin{prooftree}
  \AxiomC{$\varphi \seq{\vec x} \chi$}
  \AxiomC{$\psi \seq{\vec x} \chi$}
  \BinaryInfC{$\varphi \vee \psi \seq{\vec x} \chi$}
\end{prooftree}
besagt, dass, wenn aus~$\varphi$ die Aussage~$\chi$ ableitbar ist, und wenn
ferner auch aus~$\psi$ die Aussage~$\chi$ ableitbar ist, dass dann auch
aus~$\varphi \vee \psi$ die Aussage~$\chi$ ableitbar ist. Diese Regel
rechtfertigt also, bei einer Disjunktion als Voraussetzung einen Beweis durch
Unterscheidung der beiden möglichen Fälle zu führen.


\subsubsection*{Die Doppelregel für den Allquantor}

Der Doppelstrich in der Regel
\begin{prooftree}
  \Axiom$\varphi\ \fCenter\seq{\vec x, y} \psi$
  \doubleLine
  \UnaryInf$\varphi\ \fCenter\seq{\vec x\phantom{, y}} \forall y\?Y\_\! \psi$
\end{prooftree}
für den Allquantor, die nur angewendet werden darf, wenn~$y$ keine freie
Variable in~$\varphi$ ist, deutet an, dass die Regel sowohl wie üblich von oben
nach unten, als auch von unten nach oben gelesen werden kann. Sie besagt, dass
die beiden Urteile
\begin{itemize}
\item "`Im Kontext~$\vec x : \vec A, y : Y$ ist aus~$\varphi$ die
Aussage~$\psi$ ableitbar."'
\item "`Im Kontext~$\vec x : \vec A$ ist aus~$\varphi$ die Allaussage~$\forall
y\?Y{:}\, \psi$ ableitbar."'
\end{itemize}
äquivalent sind. Sie rechtfertigt daher das bekannte Standardvorgehen, um
Allaussagen nachzuweisen: Man nimmt sich ein "`beliebiges, aber festes"'~$y :
Y$, von dem man außer der Zugehörigkeit zu~$Y$ keine weiteren Eigenschaften
unterstellt, und weist die Behauptung dann für \emph{dieses}~$y$ nach.

\begin{aufg}Wieso sind die Variablenbeschränkungen in den Regeln für den
Existenz- und Allquantor nötig?\end{aufg}


\subsubsection*{Umfang der Ableitungsregeln}

\begin{motto}\label{allesformalisierbar}\emph{Alle} Beweise gewöhnlicher Mathematik, die man
gemeinhin als "`vollständig und präzise ausformuliert"', lassen sich als
Ableitungen im Sinne der Definition formalisieren (ggf. unter Hinzunahme
klassischer logischer Axiome, Mengentheorieaxiome oder Typtheorieaxiome).
\end{motto}

\begin{aufg}Überzeuge dich von dieser Behauptung. \emph{Tipp:} Formalisiere so
viele Beweise deiner Wahl, bis du keine Lust mehr hast.\end{aufg}

Wer nicht so viel Zeit hat, dem sei verraten, dass
Tafel~\ref{ableitungsregeln:int} kein Haufen ungeordneter Ableitungsregeln ist.
Stattdessen sind die Axiome nach den sie betreffenden Junktoren bzw. Quantoren
gruppiert: Sie legen für jedes sprachliche Konstrukt ist fest, wie man es
\emph{einführt} (etwa: "`aus~$\varphi \wedge \psi$ folgt schlicht~$\varphi$"')
und \emph{eliminiert} (etwa: "`kann man sowohl~$\varphi$ als auch~$\psi$
ableiten, so auch~$\varphi \wedge \psi$"').

\begin{bem}Neben den strukturellen Regeln sticht einzig das Prinzip vom
ausgeschlossenen Dritten aus diesem System von Einführungs- und
Eliminationsprinzipien heraus. Das ist ein rein formal-ästhetisches Argument
gegen klassische Logik.\end{bem}

\begin{bsp}Hier ein längeres Beispiel für eine Ableitung (ein Scan aus
dem Elephant-Buch, Seite~832):
\begin{center}\includegraphics[scale=0.3]{prooftree-elephant.png}\end{center}
Diese Ableitung beweist (eine Richtung des) \emph{Frobenius-Prinzips}.
\end{bsp}

Nicht verschwiegen werden sollte folgende Ergänzung des formalistischen Kredos:
\begin{motto}Das optimistische Motto~\ref{allesformalisierbar} stimmt nur in
erster Näherung. Es gibt mathematische Gedanken, die nicht formalisierbar
sind.\end{motto}

\XXX{Das ausführen!}


\subsection{Peano-Arithmetik und Heyting-Arithmetik}

\begin{defn}Das formale System \emph{Heyting-Arithmetik} ist gegeben durch
\begin{itemize}
\item intuitionistische Logik,
\item die Gleichheitsregeln (siehe Tafel~\ref{ableitungsregeln:weitere}),
\item einem einzigen Typ~$\NN$,
\item einer Termkonstante~$0 : \NN$,
\item einem 1-adischen Termkonstruktor~$S$ (für successor): Ist~$n : \NN$ ein
Term vom Typ~$\NN$, so ist~$S(n) : \NN$ ebenfalls ein Term vom Typ~$\NN$,
\item die Axiome \\
\vspace{-0.5em}
\phantom{a}\hfill
\AxiomC{$\phantom{\seq{\vec x}}$}
\UnaryInfC{$S(n) = 0 \seq{n} \bot$}
\DisplayProof
\hfill
\AxiomC{$\phantom{\seq{\vec x}}$}
\UnaryInfC{$S(n) = S(m) \seq{n,m} n = m$}
\DisplayProof
\hfill
\phantom{a} \\

und das Induktionsprinzip

\vspace{-1.0em}
\phantom{a}\hfill
\AxiomC{$\varphi \seq{\vec x} \psi[0/m]$}
\AxiomC{$\varphi \seq{\vec x, m} \psi \Rightarrow \psi[S(m)/m]$}
\BinaryInfC{$\varphi \seq{\vec x} \forall m\?\NN{:}\ \psi$}
\DisplayProof
\hfill\phantom{a}

\item sowie Regeln für alle primitiv-rekursiven Funktionen, insbesondere also
die erwarteten Regeln für Addition und Multiplikation.
\end{itemize}
\end{defn}

\begin{defn}Das formale System \emph{Peano-Arithmetik} ist genau wie
Heyting-Arithmetik gegeben, nur mit klassischer statt intuitionistischer
Logik.\end{defn}

\begin{defn}Ein formales System heißt genau dann \emph{inkonsistent}, wenn es in
ihm eine Ableitung der Sequenz~$\top \seq{} \bot$ (im leeren Kontext) gibt. Andernfalls heißt es
\emph{konsistent}.\end{defn}


\section[Beziehung zu klassischer Logik: die Doppelnegationsübersetzung]{%
  Beziehung zu klassischer Logik: \newline die Doppelnegationsübersetzung}

Aus den Augen einer konstruktiven Mathematikerin sind manche Aussagen ihrer
klassisch arbeitenden Kollegen falsch. Es gibt aber eine einfache Übersetzung,
die \emph{Doppel\-ne\-ga\-tions\-über\-set\-zung}, die Aussagen derart umformt, dass die
Übersetzung genau dann konstruktiv gilt, wenn die ursprüngliche Aussage
klassisch gilt. Mit dieser kann die konstruktive Mathematikern daher ihre
Kollegen verstehen, ohne ihre Logik verlassen zu müssen.

\begin{defn}Die \emph{Doppelnegationsübersetzung} (nach Kolmogorov, Gentzen,
Gödel und anderen) wird rekursiv wie folgt definiert:
\newcommand{\optnegneg}{\textcolor{grey}{\neg\neg}}
\begin{align*}
  \varphi^\circ &:\equiv \neg\neg \varphi \text{ für atomare Aussagen~$\varphi$} \\
  \top^\circ &:\equiv \top \\
  \bot^\circ &:\equiv \bot \\
  (\varphi \wedge \psi)^\circ &:\equiv \optnegneg(\varphi^\circ \wedge \psi^\circ) \\
  (\varphi \vee \psi)^\circ &:\equiv \neg\neg(\varphi^\circ \vee \psi^\circ) \\
  (\varphi \Rightarrow \psi)^\circ &:\equiv \optnegneg(\varphi^\circ \Rightarrow \psi^\circ) \\
  (\forall x\?X{:}\ \varphi)^\circ &:\equiv \optnegneg\forall x\?X{:}\ \varphi^\circ \\
  (\exists x\?X{:}\ \varphi)^\circ &:\equiv \neg\neg\exists x\?X{:}\ \varphi^\circ
\end{align*}
\end{defn}

\begin{bem}Da~$\neg\varphi :\equiv (\varphi \Rightarrow \bot)$,
gilt~$(\neg\varphi)^\circ \equiv \neg(\varphi^\circ)$.\end{bem}

\begin{aufg}Beweise durch Induktion über den Aussageaufbau, dass man auf die grau
gesetzten Doppelnegationen verzichten kann. Gewissermaßen besteht also der
einzige Unterschied zwischen klassischer und intuitionistischer Logik in der
Interpretation der Disjunktion und der Existenzquantifikation.\end{aufg}

\begin{satz}\label{dnt:proof}Seien~$\varphi$, $\psi$ beliebige Aussagen in einem Kontext~$\vec x$.
\begin{enumerate}
\item Klassisch gilt: $\varphi^\circ \Longleftrightarrow \varphi$.
\item Intuitionistisch gilt: $\neg\neg\varphi^\circ \Longrightarrow
\varphi^\circ$.
\item Wenn~$\varphi \seq{\vec x} \psi$ klassisch, dann~$\varphi^\circ \seq{\vec
x} \psi^\circ$ intuitionistisch. Wegen der Ab\-wärts\-kom\-pa\-ti\-bi\-li\-tät
intuitionistischer Logik und Teilaussage~a) gilt trivialerweise auch die
Umkehrung.
\end{enumerate}
\end{satz}
\begin{proof}
\begin{enumerate}
\item Klar, für jede Aussage~$\chi$ ist~$\neg\neg\chi \Leftrightarrow \chi$
eine klassische Tautologie.
\item Induktion über den Aussageaufbau, ausgelassen.
\item Wir müssen in einer Induktion über den Aufbau klassischer Ableitungen
nachweisen, dass wir jeden logischen Schluss klassischer Logik in der
Doppelnegationsübersetzung intuitionistisch nachvollziehen können. (Aus diesem
Grund mussten wir im vorherigen Abschnitt formal definieren, was wir unter
Ableitungen verstehen wollen.)

Etwa müssen wir zeigen, dass die übersetzte Schnittregel gültig ist:
\begin{prooftree}
  \AxiomC{$\varphi^\circ \seq{\vec x} \psi^\circ$}
  \AxiomC{$\psi^\circ \seq{\vec x} \chi^\circ$}
  \BinaryInfC{$\varphi^\circ \seq{\vec x} \chi^\circ$}
\end{prooftree}
Aber das ist klar, denn das ist wieder eine Instanz der intuitionistisch
zulässigen Schnittregel. Ein interessanteres Beispiel ist die übersetzte Form
von einer der Disjunktionsregeln:
\begin{prooftree}
  \AxiomC{}
  \UnaryInfC{$\varphi^\circ \seq{\vec x} \neg\neg(\varphi^\circ \vee \psi^\circ)$}
\end{prooftree}
Die Gültigkeit dieser Regel folgt aus der Disjunktionsregel und der
intuitionistischen Tautologie~$\chi \Rightarrow \neg\neg\chi$. Als letztes
und wichtigstes Beispiel wollen wir die Übersetzung des klassischen Axioms vom
ausgeschlossenen Dritten diskutieren:
\begin{prooftree}
  \AxiomC{}
  \UnaryInfC{$\top \seq{\vec x} \neg\neg(\varphi^\circ \vee \neg\varphi^\circ)$}
\end{prooftree}
Dass diese Regel intuitionistisch zulässig ist, haben wir in Übungsblatt~1
gesehen. Die Untersuchung aller weiteren Schlussregeln überlassen wir den Leser
(Übungsblatt~2).\qedhere
\end{enumerate}
\end{proof}

\begin{kor}Zeigt Peano-Arithmetik einen Widerspruch, so auch
Heyting-Arithmetik.\end{kor}
\begin{proof}Man kann leicht nachprüfen, dass die
Doppelnegationsübersetzungen der Peano-Axiome wiederum Instanzen den
Peano-Axiome sind und daher auch in Heyting-Arithmetik gelten. Daher kann man
eine Ableitung von~$\bot$ in Peano-Arithmetik in einer Ableitung
von~$\bot^\circ \equiv \bot$ in Heyting-Arithmetik überführen.\end{proof}

Folgendes Lemma werden wir erst später, im Abschnitt über Friedmans Trick, benötigen:
\begin{lemma}\label{dnt:geom}Sei~$\varphi$ eine Aussage, in der nur~$\top$, $\bot$,
$\wedge$, $\vee$ und $\exists$ (über
bewohnte Typen), aber nicht~$\Rightarrow$ oder~$\forall$ vorkommen. Dann gilt
intuitionistisch: $\varphi^\circ \Longleftrightarrow \neg\neg\varphi$.
\end{lemma}
\begin{proof}Induktion über den Aussageaufbau.\end{proof}


\subsection{Interpretation der übersetzten Aussagen}

Uns allen ist die Dialogmetapher zur Interpretation logischer Aussagen bekannt:
Wir stellen uns ein besonders kritisches Gegenüber vor, das unsere Behauptung
bezweifelt. In einem Dialog versuchen wir dann, das Gegenüber zu überzeugen.
Eine typische Stetigkeitsüberzeugung sieht etwa wie folgt aus:

\begin{dialogue}
\Eve Ich gebe dir~$x = \cdots$ und~$\varepsilon = \cdots$ vor.
\Alice Gut, dann setze ich~$\delta = \cdots$.
\Eve Dann ist hier ein~$\tilde x = \cdots$ zusammen mit einem Beleg von~$|x -
\tilde x| < \delta$.
\Alice Dann gilt tatsächlich~$|f(x) - f(\tilde x)| < \varepsilon$,
wie von mir behauptet, denn \ldots
\end{dialogue}

In Tafel~\ref{bhk} (Seite~\pageref{bhk}) ist festgelegt, nach welchen
Spielregeln Alice und Eve bei solchen Dialogen miteinander kommunizieren
müssen. Exemplarisch seien einige nochmal betont:
\begin{itemize}
\item Wenn Eve von Alice einen Beleg von~$\varphi \vee \psi$ fordert,
muss Alice einen Beleg von~$\varphi$ oder einen Beleg von~$\psi$ präsentieren.
Sie darf sich nicht mit einem "`angenommen, keines von beiden gälte"'
herausreden.
\item Wenn Eve von Alice einen Beleg von~$\varphi \Rightarrow \psi$
fordert, muss Alice ihr versprechen, Belege von~$\varphi$ in Belege von~$\psi$
überführen zu können. Dieses Versprechen kann Eve herausfordern, indem sie
einen Beleg von~$\varphi$ präsentiert; Alice muss dann in der Lage sein, mit
einem Beleg von~$\psi$ zu antworten.
\item Für die Negation als Spezialfall der Implikation gilt folgende Spielregel: Wenn Eve von
Alice einen Beleg von~$\neg\varphi \equiv (\varphi \Rightarrow \bot)$ verlangt,
muss Alice in der Lage sein, aus einem präsentierten Beleg von~$\varphi$ einen
Beleg von~$\bot$ zu produzieren. Wenn das betrachtete formale System konsistent
ist, gibt es keinen solchen Beleg; Alice kann unter der Konsistenzannahme also
nur dann~$\neg\varphi$ vertreten, wenn es keinen Beleg von~$\varphi$ gibt.
\end{itemize}

Als Motto können wir festhalten:

\begin{motto}
Eine Aussage~$\varphi$ intuitionistisch zu behaupten, bedeutet, in jedem
Dialog~$\varphi$ belegen zu können.
\end{motto}

Dank der Doppelnegationsübersetzung können wir damit auch eine
Dialoginterpretation klassischer Behauptungen angeben. Es stellt sich heraus,
dass die folgende Metapher sehr tragfähig ist. Diese wollen wir dann erst an
einem Beispiel veranschaulichen bevor wie sie begründen.

\begin{motto}
Eine Aussage~$\varphi$ klassisch zu behaupten (also~$\varphi^\circ$
intuitionistisch zu behaupten), bedeutet, in jedem Dialog~$\varphi$ belegen zu
können, wobei man aber beliebig oft Zeitsprünge in die Vergangenheit
durchführen kann.
\end{motto}


\subsubsection*{Beispiel: das Prinzip vom ausgeschlossenen Dritten}

Wir wollen sehen, wie man das klassische Prinzip~$\varphi \vee
\neg\varphi$ mit Hilfe von Zeitsprüngen vertreten kann.
\begin{dialogue}
\Eve Zeige mir~$\varphi \vee \neg\varphi$!
\Alice Gut! Es gilt~$\neg\varphi$.
\end{dialogue}
Wenn~$\varphi$ eine allgemeine Aussage ist, kann Alice nicht wissen,
ob~$\varphi$ oder~$\neg\varphi$ gilt. Sie muss daher an dieser Stelle bluffen.
Da sie die Implikation~$(\varphi \Rightarrow \bot)$ behauptet, ist nun Eve
wieder an der Reihe. Sie kann nur dann in ihrem Vorhaben, Alice zu widerlegen,
fortfahren, wenn sie einen Beleg von~$\varphi$ präsentiert und dann Alice
herausfordert, ihr Versprechen, daraufhin einen Beleg von~$\bot$ zu
präsentieren, einzulösen.

Wenn es keinen Beleg von~$\varphi$ gibt, ist das Streitgespräch daher an dieser
Stelle beendet, und Alice hat sogar die Wahrheit gesagt. Andernfalls geht es
weiter:
\begin{dialogue}
\Eve Aber hier ist ein Beleg von~$\varphi$: $x$. Belege mir nun~$\bot$!
\end{dialogue}
Wenn Alice nicht die Inkonsistenz des untersuchten formalen Systems nachweisen
kann, hat sie nun ein Problem: Ihre Lüge von Beginn straft sich, sie kann das
Gespräch nicht fortsetzen. Sie muss daher in einem Logikwölkchen verschwinden
und in der Zeit zurückspringen:
\begin{dialogue}
\Eve Zeige mir~$\varphi \vee \neg\varphi$!
\Alice Gut! Es gilt~$\varphi$, hier ist ein Beleg: $x$.
\end{dialogue}
Damit ist das Gespräch abgeschlossen.

Wer Zeitsprünge dieser Form betrügerisch findet, hat die
Grund\-über\-zeu\-gung konstruktiver Mathematik bereits verinnerlicht: In diesem (und
nur diesem) Sinn ist klassische Logik tatsächlich betrügerisch. Das macht
klassische Logik aber nicht trivial: Auch mit Zeitsprüngen kann man nicht jede
beliebige Aussage in einem Dialog vertreten. Wenn man etwa obiges Vorgehen
mit der im Allgemeinen ungerechtfertigten Aussage~$\varphi \vee \neg\psi$
versucht, wird man sehen, dass auch die Fähigkeit zu Zeitsprüngen
nicht hilft.


\subsubsection*{Dasselbe Beispiel, konservativer interpretiert}

Um zu sehen, dass die Zeitsprungmetapher berechtigt ist, wollen wir
exemplarisch dasselbe Beispiel erneut untersuchen. Genauer betrachten
wir einen Dialog zur
Dop\-pel\-ne\-ga\-tions\-über\-set\-zung des Prinzips vom ausgeschlossenen
Dritten, also zu $\neg\neg(\varphi^\circ \vee \neg\varphi^\circ)$. Wir können
sogar für beliebige Aussagen~$\varphi$ das Prinzip $\neg\neg(\varphi \vee
\neg\varphi)$ nachweisen, ausgeschrieben
\[ ((\varphi \vee \neg\varphi) \Rightarrow \bot) \Rightarrow \bot, \]
das ist geringfügig übersichtlicher.

\begin{dialogue}
\Eve Zeige mir~$\neg\neg(\varphi \vee \neg\varphi)$!
Präsentiere mir also einen Beleg von~$\bot$, wobei du auf mich zurückkommen
kannst, wenn du einen Beleg von~$\varphi \vee \neg\varphi$ hast; dann
würde ich Beleg von~$\bot$ produzieren.
\Alice Gut! Dann komme ich sofort auf dich zurück, denn ich habe einen Beleg
von~$\neg\varphi$. $(\star)$
\end{dialogue}

Wie oben ist das Gespräch an dieser Stelle beendet, wenn Eve nicht einen Beleg
von~$\varphi$ produzieren kann, mit dem sie Alice herausfordern könnte. Falls
sie das doch schafft, geht es wie folgt weiter:

\begin{dialogue}
\Eve Ach wirklich? Hier ist ein Beleg von~$\varphi$: $x$. Zeige mir nun einen
Beleg von~$\bot$!
\Alice Dann komme ich auf deine Verpflichtung mir gegenüber ein zweites Mal
zurück -- hier ist ein Beleg von~$\varphi \vee \neg\varphi$: $x$.
\Eve Stimmt. Dann ist hier Beleg von~$\bot$: $y$.
\Alice Danke. Dann ist hier ein Beleg von~$\bot$: $y$. Damit habe ich meine
Pflicht erfüllt.
\Eve Stimmt. Dann erfülle ich meinen Teil der Verpflichtung (Stelle~$(\star)$),
hier ist Beleg von~$\bot$: $z$.
\Alice Danke. Dann ist hier Beleg von~$\bot$, wie gefordert: $z$.
\end{dialogue}

% XXX Fazit

% XXX Bemerkung über Konsistenz von PA
% XXX Curry-Howard

Doppelnegationsübersetzung, Continuation-Passing-Style Transformation,
LCM, Stein der Weisen, \ldots

% XXX: Umgekehrte Übersetzung nicht möglich
% XXX: Feinere Unterschiede


\section[Beziehung zur theoretischen Informatik: die
Curry-Howard-Korrespondenz]{Beziehung zur theoretischen Informatik: \newline die
Curry-Howard-Korrespondenz}


\section{Hilberts Programm}

\subsection{Die mathematische Welt um 1900}

\XXX{Unvollständig und falsch.}

\emph{Hilberts Programm:} Zeige, dass man aus jedem Beweis einer konkreten
Aussage, welcher beliebige ideelle Prinzipien (Prinzip vom ausgeschlossenen
Dritten für beliebige Aussagen, Auswahlaxiom, maximale Ideale in der Algebra)
verwendet, einen finitistisch zulässigen Beweis erhalten kann.

In voller Allgemeinheit gilt Hilberts Programm als \emph{gescheitert}. Denn die
Aussage \emph{Peano-Arithmetik ist konsistent} lässt sich als "`konkrete
Aussage"' formulieren und leicht mit abstrakten Methoden beweisen (in üblicher
Mengenlehre liefert die unendliche Menge~$\NN$ ein Modell), kann aber keinen
finitistisch zulässigen Beweis besitzen, da es nach Gödels
Unvollständigkeitssatz nicht einmal einen Beweis in der stärkeren
Peano-Arithmetik geben kann.

Teilweise kann Hilberts Programm jedoch doch realisiert werden, unter anderem
in Analysis und Algebra: Mittels \emph{proof mining} kann aus klassischen
Beweisen mehr oder weniger systematisch noch konstruktiver Inhalt extrahiert
werden. Je nach Situation kann \emph{konstruktiver Inhalt} etwa
\begin{itemize}
\item explizite Schranken für Konstanten,
\item stetige (oder noch bessere) Abhängigkeit von Paramtern,
\item explizite Zeugen von Existenzbehauptungen oder
\item Algorithmen
\end{itemize}
umfassen.

% XXX: Eliminationsbegriff

\begin{motto}\label{motto:inhaltvonbeweisen}In einem \emph{Beweis} einer
Aussage steckt viel mehr Inhalt als die bloße Information, dass die Aussage
wahr ist.\end{motto}

Siehe~\cite{raatikainen:hilbert} für eine ausführlichere Darstellung von
Hilberts Programm und~\cite{kohlenbach:applprooftheory} für ein Lehrbuch zu
proof mining. Einen kurzen Überblick geben auch die
Vortragsfolien~\cite{avigad:proofmining}.


\subsection{Beispiel aus der Zahlentheorie: Friedmans Trick}

\begin{defn}Die \emph{Friedmanübersetzung}
wird für eine feste Aussage~$F$ wie folgt rekursiv definiert:
\begin{align*}
  \varphi^F &:\equiv \varphi \vee F \text{ für atomare Aussagen~$\varphi$} \\
  \top^F &:\equiv \top \\
  \bot^F &:\equiv F \\
  (\varphi \wedge \psi)^F &:\equiv (\varphi^F \wedge \psi^F) \\
  (\varphi \vee \psi)^F &:\equiv (\varphi^F \vee \psi^F) \\
  (\varphi \Rightarrow \psi)^F &:\equiv (\varphi^F \Rightarrow \psi^F) \\
  (\forall x\?X{:}\ \varphi)^F &:\equiv (\forall x\?X{:}\ \varphi^F) \\
  (\exists x\?X{:}\ \varphi)^F &:\equiv (\exists x\?X{:}\ \varphi^F)
\end{align*}
Wenn in~$F$ Variablen vorkommen, muss man ggf. manche Variablen umbenennen, um
Variablenkollisionen zu vermeiden.
\end{defn}

\begin{bem}Da~$\neg\varphi :\equiv (\varphi \Rightarrow \bot)$,
gilt~$(\neg\varphi)^F \equiv (\varphi^F \Rightarrow F)$.\end{bem}

\begin{satz}\label{friedman:proof}
\begin{enumerate}
\item Sei~$\varphi$ eine Aussage, in der Existenzquantoren nur über bewohnte
Typen gehen. Dann gilt intuitionistisch: $F \Longrightarrow \varphi^F$.

\item Sei~$\varphi$ eine Aussage, in der nur~$\top$, $\bot$,
$\wedge$, $\vee$ und $\exists$ (über
bewohnte Typen), aber nicht~$\Rightarrow$ oder~$\forall$ vorkommen.
Dann gilt intuitionistisch: $\varphi^F \Longleftrightarrow \varphi \vee F$.
% Wenn sich die Nummerierung ändert, auch unten anpassen!

\item Seien~$\varphi$ und~$\psi$ beliebige Aussagen in einem Kontext~$\vec x$,
in der Existenzquantoren nur über bewohnte Typen gehen.
Wenn~$\varphi \seq{\vec x} \psi$ intuitionistisch, dann gilt auch~$\varphi^F
\seq{\vec x} \psi^F$ intuitionistisch.
\end{enumerate}
\end{satz}
\begin{proof}
\begin{enumerate}
\item Induktion über den Aussageaufbau. Exemplarisch zeigen wir den Fall
\[ F \Longrightarrow (\exists x\?X{:}\ \varphi)^F. \]
Gelte also~$F$. Da~$X$ bewohnt ist, gibt es ein~$x : X$. Nach
Induktionsvoraussetzung können wir ~$F \Rightarrow \varphi^F$ voraussetzen.
Somit folgt~$\varphi^F$, das war zu zeigen.
\item Induktion über den Aussageaufbau.
\item Induktion über den Aufbau intuitionistischer Ableitungen. Wie beim
analogen Theorem über die Doppelnegationsübersetzung (Satz~\ref{dnt:proof})
muss man zeigen, dass die Friedmanübersetzungen der Schlussregeln gültig sind.
Das ist sogar einfacher als bei der Doppelnegationsübersetzung.\qedhere
\end{enumerate}
\end{proof}

\begin{kor}Peano-Arithmetik ist für Aussagen der Form~$\forall (\cdots
\Rightarrow \cdots)$, wobei die Teilaussagen den Beschränkungen
aus~\ref{friedman:proof}b) unterliegen müssen, konservativ über
Heyting-Arithmetik: Aus jedem Beweis in Peano-Arithmetik lässt sich ein Beweis
in Heyting-Arithmetik gewinnen.\end{kor}
\begin{proof}Gelte~$\top \seq{} (\forall x\?X{:}\ \varphi \Rightarrow \psi)$ in
Peano-Arithmetik. Dann gilt auch
\[ \varphi \seq{x} \psi \]
in Peano-Arithmetik; so schaffen wir den Allquantor und die Implikation weg.
Nach dem Satz über die Doppelnegationsübersetzung (Satz~\ref{dnt:proof}) folgt
die Ableitbarkeit der übersetzten Sequenz in Heyting-Arithmetik. Da~$\varphi$
und~$\psi$ den genannten Einschränkungen unterliegen, sind~$\varphi^\circ$
und~$\psi^\circ$ intuitionistisch äquivalent zu ihren Doppelnegationen
(Lemma~\ref{dnt:geom}); also ist die Sequenz
\[ \neg\neg\varphi \seq{x} \neg\neg\psi \]
in Heyting-Arithmetik ableitbar. Nun können wir die Friedmanübersetzung
bezüglich einer noch unspezifizierten Aussage~$F$ anwenden. Da sich leicht die
Friedmanübersetzungen der Peano-Axiome in Heyting-Arithmetik zeigen lassen,
folgt die Ableitbarkeit von
\[ ((\varphi^F \Rightarrow F) \Rightarrow F) \seq{x}
  ((\psi^F \Rightarrow F) \Rightarrow F) \]
in Heyting-Arithmetik. Dass~$\varphi$ und~$\psi$ den genannten Einschränkungen
unterliegen, können wir ein weiteres Mal ausnutzen: Heyting-Arithmetik kann die
Sequenz
\[ ((\varphi \vee F \Rightarrow F) \Rightarrow F) \seq{x}
  ((\psi \vee F \Rightarrow F) \Rightarrow F) \]
zeigen. \emph{Friedmans Trick} besteht nun darin, für~$F$ speziell~$\varphi$ zu
wählen. Dann vereinfachen sich die Ausdrücke und wir erhalten die Ableitbarkeit
von $\varphi \seq{x} \psi$, also von
\[ \top \seq{} (\forall x\?X{:}\ \varphi \Rightarrow \psi) \]
in Heyting-Arithmetik.
\end{proof}

\begin{bsp}Die Aussage der Zahlentheorie, dass es unendlich viele Primzahlen
gibt, lässt sich in der Form
\[ \forall n\?\NN{:}\ \exists p\?\NN{:}\ p \geq n \wedge \text{$p$ ist prim} \]
schreiben. Zur Formalisierung der Primalitätsaussage benötigt man nur
\emph{beschränkte Allquantifikation}, für welche die Konservativitätsaussage
ebenfalls gilt. Also kann man aus jedem klassischen Beweis der Unendlichkeit
der Primzahlen einen konstruktiven extrahieren.
\end{bsp}
% XXX: Wie muss /p prim/ formuliert werden?

\begin{bsp}Das Konservativitätsresultat trifft insbesondere
auf~$\Pi^0_2$-Aussagen zu -- das sind solche der Form
\[ \forall \cdots \forall\ \exists \cdots \exists\ (\cdots), \]
wobei die abschließende Teilaussage keine Quantoren mehr enthält. Zu diesen
gehört die Aussage, dass eine gegebene Turingmaschine bei jeder
beliebigen Eingabe schlussendlich terminiert ("`$\forall\,\text{Eingaben}\
\exists\,\text{Stoppzeitpunkt}$"').
\end{bsp}

% XXX: Markovs Regel beim Namen erwähnen.

% XXX: Nochmal betonen, dass dadurch eine partielle Realisierung von Hilberts
% Programm erfolgt ist.


\subsection{Beispiel aus der Algebra: dynamische Methoden}

In der kommutativen Algebra sind einige Techniken gebräuchlich, mit deren Hilfe
man viele konkrete Aussagen beweisen kann, deren Gültigkeit man aber nur
in klassischer Logik und unter Verwendung starker Auswahlprinzipien beweisen
kann. Drei Beispiele sind folgende:

\begin{itemize}
\item Um zu zeigen, dass ein Element~$x$ eines Rings~$R$ nilpotent ist (dass
also eine gewisse Potenz~$x^n$ Null ist), genügt es zu zeigen, dass~$x$ in
allen Primidealen von~$R$ liegt (siehe Proposition~\ref{intersectprim}).
\item Um zu zeigen, dass ein Element~$x$ im Jacobson-Radikal liegt (dass
also~$1-rx$ für alle~$r \in R$ invertierbar ist), genügt es zu zeigen, dass~$x$
in allen maximalen Idealen von~$R$ liegt.
\item Um zu zeigen, dass ein Element~$x$ eines Körpers~$K$ ganz über einem
Unterring~$R$ ist, genügt es zu zeigen, dass~$x$ in allen Bewertungsringen
liegt.
\item Um zu zeigen, dass zwischen Polynomen~$f_1,\ldots,f_m \in
K[X_1,\ldots,X_n]$, wobei~$K$ ein algebraisch abgeschlossener Körper ist, eine
Relation der Form~$1 = p_1 f_1 + \cdots + p_m f_m$ besteht, genügt es zu
zeigen, dass die~$f_i$ keine gemeinsame Nullstelle besitzen.
\end{itemize}

Mit sog. \emph{dynamischen Methoden} kann man aus Beweisen, die diese
Prinzipien verwenden, noch konstruktiven Inhalt retten.
Siehe~\cite{clr:dynamicalmethod,cl:logical} für relevante Originalartikel
und~\cite{coquand:sitesur,lombardi:hilbertworks} für Vortragsfolien zum Thema.


\subsubsection*{Standardbeispiel: Nilpotente Polynome}
\label{bsp:nilpotentepolynome}

Die Nützlichkeit des Nilpotenzkriteriums wird oft an folgendem Standardbeispiel
demonstriert. Alle benötigten Vorkenntnisse aus der Idealtheorie sind in
Anhang~\ref{appendix:ideale} zusammengefasst.

\begin{prop}[auch konstruktiv]Sei~$f \in R[X]$ ein Polynom über einem Ring~$R$. Dann gilt:
\[ \text{$f$ ist nilpotent} \quad\Longleftrightarrow\quad
  \text{alle Koeffizienten von~$f$ sind nilpotent}. \]
\end{prop}
\begin{proof}[Beweis (nur klassisch)]Die Rückrichtung ist einfach: Sei~$f = \sum_{i=0}^n a_i X^i$
mit~$a_i^m = 0$ für alle~$i = 0,\ldots,n$. Dann überzeugt man sich durch
Ausmultiplizieren und dem Schubfachprinzip, dass die Potenz~$f^{(m-1)(n+1) +
1}$ Null ist.

Interessant ist die Hinrichtung. Gelte~$f^m = 0$. Sei ein beliebiges
Primideal~$\pp \subseteq R$ gegeben. Dann liegen alle Koeffizienten von~$f^m$
in~$\pp$. Nach einem allgemeinem Lemma (Lemma~\ref{produktprim}) liegen dann
schon alle Koeffizienten von einem der Faktoren, also von~$f$, in~$\pp$. Das
zeigt schon die Behauptung.
\end{proof}

Der Beweis gelingt also völlig mühelos: Man muss nur
nur das Nilpotenzkriterium und das auch anderweitig nützliche
Lemma~\ref{produktprim} verwenden. Allerdings ist der Beweis in dieser Form
\emph{ineffektiv}: Man erhält keine Abschätzung der Nilpotenzindizes der
Koeffizienten, also der minimal möglichen Exponenten~$m_i$ mit~$a_i^{m_i} = 0$.
Auch ist die Abhängigkeit der~$m_i$ von den Daten nicht klar: Gibt es eine
universelle Schranke, die für jeden Ring und jedes Polynom gültig wäre? Oder
kann der Nilpotenzindex bei schlimmen Ringen oder Polynomen beliebig hoch
werden?

Diese Fragen könnte man durch eine manuelle Untersuchung, etwa mit
verschachtelten Induktionsbeweisen, klären. Es gibt aber auch ein
systematisches Verfahren, das ganz ohne weitere Arbeit direkt aus obigem Beweis die
gesuchten Schranken extrahiert. Der Schlüssel zu diesem Verfahren liegt in
folgender Erkenntnis: Der Beweis verwendet gar nicht die speziellen
Eigenschaften der Primideale des Rings~$R$ (welche das auch immer sein mögen).
Stattdessen verwendet er nur die \emph{allgemeinen Primidealaxiome}. Gewissermaßen
zeigt er also nicht nur, dass die Koeffizienten in allen Primidealen enthalten
sind, sondern dass sie in \emph{dem generischen Primideal} enthalten sind.

\begin{motto}Die \emph{generische} Verwendung ideeller Konzepte (Primideale,
maximale Ideale, Bewertungen, \ldots) lässt sich eliminieren.\end{motto}
% XXX: Trivialität des Mottos anmerken und erklären, wie es gemeint ist.


\subsubsection*{Axiomatisierung des generischen Primideals}

Sei~$R$ ein Ring.

\begin{defn}\label{defn:genprime}Die Axiome für das \emph{generische Primideal}
sind folgende.
\begin{enumerate}
\item[1.] $\top \seq{} Z(0).$
\item[2.] $Z(x) \wedge Z(y) \seq{} Z(x+y)$ für alle~$x,y \in R$.
\item[3.] $Z(x) \seq{} Z(rx)$ für alle~$r,x \in R$.
\item[4.] $Z(1) \seq{} \bot.$
\item[5.] $Z(xy) \seq{} Z(x) \vee Z(y)$ für alle~$x,y \in R$.
\end{enumerate}
\end{defn}

% XXX: Zitieren!
\begin{satz}\label{satz:genprime}Aus einem Beweis der Sequenz
\[ Z(a_1) \wedge \cdots \wedge Z(a_n) \seq{} Z(b) \]
welcher als sprachliche Mittel nur~$\top$, $\bot$, $\wedge$ und~$\vee$, nicht
aber~$\Rightarrow$ oder die Quantoren, und als Schlussregeln neben den Axiomen
aus Definition~\ref{defn:genprime} nur die
strukturellen Regeln und die Regeln für Konjunktion und Disjunktion verwendet (siehe
Tafel~\ref{ableitungsregeln:int}), kann man einen expliziten Zeugen der Aussage
\[ b \in \sqrt{(a_1,\ldots,a_n)} \]
extrahieren (siehe Definitionen~\ref{def:idealerz} und~\ref{def:radikal} für
die Notation), also eine natürliche Zahl~$m \geq 0$ und
Ringelemente~$u_1,\ldots,u_n \in R$ mit
\[ b^m = u_1 a_1 + \cdots + u_n a_n. \]
\end{satz}

Der Satz ist eine beeindruckende
Demonstration von Motto~\ref{motto:inhaltvonbeweisen}, demnach in
Beweisen von Aussagen viel mehr Inhalt steckt als die bloße Information
über die Wahrheit der Behauptung. Bevor wir den Beweis führen (welcher
erstaunlich einfach ist), wollen wir das Resultat aber noch genauer
diskutieren.

\begin{kor}Aus einem Beweis der Sequenz
\[ \top \seq{} Z(x) \]
folgt die Nilpotenz von~$x$, und man kann sogar eine explizite Schranke
für den Nilpotenzindex von~$x$, d.\,h. eine Zahl~$m \geq 0$ mit $x^m = 0$,
extrahieren.
\end{kor}
\begin{proof}[Beweis des Korollars] Mit den Axiomen kann man die Äquivalenz
von~$\top$ mit~$Z(0)$ zeigen. Nach dem Satz folgt daher, dass man einen
expliziten Zeugen der Zugehörigkeit~$x \in \sqrt{(0)}$ extrahieren
kann.\end{proof}

Mit der Interpretation des Korollars und des Satzes muss man ein wenig
vorsichtig sein. Die Aussage ist \emph{nicht}, dass aus der Zugehörigkeit
von~$x$ zu allen Primidealen die Nilpotenz von~$x$ folgt. Diese stärkere
Aussage kann man (bewiesenermaßen) nur in einem klassischen Rahmen zeigen.
Stattdessen kann man lediglich aus einem entsprechend formalisierten
\emph{Beweis}, dass~$x$ in allen Primidealen enthalten ist, die Nilpotenz
von~$x$ folgern.

\begin{bem}
Man kann sich die Frage stellen, ob das generische Primideal durch ein
gewöhnliches Primideal realisiert werden kann, ob es also ein Primideal~$\pp
\subseteq R$ gibt, das genau die Eigenschaften hat, die auch das generische
Primideal hat. Das ist nicht zu erwarten -- jedes konkrete Primideal kann nicht
die Vorstellung des generischen Primideals fassen -- und in der Tat im
Allgemeinen auch nicht der Fall. Denn wenn ein Primideal~$\pp$ für alle~$x \in
R$ die Äquivalenz
\[ x \in \pp \quad\Longleftrightarrow\quad \top \seq{} Z(x) \]
erfüllt, gilt schon~$\pp = \sqrt{(0)} = (\text{Ideal aller nilpotenten
Elemente})$. Also ist jeder Nullteiler in~$R$ nilpotent. Das ist aber eine
besondere Eigenschaft, die nur wenige Ringe haben. (Etwa ist in~$\ZZ \times
\ZZ$ das Element~$(1,0)$ ein Nullteiler, aber nicht nilpotent.)
\end{bem}


\subsubsection*{Beweis des Satzes}

\begin{proof}[Beweis von Satz~\ref{satz:genprime}]
Wir geben ein explizites \emph{Modell} der in der Formulierung des Satzes
beschriebenen Axiomensystems an. Die Aussagen~$\varphi$ der Sprache wollen wir als
gewisse Radikalideale~$\brak{\varphi} \subseteq R$ interpretieren, die
Ableitungsrelation~$\seq{}$ als umgekehrte Idealinklusion.
Lemma~\ref{lemma:rad} ist für das Verständnis der folgenden Übersetzungstabelle
hilfreich. Konkret definieren wir
\begin{align*}
  \brak{Z(x)} &:= \sqrt{(x)} \\
  \brak{\top} &:= \sqrt{(0)} \\
  \brak{\bot} &:= (1) \\
  \brak{\varphi \wedge \psi} &:= \sup\{\brak{\varphi},\brak{\psi}\} = \sqrt{\brak{\varphi} + \brak{\psi}} \\
  \brak{\varphi \cap \psi} &:= \inf\{\brak{\varphi},\brak{\psi}\} = \brak{\varphi} \cap \brak{\psi}
\end{align*}
und
\[ \varphi \models \psi \quad:\Longleftrightarrow\quad
  \brak{\varphi} \supseteq \brak{\psi}. \]
Dann kann man nachrechnen, dass diese semantisch definierte Relation~$\models$
die geforderten Axiome erfüllt. Etwa gilt
\begin{align*}
  Z(x) \wedge Z(y) \models Z(x+y), &
    \quad\text{denn } \sqrt{\sqrt{(x)} + \sqrt{(y)}} \supseteq \sqrt{(x+y)},\ \text{und} \\
  Z(xy) \models Z(x) \vee Z(y), &
    \quad\text{denn } \sqrt{(xy)} \supseteq \sqrt{(x)} \cap \sqrt{(y)},
\end{align*}
die restlichen Nachweise überlassen wir dem Leser. Jeden Beweis, der nur die
angegebenen Schlussregeln verwendet, kann man also in der Menge der
Radikalideale nachbauen. Nun ist es leicht, die Behauptung zu zeigen:
\begin{align*}
  Z(a_1) \wedge \cdots \wedge Z(a_n) \seq{} Z(b)
  &\Longleftrightarrow
  \sqrt{(b)} \subseteq \sqrt{(a_1,\ldots,a_n)} \\
  &\Longleftrightarrow
  b^m = u_1 a_1 + \cdots + u_n a_n \\
  &\qquad\qquad
    \text{für gewisse~$m \geq 0$, $u_1,\ldots,u_n \in R$.} \qedhere
\end{align*}
\end{proof}

% XXX: Wieso heißt es "dynamische Methoden"?
% XXX: Fehlt: Beispiel.
% XXX: Erweiterung mit Quantoren und V(...)
% XXX: Beweis in klassischer Logik, der natürlich witzlos ist.


\appendix
\section{Ideale in Ringen}
\label{appendix:ideale}

\subsection{Grundlegende Konzepte}

\begin{defn}Ein \emph{kommutativer Ring mit Eins} (kurz \emph{Ring}) besteht aus
\begin{itemize}
\item einer Menge~$R$,
\item einer additiv geschriebenen Verknüpfung~${+} : R \times R \to R$,
\item einer multiplikativ geschriebenen Verknüpfung~${\cdot} : R \times R \to R$,
\item einem ausgezeichneten Element~$0 \in R$ und
\item einem ausgezeichneten Element~$1 \in R$,
\end{itemize}
sodass
\begin{itemize}
\item Addition und Multiplikation assoziativ sind,
\item Addition und Multiplikation multiplikativ sind,
\item Addition über Multiplikation distribuiert,
\item das Element~$0$ neutral bezüglich der Addition ist,
\item das Element~$1$ neutral bezüglich der Multiplikation ist und
\item jedes Element ein bezüglich der Addition inverses Element besitzt.
\end{itemize}
\end{defn}

\begin{defn}Ein \emph{Körper} ist ein Ring, in dem jedes Element
\emph{entweder} Null \emph{oder} (bezüglich der Multiplikation) invertierbar
ist.\end{defn}

\begin{bsp}Die Mengen~$\ZZ$, $\QQ$, $\RR$, $\ZZ/(n)$, $\ZZ[X]$, $\QQ[X]$ bilden
bezüglich ihrer üblichen Additionen und Multiplikationen Ringe. Für~$n$ prim
ist~$\ZZ/(n)$ sogar ein Körper. Die Menge~$\NN$ bildet bezüglich der üblichen
Addition und Multiplikation noch keinen Ring, da bis auf die Null kein Element
ein Negatives besitzt.\end{bsp}

\begin{defn}Ein \emph{Ideal} eines Rings~$R$ ist eine Teilmenge~$\aa \subseteq R$, die
\begin{itemize}
\item die Null enthält: $0 \in \aa$,
\item abgeschlossen unter Addition ist: $x + y \in \aa$ für alle~$x,y \in \aa$, und
\item die \emph{Magneteigenschaft} erfüllt: $r x \in \aa$ für alle~$r \in R, x
\in \aa$.
\end{itemize}
\end{defn}

Die Axiome werden durch folgendes Beispiel motiviert:

\begin{bsp}Sei~$R$ ein Ring (zum Beispiel~$R = \ZZ$) und~$u \in R$ ein Element
(zum Beispiel deine Lieblingszahl). Dann ist die Menge
\[ (u) := \{ r u \,|\, r \in R \} \subseteq R \]
aller Vielfachen von~$u$ ein Ideal, das sog. \emph{von~$u$ erzeugte Ideal}.
Denn die Null ist ein Vielfaches von~$u$ (das Null-fache), die Summe zweier
Vielfachen von~$u$ ist ein Vielfaches von~$u$, und ist~$x$ ein Vielfaches
von~$u$, so ist~$r x$ für ein beliebiges Element~$r \in R$ "`umso mehr"' ein
Vielfaches von~$u$.
\end{bsp}

In Körpern~$K$ ist der Idealbegriff dagegen langweilig: Körper besitzen stets nur
genau zwei Ideale, nämlich das sog. Nullideal~$(0) = \{ 0 \}$ und das sog.
Einsideal~$(1) = K$.

\begin{defn}\label{def:idealerz}Seien~$x_1,\ldots,x_n$ Elemente eines
Rings~$R$. Dann heißt das Ideal
\[ (x_1,\ldots,x_n) := \{ r_1 x_1 + \cdots + r_n x_n \,|\, r_1,\ldots,r_n \in R
\} \subseteq R \]
das \emph{von~$x_1,\ldots,x_n$ erzeugte Ideal}.\end{defn}

\begin{bsp}Für den Ring der ganzen Zahlen gilt~$(2,3) = (1) = \ZZ$.\end{bsp}

\begin{defn}Ein Ideal~$\pp \subseteq R$ heißt genau dann \emph{Primideal}, wenn
\begin{itemize}
\item die Eins nicht enthalten ist: $1 \not\in \pp$, und
\item wann immer ein Produkt in~$\pp$ enthalten ist, schon ein Faktor in~$\pp$
liegt:
\[ xy \in \pp \quad\Longrightarrow\quad x \in \pp \,\vee\, y \in \pp
  \qquad\text{für alle~$x,y \in R$.} \]
\end{itemize}
\end{defn}

\begin{bsp}Sei~$u \in \ZZ$. Dann ist das von~$u$ erzeugte Ideal~$(u) \subseteq
\ZZ$ genau dann ein Primideal, wenn~$u$ Null ist oder wenn~$u$ oder~$-u$ eine
Primzahl ist.\end{bsp}


\subsection{Historische Motivation für Idealtheorie}

Historisch gab es eine große Motivation, dieses Konzept einzuführen. Vom Ring
der ganzen Zahlen war natürlich bekannt, dass sich (bis auf die Null) jedes
Element auf eindeutige Weise als Produkt von Primfaktoren schreiben lässt. Man
fragte sich nun, ob gewisse für die Zahlentheorie relevante Ringe dieselbe
Eigenschaft hatten: Das wäre zum einen recht "`nett"', zum anderen aber auch
enorm nützlich: Denn man kannte schon einfache Beweise von Fermats letztem
Satz, welche als einzige ungesicherte Voraussetzung diese Eigenschaft hatten.

Leider stellte es sich heraus, dass diese Eigenschaft vielen der interessanten
Ringe \emph{nicht} zu kam. Kronecker hatte nun die geniale Einsicht, von
Zahlen zu Idealen und von Primzahlen zu Primidealen zu verallgemeinern. Denn in
diesen Ringen gilt zumindest noch, dass sich jedes \emph{Ideal} eindeutig als
Produkt von Prim\emph{idealen} schreiben lässt. Mit dieser schwächeren Eigenschaft
lässt sich zwar kein allgemeiner Beweis von Fermats letztem Satz führen,
zumindest lässt sich jedoch eine große Klasse von Spezialfällen damit
behandeln.

Wer sich für dieses Thema interessiert, dem sei das deutschsprachige
Buch~\cite{schmidt:zahlentheorie} von Alexander Schmidt empfohlen. Als Vorwissen
setzt es nur Schulkenntnisse voraus.


\subsection{Die Ideale im Ring der ganzen Zahlen}

Tafel~\ref{ideale:z} zeigt die Ideale des Rings~$\ZZ$.
Ergänzt man die aus Platzgründen ausgelassenen Ideale, ist das
sogar eine vollständig Übersicht über \emph{alle} Ideale von~$\ZZ$ --
wenn man klassische Logik voraussetzt.

\begin{table}
  \centering
  \includegraphics[scale=0.4]{ideale-z}
  \caption{\label{ideale:z}Modulo Platz und klassische Logik eine
  vollständige Übersicht über alle Ideale von~$\ZZ$.}
\end{table}

\begin{aufg}Zeige, dass wenn alle Ideale von~$\ZZ$ von der Form~$(a)$ für
eine ganze Zahl~$x$ sind, das Prinzip vom ausgeschlossenen Dritten gilt.
\emph{Tipp:} Betrachte für eine beliebige Aussage~$\varphi$ das Ideal (wieso
sind die Idealaixome erfüllt?) $\{ x \in \ZZ
\,|\, x = 0 \vee \varphi \} \subseteq \ZZ$.\end{aufg}


\subsection{Primideale und Nilpotenz}

\begin{defn}Ein Element~$x \in R$ eines Rings~$R$ heißt genau dann
\emph{nilpotent}, wenn eine gewisse Potenz Null ist:
\[ x^n = 0 \qquad\text{für ein~$n \geq 0$}. \]
\end{defn}

\begin{bsp}Im Ring~$\ZZ/(4)$ ist das Element~$[2]$ nilpotent.\end{bsp}

\begin{prop}Die nilpotenten Elemente eines Rings liegen in allen Primidealen
des Rings.\end{prop}
\begin{proof}Sei~$x$ mit~$x^n = 0$ ein nilpotentens Element. Sei~$\pp$ ein
beliebiges Primideal. Dann ist also~$x^n$ in~$\pp$ enthalten. Wegen der
Primalitätsbedingung ist daher auch~$x$ in~$\pp$ enthalten. Das war zu
zeigen.\end{proof}

Interessant ist nun, dass -- in einem klassischen Kontext -- auch die Umkehrung
dieser Proposition gilt. Somit hat man ein einfaches Kriterium an der Hand, um
die Nilpotenz eines Ringelements nachzuweisen.

\begin{prop}[nur klassisch]\label{intersectprim}%
Im Schnitt aller Primideale eines Rings liegen nur
die nilpotenten Elemente.\end{prop}
\begin{proof}Sei~$x$ ein Element von~$R$, welches in allen Primidealen liegt.
Wir wollen zeigen, dass~$x$ nilpotent ist; dazu führen wir einen
Widerspruchsbeweis, nehmen also an, dass~$x$ nicht nilpotent ist. Dann enthält
die Menge
\[ S := \{ x^n \,|\, n \geq 0 \} \subseteq R \]
also nicht die Null. Wir betrachten nun das bezüglich der
Teilmengeninklusionsrelation partiell geordnete Mengensystem
\[ \U := \{ \aa \subseteq R \,|\, \text{$\aa$ ist ein Ideal mit~$\aa \cap S =
\emptyset$} \}. \]
Dieses ist bewohnt: Das Nullideal liegt wegen~$0 \not\in S$ in~$\U$. Außerdem
liegt die Vereinigung~$\bigcup_i \aa_i$ einer in~$\U$ liegenden Kette von
Elementen aus~$\U$ wieder in~$\U$. Damit sind alle Voraussetzung des Lemmas von
Zorn erfüllt, womit~$\U$ also ein maximales Element~$\mm$ enthält.

Man kann nun nachrechnen, dass~$\mm$ ein Primideal ist. Da~$x \not\in \mm$
(wegen~$x \in S$), ist das ein Widerspruch zur Voraussetzung.
\end{proof}

Dieser Beweis ist aus zwei Gründen inhärent klassisch: Zum einen, weil
er ein echter Widerspruchsbeweis ist; zum anderen, weil das Lemma von Zorn
verwendet wird (dieses ist zum Auswahlaxiom äquivalent). Man kann sogar zeigen,
dass ein konstruktiver Beweis dieser Proposition nicht möglich ist. Daher
ist folgendes Metatheorem absolut erstaunlich:

\begin{wunder}Sei~$x \in R$ ein Element eines Rings. Sei ein \emph{klassischer
Beweis} (einer gewissen Form) der Aussage~$x \in \pp$, wobei man von~$\pp$ nur
die Axiome eines Primideals voraussetzen darf, gegeben. Dann ist~$x$ nilpotent
(konstruktiv!). Aus dem klassischen Beweis kann man also auf konstruktive Art
und Weise einen konstruktiven Beweis der Nilpotenzbehauptung extrahieren.
\end{wunder}


\subsubsection*{Polynome mit Koeffizienten in Primidealen}

Für das Beispiel in Abschnitt~\ref{bsp:nilpotentepolynome} benötigen wir
folgendes Lemma.
\begin{lemma}\label{produktprim}Seien~$f,g \in R[X]$ Polynome über einem Ring~$R$. Sei~$\pp
\subseteq R$ ein Primideal. Wenn alle Koeffizienten von~$fg$ in~$\pp$ liegen,
so liegen schon alle Koeffizienten von~$f$ oder alle Koeffizienten von~$g$
in~$\pp$.\end{lemma}
Wenn man mit der Faktorringkonstruktion vertraut ist, lässt sich das Lemma
einfacher formulieren: Ist ein Produkt in~$(R/\pp)[X]$ Null, so ist schon einer
der Faktoren in~$(R/\pp)[X]$ Null. Diese Aussage ist Instanz eines noch
allgemeineren Lemmas: Ist ein Ring~$S$ ein Integritätsbereich, so auch~$S[X]$.


\subsection{Radikalideale}

\begin{defn}\label{def:radikal}\begin{enumerate}
\item Ein Ideal~$\aa \subseteq R$ eines Rings~$R$ heißt genau dann
\emph{Radikalideal}, wenn für alle~$x
\in R$ und~$n \geq 0$ aus~$x^n \in \aa$ schon~$x \in \aa$ folgt.
\item Sei~$\aa \subseteq R$ ein Ring. Dann heißt das Ideal
\[ \sqrt{\aa} := \{ x \in R \,|\, \exists n \geq 0{:}\ x^n \in \aa \} \]
das \emph{Radikal von~$\aa$}. Es ist stets ein Radikalideal, und zwar das
kleinste, das~$\aa$ umfasst.
\end{enumerate}\end{defn}

\begin{bsp}Das Ideal~$(12) \subseteq \ZZ$ ist kein Radikalideal, $\sqrt{(12)} =
(6)$ dagegen schon.\end{bsp}

\begin{bem}Die Zuordnung von Radikalen zu Idealen bildet einen
Linksadjungierten zum Vergissfunktor der Kategorie der Radikalideale von~$R$ in
die Kategorie aller Ideale von~$R$.\end{bem}

\begin{lemma}\label{lemma:rad}%
Für die bezüglich der Inklusionsbeziehung partiell geordnete Menge~$\Rad(R)$
der Radikalideale eines Rings~$R$ gilt:
\begin{enumerate}
\item Das kleinste Element ist~$\sqrt{(0)}$, das Ideal aller nilpotenten
Elemente.
\item Das größte Element ist~$(1)$, das Einsideal.
\item Das Supremum zweier Elemente~$\aa,\bb$, also das kleinste Radikalideal,
das~$\aa$ und~$\bb$ umfasst, ist
\[ \sqrt{\aa + \bb} := \{ x \in R \,|\, \text{$x^n = u + v$ für ein~$n \geq 0,
u \in \aa, v \in \bb$} \}. \]
\item Das Infimum zweier Elemente~$\aa,\bb$, also das größte Radikalideal,
das in~$\aa$ und~$\bb$ enthalten ist, ist
\[ \aa \cap \bb. \]
\end{enumerate}
\end{lemma}
\begin{proof}Nachrechnen.\end{proof}

\begin{bsp}Für den Ring der ganzen Zahlen gilt~$(6) \cap (5) = (30)$ und~$(6)
\cap (15) = (30)$.\end{bsp}

\begin{bsp}Allgemein gilt
\[ \sup\bigl\{ \sqrt{(x)}, \sqrt{(y)} \bigr\} = \sqrt{\sqrt{(x)} + \sqrt{(y)}} =
\sqrt{(x,y)}. \]
\end{bsp}

\nocite{*}
\printbibliography

\end{document}

% XXX: \seq{} erzeugt falschen horizontalen Freiraum.

% Mögliche weitere Themen:
% * Lindenbaumzeugs
% * Suche auf unendlichen Datentypen
% * Andrejs Gems and Stones
% * Arbeit mit universellen Objekten und Elementen
%   (vielleicht erstmal universelle Ringelemente,
%   dann universelle Objektelemente (Scheibentopoi)
%   und zuletzt universelle Objekte (E[X]))
% * Mutter aller Monaden verstehen. Was ist die logische Interpretation?

% Im nächsten Vortrag erklären:
% * konkrete vs. abstrakte Aussagen
% * Hilberts Programm
% * verschiedene Stufen für Schlimmheit von LEM
% * "Int. Logik schwächer, aber feiner"
% * "Von konkreten Aussagen sollte es konkrete Beweise geben"
% * Proof mining auch für Beweise aus der Analysis relevant,
%   siehe etwa Kohlenbach, Applied Proof Theory.
% * Grund, wieso PA/HA inkonsistent sein KÖNNTE
% * Gelfand-Schneider-FUD wieder auflösen
%
% Dazu noch herausfinden:
% * historischer Umgang mit LEM
% * Theorem von Barr verstehen. Ist es wirklich nicht konstruktiv?
%   Bezug zu Friedmans Trick?
%
% Ideen für Übungsaufgaben:
% * Bei Coquand und Troelstra inspirieren lassen.

% Literatur:
% * http://math.stanford.edu/~feferman/papers/highlights.pdf

% Steve Russ. The Mathematical Works of Bernard Bolzano.
% Google Docs Seite 276 und vorher.
% http://books.google.de/books?id=zp7cLQn0x3gC&lpg=PA146&ots=riM_BDSt8G&dq=bolzano%20early%20analysis%20rb&hl=de&pg=PA276#v=onepage&q&f=false
