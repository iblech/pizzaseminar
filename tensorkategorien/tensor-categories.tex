\documentclass[a4paper,english,12pt]{scrartcl}

\usepackage[utf8]{inputenc}

\usepackage[english]{babel}

\usepackage{amsmath,amsthm,amssymb,stmaryrd,mathtools,color,graphicx}
\usepackage{xspace}
\usepackage{array}
\usepackage[all]{xy}

\usepackage[protrusion=true,expansion=true]{microtype}

\usepackage{lmodern}
\usepackage{tabto}

\usepackage{hyperref}

\theoremstyle{definition}
\newtheorem{defn}{Definition}[section]
\newtheorem{ex}[defn]{Example}

\theoremstyle{plain}

\newtheorem{prop}[defn]{Proposition}
\newtheorem{lemma}[defn]{Lemma}
\newtheorem{thm}[defn]{Theorem}
\newtheorem{cor}[defn]{Corollary}

\theoremstyle{remark}
\newtheorem{rem}[defn]{Remark}

\clubpenalty=10000
\widowpenalty=10000
\displaywidowpenalty=10000

%\setlength\parskip{\medskipamount}
%\setlength\parindent{0pt}

\newcommand{\NN}{\mathbb{N}}
\newcommand{\ZZ}{\mathbb{Z}}
\renewcommand{\aa}{\mathfrak{a}}
\newcommand{\bb}{\mathfrak{b}}
\newcommand{\pp}{\mathfrak{p}}
\renewcommand{\O}{\mathcal{O}}
\newcommand{\C}{\mathcal{C}}
\newcommand{\D}{\mathcal{D}}
\newcommand{\id}{\mathrm{id}}
\newcommand{\op}{\mathrm{op}}
\newcommand{\xra}[1]{\xrightarrow{#1}}
\DeclareMathOperator{\Spec}{Spec}
\renewcommand{\_}{\mathpunct{.}\,}
\newcommand{\?}{\,{:}\,}
\newcommand{\defeq}{\vcentcolon=}

\newcommand{\hilight}[2]{\begin{center}\framebox{#2}\par#1\end{center}}

\newcommand{\speak}[1]{\ulcorner\text{#1}\urcorner}

\setcounter{tocdepth}{2}

\newlength{\aufgabenskip}
\setlength{\aufgabenskip}{1.5em}
\newcounter{aufgabennummer}
\newenvironment{aufgabe}[1]{
  \addtocounter{aufgabennummer}{1}
  \textbf{Aufgabe \theaufgabennummer{}.} \emph{#1} \par
}{\vspace{\aufgabenskip}}

\newenvironment{indentblock}{%
  \list{}{\leftmargin\leftmargin}%
  \item\relax
}{%
  \endlist
}

\newcommand{\Mod}{\mathrm{Mod}}
\newcommand{\Set}{\mathrm{Set}}
\newcommand{\Vect}{\mathrm{Vect}}
\newcommand{\Hom}{\underline{\mathrm{Hom}}}
\renewcommand{\hom}{\mathrm{Hom}}
\newcommand{\End}{\mathrm{End}}
\newcommand{\ev}{\mathrm{ev}}
\newcommand{\freist}{\underline{\ \ }}
\newcommand{\Rep}{\mathrm{Rep}}
\newcommand{\Cob}{\mathrm{Cob}}
\newcommand{\tr}{\operatorname{tr}}
\newcommand{\rk}{\operatorname{rk}}
\newcommand{\im}{\operatorname{im}}
\newcommand{\ie}{i.\,e.\@\xspace}

\begin{document}

\title{Tensor categories}
\author{Ingo Blechschmidt}
\date{February 8, 2014}
\maketitle

\begin{center}\begin{minipage}{0.8\textwidth}
Tensor categories are categories equipped with a tensor operation between
objects and between morphisms. They arise in abstract category theory,
representation theory, quantum field theory and certain other subjects and
provide an abstract setting for studying the general notions of the dual of
an object and the trace of a morphism. Thanks to a rich graphical calculus,
working with commutative diagrams in tensor categories is very
enjoyable.\medskip

These are informal notes prepared for the February 2014 meeting of the \emph{Kleine
Bayerische AG} at TU München. The notes summarize the first chapter of the
article \emph{Tannakian Categories} (LNM 900, 1982;
\href{http://www.jmilne.org/math/xnotes/tc.pdf}{2012}) by P.~Deligne and
J.~S.~Milne.
\end{minipage}\end{center}
\vspace{1em}

\tableofcontents
\newpage

\section{Basics}

\begin{ex}The category~$\Mod_R$ of modules over a commutative ring~$R$ is the
archetypical example of a tensor category.\end{ex}

\begin{defn}A \emph{tensor category}~$(\C,\otimes)$ (\ie a monoidal category with
symmetric braiding) consists of
\begin{itemize}
\item a category~$\C$,
\item a functor~${\otimes} : \C \times \C \to \C$ (the tensor operation),
\item an object~$1 \in \C$ (the unit object),
\item natural isomorphisms~$X \otimes (Y \otimes Z) \xra{\phi_{XYZ}} (X \otimes
Y) \otimes Z$ (the associator),
\item natural isomorphisms~$X \otimes Y \xra{\psi_{XY}} Y \otimes X$ (the
braiding),
\item natural isomorphisms~$1 \otimes X \xra{\lambda_X} X$ and~$X \otimes 1
\xra{\rho_X} X$ (the unitors),
\end{itemize}
such that
\begin{itemize}
\item the braiding is symmetric: $\psi_{YX} \circ \psi_{XY} = \id_{X \otimes
Y}$ and
\item the following coherence conditions are satisfied: the pentagon axiom
(displayed); the hexagon axiom (relating the associator with the braiding); and
the triangle axiom (relating the associator with the unitors).
\[ \xy
(0,20)*{X \otimes (Y \otimes (Z \otimes T))}="1"; 
(40,0)*{(X \otimes Y) \otimes (Z \otimes T)}="2"; 
(27,-20)*{\quad\ ((X \otimes Y) \otimes Z) \otimes T}="3"; 
(-27,-20)*{(X \otimes (Y \otimes Z)) \otimes T\!\!\!}="4"; 
(-40,0)*{X \otimes ((Y \otimes Z) \otimes T)}="5"; 
{\ar^{\phi_{X,Y,Z \otimes T}} "1";"2"} 
{\ar^{\phi_{X\otimes Y,Z,T}} "2";"3"} 
{\ar_{\id_X \otimes \phi_{Y,Z,T}\quad} "1";"5"} 
{\ar_{\phi_{X,Y\otimes Z,T}} "5";"4"} 
{\ar_{\!\!\!\!\!\!\!\!\phi_{X,Y,Z} \otimes \id_T} "4";"3"} 
\endxy \]
\end{itemize}
\end{defn}

The definition mimics the definition of a monoid, only ``one level up''.
Demanding that the tensor operation is associative and commutative on the nose
(e.\,g.\@ $X \otimes (Y \otimes Z) = (X \otimes Y) \otimes Z$) would be
\emph{evil} in the technical sense, \ie not preserved under equivalence of
categories. This is the reason why we demand (natural) isomorphisms, \ie
natural transformations between easily inferred functors.

The isomorphism classes of objects of a tensor category form a (possibly large) commutative monoid.

\begin{ex}In~$\Mod_R$, $\phi_{XYZ}$ is given by
\[ x \otimes (y \otimes z) \longmapsto (x \otimes y) \otimes z. \]
If~$-1 \neq 1 \in R$, introducing an artificial sign here will cause the pentagon diagram
to fail to commute by a sign.\end{ex}

\begin{ex}Let~$\C$ be a category with finite products. Then~$(\C,\times)$ is a
tensor category, with~$\phi, \psi, \lambda, \rho$ given by the universal
property of the product. In particular,~$(\Set,\times)$ is a tensor
category.\end{ex}

\begin{ex}Let~$\C$ be a tensor category. Then~$\C^\op$ is in a natural way a
tensor category, by using the inverses of the given natural
isomorphisms.\end{ex}

\begin{rem}Unlike in~$(\Set,\times)$, in a general tensor category, there is no
natural morphism~$X \to X \otimes X$.\end{rem}

The coherence conditions in the definition are needed for the following reason:
In the category of modules, we are used to dropping all parentheses when
dealing with iterated tensor products. This is justified only because between any
two given groupings, e.\,g.\@
\[ X \otimes ((Y \otimes Z) \otimes (T \otimes U)) \quad\text{and}\quad
  ((X \otimes Y) \otimes (Z \otimes T)) \otimes U, \]
we have \emph{exactly one} canonical isomorphism. A classical theorem of Mac~Lane
guarantees that the stated coherence conditions suffice to render any
``reasonable'' diagram commutative:

\begin{thm}[Mac Lane]Any (formal) diagram in a tensor category built
by \[
\otimes,\id,\phi,\psi,\lambda,\rho,\phi^{-1},\psi^{-1},\lambda^{-1},\rho^{-1}
\]
(in which both sides have the same permutation) commutes.\end{thm}

The two caveats are the following: Firstly, in a given tensor category, there
may hold certain \emph{identities} between objects for no general abstract
reason. For example, for some totally unrelated objects~$X,Y,A,B$, it might
hold that $X \otimes Y = A \otimes B$. Using those identities we can form
diagrams which we do \emph{not} expect to commute. Mac~Lane's coherence theorem
does not make any statement about those diagrams.

To understand the restriction about permutations, consider the diagram
\[ \xymatrix{
  X \otimes X \ar@/^1pc/[r]^{\psi_{XX}} \ar@/_1pc/[r]_{\id_{X \otimes X}} & X \otimes X.
} \]
We do not expect this diagram to be commutative; the permutations associated to
both sides are not equal: $(1,2) \neq \id$.

\begin{thm}[Joyal, Street]If the graphical depictions of given morphisms of a
tensor category are ``the same'' (in~4D space), the morphisms are
equal.\end{thm}
See \emph{A survey of graphical languages for monoidal categories}
by P.~Selinger (LNP 813, \href{http://arxiv.org/abs/0908.3347}{arXiv:0908.3347})
for a precise formulation and pointers to the relevant literature.

\begin{ex}In four-dimensional space, a double braiding can be unknotted (while
holding the endpoints fixed):
\begin{center}\scalebox{0.5}{\input{braiding.pspdftex}}\end{center}\end{ex}

\begin{thm}[Mac Lane]Any tensor category can be \emph{strictified}, \ie is
equivalent as a tensor category to a \emph{strict} tensor category: a category
in which $\phi, \lambda, \rho$ (but not~$\psi$) are identities.\end{thm}

The strictification of a tensor category should \emph{not} be
thought of as some kind of quotienting process, \ie one does not obtain the
strictification by identifying certain morphisms. Instead, the objects of the
strictification are ``cliques'' of possible groupings.

Also, skeletons of tensor categories are usually \emph{not} strict: For instance,
consider a skeleton of the category of sets with its unique object~$N$
isomorphic to the set of natural numbers. Then the objects~$N \times (N \times N)$
and~$(N \times N) \times N$ are equal. But the unique morphism $N \times (N \times
N) \to (N \times N) \times N$ compatible with the projections is not equal to
the identity.


\section{Structure in tensor categories}

\begin{defn}An \emph{internal Hom} between objects $X, Y$ of a tensor
category~$\C$ consists of
\begin{itemize}
\item an object~$\Hom(X,Y) \in \C$ and
\item a morphism~$\Hom(X,Y) \otimes X \xra{\ev} Y$ (evaluation morphism)
\end{itemize}
such that this pair is terminal among all such pairs, \ie such that for any
object~$T \in \C$ and any [fake] evaluation morphism~$T \otimes X
\xra{\widetilde\ev} Y$ there exists a unique morphism~$f : T \to \Hom(X,Y)$
such that the following diagram commutes:
\[ \xymatrix{
  T \otimes X \ar[rrd]^{\widetilde\ev} \ar[d]_{f \otimes \id_X} \\
  \Hom(X,Y) \otimes X \ar[rr]_{\quad\qquad\ev} && Y
} \]
\end{defn}

\begin{ex}In~$(\Set,\times)$, the internal Homs are given by the usual Hom
sets. The evaluation morphism is given by~$(f,x) \mapsto f(x)$.\end{ex}

\begin{ex}In~$\Mod_R$, the internal Homs are given by the Hom sets equipped
with the usual module structure.\end{ex}

\begin{rem}If an internal Hom~$\Hom(X,Y)$ exists for all objects~$Y \in \C$
(and if appropriate choice principles are available), the internal Hom can be
made into a functor~$\Hom(X,\freist) : \C \to \C$. This functor is right adjoint to
taking tensor product with~$X$:
\[ \freist \otimes X \ \dashv\  \Hom(X,\freist) \]
\end{rem}

\begin{rem}The relation with the usual Hom (which is only a set) is the
following:
\[ \hom(1, \Hom(X,Y)) \cong \hom(1 \otimes X, Y) \cong \hom(X,Y). \]
\end{rem}

\begin{rem}In~$(\Set,\amalg)$, internal Homs do not exist in general. This is
related to the fact that the functor~$\freist \amalg X$ does not preserve colimits and so cannot
be a left adjoint.\end{rem}
% XXX: "general" präzisieren

\begin{defn}\begin{enumerate}\item A \emph{dual} of an object~$X$ is an internal
Hom~$X^\vee \defeq \Hom(X,1)$. \item The \emph{dual} of a morphism~$f : X \to Y$ is
the unique morphism~$f^t : Y^\vee \to X^\vee$ rendering the diagram
\[ \xymatrixcolsep{5pc}\xymatrix{
  Y^\vee \otimes X \ar[r]^{f^t \otimes \id_X} \ar[d]_{\id_{Y^\vee} \otimes f} &
  X^\vee \otimes X \ar[d]^{\ev_X} \\
  Y^\vee \otimes Y \ar[r]_{\ev_Y} & 1
} \]
commutative (if~$X^\vee$ and~$Y^\vee$ exist).\end{enumerate}\end{defn}

\begin{ex}In~$\Mod_R$, $f^t : \theta \mapsto \theta \circ f$.\end{ex}

If all objects possess a dual (and appropriate choice principles are
available), the rule~$X \mapsto X^\vee$ can be made into a
functor~$\C \to \C^\op$.

\begin{prop}In any tensor category, the set~$\End(1) = \hom(1,1)$ is a
commutative monoid with respect to composition of morphisms.\end{prop}
\begin{proof}The result holds even if there were no braiding:
On~$\End(1)$, the tensor product induces a second binary operation. By the
coherence conditions, this operation commutes with the operation given by
composition, so by the famous Eckmann--Hilton argument, both operations coincide
and are commutative.\end{proof}

\begin{ex}In~$\Mod_R$, $\End(1) \cong R$.\end{ex}

\begin{ex}In~$(\Set,\times)$, $\End(1) = \{\id\}$.\end{ex}


\section{Tensor functors}

\begin{defn}A \emph{tensor functor}~$F : (\C,\otimes) \to (\C',\otimes')$
consists of
\begin{enumerate}
\item a functor~$F : \C \to \C'$,
\item natural isomorphisms~$FX \otimes' FY \xra{c_{XY}} F(X \otimes Y)$ and
\item an isomorphism~$1' \xra{e} F1$
\end{enumerate}
such that the following coherence condition
\[ \xymatrixcolsep{2.5pc}\xymatrix{
  FX \otimes' (FY \otimes' FZ) \ar[r]^{\id \otimes c} \ar[d]_{\phi'} &
  FX \otimes' F(Y \otimes Z) \ar[r]^c & F(X \otimes (Y \otimes Z)) \ar[d]^{F\phi} \\
  (FX \otimes' FY) \otimes' FZ \ar[r]_{c \otimes \id} &
  F(X \otimes Y) \otimes' Z \ar[r]_c &
  F((X \otimes Y) \otimes Z)
} \]
and similar ones relating~$c$ with the braidings and with the unitors are
satisfied.
\end{defn}

\begin{ex}The forgetful functor~$\Rep_k(G) \to \Vect_k$ of the category
of finite-di\-men\-sio\-nal~$k$-linear representations of a group (or group
scheme)~$G$ is a tensor functor.\end{ex}

\begin{ex}Extension of scalars defines a tensor functor~$\Mod_R \to
\Mod_S$; restriction of scalars does not.\end{ex}

\begin{ex}A quantum field theory determines a tensor functor~$\Cob_d \to
\Vect_k$.\end{ex}

\begin{defn}A \emph{morphism of tensor functors} $\eta : (F,c,e) \to
(\widetilde F,\tilde c,\tilde e)$ consists of a natural transformation~$\eta : F \to
\widetilde F$ which is compatible with the coherence isomorphisms:
\[ \xymatrixcolsep{2.5pc}\xymatrix{
  FX \otimes' FY \ar[d]_{\eta_X \otimes \eta_Y} \ar[r]^c &
  F(X \otimes Y) \ar[d]^{\eta_{(X \otimes Y)}} &&
  1' \ar[r]^e \ar[d]_{\id_1} & F1 \ar[d]^{\eta_1} \\
  \widetilde F X \otimes' \widetilde F Y \ar[r]_{\tilde c} &
  \widetilde F(X \otimes Y) &&
  1' \ar[r]_{\tilde e} & \widetilde F1
} \]
\end{defn}


\section{Rigid tensor categories}

\begin{defn}A tensor category~$\C$ is \emph{rigid} iff
\begin{itemize}
\item all internal Homs exist,
\item the natural morphisms
\[ \Hom(X_1,Y_1) \otimes \Hom(X_2,Y_2) \longrightarrow
  \Hom(X_1 \otimes Y_1, X_2 \otimes Y_2) \]
are isomorphisms and
\item all objects~$X \in \C$ are \emph{reflexive} (\ie the natural maps~$X
\to X^{\vee\vee}$ are isomorphisms).
\end{itemize}
\end{defn}

\begin{ex}The category~$\Vect^\mathrm{fd}_k$ of finite-dimensional is
rigid.\end{ex}

\begin{ex}More
generally, the category~$\Mod^\mathrm{fin.\,free}_R$ of finitely free~$R$-modules
is rigid.\end{ex}

\begin{ex}Switching toposes, the category of locally free~$\O_{\Spec R}$-modules
is rigid. (This category is equivalent to category of finitely generated
projective~$R$-modules, by the tilde construction.)\end{ex}

\begin{ex}The category~$\Rep_k(G)$ is rigid.\end{ex}

\begin{rem}In a rigid tensor category, internal Homs can be calculated using the
tensor product: $\Hom(X,Y) \cong X^\vee \otimes Y$.
\end{rem}

\begin{rem}Let~$\C$ be a rigid category. Then the functor~$\C \to \C^\op,\,X
\mapsto X^\vee$ is an equivalence of tensor categories.\end{rem}

In a rigid tensor category, we can define the notion of traces:
\begin{defn}\begin{enumerate}
\item The \emph{trace} of an endomorphism~$f : X \to X$ in a rigid tensor
category is the following element of~$\End(1)$:
\[ f \in \hom(X,X) \cong
  \hom(1,\Hom(X,X)) \cong
  \hom(1,X^\vee \otimes X) \to
  \hom(1,1) \ni\vcentcolon \tr f. \]
\item The \emph{rank} (or \emph{Euler characteristic}) of an object~$X$ is the
trace of the identity morphism on~$X$.
\end{enumerate}\end{defn}

Trace and rank satisfy the relations you expect:
\begin{lemma}\begin{enumerate}
\item $\tr(f \otimes f') = \tr(f) \circ \tr(f'), \quad
  \tr(f \circ g) = \tr(g \circ f).$
\item $\rk(X \otimes X') = \rk(X) \circ \rk(X'), \quad
  \rk(1) = \id_1.$
\end{enumerate}\end{lemma}

\begin{lemma}In a rigid tensor category, a pair~$(Y, Y \otimes X \xra{\ev} 1)$
is a dual of~$X$ iff there exists a morphism $1 \xra{\varepsilon} X \otimes Y$
such that the following triangle identities hold:
\begin{center}
\scalebox{0.7}{\input{dual-conditions.pspdftex}}
\end{center}
Suppressing coherence isomorphisms, these conditions can also be expressed as
\[ (\id_X \otimes \ev) \circ (\varepsilon \otimes \id_X) = \id_X, \quad
  (\ev \otimes \id_Y) \circ (\id_Y \otimes \varepsilon) = \id_Y. \]
\end{lemma}
\begin{proof}Given a pair~$(T, T \otimes X \xra{\widetilde\ev} 1)$,
construct~$f : T \to Y$ as follows:
\begin{center}
\scalebox{0.7}{\input{dual-construction.pspdftex}}
\end{center}
Using the graphical calculus it's fun to check that with this definition of~$f$, the
diagram in the definition of the internal Hom commutes (try it!). Use the
second triangle identity to show uniqueness.

For the converse direction, construct~$\varepsilon$ by dualizing~$\ev$.
\end{proof}

\begin{rem}Rigidity is only necessary for the converse direction: A pair~$(Y, Y
\otimes X \xra{\ev} 1)$ which satisfies the condition stated in lemma is always
a dual---and in fact, because the condition is symmetric, the coevaluation
morphism exhibits~$X$ as a dual of~$Y$. In particular, $X$ is
reflexive.\end{rem}

\begin{ex}Let~$M$ be a finitely free~$R$-module with basis~$(x_1,\ldots,x_n)$.
Let~$(\vartheta_1,\ldots,\vartheta_n)$ be the associated dual basis of~$M^\vee$.
Then~$\varepsilon$ is given by
\[ \varepsilon : R \longrightarrow M \otimes M^\vee,\ 
  1 \longmapsto \sum_i x_i \otimes \vartheta_i. \]
The condition in the lemma expresses that for any~$x \in M$ and any~$\vartheta \in
M^\vee$, it holds that
\[
  x = \sum_i \vartheta_i(x) \, x_i \quad\text{and}\quad
  \vartheta = \sum_i \vartheta(x_i) \, \vartheta_i. \]
\end{ex}

\begin{prop}Let~$F : \C \to \C'$ be a tensor functor, with~$\C$ rigid.
Let~$X,Y \in \C$. Then an internal Hom~$\Hom(FX,FY)$ exists in~$\C'$ and the
natural morphism
\[ F(\Hom(X,Y)) \longrightarrow \Hom(FX,FY) \]
is an isomorphism.\end{prop}
\begin{proof}It is enough to show that in~$\C'$ a dual of~$FX$ exists and that
the natural morphism
\[ F(X^\vee) \longrightarrow (FX)^\vee \]
is an isomorphism. This is obvious by the lemma, as the given condition is
preserved by~$F$.\end{proof}

\begin{cor}$\tr F(f) = F(\tr f), \quad \rk FX = F(\rk X)$.\end{cor}

\begin{prop}In a rigid category, $\freist \otimes X$ is not only a left, but
also a right adjoint:
\[ \Hom(X,\freist) \ \dashv\  \freist \otimes X \ \dashv\  \Hom(X,\freist). \]
\end{prop}
\begin{cor}In a rigid category, the tensor operation is continuous and
cocontinuous, \ie commutes with arbitrary limits and colimits.\end{cor}
\begin{proof}This is a general fact of functors which are both left and right
adjoints.\end{proof}


\section{Abelian tensor categories}

\begin{defn}An \emph{abelian tensor category} is a tensor
category~$(\C,\otimes)$ such that~$\C$ is abelian and~$\otimes$ is
bi-additive.\end{defn}

\begin{rem}In an abelian category,~$\End(1)$ is a commutative ring with respect
to addition and composition of morphisms.\end{rem}

\begin{lemma}Let~$\C$ be an abelian tensor category. Let~$X \in \C$.
\begin{enumerate}
\item Assume that the unit~$1$ is a simple object and that~$\C$ is rigid. Then:
\[ X \not\cong 0 \quad\Longrightarrow\quad
  \text{$X^\vee \otimes X \to 1$ is epic.} \]
\item Assume that~$1 \not\cong 0$ in~$\C$. Then:
\[ X \not\cong 0 \quad\Longleftarrow\quad
  \text{$X^\vee \otimes X \to 1$ is epic.} \]
\end{enumerate}
\end{lemma}
\begin{proof}
\begin{enumerate}
\item Because the unit is simple, the natural morphism~$X^\vee \otimes X \to 1$
is either zero or epic. (This follows by considering the image of the
morphism.) Under the correspondence
\[ \hom(X^\vee \otimes X,1) \cong \hom(X,X^{\vee\vee}), \]
the morphism corresponds to the natural isomorphism~$X \to X^{\vee\vee}$. If this
was zero,~$X$ would be zero as well.
\item Assume~$X \cong 0$. Then we have an epic morphism~$0 \to 1$. By
terminality of the zero object, this is an isomorphism.\qedhere
\end{enumerate}
\end{proof}

\begin{rem}The rigidity is necessary: In~$\Mod_\ZZ$, the object~$\ZZ/(2)$ is
not isomorphic to the zero object, but the morphism~$(\ZZ/(2))^\vee \otimes
\ZZ/(2) \to \ZZ$ has zero domain.\end{rem}

\begin{prop}Let~$\C$ be a rigid abelian tensor category with~$1 \in \C$ simple.
Let~$\C'$ be an abelian tensor category with~$1 \not\cong 0$. Then any exact
tensor functor~$F : \C \to \C'$ is faithful.\end{prop}
\begin{proof}Let~$Ff = 0$. If~$f \neq 0$, then~$\im(f) \not\cong 0$. So by the
lemma, the natural morphism~$\im(f)^\vee \otimes \im(f) \to 1$ is epic.
Because~$F$ is exact and preserves duals, the natural morphism
$\im(Ff)^\vee \otimes \im(Ff) \to 1$ is epic as well. Again by the lemma, it
follows that~$\im(Ff) \not\cong 0$. This is a contradiction.\end{proof}

\begin{rem}The simpleness of the unit is necessary: Let~$\C$ and~$\D$ be rigid abelian
categories. Then the projection functor~$\C \times \D \to \C$
is exact, but in general not faithful.\end{rem}

The following corollary shows that the proposition can be interpreted as a
\emph{categorification} of a certain well-known lemma of commutative algebra:

\begin{cor}Let~$k$ be a field and~$R$ be a ring with~$1 \neq 0 \in R$.
Let~$\varphi : k \to R$ be a ring homomorphism. Then~$\varphi$ is
injective.\end{cor}
\begin{proof}By the proposition, the functor~$F : \Vect_k^\mathrm{fd} \to
\Mod_R$ (extension of scalars) is faithful. Let~$\varphi(x) = 0$. Then the
image of the map~$k \to k$ given by multiplication with~$x$ is zero in~$\Mod_R$. So by
faithfulness, the map is already zero on~$k$, so~$x = 0$.\end{proof}

\begin{prop}Let~$\C$ be a rigid abelian tensor category. Let~$U \hookrightarrow
1$ be a subobject. Then the unit object decomposes as a direct sum
\[ 1 \cong U \oplus U^\perp, \]
where~$U^\perp = \ker(1 \to U^\vee)$.\end{prop}
\begin{proof}The argument is given in six steps. Rigidity mainly enters in that
tensoring is exact instead of merely right-exact.\begin{enumerate}
% XXX: korrektes Englisch
\item Consider the cokernel~$V$ of~$U \hookrightarrow 1$; we then have a short
exact sequence
\[ 0 \longrightarrow U \longrightarrow 1 \longrightarrow V \longrightarrow 0.
\]
\item By rigidity, tensoring with~$U$ is exact; so we obtain the commutative diagram
\[ \xymatrix{
  0 \ar[r] & U \ar[r] & 1 \ar[r] & V \ar[r] & 0 \\
  0 \ar[r] & U \otimes U \ar[r] \ar@{^{(}->}[u] & U \ar[r] \ar@{^{(}->}[u] & V \otimes U \ar[r]
  \ar@{^{(}->}[u] & 0
} \]
with exact rows. Because the morphism~$U \twoheadrightarrow V \otimes U \hookrightarrow V$ is zero,~$V
\otimes U$ is zero; and by exactness of the bottom row, $U \otimes U \cong U$.
\item For any subobject~$T \hookrightarrow X$ the following chain of
equivalences holds:
\[ T \otimes U \cong 0
  \ \Longleftrightarrow\ 
  \text{$T \otimes U \hookrightarrow T$ is zero}
  \ \Longleftrightarrow\ 
  \text{$T \twoheadrightarrow U^\vee \otimes T \to U^\vee \otimes X$ is zero}.
  \]
The first~``$\Leftarrow$'' is because~$T \otimes U \hookrightarrow T$ is a
monomorphism (by exactness of tensoring with~$T$) and the second
``$\Leftrightarrow$'' is by the isomorphisms
\[ \Hom(T \otimes U, T) \cong U^\vee \otimes T^\vee \otimes T \cong
  \Hom(T, U^\vee \otimes T). \]
So the largest such subobject~$T \hookrightarrow X$ is given by
\[ T = \ker(X \to U^\vee \otimes X) \cong U^\perp \otimes X. \]
(The isomorphism is by exactness of tensoring with~$X$.)
\item Applying this observation to~$X = V$, it follows that~$U^\perp \otimes V
\cong V$, because even~$V \otimes U \cong 0$ holds.

Applying it to~$X = U$, it follows that~$U^\perp \otimes U \cong 0$: Let~$T
\hookrightarrow U$ with~$T \otimes U \cong 0$. By exactness of tensoring
with~$T$, the sequence
\[ 0 \longrightarrow T \otimes U \longrightarrow T \longrightarrow T \otimes V
\longrightarrow 0 \]
is exact. Since~$T \otimes U \cong 0$, we have
\[ T \cong T \otimes V \hookrightarrow U \otimes V \cong 0. \]
The ``$\hookrightarrow$'' is---again---by exactness of tensoring (with~$V$).

\item By exactness of tensoring with~$U^\perp$, the sequence
\[ 0 \longrightarrow U^\perp \otimes U \longrightarrow U^\perp \longrightarrow
U^\perp \otimes V \longrightarrow 0 \]
is exact. By the previous step, this shows~$U^\perp \cong V$.

\item By applying the five lemma to the diagram
\[ \xymatrix{
  0 \ar[r] & U \ar[r] \ar[d] & U \oplus U^\perp \ar[r] \ar[d] & U^\perp \ar[r] \ar[d] & 0 \\
  0 \ar[r] & U \ar[r] & 1 \ar[r] & V \ar[r] & 0
} \]
it follows that~$1 \cong U \oplus U^\perp$. \qedhere
\end{enumerate}
\end{proof}

\begin{rem}Rigidity is necessary: In~$\Mod_\ZZ$, the unit object is not simple
but admits no non-trivial decompositions either.\end{rem}

\begin{cor}In a rigid abelian category, the unit object is simple iff~$\End(1)$
is a field.\end{cor}
\begin{proof}For the ``only if'' direction, let~$f \in \End(1)$. As the unit
object is simple, $f$ is either zero or an isomorphism (\ie invertible in the
ring~$\End(1)$). This follows by considering kernel and image of~$f$.

For the ``if'' direction, let~$U \hookrightarrow 1$. By the lemma, there exists
a projection operator~$P : 1 \to 1$ with~$\im(P) = U$. If~$P$ is zero
in~$\End(1)$, $U = 0$; if~$P$ is invertible, $U = 1$ (as subojects
of~$1$).\end{proof}


\section{The Tannaka reconstruction theorem}

\begin{thm}Let~$\C$ be a rigid abelian tensor category. Let~$k \defeq \End(1)$ be a
field. Let~$\omega : \C \to \Vect_k^\mathrm{fd}$ be an exact faithful~$k$-linear
tensor functor. Then:
\begin{enumerate}
\item The functor~$\underline{\mathrm{Aut}}^\otimes(\omega) : \mathrm{Alg}_k
\to \Set$, given by
\[ R \longmapsto \text{set of tensor automorphisms of~$(\freist \otimes_k R)
\circ \omega$}, \]
is the functor of points of an affine group scheme~$G$.
\item A certain functor~$\C \to \Rep_k(G)$ induced by~$\omega$ is an
equivalence of tensor categories.
\end{enumerate}
\end{thm}

\end{document}
