\documentclass{pizzablatt}

\geometry{tmargin=2cm,bmargin=1.3cm,lmargin=2.9cm,rmargin=2.9cm}

\setlength{\aufgabenskip}{1em}

\begin{document}

\maketitle{3}{Pizzaseminar zu konstruktiver Mathematik}{4. September 2013}

\begin{aufgabe}{Schranken für die Größe der~$n$-ten Primzahl}
Folgender Beweis der Unendlichkeit der Primzahlen wird Euklid zugeschrieben:

\begin{quote}
Angenommen, $p_1, \ldots, p_r$ seien alle Primzahlen. Wir setzen~$N :=
p_1 \cdots p_r + 1$. Nach dem Fundamentalsatz der Arithmetik lässt sich~$N$ in
Primfaktoren zerlegen. Das ist ein Widerspruch, denn die~$p_i$ sind keine
Teiler von~$N$, andere Primzahlen gibt es aber nach Widersprungsvoraussetzung
nicht.
\end{quote}

\begin{enumerate}
\item Formuliere den Beweis so um, dass er konstruktiv folgende stärkere
Aussage zeigt: \emph{Seien~$p_1,\ldots,p_r$ gegebene Primzahlen. Dann gibt es eine
weitere Primzahl ungleich den~$p_i$.} (Das war auch Euklids ursprüngliche
Formulierung.)

\item Sei nun~$p_1,p_2,\ldots$ die aufsteigende Folge aller Primzahlen. Extrahiere
aus deinem Beweis die Abschätzung
\[ p_{n+1} \leq p_1 \cdots p_n + 1. \]

\item Zeige folgende Schranke für die Größe der~$n$-ten Primzahl:
\[ p_n \leq 2^{2^{n-1}}. \]
\end{enumerate}

Tatsächlich ist diese Schranke sehr pessimistisch. Aus Eulers
Alternativbeweis der Unendlichkeit der Primzahlen, der nicht nur die Existenz,
sondern auch die Eindeutigkeit der
Primfaktorzerlegung verwendet, kann man eine bessere Schranke extrahieren;
siehe etwa die Analyse in U.~Kohlenbach, \emph{Applied Proof Theory}, Kapitel~2,
Seite~15f.
\end{aufgabe}

\begin{aufgabe}{Friedmans Trick}
Beweise folgende fundamentale Eigenschaften der Friedmanübersetzung:
\begin{enumerate}
\item Sei~$\varphi$ eine Aussage, in der Existenzquantoren nur über bewohnte
Typen gehen. Dann gilt intuitionistisch: $F \Longrightarrow \varphi^F$.

\item Sei~$\varphi$ eine Aussage, in der nur~$\top$, $\bot$,
$\wedge$, $\vee$ und $\exists$ (über
bewohnte Typen), aber nicht~$\Rightarrow$ oder~$\forall$ vorkommen.
Dann gilt intuitionistisch: $\varphi^F \Longleftrightarrow \varphi \vee F$.

\emph{Tipp:} Induktion über den Aussageaufbau.

\item Seien~$\varphi$ und~$\psi$ beliebige Aussagen in einem Kontext~$\vec x$
(mit~$\exists$ nur über bewohnte Typen).
Wenn~$\varphi \seq{\vec x} \psi$ intuitionistisch, dann gilt auch~$\varphi^F
\seq{\vec x} \psi^F$ intuitionistisch.

\emph{Tipp:} Induktion über den Aufbau von Ableitungen -- zu zeigen ist, dass
die Friedmanübersetzungen aller Schlussregeln intuitionistisch gültig sind.

\item Die Peano-Axiome implizieren ihre Friedmanübersetzungen.
\end{enumerate}
\end{aufgabe}

\begin{aufgabe}{Formaler Nullstellensatz}
Seien~$f_1,\ldots,f_n \in R[X_1,\ldots,X_m]$ Polynome in~$m$ Variablen über
einem Ring~$R$ mit~$1 \neq 0$.
\begin{enumerate}
\item
Gelte~$1 = p_1 f_1 + \cdots + p_n f_n$ für gewisse Polynome~$p_1,\ldots,p_n$.
Zeige, dass die Polynome~$f_i$ keine gemeinsame Nullstelle besitzen.
\item Zeige umgekehrt, dass man aus einem Beweis, dass die~$f_i$ keine
gemeinsame Nullstelle besitzen, genauer einem Beweis der Sequenz
\[ Z(f_1) \wedge \cdots \wedge Z(f_n) \seq{} \bot \]
welcher von der im Skript beschriebenen Form ist, explizit eine Darstellung des
Eins\-polynoms wie in~a) erhalten kann.
\end{enumerate}
\end{aufgabe}

\end{document}
