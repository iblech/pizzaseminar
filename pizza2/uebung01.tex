\documentclass{pizzablatt}

\begin{document}

\maketitle{1}{13. August 2013}

\begin{aufgabe}{Diskretheit der natürlichen Zahlen}
Zeige für alle natürlichen Zahlen~$n \in \NN$:
\[ n = 0 \quad\vee\quad \neg(n = 0). \]
Verwende dazu nur die fünf \emph{Peano-Axiome}:
\begin{enumerate}
\item[1.] $0$ ist eine natürliche Zahl.
\item[2.] Jede natürliche Zahl~$n$ hat eine Zahl~$S(n)$ als Nachfolger.
\item[3.] Die Zahl~$0$ ist Nachfolger einer Zahl.
\item[4.] Natürliche Zahlen mit gleichem Nachfolger sind gleich.
\item[5.] Enthält eine Teilmenge der natürlichen Zahlen die Zahl~$0$ und mit jeder
Zahl auch ihren Nachfolger, so enthält sie schon alle natürlichen Zahlen.
\end{enumerate}
\end{aufgabe}

\begin{aufgabe}{Konstruktive Tautologien}
Zeige für beliebige Aussagen~$\varphi, \psi$ und mache dir ggf. Gedanken über die
Rückrichtung:
\begin{enumerate}
\item $\varphi \Longrightarrow \neg\neg\varphi$
\item $(\varphi \Rightarrow \psi) \Longrightarrow (\neg\psi \Rightarrow
\neg\varphi)$
\item $\neg\neg(\varphi \vee \neg\varphi)$
\item $\neg\neg\exists x \in A{:}\, \psi(x) \Longleftrightarrow
  \neg\forall x \in A{:}\, \neg\psi(x)$
\end{enumerate}
\emph{Tipp:} Die Negation ist als~$\neg\varphi :\equiv (\varphi \Rightarrow
\bot)$ definiert. Wahrheitstafeln haben hier nichts zu suchen.
\end{aufgabe}

\begin{aufgabe}{Doppelnegationselimination}
Zeige, dass folgende zwei Prinzipien äquivalent sind:
\begin{enumerate}
\item[1.] Für alle Aussagen~$\varphi$ gilt: $\varphi \vee \neg\varphi$.
\item[2.] Für alle Aussagen~$\psi$ gilt: $\neg\neg\psi \Rightarrow
\psi$.
\end{enumerate}
\emph{Tipp für 2. $\Rightarrow$ 1.:} Verwende die Voraussetzung \emph{nicht} für~$\psi
:= \varphi$.
\end{aufgabe}

\begin{aufgabe}{Teilmengen von~$\{\star\}$}
Klassisch gilt:
\begin{quote}
  Jede Teilmenge von~$X := \{ \star \}$ ist gleich~$\emptyset$ oder gleich~$X$.
\end{quote}
Konstruktiv lässt sich das nicht zeigen, die Potenzmenge von~$X$ hat
(potenziell) viel mehr Struktur. Beweise das durch ein \emph{brouwersches
Gegenbeispiel}: Zeige, dass aus dieser Aussage das
Prinzip vom ausgeschlossenen Dritten folgt.
\end{aufgabe}

\end{document}
