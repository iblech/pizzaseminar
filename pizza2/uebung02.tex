\documentclass{pizzablatt}

\begin{document}

\maketitle{2}{21. August 2013}

\begin{aufgabe}{Doppelnegationsübersetzung}
Beweise die fundamentalen Eigenschaften der Doppelnegationsübersetzung, jeweils für alle
Aussagen~$\varphi$ und~$\psi$ in beliebigen Kontexten~$\vec x$.
\begin{enumerate}
\item Klassisch gilt: $\varphi \Longleftrightarrow \varphi^\circ$.
\item Intuitionistisch gilt: $\neg\neg\varphi^\circ \Longrightarrow
\varphi^\circ$.
\item Wenn~$\varphi \seq{\vec x} \psi$ klassisch, dann~$\varphi^\circ \seq{\vec
x} \psi^\circ$ intuitionistisch.

\emph{Bemerkung:} Du kannst sogar zeigen, dass~$\varphi^\circ \seq{\vec x}
\psi^\circ$ in \emph{minimaler Logik} gilt, das ist intuitionistische Logik
ohne das Prinzip \emph{ex falso quodlibet} ($\bot \seq{\vec x} \chi$).
\end{enumerate}
\end{aufgabe}

\begin{aufgabe}{Beweisbäume}
Finde für folgende Sequenzen formale Ableitungsbäume:
\begin{enumerate}
\item $(\varphi \Rightarrow \psi) \seq{\vec x} ((\psi \Rightarrow \chi)
\Rightarrow (\varphi \Rightarrow \chi))$
\item $(\exists y\?Y{:}\ \varphi) \seq{\vec x,z} \varphi[z/y]$
\item $(x = y) \seq{x,y} (y = x)$
\end{enumerate}
\end{aufgabe}

\begin{aufgabe}{Minimumsprinzip}
Sei~$(a_n)_{n \geq 0}$ eine Folge natürlicher Zahlen. Wir wollen die Aussage
\[ A :\equiv \exists n\?\NN{:}\ \forall m\?\NN{:}\ a_n \leq a_m \]
betrachten, die besagt, dass die Folge ein Minimum annimmt.

\begin{enumerate}
\item Zeige konstruktiv, dass~$\neg\neg A$.
\item Formuliere deinen Beweis als Streitgespräch für~$\neg\neg A$ (ohne
Zeitsprünge).
\item Formulieren deinen Beweis als Streitgespräch für~$A$ (notwendigermaßen
mit Zeitsprüngen).
\end{enumerate}

\emph{Bemerkung:} Man kann nicht erwarten,
konstruktiv~$A$ zeigen zu können. Die Aussage, dass das für alle Folgen doch
ginge, ist übrigens ein klassisches Prinzip, das aus dem Prinzip vom
ausgeschlossenen Dritten folgt, aber echt schwächer ist.

\end{aufgabe}

\end{document}
