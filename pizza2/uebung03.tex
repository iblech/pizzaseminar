\documentclass{pizzablatt}

\begin{document}

\maketitle{3}{4. September 2013}

\begin{aufgabe}{Schranken für die Größe der~$n$-ten Primzahl}
Folgender Beweis der Unendlichkeit der Primzahlen wird Euklid zugeschrieben:

\begin{quote}
Angenommen, $p_1, \ldots, p_r$ seien alle Primzahlen. Wir setzen~$N :=
p_1 \cdots p_r + 1$. Nach dem Fundamentalsatz der Arithmetik lässt sich~$N$ in
Primfaktoren zerlegen. Das ist ein Widerspruch, denn die~$p_i$ sind keine
Teiler von~$N$, andere Primzahlen gibt es aber nach Widersprungsvoraussetzung
nicht.
\end{quote}

\begin{enumerate}
\item Formuliere den Beweis so um, dass er konstruktiv folgende stärkere
Aussage zeigt: \emph{Seien~$p_1,\ldots,p_r$ gegebene Primzahlen. Dann gibt es eine
weitere Primzahl ungleich den~$p_i$.}

\item Sei nun~$p_1,p_2,\ldots$ die aufsteigende Folge aller Primzahlen. Extrahiere
aus deinem Beweis die Abschätzung
\[ p_{n+1} \leq p_1 \cdots p_n + 1. \]

\item Zeige folgende Schranke für die Größe der~$n$-ten Primzahl:
\[ p_n \leq 2^{2^{n-1}}. \]

\item Tatsächlich ist diese Schranke sehr pessimistisch. Extrahiere aus Eulers
Alternativbeweis der Unendlichkeit der Primzahlen eine
bessere! (Details kommen noch!)
\end{enumerate}
\end{aufgabe}

\end{document}
