\documentclass{pizzablatt}

\begin{document}

\maketitle{2}{Pizzaseminar zu konstruktiver Mathematik}{21. August 2013}

\begin{aufgabe}{Doppelnegationsübersetzung}
Beweise die fundamentalen Eigenschaften der Doppelnegationsübersetzung, jeweils für alle
Aussagen~$\varphi$ und~$\psi$ in beliebigen Kontexten~$\vec x$.
\begin{enumerate}
\item Klassisch gilt: $\varphi \Longleftrightarrow \varphi^\circ$.
\item Intuitionistisch gilt: $\neg\neg\varphi^\circ \Longrightarrow
\varphi^\circ$.
\item Wenn~$\varphi \seq{\vec x} \psi$ klassisch, dann~$\varphi^\circ \seq{\vec
x} \psi^\circ$ intuitionistisch.

\emph{Bemerkung:} Du kannst sogar zeigen, dass~$\varphi^\circ \seq{\vec x}
\psi^\circ$ in \emph{minimaler Logik} gilt, das ist intuitionistische Logik
ohne das Prinzip \emph{ex falso quodlibet} ($\bot \seq{\vec x} \chi$).
\end{enumerate}
\end{aufgabe}

\begin{aufgabe}{Beweisbäume}
Finde für folgende Sequenzen formale Ableitungsbäume:
\begin{enumerate}
\item $(\varphi \Rightarrow \psi) \seq{\vec x} ((\psi \Rightarrow \chi)
\Rightarrow (\varphi \Rightarrow \chi))$
\item $(\exists y\?Y{:}\ \varphi) \seq{\vec x,z} \varphi[z/y]$
\item $(x = y) \seq{x,y} (y = x)$
\end{enumerate}
\end{aufgabe}

\begin{aufgabe}{Minimumsprinzip}
Sei~$(a_n)_{n \geq 0}$ eine Folge natürlicher Zahlen. Wir wollen die Aussage
\[ A :\equiv (\exists n\?\NN{:}\ \forall m\?\NN{:}\ a_n \leq a_m) \]
betrachten, die besagt, dass die Folge ein Minimum annimmt.

\begin{enumerate}
\item Zeige konstruktiv, dass~$\neg\neg A$.
\item Formuliere deinen Beweis als Streitgespräch für~$\neg\neg A$ (ohne
Zeitsprünge).
\item Formulieren deinen Beweis als Streitgespräch für~$A$ (notwendigermaßen
mit Zeitsprüngen).
\item Welcher Algorithmus zur Minimumssuche ist in deinem Beweis versteckt?
\end{enumerate}

\emph{Bemerkung:} Man kann nicht erwarten,
konstruktiv~$A$ zeigen zu können. Die Aussage, dass das für alle Folgen doch
ginge, ist ein klassisches Prinzip, das übrigens aus dem Prinzip vom
ausgeschlossenen Dritten folgt, aber echt schwächer ist.
\end{aufgabe}

\begin{aufgabe}{Wissen im Grenzwert}
Experimentell kann man das Minimum einer Folge natürlicher Zahlen wie folgt
bestimmen: Man läuft die Folgeglieder sequenziell ab. Stößt man dabei auf einen Wert,
der kleiner als alle vorherigen Werte ist, proklamiert man diesen als das
Minimum der Folge (und entschuldigt sich ggf. für vorherige falsche
Ankündigungen). Klassische Metalogik vorausgesetzt, wird man \emph{irgendwann}
auf das wahre Minimum stoßen und dann seine Aussage nie wieder ändern;
\emph{wann} der richtige Minimalwert gefunden wurde, kann man jedoch selbst
nicht sagen.

Finde Alltagsbeispiele für dieses lerntheoretische Phänomen, also für
Situationen, in denen man wiederholt Vermutungen abgibt und tatsächlich
schlussendlich den richtigen Sachverhalt lernt, jedoch selbst nie sagen kann,
ab wann die eigene Vermutung korrekt ist.
\end{aufgabe}

\end{document}
