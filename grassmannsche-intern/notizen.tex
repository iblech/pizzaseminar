\documentclass[a4paper,ngerman,12pt]{scrartcl}

\usepackage[utf8]{inputenc}

\usepackage[ngerman]{babel}

\usepackage{amsmath,amsthm,amssymb,stmaryrd,color,graphicx,mathtools}
\usepackage{array}

\usepackage[protrusion=true,expansion=true]{microtype}

\usepackage{hyperref}

\theoremstyle{definition}
\newtheorem{defn}{Definition}
%\newtheorem{defn}{Definition}[section]
\newtheorem{ex}[defn]{Beispiel}
\newtheorem{motto}[defn]{Motto}

\theoremstyle{plain}

\newtheorem{defnprop}[defn]{Definition/Proposition}
\newtheorem{prop}[defn]{Proposition}
\newtheorem{fact}[defn]{Fakt}
\newtheorem{lemma}[defn]{Lemma}
\newtheorem{thm}[defn]{Satz}
\newtheorem{cor}[defn]{Korollar}

\theoremstyle{remark}
\newtheorem{rem}[defn]{Bemerkung}
\newtheorem{warning}[defn]{Warnung}

\clubpenalty=10000
\widowpenalty=10000
\displaywidowpenalty=10000

\renewcommand{\AA}{\mathbb{A}}
\newcommand{\CC}{\mathbb{C}}
\newcommand{\NN}{\mathbb{N}}
\newcommand{\ZZ}{\mathbb{Z}}
\newcommand{\QQ}{\mathbb{Q}}
\newcommand{\FF}{\mathbb{F}}
\newcommand{\PP}{\mathbb{P}}
\newcommand{\C}{\mathcal{C}}
\newcommand{\E}{\mathcal{E}}
\newcommand{\F}{\mathcal{F}}
\newcommand{\G}{\mathcal{G}}
\renewcommand{\H}{\mathcal{H}}
\newcommand{\N}{\mathcal{N}}
\newcommand{\J}{\mathcal{J}}
\newcommand{\K}{\mathcal{K}}
\renewcommand{\L}{\mathcal{L}}
\renewcommand{\O}{\mathcal{O}}
\newcommand{\id}{\mathrm{id}}
\newcommand{\op}{\mathrm{op}}
\newcommand{\ppp}{\mathfrak{p}}
\newcommand{\mmm}{\mathfrak{m}}
\newcommand{\xra}[1]{\xrightarrow{#1}}
\newcommand{\Mod}{\mathrm{Mod}}
\newcommand{\Set}{\mathrm{Set}}
\newcommand{\Cat}{\mathrm{Cat}}
\newcommand{\Vect}{\mathrm{Vect}}
\newcommand{\VB}{\mathrm{VB}}
\newcommand{\Id}{\mathrm{Id}}
\newcommand{\Coh}{\mathrm{Coh}}
\newcommand{\GL}{\mathrm{GL}}
\newcommand{\pt}{\mathrm{pt}}
\newcommand{\Ob}{\operatorname{Ob}}
\newcommand{\rank}{\operatorname{rank}}
\newcommand{\Hom}{\mathrm{Hom}}
\newcommand{\Pic}{\mathrm{Pic}}
\newcommand{\ul}[1]{\underline{#1}}
\newcommand{\placeholder}{\underline{\ \ }}
\newcommand{\lra}{\longrightarrow}
\renewcommand{\div}{\operatorname{div}}
\newcommand{\Div}{\mathrm{Div}}
\newcommand{\ord}{\operatorname{ord}}
\newcommand{\PD}{\operatorname{PD}}
\newcommand{\Sh}{\operatorname{Sh}}
\DeclareMathOperator{\Spec}{Spec}
\DeclareMathOperator{\Proj}{Proj}
\DeclareMathOperator{\Sym}{Sym}
\DeclareMathOperator{\colim}{colim}
\newcommand{\defeq}{\vcentcolon=}
\newcommand{\defeqv}{\vcentcolon\equiv}

\newcommand{\Gr}{\mathrm{Gr}}

\let\raggedsection\centering

\begin{document}

\title{Ein synthetischer Zugang zur Grassmannschen}
\author{}
%\maketitle

\section*{Ein synthetischer Zugang zur Grassmannschen}

Wir möchten zeigen, dass für lokal freie Garben~$\mathcal{V}$ endlichen Rangs
über einem beliebigen Basisschema~$S$ die Grassmannsche~$\Gr(\mathcal{V},r)$
der lokal freien Quotienten von~$\mathcal{V}$ vom Rang~$r$ (definiert als
Funktor wie in Carens Vortrag) durch ein~$S$-Schema von endlichem Typ
darstellbar ist.

Dazu arbeiten wir intern im großen Zariski-Topos von~$S$. Dort sehen
dann~$\O_S$ wie ein gewöhnlicher Ring, die Modulgarbe~$\mathcal{V}$ wie ein
gewöhnlicher freier Modul und der Punktefunktor der Grassmannschen wie eine
gewöhnlichen Menge aus; wir dürfen synthetisch arbeiten und müssen zu keinem
Zeitpunkt topologische Struktur beachten, konstruieren oder mitschleppen. Das
Spektrum einer Algebra wird zur Menge ihrer rationalen Punkte. Der Preis für
diese Vorteile ist, dass wir konstruktiv arbeiten müssen.

Sei also~$V$ ein freier~$k$-Modul vom Rang~$n$. Wir werden von~$k$
nur voraussetzen, dass er ein lokaler Ring ist (obwohl er in der Anwendung
sogar ein Körper ist).

\begin{defn}Die \emph{Grassmannsche der Quotienten vom Rang~$r$ von~$V$} ist
die Menge
\[ \Gr(V,r) \defeq \{ \text{$U \subseteq V$ Untermodul} \,|\, \text{$V/U$ frei
vom Rang~$r$} \}. \]
\end{defn}

Ferner definieren wir für jeden freien Untermodul~$W \subseteq V$ vom Rang~$r$
die Teilmenge
\[ G_W \defeq \{ U \in \Gr(V,r) \,|\, \text{$W \to V \to V/U$ ist bijektiv} \}. \]
Diese Menge besitzt noch eine konkretere Beschreibung, denn sie steht in
kanonischer Bijektion zur Menge
\[ G_W' \defeq \{ \pi : V \to W \,|\, \pi \circ \iota = \id \} \]
aller Zerfällungen der Inklusion~$\iota : W \hookrightarrow V$: Ein Element~$U
\in G_W$ können wir auf die Surjektion~$V \twoheadrightarrow V/U
\xrightarrow{({\cong})^{-1}} W$ schicken. Umgekehrt können wir eine
Zerfällung~$\pi$ auf~$U \defeq \ker(\pi)$ schicken.

\begin{prop}Die Vereinigung der~$G_W$ ist ganz~$\Gr(V,r)$.\end{prop}

\begin{proof}Sei~$U \in \Gr(V,r)$ beliebig. Dann gibt es eine
Basis~$([v_1],\ldots,[v_r])$ von~$V/U$. Die Familie~$(v_1,\ldots,v_r)$ ist
in~$V$ linear unabhängig, daher ist der Untermodul~$W \defeq
\operatorname{span}(v_1,\ldots,v_r) \subseteq V$ frei vom Rang~$r$. Die
kanonische lineare Abbildung~$W \hookrightarrow V \twoheadrightarrow V/U$ bildet die
Basis~$(v_i)_i$ auf die Basis~$([v_i])_i$ ab und ist daher bijektiv. Somit
liegt~$U$ in~$G_W$.\end{proof}

\begin{prop}Die~$G_W$ sind (quasikompakt-)offene Teilmengen von~$\Gr(V,r)$, in
folgendem Sinn: Für jedes Element~$U$ aus~$\Gr(V,r)$ gibt es
Zahlen~$f_1,\ldots,f_n$ aus~$k$ sodass~$U$ genau dann in~$G_W$ liegt, wenn
eine dieser Zahlen invertierbar ist.
\end{prop}

\begin{proof}Sei~$U \in \Gr(V,r)$ beliebig. Genau dann liegt~$U$ in~$G_W$, wenn
die kanonische lineare Abbildung~$W \hookrightarrow V \twoheadrightarrow V/U$
bijektiv ist. Da~$W$ und~$V/U$ beide frei vom Rang~$r$ sind, ist diese
Abbildung durch eine~$(r \times r)$-Matrix~$M$ über~$k$ gegeben und daher genau
dann bijektiv, wenn die Determinante dieser Matrix invertierbar ist. Wir können
also~$n \defeq 1$ und~$f_1 \defeq \det(M)$ setzen.
\end{proof}

\begin{prop}Die $G_W$ sind affin in dem Sinn, dass es eine~$k$-Algebra~$A$ gibt,
sodass~$G_W$ in Bijektion zu den~$k$-Algebrenhomomorphismen~$A \to k$
steht:
\[ G_W \cong \Hom_k(A, k) =\vcentcolon \Spec(A). \]
Im Beweis werden wir sehen, dass wir~$A$ sogar als endlich präsentiert wählen
können.\end{prop}

\begin{proof}Die Menge aller linearen Abbildungen~$V \to W$ ist das Spektrum
von~$R \defeq \Sym(\Hom_k(V,W)^\vee)$, denn
\begin{align*}
  \Spec(R) &=
  \Hom_{\mathrm{Alg}(k)}(\Sym(\Hom_{\mathrm{Mod}(k)}(V,W)^\vee), k) \\
  &\cong \Hom_{\mathrm{Mod}(k)}(\Hom_{\mathrm{Mod}(k)}(V,W)^\vee, k) \\
  &=\Hom_{\mathrm{Mod}(k)}(V,W)^{\vee\vee} \\
  &\cong \Hom_{\mathrm{Mod}(k)}(V,W).
\end{align*}
Im letzten Schritt geht ein, dass nicht nur~$W$, sondern auch~$V$ ein freier Modul
endlichen Rangs ist. Es ist das erste Mal, dass wir diese Voraussetzung
für~$V$ benötigen.

Die Menge~$G_W'$ ist eine abgeschlossene Teilmenge dieses Spektrums, nämlich
der Ort, wo die generische lineare Abbildung~$V \to W$ eine Zerfällung der
Inklusion~$\iota : W \hookrightarrow V$ ist. Weg mag, kann Basen von~$V$
und~$W$ wählen: Dann ist~$\Sym(\Hom_k(V,W)^\vee)$ isomorph
zu~$k[M_{11},\ldots,M_{rn}]$ und~$G_W'$ zu
\[ \Spec(k[M_{11},\ldots,M_{rn}]/(MN-I)). \]
Dabei ist~$I$ die~$(r \times r)$-Einheitsmatrix, $M$ die generische Matrix~$M =
(M_{ij})_{ij}$ und~$N$ die Darstellungsmatrix von~$\iota$ bezüglich der
gewählten Basen. Mit~$(MN-I)$ ist das von den Einträgen von~$MN-I$ erzeugte
Ideal gemeint.
\end{proof}

\begin{cor}Die Grassmannsche~$\Gr(V,r)$ ist ein Schema von endlichem
Typ.\end{cor}

\begin{proof}Die Grassmannsche~$\Gr(V,r)$ besitzt eine offene
Überdeckung durch die affinen Schemata~$G_W$ und ist daher ein Schema.

Wenn wir einen Isomorphismus~$V \cong k^n$ wählen, sehen wir, dass
schon~$\binom{n}{r}$ viele dieser affinen Schemata genügen: nämlich diese,
wo~$W$ einer der Standarduntermoduln von~$k^n$ ist (erzeugt durch
Einheitsvektoren).

Denn ist~$U \in \Gr(k^n,r)$, so bildet die Surjektion~$V \to V/U$ die Basis von
mindestens einem dieser Standarduntermoduln auf eine Basis ab und ist daher
bijektiv. (Aus einer surjektiven~$(r \times n)$-Matrix über einem lokalen Ring
kann man stets~$r$ Spalten auswählen, die eine linear unabhängige Familie
bilden.)
\end{proof}

\begin{prop}Für den Tangentialraum an~$U \in \Gr(V,r)$ gilt:
$T_U \Gr(V,r) \cong \Hom(U, V/U)$.\end{prop}

\begin{proof}Die Menge der Tangentialvektoren an~$U$ kann kanonisch mit den
Abbildungen~$\gamma : \Delta \to \Gr(V,r)$ mit~$\gamma(0) = U$ identifiziert
werden. Dabei ist~$\Delta \defeq \{ \varepsilon \in k \,|\, \varepsilon^2 = 0 \}$.
Eine solche Abbildung liftet stets zu einer Abbildung von~$\Delta$ in die Menge
der linear unabhängigen Familien der Länge~$r$ in~$V$. Der Rest sei als
Übungsaufgabe überlassen. Willkommen in der wunderbaren Welt synthetischer
Geometrie.
\end{proof}

\enlargethispage{2em}

\begin{rem}Wiederholt man genau dieselben Argumente in einem anderen Topos --
einem, der für Differentialgeometrie angepasst ist -- erhält man mehr oder
weniger die Darstellbarkeit der Grassmannschen als Mannigfaltigkeit. Das
einzige, was fehlt, ist ein Nachweis der Glattheit.
\end{rem}

\end{document}
