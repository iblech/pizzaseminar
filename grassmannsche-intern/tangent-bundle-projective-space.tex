\documentclass[a4paper,ngerman,12pt]{scrartcl}

\usepackage[utf8]{inputenc}

\usepackage[english]{babel}

\usepackage{amsmath,amsthm,amssymb,stmaryrd,color,graphicx,mathtools}
\usepackage{array}

\usepackage[protrusion=true,expansion=true]{microtype}

\usepackage{hyperref}

\theoremstyle{definition}
\newtheorem{defn}{Definition}
\newtheorem{ex}[defn]{Example}
\theoremstyle{plain}
\newtheorem{thm}[defn]{Theorem}

\clubpenalty=10000
\widowpenalty=10000
\displaywidowpenalty=10000

\renewcommand{\AA}{\mathbb{A}}
\newcommand{\CC}{\mathbb{C}}
\newcommand{\NN}{\mathbb{N}}
\newcommand{\ZZ}{\mathbb{Z}}
\newcommand{\QQ}{\mathbb{Q}}
\newcommand{\RR}{\mathbb{R}}
\newcommand{\FF}{\mathbb{F}}
\newcommand{\PP}{\mathbb{P}}
\newcommand{\C}{\mathcal{C}}
\newcommand{\E}{\mathcal{E}}
\newcommand{\F}{\mathcal{F}}
\newcommand{\G}{\mathcal{G}}
\renewcommand{\H}{\mathcal{H}}
\newcommand{\N}{\mathcal{N}}
\newcommand{\J}{\mathcal{J}}
\newcommand{\K}{\mathcal{K}}
\renewcommand{\L}{\mathcal{L}}
\renewcommand{\O}{\mathcal{O}}
\newcommand{\id}{\mathrm{id}}
\newcommand{\op}{\mathrm{op}}
\newcommand{\ppp}{\mathfrak{p}}
\newcommand{\mmm}{\mathfrak{m}}
\newcommand{\xra}[1]{\xrightarrow{#1}}
\newcommand{\Mod}{\mathrm{Mod}}
\newcommand{\Set}{\mathrm{Set}}
\newcommand{\Cat}{\mathrm{Cat}}
\newcommand{\Vect}{\mathrm{Vect}}
\newcommand{\VB}{\mathrm{VB}}
\newcommand{\Id}{\mathrm{Id}}
\newcommand{\Coh}{\mathrm{Coh}}
\newcommand{\GL}{\mathrm{GL}}
\newcommand{\pt}{\mathrm{pt}}
\newcommand{\Ob}{\operatorname{Ob}}
\newcommand{\rank}{\operatorname{rank}}
\newcommand{\Hom}{\mathrm{Hom}}
\newcommand{\Pic}{\mathrm{Pic}}
\newcommand{\ul}[1]{\underline{#1}}
\newcommand{\placeholder}{\underline{\ \ }}
\newcommand{\lra}{\longrightarrow}
\renewcommand{\div}{\operatorname{div}}
\newcommand{\Div}{\mathrm{Div}}
\newcommand{\ord}{\operatorname{ord}}
\newcommand{\PD}{\operatorname{PD}}
\newcommand{\Sh}{\operatorname{Sh}}
\DeclareMathOperator{\Span}{span}
\newcommand{\defeq}{\vcentcolon=}
\newcommand{\defeqv}{\vcentcolon\equiv}

\newcommand{\Gr}{\mathrm{Gr}}

\let\raggedsection\centering

\begin{document}

\title{The tangent bundle of the projective space}
\author{}
%\maketitle

\section*{The tangent bundle of the projective space}

The aim of this short note is to calculate the tangent bundle of the projective
space in the setting of synthetic differential geometry.

\begin{center}
  \setlength{\fboxrule}{1pt}
  \setlength{\fboxsep}{8pt}
  \fbox{\parbox{0.7\textwidth}{
    \textbf{Axiom of microaffinity.} For any function~$f : \Delta \to \RR$,
    where~$\Delta = \{ \varepsilon \in \RR \,|\, \varepsilon^2 = 0 \}$, there
    exists a unique pair~$(a,b)$ of real numbers such that
    \[ f(\varepsilon) = a + b \varepsilon \]
    for all~$\varepsilon \in \Delta$.
  }}
\end{center}

\begin{ex}For all~$\varepsilon \in \Delta$, the number~$1 + \varepsilon$ is
invertible with inverse given by
\[ \frac{1}{1 + \varepsilon} = \frac{1 - \varepsilon}{(1 + \varepsilon) (1 -
\varepsilon)} = \frac{1 - \varepsilon}{1 - \varepsilon^2} = 1 - \varepsilon. \]
\end{ex}

\begin{defn}The \emph{derivative}~$f'(x)$ of a set-theoretical function~$f :
\RR \to \RR$ at a point~$x$ is the unique number~$b$ such that
\[ f(x + \varepsilon) = f(x) + b \varepsilon \]
for all~$\varepsilon \in \Delta$.\end{defn}

\begin{ex}Let~$f(x) = x^3$. Then~$f'(x) = 3x^2$, since
\[ f(x + \varepsilon) = (x + \varepsilon)^3 =
  x^3 + 3x^2\varepsilon + 3x\varepsilon^2 + \varepsilon^3 =
  x^3 + 3x^2\varepsilon \]
for all~$\varepsilon \in \Delta$.\end{ex}

\begin{defn}The \emph{projective space}~$\PP V$ associated to a vector
space~$V$ is the set
\[ \PP V = \{ \ell \subseteq V \,|\, \text{$\ell$ is a one-dimensional subspace
of~$V$} \}. \]\end{defn}

\begin{defn}The \emph{tangent bundle}~$TX$ of a set~$X$ is the set
\[ TX = X^\Delta = \{ \gamma : \Delta \to X \} \]
of all set-theoretical maps~$\Delta \to X$.\end{defn}
\enlargethispage{4em}\thispagestyle{empty}

\begin{thm}Let~$V$ be a vector space. Then there is a canonical isomorphism
(bijection)
\[ T(\PP V) \cong \coprod_{\ell \in \PP V} \Hom_\RR(\ell,V/\ell). \]
\end{thm}

The symbol~``$\coprod$'' for the disjoint union might seem dubious on first
sight: Aren't the fibers of the tangent bundle supposed to have some cohesion
and fit continuously together? In synthetic differential geometry, all sets
have automatically a continuous/geometric flavour. The cohesion present in the
index set over which the disjoint union is taken over, the set~$\PP V$, induces
a kind of cohesion in the disjoint union.

A simpler example is given by the tangent bundle of~$\RR^n$: This admits the
description
\[ T\RR^n \cong \coprod_{x \in \RR^n} \RR^n \cong \RR^n \times \RR^n. \]
The description using the disjoint union might feel weird while the description
using the cartesian product seems right, but there is in fact a canonical
bijection between these sets, sending an element~$\langle x,v \rangle \in
\coprod_{x \in \RR^n} \RR^n$ to~$(x,v) \in \RR^n \times \RR^n$.

The point of synthetic differential geometry is, however, that one doesn't need
to make this idea of ``cohesion'' explicit (unlike in the standard approach,
where one defines topological spaces and manifolds to capture it). The sets
themselves already exhibit the smooth behaviour we're interested in.

\begin{proof}[Proof of Theorem~6]Let~$\gamma : \Delta \to \PP V$ be a tangent
vector with base point~$\ell \defeq \gamma(0) \in \PP V$. There's a
lift~$\overline{\gamma} : \Delta \to V$ such that~$\gamma(\varepsilon) =
\Span(\overline{\gamma}(\varepsilon))$ for all~$\varepsilon \in \Delta$.
We define a linear map~$\alpha : \ell \to V/\ell$ by setting
\[ x \longmapsto \alpha(x) = [x/\overline{\gamma}(0) \cdot
\overline{\gamma}'(0)]. \]
The expression ``$x/\overline{\gamma}(0)$'' should be read as follows: The
vector~$x$, being an element of~$\ell$, is some multiple~$\lambda$
of~$\overline{\gamma}(0)$. The expression ``$x/\overline{\gamma}(0)$'' denotes
this unique number~$\lambda$. It can be checked that the vector~$\alpha(x) \in
V/\ell$ does not depend on the choice of the lifting~$\overline{\gamma}$. The
element~$\langle\ell, \alpha\rangle$ is therefore a well-defined element
of~$\coprod_{\ell \in \PP V} \Hom_\RR(\ell,V/\ell)$.

Conversely, let an element~$\langle\ell, \alpha\rangle \in \coprod_{\ell \in
\PP V} \Hom_\RR(\ell,V/\ell)$ be given. We choose vectors~$x_0 \in V$ and~$z \in V$ such
that~$\ell = \Span(x_0)$ and~$\alpha(x_0) = [z]$ and define~$\gamma : \Delta
\to \PP V$ by setting
\[ \varepsilon \longmapsto \gamma(\varepsilon) = \Span(x_0 + \varepsilon z). \]
The definition of~$\gamma(\varepsilon)$ is invariant under scaling of~$x_0$ and
also under changing~$z$ to some other vector~$z + \lambda x_0$ in its
equivalence class:
\begin{align*}
  \Span(x_0 + \varepsilon (z + \lambda x_0))
  &= \Span((1 + \varepsilon \lambda) x_0 + \varepsilon z) \\
  &= \Span(x_0 + \varepsilon / (1 + \varepsilon \lambda) z) \\
  &= \Span(x_0 + \varepsilon z),
\end{align*}
since~$\varepsilon / (1 + \varepsilon \lambda) = \varepsilon (1 - \varepsilon
\lambda) = \varepsilon$. Therefore~$\gamma$ is a well-defined element of~$T(\PP
V)$ which only depends on~$\langle\ell, \alpha\rangle$ and not on the arbitrary
choices of~$x_0$ and~$z$.

One can check that the two described constructions are mutually inverse.
\end{proof}

At no point in the proof one has to check that certain maps are continuous or
smooth. The proof can be easily adapted to verify that the tangent bundle of
the Grassmannian of~$r$-dimensional subspaces admits the description
\[ T(\mathrm{Gr}_r V) \cong \coprod_{U \in \mathrm{Gr}_r V} \Hom_\RR(U, V/U). \]

\end{document}
