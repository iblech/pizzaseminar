\documentclass{pizzablatt}

\begin{document}

\maketitle{7}{4. April 2013}

\begin{aufgabe}{Auf- und Abrundung}
Wir betrachten die drei monotonen Abbildungen
\[ \begin{array}{@{}rrclrcl@{}}
  \lceil\freist\rceil: & \QQ &\longrightarrow& \ZZ, &
  x &\longmapsto& \text{Aufrundung von~$x$ := (kleinste ganze Zahl~$\geq x$)} \\
  i: & \ZZ &\longrightarrow& \QQ, &
  z &\longmapsto& z \\
  \lfloor\freist\rfloor: & \QQ &\longrightarrow& \ZZ, &
  x &\longmapsto& \text{Abrundung von~$x$ := (größte ganze Zahl~$\leq x$)}
\end{array} \]
und ihre gemäß Aufgabe~2 von Übungsblatt~3 induzierten Funktoren
\[ \xymatrixcolsep{7pc}\xymatrix{
  B\QQ
    \ar@/^1.4pc/[r]^{B\lceil\freist\rceil}
    \ar@/_1.4pc/[r]_{B\lfloor\freist\rfloor}
  & B\ZZ. \ar[l]^{Bi}
} \]
\begin{enumerate}
\item Mache dir klar, dass für alle~$x \in \QQ$ und~$y \in \ZZ$ gilt:
\[ \lceil x \rceil \leq y \quad\Longleftrightarrow\quad
  x \leq i(y). \]
Welche analoge Beziehung gilt zwischen~$i$ und~$\lfloor\freist\rfloor$?
\item Zeige: $B\lceil\freist\rceil \dashv Bi \dashv B\lfloor\freist\rfloor$.
\end{enumerate}
\end{aufgabe}

\begin{aufgabe}{Freie Monoide}
Ein \emph{Monoid} besteht aus einer Menge~$M$, einer assoziativen
zweistelligen Verknüpfung~$\circ$ und einem neutralen Element~$e$; Monoiden
darf es also anders als Gruppen an Inversen fehlen. Monoidhomomorphismen müssen
die Verknüpfung und das neutrale Element bewahren.
\begin{enumerate}
\item
Sei~$X$ eine Menge. Ein \emph{Wort} über~$X$ ist eine endliche Folge von
Elementen aus~$X$. Wir haben keine Angst vor dem leeren Wort. Wie wird die
Menge~$F(X)$ der Wörter über~$X$ zu einem Monoid?
\item Ergänze die Konstruktion aus~a) zu einem Funktor der Kategorie der
Mengen in die Kategorie der Monoide.
\item Zeige, dass der so gebastelte Funktor linksadjungiert zum Vergissfunktor
ist.
\end{enumerate}
\end{aufgabe}

\begin{aufgabe}{Freie Körper?}
Ein Körperhomomorphismus muss Addition und Multiplikation sowie Null- und
Einselement bewahren.
\begin{enumerate}
\item Zeige: Die Kategorie der Körper besitzt kein initiales Objekt.
\item Folgere: Der Vergissfunktor von der Kategorie der Körper in die Kategorie
der Mengen besitzt keinen Linksadjungierten. Freie Körper gibt es also nicht.
\end{enumerate}
\end{aufgabe}

\end{document}
