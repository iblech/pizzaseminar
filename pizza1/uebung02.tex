\documentclass[a4paper,ngerman]{scrartcl}

%\usepackage{ucs}
\usepackage[utf8]{inputenc}

\usepackage[ngerman]{babel}

\usepackage{amsmath,amsthm,amssymb,amscd,color,graphicx}

%\usepackage[small,nohug]{diagrams}
%\diagramstyle[labelstyle=\scriptstyle]

\usepackage[protrusion=true,expansion=true]{microtype}

\usepackage{lmodern}
\usepackage{tabto}

\usepackage[natbib=true,style=numeric]{biblatex}
\usepackage[babel]{csquotes}
\bibliography{lit}

\usepackage[all]{xy}

%\usepackage{hyperref}

\setlength\parskip{\medskipamount}
\setlength\parindent{0pt}

\theoremstyle{definition}
\newtheorem{defn}{Definition}
\newtheorem{bsp}[defn]{Beispiel}

\theoremstyle{plain}

\newtheorem{prop}[defn]{Proposition}
\newtheorem{ueberlegung}[defn]{Überlegung}
\newtheorem{lemma}[defn]{Lemma}
\newtheorem{kor}[defn]{Korollar}
\newtheorem{hilfsaussage}[defn]{Hilfsaussage}
\newtheorem{satz}[defn]{Satz}

\theoremstyle{remark}
\newtheorem{bem}[defn]{Bemerkung}

\clubpenalty=10000
\widowpenalty=10000
\displaywidowpenalty=10000

\newcommand{\lra}{\longrightarrow}
\newcommand{\lhra}{\ensuremath{\lhook\joinrel\relbar\joinrel\rightarrow}}
\newcommand{\thlra}{\relbar\joinrel\twoheadrightarrow}

\newcommand{\A}{\mathcal{A}}
\newcommand{\Z}{\mathbb{Z}}
\newcommand{\Q}{\mathbb{Q}}
\newcommand{\R}{\mathbb{R}}
\newcommand{\C}{\mathcal{C}}
\newcommand{\RP}{\mathbb{R}\mathrm{P}}
\newcommand{\Hom}{\mathrm{Hom}}
\newcommand{\Set}{\mathrm{Set}}
\newcommand{\Spur}[1]{\operatorname{Spur}#1}
\newcommand{\rank}[1]{\operatorname{rank}#1}
\newcommand{\sgn}[1]{\operatorname{sgn}#1}
\newcommand{\id}{\mathrm{id}}
\newcommand{\Aut}[1]{\operatorname{Aut}(#1)}
\newcommand{\GL}[1]{\operatorname{GL}(#1)}
\newcommand{\ORTH}[1]{\operatorname{O}(#1)}
\newcommand{\freist}{\underline{\ \ }}
\newcommand{\op}{\mathrm{op}}
\DeclareMathOperator{\rk}{rk}
\DeclareMathOperator{\Spec}{Spec}
\DeclareMathOperator{\Bild}{im}
\DeclareMathOperator{\Kern}{ker}
\DeclareMathOperator{\Int}{int}
\DeclareMathOperator{\Ob}{Ob}
\newcommand{\Zzwei}{\Z_2}

\newcommand{\XXX}[1]{\textcolor{red}{#1}}

\renewcommand*\theenumi{\alph{enumi}}
\renewcommand{\labelenumi}{\theenumi)}

\pagestyle{empty}

%\newarrow{Equals}=====

\usepackage{geometry}
\geometry{tmargin=2cm,bmargin=2cm,lmargin=3cm,rmargin=3cm}

\begin{document}

\vspace*{-4em}
\begin{flushright}Universität Augsburg \\ 6. März 2013\end{flushright}

\begin{center}\Large \textbf{Pizzaseminar zur Kategorientheorie} \\
2. Übungsblatt
\end{center}
\vspace{1.5em}

\newbox{\mybox}
\setbox\mybox=\hbox{\textbf{Aufgabe 1:}}

\begin{list}{}{\labelwidth0em \leftmargin0em \itemindent0.5em \itemsep 1.3em}
\item[\textbf{Aufgabe 1:}]
Sei~$X$ ein Objekt einer Kategorie~$\C$.
\begin{enumerate}
\item Zeige: Besitzt~$\C$ ein terminales Objekt~$1$, so gilt
\[ X \times 1 \cong X. \]
Diese Aussage ist nicht wörtlich zu verstehen: Genauer ist zu zeigen, dass~$X$
(mit welchen Morphismen?) als Produkt von~$X$ und~$1$ dienen kann.
\item Was ist die duale Aussage zu a)?
\end{enumerate}

\item[\textbf{Aufgabe 2:}]
Seien $X$ und $Y$ Objekte einer Kategorie~$\C$. Wir definieren folgende
Kategorie der \emph{Möchtegern-Produkte} von~$X$ und~$Y$:
\begin{align*}
  \text{Objekte: } & \text{Diagramme der Form $X \leftarrow Q \to Y$ in~$\C$} \\
  \text{Morphismen: } &
    \Hom(X \leftarrow Q \to Y, X \leftarrow R \to Y) := \\ & \left\{
      \text{(kommutative) Diagramme der Form $\vcenter{\xymatrix{
        & Q \ar[ld] \ar[dd] \ar[rd] \\
      X & & Y \\
        & R \ar[lu] \ar[ru]
      }}$} \right\}
\end{align*}
\begin{enumerate}
\item Zeige: Terminale Objekte beliebiger Kategorien sind "`eindeutig bis auf
eindeutige Isomorphie"', d.\,h. zwischen je zwei terminalen Objekten einer
Kategorie existiert genau ein Isomorphismus.
\item Mache dir klar: Die Angabe eines Produkts von~$X$ und~$Y$ in~$\C$ ist
gleichwertig zur Angabe eines terminalen Objekts in der Kategorie der
Möchtegern-Produkte von~$X$ und~$Y$. Was folgt daher in Kombination mit
Teilaufgabe~a)?
\end{enumerate}
% * X x 1 existiert, ist gleich X.
%   Was folgt fürs Koprodukt?
% * Kategorie der Möchtegernprodukte. Was ist terminales Objekt dadrin?

\item[\textbf{Aufgabe 3:}]
Eine \emph{Quasiordnung} besteht aus einer Menge~$X$ und einer reflexiven und
transitiven (aber nicht unbedingt antisymmetrischen) Relation~$\preceq$
auf~$X$. Zum Beispiel bildet die Menge der ganzen Zahlen mit der
Teilbarkeitsrelation eine Quasiordnung.
\begin{enumerate}
\item Bastele auf sinnvolle Art und Weise aus~$X$ eine Kategorie. Weshalb sind
die Ka\-te\-go\-rien\-axi\-ome erfüllt?
\item Wann sind zwei Objekte dieser Kategorie zueinander isomorph?
\item Ein \emph{Infimum} zweier Elemente~$a,b\in X$ ist ein Element~$p \in X$
mit
\[ \forall x \in X{:}\qquad x \preceq a \text{ und } x \preceq b \quad\Longleftrightarrow\quad x \preceq p.
\]
Zeige: Die Angabe eines Infimums von~$a$ und~$b$ ist gleichwertig mit der
Angabe eines Produkts von~$a$ und~$b$ in dieser Kategorie.
\end{enumerate}

\item[\textbf{Aufgabe 4:}] Seien~$X$, $Y$ und~$Z$ Objekte einer Kategorie~$\C$.
Existiere ein Produkt~$X \times Y$ von~$X$ und~$Y$ und existiere ein
Produkt~$(X \times Y) \times Z$ von~$X \times Y$ und~$Z$.
\begin{enumerate}
\item Zeige: Dann existiert auch ein Dreier-Produkt~$X \times Y \times Z$.
\item Welches Assoziativ- und Kommutativgesetz folgt damit?
\end{enumerate}
\end{list}

\end{document}
