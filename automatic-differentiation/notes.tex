\documentclass[a4paper,ngerman,12pt]{scrartcl}

\usepackage[utf8]{inputenc}

\usepackage[ngerman]{babel}

\usepackage{amsmath,amsthm,amssymb,stmaryrd,color,graphicx,mathtools}
\usepackage{array}
\usepackage[all]{xy}

\usepackage{shadethm}

\usepackage[protrusion=true,expansion=true]{microtype}

\usepackage[T1]{fontenc}
\usepackage{libertine}

\usepackage{hyperref}

\setlength{\shadeboxsep}{6pt}
\setlength{\shadeleftshift}{-\shadeboxsep}
\setlength{\shaderightshift}{-\shadeboxsep}

\theoremstyle{definition}
\newtheorem{defn}{Definition}[section]
%\newtheorem{defn}{Definition}[section]
\newtheorem{ex}[defn]{Beispiel}
\newtheorem{motto}[defn]{Motto}

\theoremstyle{plain}

\newtheorem{defnprop}[defn]{Definition/Proposition}
\newtheorem{prop}[defn]{Proposition}
\newtheorem{fact}[defn]{Fakt}
\newtheorem{lemma}[defn]{Lemma}
\newtheorem{thm}[defn]{Satz}
\newtheorem{cor}[defn]{Korollar}

\theoremstyle{remark}
\newtheorem{rem}[defn]{Bemerkung}
\newtheorem{warning}[defn]{Warnung}

\clubpenalty=10000
\widowpenalty=10000
\displaywidowpenalty=10000

\newcommand{\RR}{\mathbb{R}}
\newcommand{\CC}{\mathbb{C}}
\newcommand{\defeq}{\vcentcolon=}

\begin{document}

\title{Automatic Differentiation}
\author{Ingo Blechschmidt}
\date{July 9th, 2015}
\maketitle

\section{What automatic differentiation achives}

Given code which implements a function~$x \mapsto f(x)$, automatic
differentation gives us code which implements its derivative~$x \mapsto
f'(x)$. The code obtained this way is exactly the same as if we'd worked out
the derivative on paper.

The input code to automatic differentiation may contain arbitrary control
structures such as \emph{if} conditionals or \emph{while} loops. Therefore it
is applicable to a wide range of problems:

\begin{itemize}
\item The basic use case is to find out the derivative of a function.
\item Assume that we're using Newton's method to solve a parameter-dependent
equation~$f(x,\theta) = 0$ for~$x$. We might be interested in how the
solution~$x(\theta)$ depends on~$\theta$. This is trivial with automatic
differentiation, we just feed it our Newton code.
\item Assume that we're solving some parameter-dependent ordinary differential
equation. We are interested in the dependence of the solution (at the final
time, say) on the parameter. For this, we just hand our differential equation
solving code to automatic differentiation.
\end{itemize}

Automatic differentiation is totally unlike to \emph{numerical
differentiation}. With numerical differentiation, we approximate~$f'(x)$ by
some difference quotient like
\[ \frac{f(x + h) - f(x)}{h} \qquad\text{or}\qquad
  \frac{f(x + h) - f(x - h)}{2h}. \]
This approach faces severe problems: If~$h$ is large, the quotient won't be a
good approximation to the true derivative. Instead, it will give the slope of
some unrelated secant. If~$h$ is small, the approximation will be good in
theory. But practically, with floating-point arithmetic, a huge loss of
precision may occur, since we are subtracting two nearly equal numbers.

Automatic differentiation is also unlike to \emph{symbolic differentiation},
which operates on the level of \emph{terms}. Symbolic differentiation is useful
if our goal is to obtain \emph{formulas} for various quantities, but it isn't
particularly suited for efficient evaluation to floating-point numbers.


\section{The basic idea of automatic differentiation}

To grasp the basic idea of automatic differentiation, assume that there exists
a magical number~$\varepsilon$ such that~$\varepsilon^2 = 0$. This
number~$\varepsilon$ should not itself be zero, as else we couldn't extract any
meaningful information from calculations with~$\varepsilon$.

The set of real numbers doesn't contain such a number. Nevertheless, watch:
\begin{align*}
  (x+\varepsilon)^2 &= x^2 + 2x\varepsilon + \varepsilon^2 = x^2 + 2x\varepsilon \\
  (x+\varepsilon)^3 &= x^2 + 3x^2\varepsilon + 3x\varepsilon^2 + \varepsilon^3 = x^3 + 3x^2\varepsilon \\
  \frac{1}{x+\varepsilon} &= \frac{x-\varepsilon}{(x+\varepsilon) \cdot
  (x-\varepsilon)} = \frac{x-\varepsilon}{x^2} = \frac{1}{x} - \frac{1}{x^2}
  \varepsilon
\end{align*}

So it appears that plugging in~$x + \varepsilon$ into a function~$f$ yields
\emph{the derivative~$f'(x)$ along with the function value~$f(x)$}, as the
coefficient of the magical number~$\varepsilon$.

We exploit this observation with automatic differentiation. To calculate the
derivative~$f'(x)$, given code for~$f(x)$, we feed the code with~$x +
\varepsilon$ and then extract the coefficient of~$\varepsilon$ from the result.
Of course, the given code didn't expect to be called with
magical~$\varepsilon$'s instead of ordinary floating-point numbers. But in a
language with operator overloading, there's no way for the code to prevent such
unusual evaluations. We discuss this in more detail below.


\section{A closer look: the dual numbers}

Recall how we construct the complex numbers~$\CC$ from the real numbers~$\RR$.
We define~$\CC \defeq \RR \times \RR$ and set
\begin{align*}
  (x,a) + (y,b) &\defeq (x+y, a+b), \\
  (x,a) \cdot (y,b) &\defeq (xy - ab, xb+ay).
\end{align*}
These formulas don't appear from nowhere. Instead, setting~$i \defeq (0,1)$,
they are precisely the formulas needed such that the identity~$i^2 = -1$
holds. Writing~$(x,a) = x + ai$, we call~$x$ the \emph{real part} and~$a$ the
\emph{imaginary part}.

In a similar way, we can construct the \emph{dual
numbers}~$\RR[\varepsilon]/(\varepsilon^2)$. We
set~$\RR[\varepsilon]/(\varepsilon^2) \defeq \RR \times \RR$ and set
\begin{align*}
  (x,a) + (y,b) &\defeq (x+y, a+b), \\
  (x,a) \cdot (y,b) &\defeq (xy, xb+ay).
\end{align*}
Again, these formulas can be motivated. We write~$\varepsilon \defeq (0,1)$
and~$x + a \varepsilon \defeq (x,a)$. Then these rules can be obtained by
formally expanding~$(x + a \varepsilon) + (y + b \varepsilon)$ respectively~$(x
+ a \varepsilon) \cdot (y + b \varepsilon)$ and imposing the
relation~$\varepsilon^2 = 0$.

% Definition
% Haskell and Python code


\section{Why automatic differentiation works}


\section{Caveats and outlook}

% "poor man's automatic differentiation: Imag(f(x+ih)/h)"

\end{document}
