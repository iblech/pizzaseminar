\documentclass[a4paper,ngerman]{scrartcl}

%\usepackage{ucs}
\usepackage[utf8]{inputenc}

\usepackage[ngerman]{babel}

\usepackage{amsmath,amsthm,amssymb,amscd,color,graphicx}

%\usepackage[small,nohug]{diagrams}
%\diagramstyle[labelstyle=\scriptstyle]

\usepackage[protrusion=true,expansion=true]{microtype}

\usepackage{lmodern}
\usepackage{tabto}

\usepackage[natbib=true,style=numeric]{biblatex}
\usepackage[babel]{csquotes}
\bibliography{lit}

\usepackage[all]{xy}

%\usepackage{hyperref}

\setlength\parskip{\medskipamount}
\setlength\parindent{0pt}

\theoremstyle{definition}
\newtheorem{defn}{Definition}
\newtheorem{bsp}[defn]{Beispiel}

\theoremstyle{plain}

\newtheorem{prop}[defn]{Proposition}
\newtheorem{ueberlegung}[defn]{Überlegung}
\newtheorem{lemma}[defn]{Lemma}
\newtheorem{kor}[defn]{Korollar}
\newtheorem{hilfsaussage}[defn]{Hilfsaussage}
\newtheorem{satz}[defn]{Satz}

\theoremstyle{remark}
\newtheorem{bem}[defn]{Bemerkung}

\clubpenalty=10000
\widowpenalty=10000
\displaywidowpenalty=10000

\newcommand{\lra}{\longrightarrow}
\newcommand{\lhra}{\ensuremath{\lhook\joinrel\relbar\joinrel\rightarrow}}
\newcommand{\thlra}{\relbar\joinrel\twoheadrightarrow}

\newcommand{\A}{\mathcal{A}}
\newcommand{\Z}{\mathbb{Z}}
\newcommand{\Q}{\mathbb{Q}}
\newcommand{\R}{\mathbb{R}}
\newcommand{\C}{\mathcal{C}}
\newcommand{\RP}{\mathbb{R}\mathrm{P}}
\newcommand{\Hom}{\mathrm{Hom}}
\newcommand{\Set}{\mathrm{Set}}
\newcommand{\Spur}[1]{\operatorname{Spur}#1}
\newcommand{\rank}[1]{\operatorname{rank}#1}
\newcommand{\sgn}[1]{\operatorname{sgn}#1}
\newcommand{\id}{\mathrm{id}}
\newcommand{\Aut}[1]{\operatorname{Aut}(#1)}
\newcommand{\GL}[1]{\operatorname{GL}(#1)}
\newcommand{\ORTH}[1]{\operatorname{O}(#1)}
\newcommand{\freist}{\underline{\ \ }}
\newcommand{\op}{\mathrm{op}}
\DeclareMathOperator{\rk}{rk}
\DeclareMathOperator{\Spec}{Spec}
\DeclareMathOperator{\Bild}{im}
\DeclareMathOperator{\Kern}{ker}
\DeclareMathOperator{\Int}{int}
\DeclareMathOperator{\Ob}{Ob}
\newcommand{\Zzwei}{\Z_2}

\newcommand{\XXX}[1]{\textcolor{red}{#1}}

\renewcommand*\theenumi{\alph{enumi}}
\renewcommand{\labelenumi}{\theenumi)}

\usepackage{enumerate}

\pagestyle{empty}

%\newarrow{Equals}=====

\usepackage{geometry}
\geometry{tmargin=2cm,bmargin=3cm,lmargin=3cm,rmargin=3cm}

\begin{document}

\vspace*{-4em}
\begin{flushright}Universität Augsburg \\ 8. März 2013\end{flushright}

\begin{center}\Large \textbf{Pizzaseminar zur Kategorientheorie} \\
Lösung zum 1. Übungsblatt
\end{center}
\vspace{2em}

\newbox{\mybox}
\setbox\mybox=\hbox{\textbf{Aufgabe 1:}}

%\begin{list}{}{\labelwidth\wd\mybox \leftmargin\wd\mybox \itemsep 1.3em}
\begin{list}{}{\labelwidth0em \leftmargin0em \itemindent0.5em \itemsep 1.3em}
\item[\textbf{Aufgabe 1:}]\mbox{}
\begin{enumerate}
\item Eine mögliche Antwort ist die Kategorie \textbf{Grp}, deren Objekte die Klasse aller
Gruppen ist, und  die als Morphismen die Gruppenhomomorphismen mit der üblichen Komposition von Abbildungen besitzt. Initiale und terminale Objekte in \textbf{Grp} sind alle trivialen Gruppen.
\item Wir nennen die Kategorie~$K$ und geben den Objekten und den Morphismen im Diagramm Namen und können mit diesen die Objekte, Morphismen und die Kompositionsvorschrift direkt angeben:\vspace{-2em}

    ${\xymatrix{
      & A \ar[d]^f \ar@(ur,ul)_{\id_A} \\
      B \ar[r]^g \ar@(ul,dl)_{\id_B} & C \ar@(dr,ur)_{\id_C}
    }}$
    {\small
    \setlength{\tabcolsep}{3pt}
    $\vtop{\begin{tabular}{l l l}
      \\
      \\
      $\Ob(K) = \{A, B, C\}$ & & \\
      \noalign{\smallskip}
      $\Hom(A, A) = \{\id_A\}$ & $\Hom(B, A) = \varnothing$ & $\Hom(C, B) = \varnothing$ \\
      $\Hom(A, B) = \varnothing$ & $\Hom(B, B) = \{\id_B\}$ & $\Hom(C, B) = \varnothing$ \\
      $\Hom(A, C) = \{f\}$ & $\Hom(B, C) = \{g\}$ & $\Hom(C, C) = \{\id_C\}$ \\
      \noalign{\smallskip}
      $\id_A \circ \id_A = \id_A$ & $\id_B \circ \id_B = \id_B$ & $\id_C \circ \id_C = \id_C$ \\
      $f \circ \id_A = f$ & $g \circ \id_B = g$ & $\id_C \circ f = f$ \\
      & & $\id_C \circ g = g$
    \end{tabular}}$
    }

\item Angenommen $\id_X$ und $\widetilde{\id}_X$ sind beides Identitätsmorphismen für ein Objekt~$X$ einer Kategorie. Dann gilt für alle passende Morphismen $f$ und $g$

  $f \circ \widetilde{\id}_X = f$ und $\id_X \circ g = g$.

  Insbesondere ist $\id_X = \id_X \circ \widetilde{\id}_X = \widetilde{\id}_X$.
\end{enumerate}

\item[\textbf{Aufgabe 2:}]
Sei~$f:X \to Y$ eine Abbildung zwischen Mengen.
\begin{enumerate}
\item $f$ injektiv $\Rightarrow$ $f$ Monomorphismus:

  \begin{addmargin}{2em}
    Seien $g, g':W \to X$ zwei Abbildungen mit $f \circ g = f \circ g'$. Zu zeigen: $g = g'$.
    Sei dazu $w \in W$ beliebig. Dann ist $f(g(w)) = f(g'(w))$ und weil f injektiv ist,
    $g(w) = g'(w)$ für alle $w \in W$.
  \end{addmargin}

  $f$ Monomorphismus $\Rightarrow$ $f$ injektiv:

  \begin{addmargin}{2em}
    Seien $x, x' \in X$ beliebig mit $f(x) = f(x')$. Zu zeigen: $x = x'$. Definiere dazu
    \begin{align*}
      g &\colon\{\star\} \to X, \enskip\star \mapsto x \\
      g' &\colon\{\star\} \to X, \enskip\star \mapsto x'.
    \end{align*}
    Es gilt offensichtlicherweise $f \circ g = f \circ g'$ und da $f$ Monomorphismus ist,
    auch $g = g'$. Also ist $x = g(\star) = g'(\star) = x'$.
  \end{addmargin}

\item $f$ surjektiv $\Rightarrow$ $f$ Epimorphismus:

  \begin{addmargin}{2em}
    Seien $g, g':Y \to Z$ zwei Abbildungen mit $g \circ f = g' \circ f$. Zu zeigen: $g = g'$.
    Sei dazu $y \in Y$ beliebig. Da $f$ surjektiv ist, gibt es ein $x \in X$ mit $f(x) = y$. Rechne:
    $$g(y) = g(f(x)) = (g \circ f)(x) = (g' \circ f)(x) = g'(f(x)) = g'(y)$$
  \end{addmargin}

  $f$ Epimorphismus $\Rightarrow$ $f$ surjektiv:

  \begin{addmargin}{2em}
    Sei $y \in Y$. Zu zeigen: $y \in \Bild(f)$. Definiere dazu
    \begin{align*}
      g&:Y \to \mathcal{P}(\{\star\}), \enskip\widetilde{y} \mapsto \{\star\} \\
      g'&:Y \to \mathcal{P}(\{\star\}), \enskip\widetilde{y} \mapsto \{\star~|~\widetilde{y} \in \Bild(f) \}.
    \end{align*}
    Es gilt offensichtlicherweise $g \circ f = g' \circ f$ und da $f$ Epimorphismus ist,
    auch $g = g'$. Also ist $\{\star\} = g(y) = g'(y) = \{\star~|~y \in \Bild(f)\}$ und damit beide Mengen gleich sind, muss $y \in \Bild(f)$ gelten.
  \end{addmargin}
\end{enumerate}

\item[\textbf{Aufgabe 3:}]
Seien $f:X \to Y$ und~$g:Y \to Z$ Morphismen einer beliebigen Kategorie~$\C$.
\begin{enumerate}
\item Zu zeigen: $f$ ist Monomorphismus, wenn $g \circ f$ Monomorphismus ist.
  Seien dazu $h, h' : W \to X$ mit $f \circ h = f \circ h'$. Dann ist 
  $g \circ f \circ h = g \circ f \circ h'$ und da $g \circ f$ Monomorphismus ist,
  folgt $h = h'$ wie gewünscht.
\item Die zu a) duale Aussage ist:

  Seien $f:Y \to X$ und~$g:Z \to Y$ Morphismen einer beliebigen Kategorie. Wenn $f \circ g$ Epimorphismus ist, so ist auch $f$ Epimorphismus.
\end{enumerate}


\item[\textbf{Aufgabe 4:}]\mbox{}
\begin{enumerate}
\item $f$ Isomorphismus in \textbf{Grp} $\Rightarrow$ $f$ Gruppenisomorphismus:

    \begin{addmargin}{2em}
      Wenn $f$ Isomorphismus in \textbf{Grp} ist, so ist $f$ ein Gruppenhomomorphismus und besitzt eine Umkehrabbildung $f^{-1}$, ist also bijektiv. Damit ist $f$ nach Definition Gruppenisomorphismus.
    \end{addmargin}

  $f$ Gruppenisomorphismus $\Rightarrow$ $f$ Isomorphismus in \textbf{Grp}:

    \begin{addmargin}{2em}
      Da $f : H \to G$ Gruppenisomorphismus ist, ist $f$ insbesondere bijektiv und besitzt daher eine Umkehrabbildung $f^{-1}$.
      Diese Umkehrabbildung ist sogar ein Gruppenhomomorphismus:
      $$f^{-1}(h \circ \widetilde{h}) = f^{-1}(f(g) \circ f(\widetilde{g})) = f^{-1}(f(g \circ \widetilde{g})) = g \circ \widetilde{g} = f^{-1}(h) \circ f^{-1}(\widetilde{h})$$
      Dabei haben wir verwendet, dass $f$ surjektiv ist, und daher $g, \widetilde{g} \in G$ existieren mit
      $f(g) = h$ und $f(\widetilde{g}) = \widetilde{h}$. Damit befindet sich $f^{-1}$ auch in \textbf{Grp} und bildet dort das Inverse zu $f$.
    \end{addmargin}
\item
  \emph{Beobachtung}: In jeder beliebigen Kategorie sind die Identitätsmorphismen sowohl Mono- als auch Epimorphismus, denn wenn $\id \circ f = \id \circ \widetilde{f}$ bzw. $g \circ \id = \widetilde{g} \circ \id$ gilt, folgt $f = \widetilde{f}$ bzw. $g = \widetilde{g}$ aus den Kategorienaxiomen.

  Wenn $f$ Isomorphismus ist, so gibt es einen Morphismus $f^{-1}$ mit
  \begin{enumerate}[(1)]
    \item $f \circ f^{-1} = id$
    \item $f^{-1} \circ f = id$
  \end{enumerate}
  Aus (1) folgt mit 3b), dass f Epimorphismus ist, und aus (2) folgt mit 3a), dass f Monomorphismus ist.

  Die Umkehrung gilt nicht, wie folgendes Gegenbeispiel zeigt:

  \begin{center}
    ${\xymatrix{
      & A \ar@(dl,ul)^{\id_A} \ar[r]^f & B \ar@(dr,ur)_{\id_B} \\
    }}$
  \end{center}

  Hier ist $f$ Monomorphismus und Epimorphismus, da wir $f$ nur mit den Identitätsmorphismen $\id_A$ und $\id_B$ verknüpfen können und somit wieder $f$ erhalten. Allerdings ist $f$ kein Isomorphismus, da es keinen Morphismus von $B$ nach $A$ gibt.
\end{enumerate}

\item[\textbf{Aufgabe 5:}]
Sei~$G$ eine Gruppe. Wir definieren die Kategorie $BG$ durch

  \begin{enumerate}[(1)]
    \item $\Ob(BG) = \{A\}$
    \item $\Hom(A, A) = G$
    \item Die Komposition in BG entspricht der Komposition der Gruppe.
  \end{enumerate}

  Diese Definition ergibt tatsächlich eine Kategorie, da die Komposition in jeder Gruppe assoziativ ist, und jede Gruppe ein neutrales Element besitzt, das in $BG$ den Identitätsmorphismus auf $A$ ergibt.
  Diese Definition ist auch sinnvoll, da sich aus der Komposition von Morphismen wieder die gesamte Struktur der Gruppe ablesen lässt.
\end{list}

\end{document}
