\documentclass[a4paper,ngerman]{scrartcl}

%\usepackage{ucs}
\usepackage[utf8]{inputenc}

\usepackage[ngerman]{babel}

\usepackage{amsmath,amsthm,amssymb,amscd,color,graphicx}

%\usepackage[small,nohug]{diagrams}
%\diagramstyle[labelstyle=\scriptstyle]

\usepackage[protrusion=true,expansion=true]{microtype}

\usepackage{lmodern}
\usepackage{tabto}

\usepackage[natbib=true,style=numeric]{biblatex}
\usepackage[babel]{csquotes}
\bibliography{lit}

\usepackage[all]{xy}

%\usepackage{hyperref}

\setlength\parskip{\medskipamount}
\setlength\parindent{0pt}

\theoremstyle{definition}
\newtheorem{defn}{Definition}
\newtheorem{bsp}[defn]{Beispiel}

\theoremstyle{plain}

\newtheorem{prop}[defn]{Proposition}
\newtheorem{ueberlegung}[defn]{Überlegung}
\newtheorem{lemma}[defn]{Lemma}
\newtheorem{kor}[defn]{Korollar}
\newtheorem{hilfsaussage}[defn]{Hilfsaussage}
\newtheorem{satz}[defn]{Satz}

\theoremstyle{remark}
\newtheorem{bem}[defn]{Bemerkung}

\clubpenalty=10000
\widowpenalty=10000
\displaywidowpenalty=10000

\newcommand{\lra}{\longrightarrow}
\newcommand{\lhra}{\ensuremath{\lhook\joinrel\relbar\joinrel\rightarrow}}
\newcommand{\thlra}{\relbar\joinrel\twoheadrightarrow}

\newcommand{\A}{\mathcal{A}}
\newcommand{\Z}{\mathbb{Z}}
\newcommand{\Q}{\mathbb{Q}}
\newcommand{\R}{\mathbb{R}}
\newcommand{\C}{\mathcal{C}}
\newcommand{\RP}{\mathbb{R}\mathrm{P}}
\newcommand{\Hom}{\mathrm{Hom}}
\newcommand{\Set}{\mathrm{Set}}
\newcommand{\Spur}[1]{\operatorname{Spur}#1}
\newcommand{\rank}[1]{\operatorname{rank}#1}
\newcommand{\sgn}[1]{\operatorname{sgn}#1}
\newcommand{\id}{\mathrm{id}}
\newcommand{\Aut}[1]{\operatorname{Aut}(#1)}
\newcommand{\GL}[1]{\operatorname{GL}(#1)}
\newcommand{\ORTH}[1]{\operatorname{O}(#1)}
\newcommand{\freist}{\underline{\ \ }}
\newcommand{\op}{\mathrm{op}}
\DeclareMathOperator{\rk}{rk}
\DeclareMathOperator{\Spec}{Spec}
\DeclareMathOperator{\Bild}{im}
\DeclareMathOperator{\Kern}{ker}
\DeclareMathOperator{\Int}{int}
\DeclareMathOperator{\Ob}{Ob}
\newcommand{\Zzwei}{\Z_2}

\newcommand{\XXX}[1]{\textcolor{red}{#1}}

\renewcommand*\theenumi{\alph{enumi}}
\renewcommand{\labelenumi}{\theenumi)}

\pagestyle{empty}

%\newarrow{Equals}=====

\usepackage{geometry}
\geometry{tmargin=2cm,bmargin=3cm,lmargin=3cm,rmargin=3cm}

\begin{document}

\vspace*{-4em}
\begin{flushright}Universität Augsburg \\ 27. Februar 2013\end{flushright}

\begin{center}\Large \textbf{Pizzaseminar zur Kategorientheorie} \\
1. Übungsblatt
\end{center}
\vspace{2em}

\newbox{\mybox}
\setbox\mybox=\hbox{\textbf{Aufgabe 1:}}

\begin{list}{}{\labelwidth\wd\mybox \leftmargin\wd\mybox \itemsep 1.3em}
\item[\textbf{Aufgabe 1:}]
\begin{enumerate}
\item Gib die Kategorie zu deinem Lieblingsgebiet an: Was sind ihre Objekte,
was ihre Morphismen? Was sind ihre initialen und terminalen Objekte?
\item Gib eine formale Definition folgender Kategorie:
    \vspace{0.5em}
    \[ \xymatrix{
      & \bullet \ar[d] \ar@(ur,ul) \\
      \bullet \ar[r] \ar@(ul,dl) & \bullet \ar@(dr,ur)
    } \]
Was genau sind also die Objekte und die Morphismen? Wie lautet die
Kompositionsvorschrift?
\item Zeige: Identitätsmorphismen in Kategorien sind eindeutig, d.\,h. sind
$\id_X$ und $\widetilde{\id}_X$ beides Identitätsmorphismen für ein Objekt~$X$
einer Kategorie~$\C$, so gilt~$\id_X = \widetilde{\id}_X$.
\end{enumerate}

\item[\textbf{Aufgabe 2:}]
Sei~$f:X \to Y$ eine Abbildung zwischen Mengen.
\begin{enumerate}
\item Zeige: Es ist~$f$ genau dann ein Monomorphismus, wenn $f$
injektiv ist.
\item Zeige: Es ist~$f$ genau dann ein Epimorphismus, wenn $f$
surjektiv ist.

\emph{Tipp:} Widerspruchsbeweis oder geeignete Abbildungen nach $\mathcal{P}(\{\star\})
= \{ \emptyset, \{ \star\} \}$ betrachten.
\end{enumerate}

\item[\textbf{Aufgabe 3:}]
Seien $f:X \to Y$ und~$g:Y \to Z$ Morphismen einer beliebigen Kategorie~$\C$.
\begin{enumerate}
\item Zeige: Ist $g \circ f$ ein Monomorphismus, so ist auch $f$ ein
Monomorphismus.
\item Dein Beweis von a) funktioniert in allen Kategorien, daher auch
in~$\C^\op$. Was besagt er dann?
\end{enumerate}


\item[\textbf{Aufgabe 4:}]
Ein \emph{Isomorphismus} $f:X \to Y$ in einer Kategorie ist ein
Morphismus, zu dem es einen Morphismus $g:Y \to X$ mit
\[ g \circ f = \id_X, \quad f \circ g = \id_Y \]
gibt. Statt "`$g$"' schreibt man auch "`$f^{-1}$"'.
\begin{enumerate}
\item Zeige: Die Isomorphismen in der Kategorie der Gruppen sind genau die
üblichen Gruppenisomorphismen.
\item Zeige: In beliebigen Kategorien sind Isomorphismen stets Mono- und
Epimorphismen, aber die Umkehrung gilt nicht.

\emph{Tipp:} Für die Rückrichtung kann man ein Gegenbeispiel in einer
über\-schau\-ba\-ren Kategorie angeben.
\end{enumerate}

\item[\textbf{Aufgabe 5:}]
Sei~$G$ eine Gruppe. Bastele auf sinnvolle Art und Weise aus~$G$ eine Kategorie
-- so, dass die Gruppenverknüpfung eine Rolle spielt. (Diese Kategorie wird oft
mit "`$BG$"' bezeichnet und ist in der algebraischen Topologie wichtig.)
\end{list}

\end{document}
