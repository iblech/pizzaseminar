\documentclass{pizzablatt}

%\geometry{tmargin=2cm,bmargin=1.3cm,lmargin=2.9cm,rmargin=2.9cm}
%\setlength{\aufgabenskip}{1em}

\begin{document}

\maketitle{5}{Pizzaseminar zu konstruktiver Mathematik}{25. Oktober 2013}

\begin{aufgabe}{Monomorphismen und Epimorphismen aus interner Sicht}
Sei~$X$ ein topologischer Raum (oder eine Örtlichkeit). Sei~$\alpha : \F \to
\G$ ein Morphismus von Garben auf~$X$. Zeige, \ldots
\begin{enumerate}
\item \ldots dass $\alpha$ genau dann ein Monomorphismus ist, wenn
$X \models \forall x,y\?\F\_ \alpha(x) = \alpha(y) \Rightarrow x = y$.
\item \ldots dass $\alpha$ genau dann ein Epimorphismus ist, wenn
$X \models \forall y\?\G\_ \exists x\?\F\_ \alpha(x) = y$.
\end{enumerate}
Aus Sicht der internen Sprache des Garbentopos~$\Sh(X)$ sehen also
Monomorphismen wie gewöhnliche injektive und Epimorphismen wie
gewöhnliche surjektive Abbildungen aus.
\end{aufgabe}

\begin{aufgabe}{Vereinfachungsregeln für die interne Sprache}
Sei~$X$ ein topologischer Raum.
\begin{enumerate}
\item \emph{Eindeutige Existenz ist globale Existenz.}
Sei~$\F$ eine Garbe auf~$X$ und~$\varphi$ eine Aussage, in der eine
Variable~$x\?\F$ frei vorkommt. Zeige: Genau dann gilt~$X \models \exists!
x\?\F\_ \varphi$, wenn auf jeder offenen Teilmenge~$U \subseteq X$ genau ein
Schnitt~$s \in \Gamma(U,\F)$ mit~$U \models \varphi(s)$ existiert.
\item \emph{Topologische Interpretation der Doppelnegation.}
Sei~$\varphi$ eine Aussage. Zeige: Genau dann gilt~$X \models
\neg\neg\varphi$, wenn es eine dichte offene Teilmenge~$U \subseteq X$ mit~$U
\models \varphi$ gibt.
\end{enumerate}
\end{aufgabe}

\begin{aufgabe}{Die Ringgarbe stetiger Funktionen als Körper}
Sei~$X$ ein topologischer Raum. Sei~$\C^0$ die Garbe der stetigen Funktionen
auf~$X$.
\begin{enumerate}
\item Sei~$f \in \Gamma(U,\C^0)$. Zeige: Die Funktion~$f$ besitzt genau dann
ein multiplikatives Inverses in~$\Gamma(U,\C^0)$, wenn $X \models \speak{$f$
invertierbar}$, d.\,h. wenn $X \models \exists g\?\C^0\_ fg = 1$.
\item Zeige, dass~$\C^0$ aus interner Sicht in folgendem Sinn ein Körper ist:
\[ X \models \forall f\?\C^0\_ \neg(\speak{$f$ invertierbar})
\Rightarrow f = 0. \]
\item Zeige, dass~$\C^0$ aber nicht folgende Körperbedingung erfüllt:
\[ X \models \forall f\?\C^0\_ f = 0 \vee \speak{$f$ invertierbar}. \]
\end{enumerate}
\emph{Bemerkung.} Man kann zeigen, dass~$\C^0$ aus interner Sicht die Rolle der
über dedekindsche Schnitte konstruierten reellen Zahlen erfüllt.
\end{aufgabe}

\begin{aufgabe}{Basen endlich erzeugter Vektorräume}
\begin{enumerate}
\item Sei~$V$ ein endlich erzeugter Vektorraum über einem Ring~$k$, der die Körperbedingung aus
Aufgabe~3b) erfüllt. Zeige konstruktiv, dass~$V$ \emph{nicht nicht} eine Basis
besitzt.

\emph{Tipp.} Verwende, dass die Menge~$\{ n \in \NN \,|\, \text{$V$ besitzt ein
Erzeugendensystem der Länge~$n$} \}$ \emph{nicht nicht} ein Minimum
besitzt.

\item Sei~$X$ ein topologischer Raum. Sei~$\F$ eine~$\C^0$-Modulgarbe, die
lokal von endlichem Typ ist (das ist gleichbedeutend damit, dass~$\F$ aus
interner Sicht ein endlich erzeugter~$\C^0$-Modul ist). Folgere direkt aus~a),
dass~$\F$ auf einer dichten offenen Teilmenge~$U$ lokal frei ist (dass es also
eine offene Überdeckung~$U = \bigcup_i U_i$ gibt, sodass die~$\F|_{U_i}$ isomorph
zu Modulgarben der Form~$(\C^0|_{U_i})^{n_i}$ sind).
\end{enumerate}
\end{aufgabe}

\end{document}

Auch noch schön wäre: Beispiel aus meiner Masterarbeit zum Thema exists vs.
negnegexists.
