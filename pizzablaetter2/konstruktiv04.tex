\documentclass{pizzablatt}

%\geometry{tmargin=2cm,bmargin=1.3cm,lmargin=2.9cm,rmargin=2.9cm}
%\setlength{\aufgabenskip}{1em}

\newcommand{\bra}[1]{\left\langle #1 \right|}
\newcommand{\ket}[1]{\left| #1 \right\rangle}

\begin{document}

\maketitle{4}{Pizzaseminar zu konstruktiver Mathematik}{16. Oktober 2013}

\begin{aufgabe}{Distributionen}
Der~$\RR$-Vektorraum
\[ C^\infty_c(\RR) := \{ \phi \in C^\infty(\RR) \,|\,
  \text{$\operatorname{supp}(\phi)$ ist eine kompakte Teilmenge von~$\RR$}
  \} \]
ist das wichtigste Beispiel für einen \emph{Raum von Testfunktionen}. Man
topologisiert~$C^\infty_c(\RR)$ so, dass eine Funktionenfolge~$(\phi_n)_n$
in~$C^\infty_c(\RR)$ genau dann gegen eine Funktion~$\phi$ konvergiert, wenn
es ein Kompaktum~$K \subseteq \Omega$ gibt, sodass alle~$\phi_n$ Träger in~$K$
haben und sodass für jede Ableitungsordnung~$k$ die Konvergenz
\[ \sup_{x \in K}\ \Bigl|\tfrac{\partial^k}{\partial x^k} (\phi_n(x) -
\phi(x))\Bigr| \xrightarrow{n \to \infty} 0 \]
vorliegt. Eine \emph{Distribution auf~$\RR$} ist eine stetige lineare
Abbildung~$C^\infty_c(\RR) \to \RR$.

\begin{enumerate}
\item Sei~$f : \RR \to \RR$ eine lokal integrable Funktion. Zeige, dass die Abbildung
\[ \begin{array}{@{}rrcl@{}}
  T_f\ : & C^\infty_c(\RR) &\longrightarrow& \RR \\
  & \phi &\longmapsto& \int_\RR f(x) \, \phi(x) \,dx
\end{array} \]
eine Distribution auf~$\RR$ ist. (Wieso ist das unbeschränkte Integral
wohldefiniert?) In diesem Sinn kann man also jede (lokal integrable) Funktion
auch als Distribution auffassen.
\item Die \emph{Ableitung~$D'$} einer (beliebigen!) Distribution~$D$ ist über die Vorschrift
\[ D'(\phi) := -D(\phi') \]
definiert, wobei auf der rechten Seite die gewöhnliche Ableitung von~$\phi$
auftritt. Zeige, dass die so definierte Abbildung tatsächlich eine Distribution
ist.
\item Sei~$f \in C^\infty(\RR)$. Zeige: $(T_f)' = T_{f'}$.

Die Einbettung von Funktionen in Distributionen respektiert also den
Ableitungsbegriff.
\end{enumerate}

In der Literatur schreibt man oft~"`$\int D(x) \phi(x)
\,dx$"' für~$D(\phi)$, falls~$D$ eine Distribution und~$\phi$ eine Testfunktion
ist. Diese Notation spiegelt folgendes Motto wieder:

\textbf{Motto.} Distributionen sind \emph{verallgemeinerte Funktionen}; sie
definieren sich nicht durch ihre Funktionswerte, sondern dadurch, wie sie sich
"`unter dem Integral"' mit beliebigen Testfunktionen verhalten.

Dieses Motto harmoniert mit der kategorientheoretischen Vorstellung von
Prägarben als ideelle Objekte, welche allein dadurch bestimmt werden, welche Beziehungen sie zu
anderen Objekten haben.
\end{aufgabe}

\begin{aufgabe}{Die Delta-Distribution}
Sei~$\delta : C^\infty_c(\RR) \to \RR,\, \phi \mapsto \phi(0)$.
\begin{enumerate}
\item Zeige, dass~$\delta$ tatsächlich eine Distribution ist.
\item Rechtfertige folgende Notation: $\int_\RR \delta(x) \phi(x) \,dx =
\phi(0)$.
\end{enumerate}
Eine Folge~$(\delta_n)_n$ nichtnegativer integrierbarer Funktionen
auf~$\RR$ heißt genau dann \emph{Dirac-Folge}, wenn~$\int_\RR \delta_n(x) \,dx
= 1$ für alle~$n \geq 0$ und~$\int_{\RR \setminus (-\varepsilon,\varepsilon)}
\delta_n(x) \,dx \xrightarrow{n \to \infty} 0$ für alle~$\varepsilon > 0$.
\begin{enumerate}
\addtocounter{enumi}{2}
\item Zeige, dass~$(\delta_n)_n$ mit~$\delta_n(x) := \frac{1}{n \sqrt{\pi}}
\exp(-x^2 / n^2)$ eine Dirac-Folge ist.
\item Zeige für alle~$\phi \in C^\infty_c(\RR)$: $\int_\RR \delta_n(x) \phi(x) \,dx
\xrightarrow{n \to \infty} \delta(\phi)$.
\end{enumerate}
\end{aufgabe}

\begin{aufgabe}{Bra-Ket-Notation}
Sei~$H$ ein Hilbertraum über~$\CC$, also ein~$\CC$-Vektorraum mit einer
positiv definiten hermiteschen Sesquilinearform~$(\freist,\freist)$, der
bezüglich der induzierten Metrik vollständig ist. Sei~$H^\vee =
\{ A : H \to \CC \,|\, \text{$A$ linear und stetig} \}$ der topologische
Dualraum von~$H$. Die Konvention sei so, dass die Sesquilinearform im rechten
Argument linear ist.

\begin{enumerate}
\item Der für die Hilbertraumtheorie fundamentale \emph{Satz von Riesz} besagt,
dass die kanonische Abbildung
\[ \begin{array}{@{}rcl@{}}
  H &\longrightarrow& H^\vee \\
  x &\longmapsto& (x, \freist)
\end{array} \]
eine antilineare Bijektion ist. Beweise diese Behauptung für den Fall, dass~$H$
endlich-dimensional ist.
\end{enumerate}

Man schreibt nun~"`$\ket{x}$"' für~$x \in H$ und~"`$\bra{x}$"' für~$(\freist,x)
\in H^\vee$.

\begin{enumerate}
\addtocounter{enumi}{1}
\item Zeige: $\langle x|y \rangle := (\bra{x})(\ket{y}) = (x,y)$ für
alle~$x,y \in H$.
\item Zeige: $\langle x|A|y \rangle := \bigl(\bra{x} \circ A\bigr)
\bigl(\ket{y}\bigr) = \bigl(\bra{x}\bigr) \bigl(A(\ket{y})\bigr) = (x,Ay)$ für
alle~$x,y \in H$, $A : H \to H$ linear und stetig.
\item Sei~$x \in H$ ein Vektor von Norm~$1$. Zeige: Der Operator~$P := \ket{x} \, \bra{x} : H \to H$, definiert
über~$P(y) = \bra{x}(y) \cdot \ket{x}$ für~$y \in H$, erfüllt~$P \circ P = P$
und hat~$\operatorname{span}(x)$ als Bild.
\item Sei~$e_1,\ldots,e_n$ eine Basis von~$H$. Zeige: Der
Operator~$\sum_{i=1}^n \ket{e_i} \bra{e_i}$ ist der Identitätsoperator.
\item Sei weiterhin~$e_1,\ldots,e_n$ eine Basis von~$H$. Beobachte, wie
suggestiv die Notation dich in einem Beweis der Identität $\ket{x} =
\sum_{i=1}^n \langle e_i|x \rangle \cdot \ket{e_i}$ leitet.
\end{enumerate}
\end{aufgabe}

\end{document}
