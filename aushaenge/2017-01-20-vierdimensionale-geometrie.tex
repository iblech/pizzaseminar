\documentclass[a4paper,ngerman,landscape]{scrartcl}

\usepackage[utf8]{inputenc}

\usepackage[ngerman]{babel}
\usepackage{hyperref}

\usepackage{graphicx}

\usepackage[protrusion=true,expansion=true]{microtype}

\usepackage{libertine}
\usepackage{tabto}

\setlength\parskip{\medskipamount}
\setlength\parindent{0pt}

\usepackage{geometry}
\geometry{tmargin=0.1cm,bmargin=1.0cm,lmargin=2.5cm,rmargin=2.5cm}

\pagestyle{empty}

\begin{document}

\newcommand{\blurb}[1]{\begin{center}
  \huge
  \vspace*{0.0em}
  Neujahrsvorlesung im Pizzaseminar \\
  Freitag, 20. Januar 2017, 14:00 Uhr, 1005/L1 \\
  \Huge
  \textbf{Ingo Blechschmidt und Matthias Hutzler: \\ Die kuriose Welt der
  vierdimensionalen Geometrie}
  \vfill
  \vspace{0.3em}
  \includegraphics[height=0.5\textheight]{#1}
  \vfill

  \Large
  \begin{minipage}{0.92\textwidth}
    \renewcommand{\baselinestretch}{1.3}

    \setlength\parskip{\medskipamount}
    \vspace{0.3em}
    Der Vortrag setzt keinerlei Vorkenntnisse voraus. Bringt eure Geschwister
    mit! Alle Interessierten sind herzlich eingeladen. Es gibt Kekse. Außerdem:
    Interaktive Demos, Animationen und einen schlechten Witz
    \textbullet{} Anschauliche Vorstellung mittels der
    Flachlandanalogie
    \textbullet{} Knoten und andere Verwirrungen im Vierdimensionalen
    \textbullet{} Möbiusband? Viel zu Mainstream. Wir werden uns die Kleinsche Flasche
    genauer ansehen!
    \textbullet{} Seltsame Volumenphänomene
    \textbullet{} Ein vierdimensionales Fraktal, das das bekannte Mandelbrotfraktal und
    alle Juliamengen vereinigt
    \textbullet{} Wieso die allgemeine Relativitätstheorie nicht in nur drei Dimensionen
    funktionieren kann
    \textbullet{} Platonische Körper im Vierdimensionalen
    \textbullet{} Wie man einen vierdimensionalen Würfel bastelt
    \vspace{0.3em}
    \hfill\small \textsf{https:/$\!$/pizzaseminar.speicherleck.de/}
  \end{minipage}
\end{center}}

\blurb{great-grand-120-cell}
\blurb{120-cell}
\blurb{flatlandthefilm}

\end{document}
