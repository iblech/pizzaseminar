\documentclass[a4paper,ngerman,landscape]{scrartcl}

\usepackage[utf8]{inputenc}

\usepackage[ngerman]{babel}
\usepackage{hyperref}

\usepackage{graphicx}

\usepackage[protrusion=true,expansion=true]{microtype}

\usepackage{libertine}
\usepackage{tabto}

\setlength\parskip{\medskipamount}
\setlength\parindent{0pt}

\usepackage{geometry}
\geometry{tmargin=0.1cm,bmargin=1.0cm,lmargin=2.5cm,rmargin=2.5cm}

\pagestyle{empty}

\begin{document}

\begin{center}
  \Huge
  \vspace*{0.0em}
  Mittwoch, 19. Juli 2017, 17:30 Uhr, 2004/L1 \\
  \mbox{\textbf{Einsteiger-Workshop zu Git}}
  \vfill
  % https://en.wikipedia.org/wiki/File:Sunflowers.jpg
  \vspace{0.3em}
  \includegraphics[width=0.6\textwidth]{git-logo}
  \vfill

  \Large
  \begin{minipage}{0.92\textwidth}
    \setlength\parskip{\medskipamount}
    \vspace{0.3em}
    Üblicherweise gibt es beim Schreiben einer Abschlussarbeit oder eines
    Artikels ein großes Dateinamenschaos: Man hat Dateien wie
    Abschlussarbeit.tex, Abschlussarbeitfinal.tex, Abschlussarbeitfinal2.tex,
    Abschlussarbeitfinal2korr.tex und so weiter und verliert völlig den
    Überblick darüber, was die jeweils neueste Version ist. Noch schlimmer wird
    es, wenn man an mehreren verschiedenen Computern arbeitet oder mit anderen
    Leuten zusammenarbeitet. Dieses Chaos muss nicht sein! Im dritten
    Jahrtausend gibt es eine technologische Lösung dafür, die bei
    Programmiererinnen und Programmierern allgemeint bekannt, sonst aber völlig
    unbekannt ist. Diese Lösung ist das \emph{Versionskontrollsystem Git}.

    In diesem etwa 60-minütigen Workshop zeigen wir euch
    die ersten Schritte mit Git. Alle Interessierten sind herzlich
    eingeladen, mit dem eigenen Laptop vorbeizuschauen. Anmeldung ist nicht
    nötig, Kosten entstehen auch keine. Bei guten Betriebssystemen ist Git
    schon dabei, unter Windows bitte vorab Git von
    \textsf{https:/$\!$/git-scm.com/downloads} herunterladen und installieren,
    um auf dem Workshop Zeit zu sparen.

    Euer Git-Workshop-Team:
    Ingo, Matthias, Tim, Xaver
  \end{minipage}
\end{center}

\end{document}
