\documentclass[a4paper,ngerman,landscape]{scrartcl}

\usepackage[utf8]{inputenc}

\usepackage[ngerman]{babel}
\usepackage{hyperref}

\usepackage{graphicx}

\usepackage[protrusion=true,expansion=true]{microtype}

\usepackage{lmodern}
\usepackage{tabto}

\setlength\parskip{\medskipamount}
\setlength\parindent{0pt}

\usepackage{geometry}
\geometry{tmargin=1.5cm,bmargin=0.5cm,lmargin=2.5cm,rmargin=2.5cm}

\pagestyle{empty}

\begin{document}

\begin{center}
  \Huge
  Dienstag, 13. August 2013, 10:00 Uhr, 2004/L1 \\
  \textbf{Ingo Blechschmidt: Konstruktive Mathematik I}
  \vfill
  \includegraphics[scale=1.4]{lem}
  \vfill

  \Large
  \begin{minipage}{0.83\textwidth}
    \setlength\parskip{\medskipamount}
    Konstruktive Mathematik ist Mathematik ohne Verwendung des Prinzips vom
    ausgeschlossenen Dritten -- daher sind Widerspruchsbeweise nicht pauschal
    zugelassen. Im Vortrag werden wir einen ersten Einblick in die etwas
    kuriose konstruktive Mathematik-Welt bekommen und verstehen, wieso konstruktive
    Mathematik auch dann nützlich ist, wenn man philosophisch überhaupt kein Problem
    mit klassischer Logik hat. Wir werden auch thematisieren, wie es
    historisch zur Entwicklung konstruktiver Mathematik kam.
  \end{minipage}
\end{center}

\end{document}
