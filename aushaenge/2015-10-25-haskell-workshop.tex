\documentclass[a4paper,ngerman,landscape]{scrartcl}

\usepackage[utf8]{inputenc}

\usepackage[ngerman]{babel}
\usepackage{hyperref}

\usepackage{graphicx}

\usepackage{multicol}

\usepackage[protrusion=true,expansion=true]{microtype}

\usepackage{libertine}
\usepackage{tabto}

\setlength\parskip{\medskipamount}
\setlength\parindent{0pt}

\usepackage{geometry}
\geometry{tmargin=0.1cm,bmargin=1.0cm,lmargin=2.5cm,rmargin=2.5cm}

\pagestyle{empty}

\begin{document}

\begin{center}
  \Huge
  \vspace*{0.0em}
  Sonntag, 25. Oktober 2015: \\
  \mbox{\textbf{Einsteiger-Workshop zu Haskell,}} \\
  \mbox{\textbf{der rein funktionalen Programmiersprache}} \\
  \vfill
  \includegraphics[width=0.5\textwidth]{learn-you-a-haskell-for-great-good}
  \vfill

  \Large
  \begin{minipage}{0.77\textwidth}
    \setlength\parskip{\medskipamount}
    Was ist schneller als C++, prägnanter als Perl, regelmäßiger als Python,
    flexibler als Ruby, typisierter als C\#, robuster als Java und hat absolut
    nichts mit PHP gemeinsam? Es ist Haskell!

    Der Curry Club Augsburg, vertreten durch Mathe-Studierende wie Evelyn A.,
    Ingo B., Matthias H., Maximilian H., Tim B. und Stefan H., lädt zum großen
    Haskell-Workshop ein. Zielgruppe sind in erster Linie völlige
    Haskell-Neulinge. Teilnahmevoraussetzung ist Erfahrung mit einer beliebigen
    anderen Programmiersprache.

    Wer vorhat, am Workshop teilzunehmen, kann sich auf
    http:/$\!$/curry-club-augsburg.de/ unverbindlich anmelden. Kosten gibt es
    keine.
  \end{minipage}
\end{center}

\newpage
\vspace*{1.5em}

\setlength{\columnsep}{2.5em}
\begin{multicols}{2}
\setlength\parskip{\medskipamount}
\Large
Haskell ist eine moderne und innovative Programmiersprache, die sich von
bekannten imperativen Sprachen in vielerlei Hinsicht deutlich unterscheidet:
Ein Haskell-Programm besteht nicht etwa aus einer Abfolge von auszuführenden
Anweisungen, sondern aus einer Ansammlung von Deklarationen, deren
Reihenfolge keine Rolle spielt. Auch gibt es keine veränderlichen Variablen,
und ausgewertet wird nur, was wirklich benötigt wird; unendliche
Datenstrukturen sind möglich und sinnvoll.

Dieses Denkparadigma mag anfangs sehr ungewohnt sein, zieht jedoch eine Reihe
von Vorteilen mit sich: Da es keine Nebenwirkungen wie beispielsweise globale
Variablen gibt, kann man Code rein lokal verstehen. Damit wird es einfacher,
modular Komponenten zusammenzubauen, sich Datenflüsse klarzumachen und Code
auf seine Korrektheit hin zu überprüfen. Insbesondere vereinfacht sich die
Programmierung mit Threads enorm.

Ferner ist Haskells starkes statisches Typsystem eine große Hilfe beim
Programmieren und verhindert viel mehr Fehler schon während des Kompilierens,
als man vielleicht aus anderen Sprachen gewohnt ist. Es gibt das Motto, dass,
wenn Haskell-Code erst einmal erfolgreich durchkompiliere, er dann auch schon
korrekt sei. Das ist sicherlich übertrieben, hat aber einen erstaunlich
wahren Kern.

Bei der Beschäftigung mit Haskell lernt man viele neue Herangehensweisen kennen, die
auch in anderen Sprachen nützlich sind; das ist einer der Hauptvorteile an
Haskell, der auch dann noch relevant ist, wenn man aus verschiedenen Gründen
im täglichen Leben nicht in Haskell programmieren möchte.
\ \\
\ \\

\vfill
\centering
\includegraphics[height=0.5\textheight]{haskell-spock}
\par
\end{multicols}

\end{document}
