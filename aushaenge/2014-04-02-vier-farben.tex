\documentclass[a4paper,ngerman,landscape]{scrartcl}

\usepackage[utf8]{inputenc}

\usepackage[ngerman]{babel}
\usepackage{hyperref}

\usepackage{graphicx}

\usepackage[protrusion=true,expansion=true]{microtype}

\usepackage{lmodern}
\usepackage{tabto}

\setlength\parskip{\medskipamount}
\setlength\parindent{0pt}

\usepackage{geometry}
\geometry{tmargin=0.5cm,bmargin=1.0cm,lmargin=2.5cm,rmargin=2.5cm}

\pagestyle{empty}

\begin{document}

\begin{center}
  \huge
  Mittwoch, 2. April 2014, 12:45 Uhr, 2004/L1 \\
  \mbox{\textbf{Tim Baumann: Der Vier-Farben-Satz}}
  \vfill
  \vspace{0.3em}
  \includegraphics[scale=0.55]{vier-farben-satz}
  \vfill

  \Large
  \begin{minipage}{0.85\textwidth}
    \setlength\parskip{\medskipamount}
    \vspace{0.3em}
    Der Vier-Farben-Satz besagt, dass vier Farben immer ausreichen, um eine
    beliebige Landkarte so einzufärben, dass keine zwei aneinandergrenzenden
    Länder die gleiche Farbe erhalten. Obwohl er bereits 1852 von Francis
    Guthrie als Vermutung formuliert wurde, konnte er erst im Jahre 1976 von
    Appel und Haken bewiesen werden. Das Besondere an ihrem Beweis ist, dass er
    sich auf eine aufwändige Fallunterscheidung stützt, für die sie ein
    Computerprogramm entwickelten. Im Vortrag werden wir den Ansatz ihres
    Beweises verstehen. Außerdem werden wir auf Korrektheit und
    Überprüfbarkeit von computergestützen Beweisen eingehen.

    Es wird keinerlei Vorwissen vorausgesetzt. Alle Interessierten sind
    daher herzlich eingeladen.

    \vspace{1em}
    \hfill\small Skript und Übungsblätter: \url{http://pizzaseminar.speicherleck.de/}
  \end{minipage}
\end{center}

\end{document}
