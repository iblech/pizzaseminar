\documentclass[a4paper,ngerman,landscape]{scrartcl}

\usepackage[utf8]{inputenc}

\usepackage[ngerman]{babel}
\usepackage{hyperref}

\usepackage{graphicx}
\usepackage{tikz}
\usetikzlibrary{calc}

\usepackage[protrusion=true,expansion=true]{microtype}

\usepackage{libertine}
\usepackage{tabto}

\setlength\parskip{\medskipamount}
\setlength\parindent{0pt}

\usepackage{geometry}
\geometry{tmargin=0.1cm,bmargin=1.0cm,lmargin=1.5cm,rmargin=1.5cm}

\pagestyle{empty}

\begin{document}

\begin{center}
  \includegraphics[height=0.15\textwidth]{eisbaden-5}
  \includegraphics[height=0.15\textwidth]{eisbaden-3}
  \includegraphics[height=0.15\textwidth]{eisbaden-1}
  \includegraphics[height=0.15\textwidth]{eisbaden-2}
  \includegraphics[height=0.15\textwidth]{eisbaden-qrcode4}

  \Huge
  \scalebox{3.8}{\textbf{Mathe-Eisbaden}}

  \large
  \begin{minipage}{0.92\textwidth}
    \renewcommand{\baselinestretch}{1.3}

    \setlength\parskip{\medskipamount}
    \vspace{0.3em}
    Eisbaden macht Spaß, tut gut und ist nicht so krass, wie man es sich vorstellt.
    Wir haben das schon öfter gemacht, zuletzt am vergangenen Dienstag mit fast
    50 Leuten. \textbf{Das kommende Eisbaden wird das letzte Eisbaden in diesem
    Semester sein.} Wer Eisbaden also ausprobieren möchte, hat nur noch dieses
    Mal dazu die Möglichkeit. Alle Interessierten sind herzlich eingeladen!
    Wir treffen uns am Mittwoch pünktlich um 15:45 Uhr bei der
    Schranke zwischen Mathe- und Info-Gebäude, um dann gemeinsam zum Ilsesee in
    Königsbrunn zu fahren. Um 16:20 Uhr sind wir wieder an der Uni.
    Wer mit möchte, muss sich nur auf \textsf{http:/$\!$/bit.ly/2kjjf9t} eintragen, damit
    wir die Anzahl Autos planen können.
    \vspace{0.3em}
  \end{minipage}

  \huge
  \scalebox{1.5}{Mittwoch, 1. Februar 2017, 15:45--16:20 Uhr}

  \tikz[remember picture,overlay] \node[opacity=1.0,inner sep=0pt] at (current
  page.south){\hspace*{-3cm}\vbox{\vspace*{-2.4cm}\includegraphics[width=\paperwidth]{eisbaden-6}}};
\end{center}

\end{document}
