\documentclass[a4paper,ngerman]{scrartcl}

\usepackage[utf8]{inputenc}

\usepackage[ngerman]{babel}

\usepackage{amsmath,amsthm,amssymb,amscd,color,graphicx}
\usepackage{array}

\usepackage[protrusion=true,expansion=true]{microtype}

\usepackage{lmodern}
\usepackage{tabto}

\usepackage[all]{xy}

\usepackage{hyperref}

\setlength\parskip{\medskipamount}
\setlength\parindent{0pt}

\theoremstyle{definition}
\newtheorem{defn}{Definition}[section]
\newtheorem{bsp}[defn]{Beispiel}

\theoremstyle{plain}

\newtheorem{prop}[defn]{Proposition}
\newtheorem{motto}[defn]{Motto}
\newtheorem{ueberlegung}[defn]{Überlegung}
\newtheorem{lemma}[defn]{Lemma}
\newtheorem{kor}[defn]{Korollar}
\newtheorem{hilfsaussage}[defn]{Hilfsaussage}
\newtheorem{satz}[defn]{Satz}

\theoremstyle{remark}
\newtheorem{bem}[defn]{Bemerkung}

\clubpenalty=10000
\widowpenalty=10000
\displaywidowpenalty=10000

\newcommand{\lra}{\longrightarrow}
\newcommand{\lhra}{\ensuremath{\lhook\joinrel\relbar\joinrel\rightarrow}}
\newcommand{\thlra}{\relbar\joinrel\twoheadrightarrow}

\newcommand{\ZZ}{\mathbb{ZZ}}
\newcommand{\QQ}{\mathbb{QQ}}
\newcommand{\RR}{\mathbb{RR}}
\newcommand{\C}{\mathcal{C}}
\newcommand{\Hom}{\mathrm{Hom}}
\newcommand{\id}{\mathrm{id}}
\newcommand{\freist}{\underline{\ \ }}
\DeclareMathOperator{\Ob}{Ob}
\DeclareMathOperator{\ggT}{ggT}
\newcommand{\op}{\mathrm{op}}
\newcommand{\Set}{\mathrm{Set}}
\newcommand{\Grp}{\mathrm{Grp}}
\newcommand{\Vect}{\mathrm{Vect}}

\newcommand{\XXX}[1]{\textcolor{red}{#1}}

\renewcommand*\theenumi{\alph{enumi}}
\renewcommand{\labelenumi}{\theenumi)}

%\newarrow{Equals}=====

%\usepackage{geometry}
%\geometry{tmargin=2cm,bmargin=4cm,lmargin=3cm,rmargin=3cm}

\begin{document}

\vspace*{2em}%
\begin{center}%
  \vskip 1em
  {\LARGE Pizzaseminar zur Kategorientheorie}
  \vskip 1.5em%
  {\large
   \lineskip .5em%
    \begin{tabular}[t]{c}%
      \today
    \end{tabular}\par}%
    \vskip 1em%
\end{center}\par
\par\vskip 1.5em

\begin{center}\emph{Warnung:} Die vielen Beispiele, Erklärungen und Hintergründe
fehlen bislang.\end{center}

\tableofcontents

\section[Was sollen Kategorien?]{Was sollen Kategorien? \hfill \small Ingo
Blechschmidt}

\subsection{Beispiele für kategorielles Verständnis}

\subsubsection*{Beispiel: Produkte}

Von manchen Konstruktionen in verschiedenen Teilgebieten der Mathematik wird
man das Gefühl nicht los, dass sie einem gemeinsamen Ursprung entstammen: Etwa
kennt man\ldots
\begin{itemize}
  \item das kartesische Produkt von Mengen: $X \times Y$,
  \item das kartesische Produkt von Vektorräumen: $V \times W$,
  \item das kartesische Produkt von Gruppen: $G \times H$,
  \item das kartesische Produkt von Garben: $\mathcal{F} \times \mathcal{G}$,
  \item das kartesische Produkt von Vektorbündeln: $\mathcal{E} \times \mathcal{F}$,
  \item das Minimum von Zahlen: $\min\{n,m\}$,
  \item den größten gemeinsamen Teiler von Zahlen: $\ggT(n,m)$,
  \item den Paartyp in Programmiersprachen: \texttt{(a,b)},
  \item den Mutterknoten zweier Knoten in einem Graph.
\end{itemize}
Die Ähnlichkeit untereinander ist mal mehr, mal weniger deutlich. Nur mit
Kategorientheorie versteht man: All dies sind Spezialfälle des allgemeinen
\emph{kategoriellen Produkts}. Ferner erfüllen all diese Konstruktionen sehr
ähnliche Gesetze, etwa gilt
\begin{align*}
  X \times (Y \times Z) &\cong (X \times Y) \times Z, \\
  U \times (V \times W) &\cong (U \times V) \times W, \\
  \min\{m,\min\{n,p\}\} &= \min\{\min\{m,n\},p\}, \\
  \ggT(m,\ggT(n,p)) &= \ggT(\ggT(m,n),p),
\end{align*}
wobei in der ersten Zeile~$X$, $Y$ und~$Z$ Mengen sein und das
Isomorphiezeichen für Gleichmächtigkeit stehen soll und in der zweiten
Zeile~$U$, $V$ und~$W$ Vektorräume sein und das Isomorphiezeichen für
Vektorraumisomorphie stehen soll. Mit Kategorientheorie versteht man:
All dies sind Spezialfälle der allgemeinen Assoziativität des kategoriellen
Produkts.

\subsubsection*{Beispiel: Isomorphie}

Ferner fällt auf, dass in vielen Teilgebieten der Mathematik jeweils ein
speziell zugeschnittener Isomorphiebegriff vorkommt: Etwa können\ldots
\begin{itemize}
  \item zwei Mengen $X,Y$ \tabto{4.63cm} gleichmächtig sein,
  \item zwei Vektorräume $V,W$ \tabto{4.63cm} isomorph sein,
  \item zwei Gruppen $G,H$ \tabto{4.63cm} isomorph sein,
  \item zwei top. Räume $X,Y$ \tabto{4.63cm} homöomorph sein,
  \item zwei Zahlen $n,m$ \tabto{4.63cm} gleich sein,
  \item zwei Typen \texttt{a}, \texttt{b} \tabto{4.63cm} sich verlustfrei ineinander umwandeln lassen.
\end{itemize}
All dies sind Spezialfälle des allgemeinen \emph{kategoriellen
Isomorphiekonzepts}.

\subsubsection*{Beispiel: Dualität}

Von folgenden Konzepten hat man im Gefühl, dass sie in einem gewissen Sinn
\emph{zueinander dual} sein sollten:
\begin{center}
  \setlength{\extrarowheight}{0.3em}
  \begin{tabular}{r|l}
    $f \circ g$ & $g \circ f$ \\
    $\leq$ & $\geq$ \\
    injektiv & surjektiv \\
    $\{\star\}$ & $\emptyset$ \\
    $\times$ & $\coprod$ \\
    ggT & kgV \\
    $\cap$ & $\cup$ \\
    Teilmenge & Faktormenge
  \end{tabular}
\end{center}
Mit Kategorientheorie versteht man: All dies sind Spezialfälle eines allgemeinen
\emph{kategoriellen Dualitätsprinzips} -- und diese Erkenntnis kann man nutzen,
um Ergebnisse für jeweils eines der Konzepte auf sein duales Gegenstück zu
übertragen.

\subsection{Grundlagen}

\begin{defn}
Eine \emph{Kategorie}~$\C$ besteht aus
\begin{enumerate}
  \item einer Klasse von \emph{Objekten} $\Ob \C$,
  \item zu je zwei Objekten $X,Y \in \Ob \C$ einer Klasse $\Hom_\C(X,Y)$ von
  \emph{Morphismen} zwischen ihnen und
  \item einer Kompositionsvorschrift:
  \begin{align*}
    \text{zu }\ & f \in \Hom_\C(X,Y) &
    \text{zu }\ & f : X \to Y \\
    \text{und }\ & g\in\Hom_\C(Y,Z) &
    \text{und }\ & g : Y \to Z \\
    \text{habe }\ & g\circ f\in\Hom_\C(X,Z), &
    \text{habe }\ & g\circ f : X \to Z,
  \end{align*}
\end{enumerate}
sodass
\begin{enumerate}
  \item die Komposition $\circ$ assoziativ ist und
  \item es zu jedem $X \in \Ob\C$ einen \emph{Identitätsmorphismus} $\id_X
  \in \Hom_\C(X,X)$ mit
  \[ f \circ \id_X = f, \quad \id_X \circ g = g \]
  für alle Morphismen $f,g$ gibt.
\end{enumerate}
\end{defn}

Die Morphismen von Kategorien müssen nicht unbedingt Abbildungen
sein, die Schreibweise "`$f:X \to Y$"' missbraucht also Notation. Die genaue
Bedeutung von \emph{Klassen} im Gegensatz zu \emph{Mengen} hängt von der
persönlich gewählten logischen Fundierung der Mathematik ab. Für uns genügt
folgende naive Sichtweise: Klassen können (im Gegensatz zu Mengen) beliebige
mathematische Objekte enthalten, sind aber selbst nicht mathematische Objekte.
Daher gibt es etwa widerspruchsfrei die Klasse aller Mengen, von einer Klasse
aller Klassen kann man aber nicht sprechen.

\begin{bsp}\begin{enumerate}
  \item Archetypisches Beispiel ist $\Set$, die Kategorie der Mengen und Abbildungen:
  \begin{align*}
    \Ob \Set &:= \{ M \,|\, \text{$M$ ist eine Menge} \} \\
    \Hom_\Set(X,Y) &:= \{ f:X \to Y \,|\, \text{$f$ ist eine Abbildung} \}
  \end{align*}
  \item Die meisten Teilgebiete der Mathematik studieren jeweils eine bestimmte
  Kategorie: Gruppentheoretiker beschäftigen sich etwa mit der Kategorie
  $\Grp$ der Gruppen und Gruppenhomomorphismen:
  \begin{align*}
    \Ob \Grp &:= \text{Klasse aller Gruppen} \\
    \Hom_\Grp(G,H) &:= \{ f:G \to H \,|\, \text{$f$ ist ein Gruppenhomo} \}
  \end{align*}
  \item Es gibt aber auch wesentlich kleinere Kategorien. Etwa kann man aus
  jeder Quasiordnung~$(P,\preceq)$ eine Kategorie~$\C$ basteln:
  \begin{align*}
    \Ob \C &:= P \\
    \Hom_\C(x,y) &:= \begin{cases}
      \text{einelementige Menge}, & \text{falls $x \preceq y$,} \\
      \text{leere Menge}, & \text{sonst}
    \end{cases}
  \end{align*}
  \item Auch sind gewisse endliche Kategorien bedeutsam, etwa die durch
  folgende Skizze gegebene:

  \[ \xymatrix{
    & \bullet \ar[d] \ar@(ur,ul) \\
    \bullet \ar[r] \ar@(ul,dl) & \bullet \ar@(dr,ur)
  } \]
\end{enumerate}\end{bsp}

\begin{motto}[fundamental]Kategorientheorie stellt \emph{Beziehungen zwischen
Objekten} statt etwaiger innerer Struktur in den Vordergrund.\end{motto}


\subsubsection*{Initiale und terminale Objekte}

In Kategorien sind folgende zwei Arten von Objekten aufgrund ihrer
ausgezeichneten Beziehungen zu allen (anderen) Objekten besonders wichtig:
\begin{defn}
Ein Objekt~$X$ einer Kategorie~$\C$ heißt genau dann
\begin{itemize}
  \item \emph{initial}, wenn
    \[ \forall Y \in \Ob \C{:}\ \exists! f : X \to Y. \]
  \item \emph{terminal}, wenn
    \[ \forall Y \in \Ob \C{:}\ \exists! f : Y \to X. \]
\end{itemize}
\end{defn}
Diese Definitionen geben ein erstes Beispiel für sog. \emph{universellen
Eigenschaften}.

\begin{bsp}\begin{enumerate}
\item In der Kategorie der Mengen ist genau die leere Menge initial und
genau jede einelementige Menge terminal. Diese Erkenntnis ist ein erstes
Beispiel dafür, wie das fundamentale Motto gemeint ist: Eine Definition der
leeren bzw. einer einelementigen Menge über eine Aufzählung ihrer Elemente
betont ihre innere Struktur, während eine Definition als initiales bzw.
terminales Objekt die besonderen Beziehungen zu allen Mengen hervorhebt.
\item In der Kategorie der $K$-Vektorräume ist der Nullvektorraum $K^0$ initial
und terminal.
\item Viele kategorielle Konstruktionen realisiert man als initiales oder
terminales Objekt in einer geeigneten Kategorie von Möchtegern-Konstruktionen.
Ein erstes Beispiel dazu werden wir im folgenden Kapitel über Produkte finden.
\end{enumerate}\end{bsp}


\subsubsection*{Mono-, Epi- und Isomorphismen}

\begin{defn}
Ein Morphismus $f:X \to Y$ einer Kategorie~$\C$ heißt genau dann
\begin{itemize}
  \item \emph{Monomorphismus}, \tabto{3.35cm}wenn für alle Objekte~$A \in \Ob \C$
  und $p,q:A \to X$ gilt:
  \[ f \circ p = f \circ q \quad\Longrightarrow\quad p = q. \]
  \item \emph{Epimorphismus}, \tabto{3.35cm}wenn für alle Objekte~$A \in \Ob \C$
  und $p,q:Y \to A$ gilt:
  \[ p \circ f = q \circ f \quad\Longrightarrow\quad p = q. \]
\end{itemize}
\end{defn}

\begin{bsp}\begin{enumerate}
\item In den Kategorien der Mengen, Gruppen und $K$-Vektorräumen sind die
Monomorphismen genau die injektiven und die Epimorphismen genau die
surjektiven Abbildungen. Das ist jeweils eine interessante Erkenntnis über die
Struktur dieser Kategorien und nicht ganz leicht zu zeigen.
\item In der Kategorie der metrischen Räume mit stetigen Abbildungen gibt es
Epimorphismen, die nicht surjektiv sind: nämlich alle stetigen Abbildungen mit
dichtem Bild.
\end{enumerate}\end{bsp}

\begin{defn}
Ein \emph{Isomorphismus} $f:X \to Y$ in einer Kategorie ist ein
Morphismus, zu dem es einen Morphismus $g:Y \to X$ mit
\[ g \circ f = \id_X, \quad f \circ g = \id_Y \]
gibt. Statt "`$g$"' schreibt man auch "`$f^{-1}$"'. Existiert zwischen
Objekten~$X$ und~$Y$ ein Isomorphismus, so heißen die Objekte \emph{zueinander
isomorph}:
$X \cong Y$.
\end{defn}


\subsubsection*{Die duale Kategorie}

\begin{defn}
Die zu einer Kategorie~$\C$ zugehörige \emph{duale Kategorie} $\C^\op$ ist
folgende:
\begin{align*}
  \Ob \C^\op &:= \Ob \C \\
  \Hom_{\C^\op}(X,Y) &:= \Hom_\C(Y,X)
\end{align*}
\end{defn}

\begin{bsp}\begin{enumerate}
\item Ein initiales Objekt in~$\C^\op$ ist ein terminales Objekt in~$\C$ und
umgekehrt.
\item Ein Epimorphismus in~$\C^\op$ ist ein Monomorphismus in~$\C$ und
umgekehrt.
\item Zwei Objekte sind genau dann in~$\C^\op$ zueinander isomorph, wenn sie es
in~$\C$ sind.
\end{enumerate}\end{bsp}


\section[Produkte und Koprodukte]{Produkte und Koprodukte \hfill \small
Matthias Hutzler}

\begin{defn}Seien~$X$, $Y$ Objekte einer Kategorie~$\C$. Dann besteht ein
\emph{Produkt} von~$X$ und~$Y$ aus
\begin{enumerate}
\item einem Objekt~$P \in \Ob \C$ und
\item Morphismen $\pi_X : P \to X$, $\pi_Y : P \to Y$,
\end{enumerate}
sodass für jedes andere \emph{Möchtegern-Produkt}, also
\begin{enumerate}
\item jedem Objekt~$\widetilde P \in \Ob \C$ zusammen mit
\item Morphismen $\widetilde \pi_X : \widetilde P \to X$, $\widetilde\pi_Y :
\widetilde P \to Y$
\end{enumerate}
genau ein Morphismus $\psi : \widetilde P \to P$ existiert, der das Diagramm
\[ \xymatrix{
    & \widetilde P \ar[ld]_{\widetilde \pi_X} \ar@{-->}[dd]_\psi \ar[rd]^{\widetilde \pi_Y} \\
  X & & Y \\
    & P \ar[lu]^{\pi_X} \ar[ru]_{\pi_Y}
  } \]
kommutieren lässt, also die Gleichungen
\begin{align*}
  \pi_X \circ \psi &= \widetilde \pi_X \\
  \pi_Y \circ \psi &= \widetilde \pi_Y
\end{align*}
erfüllt.
\end{defn}

\begin{motto}Ein Produkt ist ein bestes Möchtegern-Produkt.\end{motto}

Analog definiert man das Produkt von~$n$ Objekten, $n \geq 0$; und dual
definiert man das Koprodukt.

\begin{bsp}\begin{enumerate}
\item Das Produkt in der Kategorie der Mengen ist durch das kartesische Produkt
gegeben, das Koprodukt durch die disjunkt-gemachte Vereinigung.
\item Das Produkt in der Kategorie der Gruppen ist durch das direkte Produkt
mit der komponentenweisen Verknüpfung gegeben, das Koprodukt durch das sog.
freie Produkt von Gruppen.
\item Produkt und Koprodukt endlich vieler Objekte in der Kategorie
der~$K$-Vektorräume sind durch die äußere direkte Summe gegeben. Produkte und
Koprodukte von unendlich vielen Objekten unterscheiden sich allerdings.
\item Das Produkt in der von einer Quasiordnung induzierten Kategorie ist durch
das Infimum gegeben, siehe Aufgabe~3 von Übungsblatt~2. Dual ist das Koprodukt
durchs Supremum gegeben.
\end{enumerate}\end{bsp}

\begin{prop}Die Objektteile je zweier Produkte von Objekten~$X$, $Y$ sind
zueinander isomorph.\end{prop}

\begin{bem}Es gilt sogar noch mehr, siehe Aufgabe~2 von Übungsblatt~2.\end{bem}

\begin{prop}Die Angabe eines Produkts von~$X$ und~$Y$ ist gleichwertig mit der
Angabe eines Produkts von~$Y$ und~$X$.\end{prop}

\begin{prop}Die Angabe eines Produkts von null vielen Objekten ist gleichwertig
mit der Angabe eines terminalen Objekts.\end{prop}

\end{document}
