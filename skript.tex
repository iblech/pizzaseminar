\documentclass[a4paper,ngerman]{scrartcl}

\usepackage[utf8]{inputenc}

\usepackage[ngerman]{babel}

\usepackage{amsmath,amsthm,amssymb,amscd,color,graphicx}
\usepackage{array}
\usepackage{comment}

\usepackage[protrusion=true,expansion=true]{microtype}

\usepackage{lmodern}
\usepackage{tabto}

\usepackage[natbib=true,style=numeric]{biblatex}
\usepackage[babel]{csquotes}
\bibliography{literatur}

\usepackage[all]{xy}

\usepackage{hyperref}

\setlength\parskip{\medskipamount}
\setlength\parindent{0pt}

\theoremstyle{definition}
\newtheorem{defn}{Definition}[section]
\newtheorem{bsp}[defn]{Beispiel}

\theoremstyle{plain}

\newtheorem{prop}[defn]{Proposition}
\newtheorem{motto}[defn]{Motto}
\newtheorem{ueberlegung}[defn]{Überlegung}
\newtheorem{lemma}[defn]{Lemma}
\newtheorem{kor}[defn]{Korollar}
\newtheorem{hilfsaussage}[defn]{Hilfsaussage}
\newtheorem{satz}[defn]{Satz}

\theoremstyle{remark}
\newtheorem{bem}[defn]{Bemerkung}

\clubpenalty=10000
\widowpenalty=10000
\displaywidowpenalty=10000

\newcommand{\xra}[1]{\xrightarrow{#1}}
\newcommand{\lra}{\longrightarrow}
\newcommand{\lhra}{\ensuremath{\lhook\joinrel\relbar\joinrel\rightarrow}}
\newcommand{\thlra}{\relbar\joinrel\twoheadrightarrow}

\newcommand{\ZZ}{\mathbb{Z}}
\newcommand{\QQ}{\mathbb{Q}}
\newcommand{\RR}{\mathbb{R}}
\newcommand{\NN}{\mathbb{N}}
\newcommand{\I}{\mathcal{I}}
\newcommand{\C}{\mathcal{C}}
\newcommand{\D}{\mathcal{D}}
\newcommand{\E}{\mathcal{E}}
\renewcommand{\I}{\mathcal{I}}
\renewcommand{\P}{\mathcal{P}}
\newcommand{\Hom}{\mathrm{Hom}}
\newcommand{\id}{\mathrm{id}}
\newcommand{\Id}{\mathrm{Id}}
\newcommand{\freist}{\underline{\ \ }}
\DeclareMathOperator{\Ob}{Ob}
\DeclareMathOperator{\ggT}{ggT}
\newcommand{\op}{\mathrm{op}}
\newcommand{\Set}{\mathrm{Set}}
\newcommand{\Grp}{\mathrm{Grp}}
\newcommand{\Vect}{\mathrm{Vect}}
\newcommand{\AbGrp}{\mathrm{AbGrp}}
\newcommand{\Ring}{\mathrm{Ring}}
\newcommand{\Cat}{\mathrm{Cat}}
\newcommand{\Funct}{\mathrm{Funct}}
\newcommand{\Eins}{\mathbf{1}}
\newcommand{\Man}{\mathrm{Man}}
\newcommand{\Top}{\mathrm{Top}}

\newcommand{\XXX}[1]{\textcolor{red}{#1}}

\renewcommand*\theenumi{\alph{enumi}}
\renewcommand{\labelenumi}{\theenumi)}

\newcommand\subsubsubsection[1]{\subsubsection*{#1}}
\definecolor{grey}{rgb}{0.7,0.7,0.7}

\setcounter{tocdepth}{2}

%\newarrow{Equals}=====

%\usepackage{geometry}
%\geometry{tmargin=2cm,bmargin=4cm,lmargin=3cm,rmargin=3cm}

\begin{document}

\vspace*{2em}%
\begin{center}%
  \vskip 1em
  {\LARGE Pizzaseminar zur Kategorientheorie}
  \vskip 1.5em%
  {\large
   \lineskip .5em%
    \begin{tabular}[t]{c}%
      \today
    \end{tabular}\par}%
    \vskip 1em%
\end{center}\par
\par\vskip 1.5em

\begin{center}\emph{in Entstehung befindlich} \\ \ \\
\TeX{}er: Tim Baumann, Ingo Blechschmidt, Justin Gassner, Lukas Graf, Maximilian Huber, Matthias Hutzler\end{center}

\tableofcontents

\section[Was sollen Kategorien?]{Was sollen Kategorien? \hfill \small Ingo
Blechschmidt}

\subsection{Beispiele für kategorielles Verständnis}

\subsubsection*{Beispiel: Produkte}

Von manchen Konstruktionen in verschiedenen Teilgebieten der Mathematik wird
man das Gefühl nicht los, dass sie einem gemeinsamen Ursprung entstammen: Etwa
kennt man\ldots
\begin{itemize}
  \item das kartesische Produkt von Mengen: $X \times Y$,
  \item das kartesische Produkt von Vektorräumen: $V \times W$,
  \item das kartesische Produkt von Gruppen: $G \times H$,
  \item das kartesische Produkt von Garben: $\mathcal{F} \times \mathcal{G}$,
  \item das kartesische Produkt von Vektorbündeln: $\mathcal{E} \times \mathcal{F}$,
  \item das Minimum von Zahlen: $\min\{n,m\}$,
  \item den größten gemeinsamen Teiler von Zahlen: $\ggT(n,m)$,
  \item den Paartyp in Programmiersprachen: \texttt{(a,b)},
  \item den Produktautomat zweier endlicher Automaten,
  \item den Mutterknoten zweier Knoten in einem Graph.
\end{itemize}
Die Ähnlichkeit untereinander ist mal mehr, mal weniger deutlich. Nur mit
Kategorientheorie versteht man: All dies sind Spezialfälle des allgemeinen
\emph{kategoriellen Produkts}. Ferner erfüllen all diese Konstruktionen sehr
ähnliche Gesetze, etwa gilt
\begin{align*}
  X \times (Y \times Z) &\cong (X \times Y) \times Z, \\
  U \times (V \times W) &\cong (U \times V) \times W, \\
  \min\{m,\min\{n,p\}\} &= \min\{\min\{m,n\},p\}, \\
  \ggT(m,\ggT(n,p)) &= \ggT(\ggT(m,n),p),
\end{align*}
wobei in der ersten Zeile~$X$, $Y$ und~$Z$ Mengen sein und das
Isomorphiezeichen für Gleichmächtigkeit stehen soll und in der zweiten
Zeile~$U$, $V$ und~$W$ Vektorräume sein und das Isomorphiezeichen für
Vektorraumisomorphie stehen soll. Mit Kategorientheorie versteht man:
All dies sind Spezialfälle der allgemeinen Assoziativität des kategoriellen
Produkts.

\subsubsection*{Beispiel: Isomorphie}

Ferner fällt auf, dass in vielen Teilgebieten der Mathematik jeweils ein
speziell zugeschnittener Isomorphiebegriff vorkommt: Etwa können\ldots
\begin{itemize}
  \item zwei Mengen $X,Y$ \tabto{4.63cm} gleichmächtig sein,
  \item zwei Vektorräume $V,W$ \tabto{4.63cm} isomorph sein,
  \item zwei Gruppen $G,H$ \tabto{4.63cm} isomorph sein,
  \item zwei top. Räume $X,Y$ \tabto{4.63cm} homöomorph sein,
  \item zwei Zahlen $n,m$ \tabto{4.63cm} gleich sein,
  \item zwei endliche Automaten isomorph sein,
  \item zwei Typen \texttt{a}, \texttt{b} \tabto{4.63cm} sich verlustfrei ineinander umwandeln lassen.
\end{itemize}
All dies sind Spezialfälle des allgemeinen \emph{kategoriellen
Isomorphiekonzepts}.

\subsubsection*{Beispiel: Dualität}

Von folgenden Konzepten hat man im Gefühl, dass sie in einem gewissen Sinn
\emph{zueinander dual} sein sollten:
\begin{center}
  \setlength{\extrarowheight}{0.3em}
  \begin{tabular}{r|l}
    $f \circ g$ & $g \circ f$ \\
    $\leq$ & $\geq$ \\
    injektiv & surjektiv \\
    $\{\star\}$ & $\emptyset$ \\
    $\times$ & $\amalg$ \\
    ggT & kgV \\
    $\cap$ & $\cup$ \\
    Teilmenge & Faktormenge
  \end{tabular}
\end{center}
Mit Kategorientheorie versteht man: All dies sind Spezialfälle eines allgemeinen
\emph{kategoriellen Dualitätsprinzips} -- und diese Erkenntnis kann man nutzen,
um Ergebnisse für jeweils eines der Konzepte auf sein duales Gegenstück zu
übertragen.

\subsection{Grundlagen}

\begin{defn}
Eine \emph{Kategorie}~$\C$ besteht aus
\begin{enumerate}
  \item einer Klasse von \emph{Objekten} $\Ob \C$,
  \item zu je zwei Objekten $X,Y \in \Ob \C$ einer Klasse $\Hom_\C(X,Y)$ von
  \emph{Morphismen} zwischen ihnen und
  \item einer Kompositionsvorschrift:
  \begin{align*}
    \text{zu }\ & f \in \Hom_\C(X,Y) &
    \text{zu }\ & f : X \to Y \\
    \text{und }\ & g\in\Hom_\C(Y,Z) &
    \text{und }\ & g : Y \to Z \\
    \text{habe }\ & g\circ f\in\Hom_\C(X,Z), &
    \text{habe }\ & g\circ f : X \to Z,
  \end{align*}
\end{enumerate}
sodass
\begin{enumerate}
  \item die Komposition $\circ$ assoziativ ist und
  \item es zu jedem $X \in \Ob\C$ einen \emph{Identitätsmorphismus} $\id_X
  \in \Hom_\C(X,X)$ mit
  \[ f \circ \id_X = f, \quad \id_X \circ g = g \]
  für alle Morphismen $f,g$ gibt.
\end{enumerate}
\end{defn}

Die Morphismen von Kategorien müssen nicht unbedingt Abbildungen
sein, die Schreibweise "`$f:X \to Y$"' missbraucht also Notation. Die genaue
Bedeutung von \emph{Klassen} im Gegensatz zu \emph{Mengen} hängt von der
persönlich gewählten logischen Fundierung der Mathematik ab. Für uns genügt
folgende naive Sichtweise: Klassen können (im Gegensatz zu Mengen) beliebige
mathematische Objekte enthalten, sind aber selbst nicht mathematische Objekte.
Daher gibt es etwa widerspruchsfrei die Klasse aller Mengen, von einer Klasse
aller Klassen kann man aber nicht sprechen.

\begin{bsp}\begin{enumerate}
  \item Archetypisches Beispiel ist $\Set$, die Kategorie der Mengen und Abbildungen:
  \begin{align*}
    \Ob \Set &:= \{ M \,|\, \text{$M$ ist eine Menge} \} \\
    \Hom_\Set(X,Y) &:= \{ f:X \to Y \,|\, \text{$f$ ist eine Abbildung} \}
  \end{align*}
  \item Die meisten Teilgebiete der Mathematik studieren jeweils eine bestimmte
  Kategorie: Gruppentheoretiker beschäftigen sich etwa mit der Kategorie
  $\Grp$ der Gruppen und Gruppenhomomorphismen:
  \begin{align*}
    \Ob \Grp &:= \text{Klasse aller Gruppen} \\
    \Hom_\Grp(G,H) &:= \{ f:G \to H \,|\, \text{$f$ ist ein Gruppenhomo} \}
  \end{align*}
  \item Es gibt aber auch wesentlich kleinere Kategorien. Etwa kann man aus
  jeder Quasiordnung~$(P,\preceq)$ eine Kategorie~$\C$ basteln:
  \begin{align*}
    \Ob \C &:= P \\
    \Hom_\C(x,y) &:= \begin{cases}
      \text{einelementige Menge}, & \text{falls $x \preceq y$,} \\
      \text{leere Menge}, & \text{sonst}
    \end{cases}
  \end{align*}
  \item Auch sind gewisse endliche Kategorien bedeutsam, etwa die durch
  folgende Skizze gegebene:

  \[ \xymatrix{
    & \bullet \ar[d] \ar@(ur,ul) \\
    \bullet \ar[r] \ar@(ul,dl) & \bullet \ar@(dr,ur)
  } \]
\end{enumerate}\end{bsp}

\begin{motto}[fundamental]Kategorientheorie stellt \emph{Beziehungen zwischen
Objekten} statt etwaiger innerer Struktur in den Vordergrund.\end{motto}

\begin{defn}Eine Kategorie~$\C$ heißt \emph{lokal klein}, wenn ihre Hom-Klassen
jeweils schon Mengen (statt echte Klassen) sind. Eine Kategorie~$\C$ heißt
\emph{klein}, wenn zusätzlich auch ihre Klasse von Objekten schon eine Menge
bildet.\end{defn}


\subsubsection*{Initiale und terminale Objekte}

In Kategorien sind folgende zwei Arten von Objekten aufgrund ihrer
ausgezeichneten Beziehungen zu allen (anderen) Objekten besonders wichtig:
\begin{defn}
Ein Objekt~$X$ einer Kategorie~$\C$ heißt genau dann
\begin{itemize}
  \item \emph{initial}, wenn
    \[ \forall Y \in \Ob \C{:}\ \exists! f : X \to Y. \]
  \item \emph{terminal}, wenn
    \[ \forall Y \in \Ob \C{:}\ \exists! f : Y \to X. \]
\end{itemize}
\end{defn}
Diese Definitionen geben ein erstes Beispiel für sog. \emph{universellen
Eigenschaften}.

\begin{bsp}\begin{enumerate}
\item In der Kategorie der Mengen ist genau die leere Menge initial und
genau jede einelementige Menge terminal. Diese Erkenntnis ist ein erstes
Beispiel dafür, wie das fundamentale Motto gemeint ist: Eine Definition der
leeren bzw. einer einelementigen Menge über eine Aufzählung ihrer Elemente
betont ihre innere Struktur, während eine Definition als initiales bzw.
terminales Objekt die besonderen Beziehungen zu allen Mengen hervorhebt.
\item In der Kategorie der $K$-Vektorräume ist der Nullvektorraum $K^0$ initial
und terminal.
\end{enumerate}\end{bsp}
Viele kategorielle Konstruktionen realisiert man als initiales oder
terminales Objekt in einer geeigneten Kategorie von Möchtegern-Konstruktionen.
Ein erstes Beispiel dazu werden wir im folgenden Kapitel über Produkte finden.


\subsubsection*{Mono-, Epi- und Isomorphismen}

\begin{defn}
Ein Morphismus $f:X \to Y$ einer Kategorie~$\C$ heißt genau dann
\begin{itemize}
  \item \emph{Monomorphismus}, \tabto{3.35cm}wenn für alle Objekte~$A \in \Ob \C$
  und $p,q:A \to X$ gilt:
  \[ f \circ p = f \circ q \quad\Longrightarrow\quad p = q. \]
  \item \emph{Epimorphismus}, \tabto{3.35cm}wenn für alle Objekte~$A \in \Ob \C$
  und $p,q:Y \to A$ gilt:
  \[ p \circ f = q \circ f \quad\Longrightarrow\quad p = q. \]
\end{itemize}
\end{defn}

\begin{bsp}\begin{enumerate}
\item In den Kategorien der Mengen, Gruppen und $K$-Vektorräumen sind die
Monomorphismen genau die injektiven und die Epimorphismen genau die
surjektiven Abbildungen. Das ist jeweils eine interessante Erkenntnis über die
Struktur dieser Kategorien und nicht ganz leicht zu zeigen.
\item In der Kategorie der metrischen Räume mit stetigen Abbildungen gibt es
Epimorphismen, die nicht surjektiv sind: nämlich alle stetigen Abbildungen mit
dichtem Bild.
\end{enumerate}\end{bsp}

\begin{defn}
Ein \emph{Isomorphismus} $f:X \to Y$ in einer Kategorie ist ein
Morphismus, zu dem es einen Morphismus $g:Y \to X$ mit
\[ g \circ f = \id_X, \quad f \circ g = \id_Y \]
gibt. Statt "`$g$"' schreibt man auch "`$f^{-1}$"'. Existiert zwischen
Objekten~$X$ und~$Y$ ein Isomorphismus, so heißen die Objekte \emph{zueinander
isomorph}:
$X \cong Y$.
\end{defn}

\begin{bem}\label{gleichheitobj}In den meisten Kategorien ist die Frage, ob
Objekte~$X,Y$ tatsächlich gleich (statt nur isomorph) sind, keine interessante
Frage: Denn für alle praktischen Belange sind schon zueinander isomorphe
Objekte "`gleich gut"'. Diesen Gedanken werden wir noch manche Male aufgreifen
und weiter entwickeln.\end{bem}


\subsubsection*{Die duale Kategorie}

Aus jeder Kategorie~$\C$ kann man durch "`Umdrehen aller Pfeile"' eine weitere
Kategorie erhalten, die sogenannte duale Kategorie von~$\C$:

\begin{defn}
Die zu einer Kategorie~$\C$ zugehörige \emph{duale Kategorie} $\C^\op$ ist
folgende:
\begin{align*}
  \Ob \C^\op &:= \Ob \C \\
  \Hom_{\C^\op}(X,Y) &:= \Hom_\C(Y,X)
\end{align*}
\end{defn}

Das ist ein rein formaler Prozess, der mit dem Invertieren bijektiver
Abbildungen nichts zu tun hat. Die duale Kategorie ist nützlich, um sich der
Dualität mancher kategorieller Konzepte gewahr zu werden:

\begin{bsp}\begin{enumerate}
\item Ein initiales Objekt in~$\C^\op$ ist ein terminales Objekt in~$\C$ und
umgekehrt.
\item Ein Epimorphismus in~$\C^\op$ ist ein Monomorphismus in~$\C$ und
umgekehrt.
\item Zwei Objekte sind genau dann in~$\C^\op$ zueinander isomorph, wenn sie es
in~$\C$ sind. Isomorphie ist also ein \emph{selbstduales} Konzept.
\end{enumerate}\end{bsp}

Spannend ist es, wenn duale Kategorien durch andere, natürlich
auftretende Kategorien beschrieben werden können.


\section[Produkte und Koprodukte]{Produkte und Koprodukte \hfill \small
Matthias Hutzler}

\begin{defn}Seien~$X$, $Y$ Objekte einer Kategorie~$\C$. Dann besteht ein
\emph{Produkt} von~$X$ und~$Y$ aus
\begin{enumerate}
\item einem Objekt~$P \in \Ob \C$ und
\item Morphismen $\pi_X : P \to X$, $\pi_Y : P \to Y$,
\end{enumerate}
sodass für jedes andere \emph{Möchtegern-Produkt}, also
\begin{enumerate}
\item jedem Objekt~$\widetilde P \in \Ob \C$ zusammen mit
\item Morphismen $\widetilde \pi_X : \widetilde P \to X$, $\widetilde\pi_Y :
\widetilde P \to Y$
\end{enumerate}
genau ein Morphismus $\psi : \widetilde P \to P$ existiert, der das Diagramm
\[ \xymatrix{
    & P \ar[ld]_{\pi_X} \ar[rd]^{\pi_Y} \\
  X & & Y \\
    & \widetilde P \ar[lu]^{\widetilde \pi_X} \ar@{-->}[uu]_\psi \ar[ru]_{\widetilde \pi_Y}
  } \]
kommutieren lässt, also die Gleichungen
\begin{align*}
  \pi_X \circ \psi &= \widetilde \pi_X \\
  \pi_Y \circ \psi &= \widetilde \pi_Y
\end{align*}
erfüllt.
\end{defn}

\begin{motto}Ein Produkt ist ein bestes Möchtegern-Produkt.\end{motto}

Statt~"`$P$"' schreibt man gerne~"`$X \times Y$"'; es muss aus dem Kontext klar
werden, ob das Kreuzzeichen speziell das kartesische Produkt von Mengen
oder das allgemeine kategorielle Produkt bezeichnen soll.
Analog definiert man das Produkt von~$n$ Objekten, $n \geq 0$, und dual
definiert man das Koprodukt.


\subsection{Beispiele}

\begin{bsp}\begin{enumerate}
\item Das Produkt in der Kategorie der Mengen ist durch das kartesische Produkt
gegeben, das Koprodukt durch die disjunkt-gemachte Vereinigung.
\item Das Produkt in der Kategorie der Gruppen ist durch das direkte Produkt
mit der komponentenweisen Verknüpfung gegeben, das Koprodukt durch das sog.
freie Produkt von Gruppen.
\item Produkt und Koprodukt endlich vieler Objekte in der Kategorie
der~$K$-Vektorräume sind durch die äußere direkte Summe gegeben. Produkte und
Koprodukte von unendlich vielen Objekten unterscheiden sich allerdings.
\item Das Produkt in der von einer Quasiordnung induzierten Kategorie ist durch
das Infimum gegeben. Dual ist das Koprodukt
durchs Supremum gegeben.
\end{enumerate}\end{bsp}
\begin{proof}
\begin{enumerate}
\item Wir zeigen die Aussage über das kartesische Produkt. Seien also~$X$
und~$Y$ Mengen. Dann wird das kartesische Produkt~$X \times Y$ vermöge der
kanonischen Projektionsabbildungen
\begin{align*}
  \pi_X : X \times Y \to X,\ (x,y) \mapsto x \\
  \pi_Y : X \times Y \to Y,\ (x,y) \mapsto y
\end{align*}
zu einem Möchtegern-Produkt von~$X$ und~$Y$:
\[ \xymatrix{
  & X \times Y \ar[ld]_{\pi_X} \ar[rd]^{\pi_Y} \\
  X & & Y
} \]
Um zu zeigen, dass dieses Möchtegern-Produkt ein tatsächliches Produkt von~$X$
und~$Y$ ist, müssen wir noch die universelle Eigenschaft nachweisen. Sei also
ein Möchtegern-Produkt~$(X \leftarrow \widetilde P \to Y)$ gegeben. Dann müssen
wir nachweisen, dass es genau einen Morphismus~$\psi:\widetilde P \to X \times
Y$ gibt, der die beiden Dreiecke im Diagramm
\[ \xymatrix{
    & X \times Y \ar[ld]_{\pi_X} \ar[rd]^{\pi_Y} \\
  X & & Y \\
    & \widetilde P \ar[lu]^{\widetilde \pi_X} \ar@{-->}[uu]_\psi \ar[ru]_{\widetilde \pi_Y}
  } \]
kommutieren lässt. Ausgeschreiben besagen die Kommutativitätsbedingungen, dass
für alle~$p \in \widetilde P$ die Gleichungen
\begin{align*}
  \text{(erste Komponente von $\psi(p)$)} &= \widetilde \pi_X(p) \\
  \text{(zweite Komponente von $\psi(p)$)} &= \widetilde \pi_Y(p)
\end{align*}
gelten sollen.
Es ist klar, dass diese beiden Bedingung genau durch eine Abbildung~$\psi$
erfüllt werden, nämlich durch
\[ \psi : \widetilde P \to X \times Y,\ p \mapsto (\widetilde \pi_X(p),
\widetilde \pi_Y(p)). \]
\item Der Produkt-Fall geht analog: Zusätzlich kann man jetzt voraussetzen,
dass~$\widetilde \pi_X$ und~$\widetilde \pi_Y$ Gruppenhomomorphismen sind; im
Gegenzug muss man aber nachweisen, dass die konstruierte Abbildung~$\psi$ ein
Gruppenhomomorphismus wird.
\item Übungsaufgabe.
\item Siehe Übungsblatt 2, Aufgabe 3. \qedhere
\end{enumerate}
\end{proof}


\subsection{Erste Eigenschaften}

\begin{prop}Die Objektteile je zweier Produkte von Objekten~$X$, $Y$ sind
zueinander isomorph.\end{prop}

\begin{bem}Es gilt sogar noch mehr, siehe Aufgabe~2 von Übungsblatt~2.\end{bem}

\begin{prop}Die Angabe eines Produkts von~$X$ und~$Y$ ist gleichwertig mit der
Angabe eines Produkts von~$Y$ und~$X$.\end{prop}

\begin{prop}Die Angabe eines Produkts von null vielen Objekten ist gleichwertig
mit der Angabe eines terminalen Objekts.\end{prop}


\section[Funktoren]{Funktoren \hfill \small Felicitas Hörmann}

So, wie es Gruppenhomomorphismen zwischen Gruppen gibt, gibt es Funktoren
zwischen Kategorien. Ihre beeindruckendste Anwendung liegt darin, dass sie
zwischen unterschiedlichen Teilgebieten der Mathematik vermitteln können -- das
ist ein Grundgedanke der algebraischen Topologie. Man verwendet sie aber auch,
um verschiedene Arten von Konstruktionen übersichtlich zu organisieren und
einen sinnvollen Rahmen für die Frage nach "`bestmöglichen"' Konstruktionen mit
vorgegebenem Ziel zu haben.

\begin{defn}Ein \emph{Funktor}~$F : \C \to \D$ zwischen Kategorien~$\C$, $\D$
besteht aus
\begin{enumerate}
\item einer Vorschrift, die jedem Objekt~$X \in \Ob \C$ ein Objekt~$F(X) \in \Ob \D$
zuordnet, und
\item einer Vorschrift, die jedem Morphismus~$f:X \to Y$ in~$\C$ einen
Morphismus~$F(f) : F(X) \to F(Y)$ zuordnet,
\end{enumerate}
sodass
\begin{enumerate}
\item $F(\id_X) = \id_{F(X)}$ für alle Objekte~$X \in \Ob \C$ und
\item $F(g \circ f) = F(g) \circ F(f)$ für alle komponierbaren Morphismen $g$, $f$
in~$\C$.
\end{enumerate}
\end{defn}
\begin{bem}\label{gleichheitfunktoren}%
Quelle und Ziel der abgebildeten Morphismen~$F(f)$ sind also durch
den Objektteil des Funktors schon vorgegeben. Es ist nicht sinnvoll, von der
Gleichheit von Funktoren~$F,G : \C \to \D$ zu sprechen -- denn das würde
naheliegenderweise ja die Aussage umfassen, dass für alle Objekte~$X \in \Ob \C$ die
Gleichheit
\[ F(X) = G(X) \]
von Objekten in~$\D$ gilt. Aber wie schon in Bemerkung~\ref{gleichheitobj}
festgehalten, ist das keine sinnvolle Aussage.
\end{bem}

\begin{prop}
Ein Funktor überführt kommutative Diagramme in kommutative Diagramme:
\[ \vcenter{ \xymatrix@=8ex{
  X \ar[r]^{f} \ar[rd]_{h} & Y \ar[d]^{g} \\
  & Z
} }
\qquad \overset{F}{\longmapsto} \qquad
\vcenter{ \xymatrix@=8ex{
  F(X) \ar[r]^{F(f)} \ar[rd]_{F(h)} & F(Y) \ar[d]^{F(g)} \\
  & F(Z)
} } \]
\end{prop}
\begin{proof}
Gilt $h = g \circ f$, so folgt~$F(h) = F(g \circ f) = F(g) \circ F(f)$.
\end{proof}


\subsection{Funktoren als Diagramme}

Es sei $\I$ die durch die folgende Skizze gegebene Kategorie und $\C$ eine beliebige Kategorie.

\[ \xymatrix{
  & \bullet_1 \ar[d] \ar@(ur,ul) \\
  \bullet_2 \ar[r] \ar@(ul,dl) & \bullet_3 \ar@(dr,ur)
} \]

Um einen Funktor $F : \I \to \C$ anzugeben, muss man
\begin{enumerate}
  \item Objekte~$X_1 = F(\bullet_1)$, $X_2 = F(\bullet_2)$ und~$X_3 =
  F(\bullet_3)$ in~$\C$ und
  \item Morphismen $f:X_1 \to X_3$ und $g:X_2 \to X_3$ in~$\C$
\end{enumerate}
spezifizieren. Ein solcher Funktor ist also durch ein Diagramm der Form

\[ \xymatrix{
  & X_1 \ar[d]^f \ar@(ur,ul) \\
  X_2 \ar[r]_g \ar@(ul,dl) & X_3 \ar@(dr,ur)
} \]
in~$\C$ gegeben. Da diese Überlegung analog mit anderen Kategorien~$\I$
funktioniert, sehen wir folgendes Motto:
\begin{motto}Funktoren~$\I \to \C$ sind~$\I$-förmige Diagramme
in~$\C$.\end{motto}

% XXX: gerichtete Graphen sind Funktoren (* ==> *) --> Set.


\subsection{Kontravariante Funktoren}

Wie kann man sich einen Funktor $F : \C^\op \to \D$ vorstellen?
\begin{enumerate}
  \item Objekte $X \in \Ob \C^\op = \Ob \C$ werden auf Objekte $F(X) \in \mathcal{D}$
  abgebildet.
  \item Morphismen $f : X \to Y$ in $\C^\op$ (d.\,h. $f : Y \to X$ in~$\C$)
  werden auf Morphismen $F(f) : F(X) \to F(Y)$ in $\mathcal{D}$ abgebildet.
\end{enumerate}
Das zweite Funktoraxiom lautet für Morphismen~$X \xra{f} Y \xra{g} Z$
in~$\C^\op$
\[ F(g \circ f) = F(f \bullet g) = F(f) \circ F(g), \] 
wobei wir zur Verdeutlichung "`$\circ$"' für die Komposition in $\C$ und
"`$\bullet$"' in $\C^\op$ schreiben. Die Zuordnung
\[ \begin{array}{@{}rcl@{}}
  \C &\longrightarrow& \D \\
  X  &\longmapsto& F(X) \\
  f  &\longmapsto& F(f)
\end{array} \]
ist also kein Funktor in unserem Sinne, da er Quelle und Ziel von Morphismen
vertauscht und das zweite Funktoraxiom dann nur in entsprechend umgekehrter
Kompositionsreihenfolge erfüllt. Solche Zuordnen sind trotzdem wichtig; sie
heißen \emph{kontravariante Funktoren}.


\subsection{Beispiele für Funktoren}

\subsubsection{Langweilige Funktoren}

\begin{enumerate}
  \item Für jede Kategorie $\C$ gibt es den \emph{Identitätsfunktor}
  \[ F : \C \to \C, \quad X \mapsto X, \quad f \mapsto f. \]
  \item Für ein festes Object $\heartsuit \in \C$ hat man den \emph{konstanten Funktor}
  \[ F : \I \to \C, \quad X \mapsto \heartsuit, \quad f \mapsto \id_\heartsuit. \]
\end{enumerate}

Diese Funktoren als solche sind langweilig. Interessant sind aber natürliche
Transformationen zwischen ihnen -- das werden wir im folgenden Vortrag sehen.


\subsubsection{Vergissfunktoren}

Die bekannten Strukturen in der Mathematik organisieren sich in einer
Hierarchie. Zwischen den Kategorien zu Strukturen verschiedener Stufen hat man sog.
Vergissfunktoren:

\begin{enumerate}
  \item Der Funktor
  \[ V : \Grp \to \Set, \quad (G,\circ) \mapsto G, \quad f \mapsto f. \]
  bildet Gruppen auf ihre zugrundeliegenden Mengen und Gruppenhomomorphismen
  auf ihre zugrundeliegenden Mengenabbildung ab. Er vergisst also die
  \emph{Struktur} der Gruppenverknüpfung.
  \item Der Funktor
  \[ V : \RR\text{-}\Vect \to \AbGrp, \quad (V,+,\cdot) \mapsto (V,+), \quad f \mapsto f. \]
  vergisst ebenfalls algebraische Struktur, nämlich die Skalarmultiplikation.
  \item Der Funktor
  \[ V : \Man \to \Top, \quad M \mapsto M, \]
  die einer Mannigfaltigkeit ihren zugrundeliegenden topologischen Raum
  zuordnet, vergisst (differentialgeometrische) Struktur.
  \item Der Funktor
  \[ V : \AbGrp \to \Grp, \quad (G,\circ) \mapsto (G,\circ), \quad f \mapsto f. \]
  vergisst die \emph{Eigenschaft} der Gruppenverknüpfung $\circ$, kommutativ zu sein.
  \item Schreibe $1$ für die Kategorie mit $\Ob = \lbrace \bullet \rbrace$ und $\Hom(\bullet,\bullet) = \lbrace \id_\bullet \rbrace$. Der Funktor
  \[ V : \Set \to 1, \quad M \mapsto \bullet, \quad f \mapsto \id_\bullet \]
  vergisst \emph{stuff}, also Zeug.
\end{enumerate}

Die Unterscheidung zwischen Eigenschaft, Struktur und Zeug stammt übrigens
von Teilnehmern eines Seminars über
Quantengravitation~\cite[Abschn.~2.4]{lectures-on-n-categories}, siehe
auch~\cite{ncatlab:stuff}.

Obwohl die Vergissfunktoren beinahe tautologisch definiert sind, sind sie aus
zwei Gründen wichtig: Zum einen ist es eine interessante Frage, inwieweit
man die Vergissfunktoren umkehren kann -- wie man etwa aus einer Menge eine
Gruppe machen kann. Wie diese Frage zu präzisieren und zu beantworten ist,
werden wir im Vortrag über adjungierte Funktoren lernen.

Zum anderen ist es wichtig zu wissen, ob ein Vergissfunktor Produkte (oder
allgemeinere Limiten) bewahrt. Etwa gilt für Vektorräume~$U, W$ und den
Vergissfunktor~$V:\RR\text{-}\Vect \to \Set$, dass
\[ V(U \times W) \cong V(U) \times V(W), \]
aber
\[ V(U \amalg W) \not\cong V(U) \amalg V(W). \]
Was das genau bedeutet, werden wir im Vortrag über Limiten sehen.


\subsubsection{Funktoren aus algebraischen Konstruktionen}

Zu jedem Ring~$R$ gibt es seinen Polynomring~$R[X]$ der formalen Polynome mit
Koeffizienten aus~$R$,
\[ R[X] = \Bigl\{ \sum_{i=0}^n a_i X^i \,\Big|\, a_0,\ldots,a_n \in R, n \geq 0
\Bigr\}. \]
Diese Konstruktion kann man zu einem Funktor erheben, den sog.
\emph{Polynomringfunktor} $F : \Ring \to \Ring$: Dieser ordnet einem Ring $R$
den Polynomring $R[X]$ und einem Ringhomomorphismus $f : R \to S$ folgenden
induzierten Ringhomomorphismus zu:
\[ F(f) : R[X] \to S[X], \quad \sum a_n X^n \mapsto \sum f(a_n) X^n. \]

\begin{bem}Algebraiker kann man daran erkennen, dass sie im Gegensatz zu
Analytikern die Polynomvariable groß schreiben.\end{bem}

Fast jede algebraische Konstruktion kann man auf diese Art und Weise behandeln.


\subsubsection{Funktoren und Mengen}

Zu jeder Menge $M$ gibt es die \emph{diskrete Kategorie} $DM$:
\begin{align*}
  \Ob DM &:= M \\
  \Hom_{DM}(m,\tilde{m}) &:=
  \left\{ \id_m \,\middle|\, m = \tilde m \right\}
\intertext{Die Angabe der Morphismenmengen ist etwas kryptisch geschrieben, ausführlich
kann man die Definition auch wie folgt angeben:}
  \Hom_{DM}(m,\tilde{m}) &:=
  \begin{cases}
    \lbrace \id_m \rbrace, & \text{falls $m = \tilde{m}$} \\
    \emptyset, & \text{sonst}
  \end{cases}
\end{align*}
Sind nun $M$ und $N$ zwei Mengen und $\varphi : M \to N$ eine Abbildung, so ist
\[ DM \to DN, \quad m \mapsto \varphi(m), \quad \id_m \mapsto \id_{\varphi(m)} \]
ein Funktor. [Hier fehlt eine Skizze.] Somit sehen wir folgendes Motto:
\begin{motto}Das Funktorkonzept verallgemeinert das Konzept der Abbildung
zwischen Mengen.\end{motto}

\subsubsubsection{Potenzmengenfunktoren}

Der \emph{kovariante Potenzmengenfunktor} $\mathcal{P} : \Set \to \Set$ ordnet einer Menge $M$ die Potenzmenge $\mathcal{P}(M)$ zu und einer Abbildung $f : M \to N$ die Abbildung
\[ \mathcal{P}(f) : \mathcal{P}(M) \to \mathcal{P}(N), \quad U \mapsto f[U], \]
wobei $f[U] := \left\{ f(u) : u \in U \right\}$ ist.

Definiert man $\mathcal{P}(f)$ stattdessen durch $U \mapsto f[U]^c$
(Komplement), so erhält man keinen Funktor.

Außerdem gibt es noch den \emph{kontravarianten Potenzmengenfunktor}
$\mathcal{P} : \Set^\op \to \Set$, der ebenfalls jeder Menge~$M$ ihre
Potenzmenge, aber jeder Abbildung~$f : M \to N$ die \emph{Urbild}abbildung
\[ \mathcal{P}(f) : \mathcal{P}(N) \to \mathcal{P}(M), \quad V \mapsto
f^{-1}[V] \]
zuordnet, wobei~$f^{-1}[V] := \left\{ x \in M \,|\, f(x) \in V \right\}$.
Dieser ist sehr bedeutsam, denn er zeigt die Äquivalenz der dualen
Kategorie~$\Set^\op$ mit der Kategorie vollständiger atomischer boolescher
Algebren, siehe~\cite[Thm.~2.4]{oosten}. Was \emph{Äquivalenz} bedeutet, werden
wir im folgenden Kapitel lernen.


\subsubsection{Funktoren und Gruppen}

Es sei ein Gruppenhomomorphismus $\varphi : G \to H$ gegeben. Dann ist
  \[ f : BG \to BH, \quad \bullet \mapsto \bullet, \quad g \mapsto \varphi(g) \]
ein Funktor. (Zur Konstruktion der Kategorien $BG$ und $BH$ siehe Übungsblatt~1, Aufgabe~5.)
Denn das erste Funktoraxiom ist erfüllt,
\[
  F(\id_\bullet) = F(e_G) = \varphi(e_G) = e_H = \id_\bullet,
\]
und das zweite ebenso: Für alle Morphismen~$g, \tilde g : \bullet \to \bullet$
(d.\,h. für alle Gruppenelemente~$g, \tilde g \in G$) gilt
\[
  F(\tilde{g} \circ g) = F(\tilde{g} \cdot g) = \varphi(\tilde{g} \cdot g) =
  \varphi(\tilde{g}) \cdot \varphi(g) = \varphi(\tilde{g}) \circ \varphi(g) =
  F(\tilde{g}) \circ F(g). \]
Damit sehen wir folgendes Motto:
\begin{motto}Das Funktorkonzept verallgemeinert das Konzept des
Gruppenhomomorphismus.\end{motto}

\subsubsubsection{Gruppenwirkungen}

Was muss man angeben, um einen Funktor $F : BG \to \Set$ zu spezifizieren? Eine
Menge $M := \varphi(\bullet)$ und zu jedem $g \in G$ eine Abbildung $\varphi_g : M \to M$, sodass
\[ \varphi_{\id_\bullet} = \id_M \quad \text{und} \quad \varphi_{g \circ h} = \varphi_g \circ \varphi_h \]
für alle $g, h \in G$ gilt. Mit der Schreibweise $\varphi_g(x) =: g \cdot x$, $g \in G$, $x \in X$, wird dies zu
\[ e \cdot x = x \quad \text{und} \quad (g \circ h) \cdot x = g \cdot (h \cdot x).\]
Eine solche Struktur bestehend aus einer Menge~$M$ und einer
Multiplikationsabbildung~$G \times M \to M$, die diese Axiome erfüllt, ist eine
sog. \emph{Gruppenwirkung von~$G$}. Wir sehen also: Funktoren~$BG \to \Set$
sind "`dasselbe"' wie Gruppenwirkungen von~$G$.

Analog kann man Funktoren~$BG \to K\text{-}\Vect$ untersuchen. Solche haben
auch einen klassischen Namen: Das sind sog. \emph{Gruppendarstellungen}.


\subsubsection{Funktoren als Datenbanken}

Wir wollen an einem Beispiel zeigen, dass auch so konkrete Dinge wie
Datenbanken aus der Informatik kategoriell verstanden werden können. Etwa gibt
das zugrundeliegende Datenbankschema der 0815/Datenbank aus Tafel~\ref{db0815}
Anlass zu folgender Kategorie~$\C$:

\[ \xymatrixcolsep{5pc} \xymatrixrowsep{5pc} \xymatrix{
  \text{Angestellte}
    \ar@(ul,ur)^{\text{Vorg.}}
    \ar[r]^{\text{Abt.}}
    \ar@/^2pc/[d]^{\text{Nachname}}
    \ar@/_2pc/[d]_{\text{Vorname}}
  & \text{Abteilung}
    \ar@/^3pc/[dl]^{\text{Titel}} \\
  \text{String}
} \]

Die Tabelleninhalte kann man dann über einen Funktor~$\C \to \Set$ kodieren,
der jedes Objekt (also jeden Tabellennamen) auf die Menge der Primärschlüssel
ihrer Zeilen und jeden Morphismus (also jeden Spaltennamen) auf die
entsprechende Abbildung zwischen den Primärschlüsseln der beteiligten Tabellen
abbildet.

Gewisse einfache Integritätsbedingungen kann man über die Angabe eines
geeigneten Kompositionsgesetzes in~$\C$ kodieren. Wenn man etwa ausdrücken
möchte, dass der Sekretär einer Abteilung selbst in dieser sitzt, kann man
\[ \text{Abt.} \circ \text{Sekretär} = \id_{\text{Abteilung}} :
  \text{Abteilung} \to \text{Abteilung} \]
definieren. Diese Sichtweise auf Datenbanken ist unter Anderem für das
Verständnis von Datenmigrierung bei Schemaänderungen hilfreich. Details hat
David Spivak erforscht~\cite{spivak1,spivak2,spivak3}.

\begin{figure}
  \begin{center}
    \small
    \begin{tabular}{|l||l|l|l|l|}
      \hline
      \multicolumn{5}{|c|}{Angestellte} \\ \hline
      \textbf{Nr.} & \textbf{Vorname} & \textbf{Nachname} & \textbf{Vorg.} & \textbf{Abt.} \\ \hline
      101 & David & Hilbert & 103 & q10 \\
      102 & Bertrand & Russel & 102 & x02 \\
      103 & Alan & Turing & 103 & q10 \\
      \hline
    \end{tabular}
    \quad
    \begin{tabular}{|l||l|l|}
      \hline
      \multicolumn{3}{|c|}{Abteilung} \\ \hline
      \textbf{Nr.} & \textbf{Titel} & \textbf{Sekretär} \\ \hline
      q10 & Vertrieb & 101 \\
      x02 & Produktion & 102 \\
      \hline
    \end{tabular}
  \end{center}

  \caption{\label{db0815}Ein Standardbeispiel einer Datenbank.}
\end{figure}

\subsubsection{Hom-Funktoren}

\begin{defn}
Sei $\C$ eine lokal kleine Kategorie (sodass ihre Hom-Klassen sogar schon
Hom-Mengen sind) und $A \in \Ob \C$. Dann ist\ldots
\begin{enumerate}
  \item der \emph{kovariante Hom-Funktor zu $A$} der Funktor
    \[ \begin{array}{@{}rrcl@{}}
      \Hom_\C(A,\freist): & \C &\longrightarrow& \Set \\
      & X &\longmapsto& \Hom_\C(A,X) \\
      & (f:X \to Y) &\longmapsto& f_\star
    \end{array} \]
  \item und der \emph{kontravariante Hom-Funktor zu~$A$} der Funktor
    \[ \begin{array}{@{}rrcl@{}}
      \Hom_\C(\freist,A): & \C^\op &\longrightarrow& \Set \\
      & X &\longmapsto& \Hom_\C(X,A) \\
      & (f:X \xra{\C} Y) &\longmapsto& f^\star.
    \end{array} \]
\end{enumerate}
Dabei sind die Abbildungen~$f_\star, f^\star$ wie folgt definiert:
\[ \begin{array}{@{}rrcl@{}}
  f_\star: & \Hom_\C(A,X) &\longrightarrow& \Hom_\C(A,Y) \\
  & g &\longmapsto& f \circ g \\
  \\
  f^\star: & \Hom_\C(Y,A) &\longrightarrow& \Hom_\C(X,A) \\
  & g &\longmapsto& g \circ f
\end{array} \]
\end{defn}

Die Hom-Funktoren kodieren die Beziehungen von~$A$ mit den Objekten aus~$\C$.
Das zentrale \emph{Yoneda-Lemma} wird uns sagen, dass~$A$ durch Kenntnis des
ko- oder kontravarianten Hom-Funktors schon bis auf Isomorphie eindeutig
bestimmt ist.


\subsubsection{Weitere Beispiele}

\begin{enumerate}
\item Den Prozess des Differenzierens glatter Abbildungen zwischen
Mannigfaltigkeiten kann man als Funktor auffassen, der jeder Mannigfaltigkeit
ihr Tangentialbündel und jeder glatter Abbildung ihr Differential zuordnet:
\[ \begin{array}{@{}rcl@{}}
  \Man &\longrightarrow& \Man \\
  M &\longmapsto& TM \\
  f &\longmapsto& Df
\end{array} \]
Es gibt auch eine "`lokale Version"', wenn man die Kategorie der
\emph{punktierten glatten Mannigfaltigkeiten}~$\Man_\star$ betrachtet: Die
Objekte dieser Kategorie sind Tupel $(M,x)$ aus einer Mannigfaltigkeit und
einem ausgezeichneten Basispunkt~$x \in M$, Morphismen sind
basispunkterhaltende glatte Abbildungen. Dann hat man den Funktor
\[ \begin{array}{@{}rcl@{}}
  \Man_\star &\longrightarrow& \RR\text{-}\Vect \\
  (M,x) &\longmapsto& T_x M \\
  f &\longmapsto& d_x f.
\end{array} \]
In beiden Fällen ist das zweite Funktoraxiom gerade deswegen erfüllt, weil die
Kettenregel gilt!

\item Hier könnte dein Beispiel stehen.
\end{enumerate}


\subsection{Die Kategorie der Kategorien}

Nach dem fundamentalen Motto der Kategorientheorie sollen wir die Beziehungen
zwischen Untersuchungsgegenständen ernst nehmen und daher die von ihnen
gebildete \emph{Kategorie} betrachten. Als wir bisher Kategorientheorie
betrieben haben, haben wir dieses Motto bezogen auf Kategorien selbst aber
sträflich vernachlässigt! Diesen Missstand behebt folgende Definition.
\begin{defn}
Die Kategorie $\Cat$ der (kleinen) Kategorien besteht aus:
\begin{align*}
  \Ob \Cat &:= \text{Klasse aller (kleinen) Kategorien} \\
  \Hom_\Cat(\C,\mathcal{D}) &:= \text{Klasse der Funktoren zwischen $\C$ und $\mathcal{D}$}
\end{align*}
\end{defn}
Die Verkettung~$G \circ F : \C \to \E$ zweier Funktoren~$F : \C \to \D$ und~$G
: \D \to \E$ ist dabei als der Funktor
\[ \begin{array}{@{}rrcl@{}}
  G \circ F: & \C &\longrightarrow& \E \\
  & X &\longmapsto& G(F(X)) \\
  & f &\longmapsto& G(F(f))
\end{array} \]
definiert.

\begin{bem}Ironischerweise ist es keine gute Idee, die so definierte
Kategorie~$\Cat$ zu untersuchen: Denn in Kategorien muss es sinnvoll sein, von
der Gleichheit zweier Morphismen zu sprechen -- der Gleichheitsbegriff zwischen
Funktoren ist aber, wie eingangs schon bemerkt, nicht interessant. Tatsächlich ist
die Kategorie~$\Cat$ nur eine erste Approximation an eine sog. 2-Kategorie, in
der es nicht nur Morphismen (Funktoren) zwischen Objekten (Kategorien), sondern
auch "`höhere Morphismen"', sog. 2-Morphismen (hier natürliche
Transformationen), zwischen den gewöhnlichen (1-)Morphismen gibt.
\end{bem}

% TODO:
% * Was bedeutet Vorschrift in der Funktordefinition?
%   * Eindeutigkeit der Zuordnung
% * Rückfrage, dass keine Kommutativitätsbedingung vorhanden ist
% * I darf beim konstanten Funktor auch leer sein
% * Gegenbspfkt.: Komplement, V |-> Menge seiner Basen
% * Q/R-Struktur
% * BG: Basisraum


\section[Natürliche Transformationen]{Natürliche Transformationen \hfill \small
Tim Baumann}

\emph{Werbung:} Wir werden verstehen, was natürliche Transformationen sind,
weshalb ihre Definition ganz einfach ist und wozu man sie benötigt. Ihre
Bedeutung werden wir aus verschiedenen Blickwinkeln beleuchten. Mit natürlichen
Transformationen können wir dann auch Funktorkategorien definieren, die für das
Yoneda-Lemma später sehr wichtig sind. Außerdem können wir definieren, wann
zwei Kategorien zueinander äquivalent sind.

\textbf{XXX:} Hier fehlt noch Motivation für das Konzept.

\begin{defn}Eine \emph{natürliche Transformation} $\eta : F \Rightarrow G$
zwischen Funktoren $F, G : \C \to \D$ besteht aus
\begin{enumerate}
\item[] einem Morphismus~$\eta_X : F(X) \to G(X)$ für jedes Objekt~$X \in \Ob \C$
\end{enumerate}
sodass
\begin{enumerate}
\item[]
für alle Morphismen $f : X \to Y$ in~$\C$ das Diagramm
\[ \xymatrix{
  F(X) \ar[r]^{F(f)} \ar[d]_{\eta_X} & F(Y) \ar[d]^{\eta_Y} \\
  G(X) \ar[r]_{G(f)} & G(Y)
} \]
kommutiert.
\end{enumerate}
\end{defn}

\begin{motto}
Die Komponenten einer natürlichen Transformation sind \emph{gleichmäßig} über
alle Objekte $X \in \Ob\C$ definiert.
\end{motto}


\subsection{Beispiele für natürliche Transformationen}


\subsubsection*{Erste Beispiele mit Mengen}

Seien $\Id_\Set, K: \Set \to \Set$ die Funktoren mit
\[ \begin{array}{@{}rrcl@{}}
  \Id_\Set: & X &\longmapsto& X \\
  & f &\longmapsto& f \\\\
  K: & X &\longmapsto& X \times X \\
  & f:X\to Y &\longmapsto& (X \times X \to Y \times Y,\ (x_1,x_2) \mapsto (f(x_1),f(x_2)).
\end{array} \]

Dann kann man folgende Beobachtungen treffen:

\begin{enumerate}
\item Natürlich gibt es für jede konkrete Menge~$X$ im Allgemeinen viele
Abbildungen
\[ X \longrightarrow X. \]
Aber es gibt nur eine natürliche Transformation~$\eta : \Id_\Set \Rightarrow
\Id_\Set$, nämlich die mit
\[ \eta_X : X \longrightarrow X,\ x \longmapsto x. \]
Wir sehen das Motto in diesem Beispiel bestätigt: Denn der
Funktionsterm von~$\eta_X$ ist in der Tat gleichmäßig definiert, es kommt keine
Fallunterscheidung über~$X$ vor.
\item Analog gibt es für jede konkrete Menge~$X$ im Allgemeinen viele
Abbildungen~$X \to X \times X$ (also~$\Id_\Set(X) \to K(X)$). Aber es gibt nur
eine einzige natürliche Transformation $\eta : \Id_\Set \Rightarrow K$, nämlich
die mit
\[ \eta_X: X \to X \times X,\ x \mapsto (x,x) \]
für alle Mengen~$X$. Auch hier ist das Motto bestätigt.
\item Für konkrete Mengen~$X$ gibt es im Allgemeinen viele Abbildungen
\[ \P(X) \longrightarrow X, \]
aber es gibt keine natürliche Transformation $\P \Rightarrow \Id_\Set$.
Auch dieser Sachverhalt illustriert das Motto: Denn uns fällt kein
Abbildungsterm ein, der ohne Fallunterscheidung über~$X$ Funktionen des
Typs~$\P(X) \to X$ definieren könnte.
\end{enumerate}


\subsubsection*{Entgegengesetzte Gruppe}

In der Gruppentheorie trifft man folgende Beobachtung: \emph{Jede
Gruppe~$(G,\circ)$ ist natürlich isomorph zu ihrer entgegengesetzten Gruppe
$(G^\op,\bullet)$.}
Dabei hat~$G^\op$ dieselben Elemente wie~$G$, die
Gruppenverknüpfung~$\bullet$ ist aber
genau anders herum definiert,
\[ g \bullet h := h \circ g. \]
In der Tat ist die Abbildung
\[ \begin{array}{@{}rrcl@{}}
  \eta_G : & G &\longrightarrow& G^\op \\
  & g &\longmapsto& g^{-1}
\end{array} \]
bijektiv und auch wirklich ein Gruppenhomomorphismus, da für alle~$g,h \in G$
die Rechnung
\[ \eta_G(g \circ h) = (g \circ h)^{-1} = h^{-1} \circ g^{-1} =
  g^{-1} \bullet h^{-1} = \eta_G(g) \bullet \eta_G(h) \]
gilt. Ohne den Begriff der natürlichen Transformation kann man aber nicht
verstehen, wieso dieser Isomorphismus das Prädikat \emph{natürlich} verdient
hat: Man kann sich nur mit der Aussage begnügen, der Isomorphismus sei
\emph{kanonisch} definiert; das ist jedoch ein informaler Begriff.

Kategoriell verstehen wir: Die Isomorphismen~$\eta_G$ sind Komponenten einer
natürlichen Transformation, und zwar einer vom Identitätsfunktor auf~$\Grp$ in
den "`entgegengesetzte Gruppe"'-Funktor~$F$:
\[ \begin{array}{@{}rrcl@{}}
  F : & \Grp &\longrightarrow& \Grp \\
  & G &\longmapsto& G^\op \\
  & f &\longmapsto& f^\op
\end{array} \]
Der Gruppenhomomorphismus~$f^\op$ ist als Abbildung derselbe wie~$f$; durch das
doppelte Bilden der entgegengesetzten Gruppen ist er auch wirklich ein
Gruppenhomomorphismus. Das Natürlichkeitsdiagramm
\[ \xymatrixcolsep{4pc}\xymatrixrowsep{4pc}\xymatrix{
  \Id_\Grp(G)=G \ar[rr]^{f} \ar[d]_{\eta_G} && H=\Id_\Grp(H) \ar[d]_{\eta_H} \\
  F(G)=G^\op \ar[rr]_{f^\op} && H^\op=F(H)
} \]
kommutiert tatsächlich, wie eine Diagrammjagd zeigt:
\[ \xymatrixcolsep{4pc}\xymatrixrowsep{4pc}\xymatrix{
  g \ar@{|->}^f[rr] \ar@{|->}^{\eta_G}[d] & & f(g) \ar@{|->}^{\eta_H}[d] \\
  g^{-1} \ar@{|->}^f[r]  & f(g^{-1}) \ar@{=}_{\text{$f$ Homo}}[r] & (f(g))^{-1}
} \]


\begin{bem}\label{interpretnat}%
Manchmal findet man Aussagen der Art "`es gibt eine natürliche Abbildung
von $\ldots$ nach $\ldots$"' in der Literatur. Damit ist dann oft gemeint, dass
man Quelle und Ziel als Auswertungen zweier Funktoren verstehen kann und dass
zwischen diesen Funktoren eine natürliche Transformation verläuft.\end{bem}

\textbf{XXX:} Es fehlt noch ein weiteres Beispiel:
"`Doppeldualraum"'


\subsection{Funktorkategorien}

\begin{defn}
Seien Funktoren $F,G,H: \C \to \D$ und natürliche Transformationen $\alpha: F
\Rightarrow G$ und $\beta: G \Rightarrow H$ gegeben:
\[ \xymatrix{\C \ar@/^1.7em/[rr]^F_{\Downarrow \alpha} \ar[rr]^G \ar@/_1.7em/[rr]^{\Downarrow \beta}_H & & \D} \]
Dann heißt $\beta \circ \alpha:F \Rightarrow G$ die \emph{(vertikale)
Verkettung} von~$\alpha$ und~$\beta$ und ist komponentenweise durch
\[ (\beta \circ \alpha)_X := \beta_X \circ \alpha_X : F(X) \to H(X) \]
gegeben.
\end{defn}

\begin{prop}Die so definierte Zuordnung~$\beta \circ \alpha$ ist in der Tat
eine natürliche Transformation.\end{prop}
\begin{proof}Da für alle~$f:X \to Y$ in~$\C$ die beiden Teilquadrate
im Diagramm
\[ \xymatrix{
  F(X) \ar[r]^{F(f)} \ar[d]_{\alpha_X} \ar@/_ 1cm/[dd]_{(\beta \circ \alpha)_X := \beta_X \circ \alpha_X} & F(Y) \ar[d]_{\alpha_Y} \ar@/^ 1cm/[dd]^{\beta_Y \circ \alpha_Y =: (\beta \circ \alpha)_Y} \\
  G(X) \ar[r]^{G(f)} \ar[d]_{\beta_X} & G(Y) \ar[d]_{\beta_Y} \\
  H(X) \ar[r]^{H(f)} & H(Y)
} \]
kommutieren, kommutiert auch das äußere Rechteck. Das ist gerade das
Natür\-lich\-keits\-dia\-gramm für~$\beta\circ\alpha$.
\end{proof}

Außerdem gibt es für jeden Funktor~$F:\C\to\D$ eine natürliche
Identitätstransformation $\id_F : F \Rightarrow F$ mit $\Id_X := id_{F(X)}$.
Damit wird folgende Definition möglich:
\begin{defn}
Die \emph{Funktorkategorie} $\Funct(\C,\D)$ zu zwei Kategorien $\C$ und $\D$
ist die Kategorie mit Funktoren $F: \C \to \D$ als Objekten und natürlichen
Transformationen als Morphismen.
\end{defn}

\textbf{XXX:} Es fehlt noch eine Bemerkung über die horizontale Verkettung von
natürlichen Transformationen.

\begin{lemma}\label{natTransIsoLemma}
Seien $\C,\D$ Kategorien, $F,G: \C \to \D$ Funktoren und $\alpha: F \Rightarrow
G$ eine natürliche Transformation. Dann ist $\alpha$ genau dann ein
Isomorphismus in der Funktorkategorie~$\Funct(\C,\D)$, wenn alle Komponenten
$\alpha_X$, $X \in \Ob\C$, jeweils Isomorphismen in~$\D$ sind.
\end{lemma}

\begin{proof}\begin{itemize}
\item[\glqq$\Rightarrow$\grqq] Sei $\alpha$ ein Isomorphismus, dann existiert
also eine natürliche Transformation $\alpha^{-1}$ mit $\alpha \circ \alpha^{-1}
= \id_F$, $\alpha^{-1} \circ \alpha = \id_G$. Das bedeutet, dass für jedes $X
\in \Ob\C$ die Gleichheiten
\begin{align*}
	\id_{F(X)} = (\id_F)_X = (\alpha \circ \alpha^{-1})_X = \alpha_X \circ \alpha^{-1}_X \\	
	\id_{G(X)} = (\id_G)_X = (\alpha^{-1} \circ \alpha)_X = \alpha^{-1}_X \circ \alpha_X
\end{align*}
gelten. Also sind die Komponenten $\alpha_X$ jeweils Isomorphismen in $\D$.
\item[\glqq$\Leftarrow$\grqq] Seien alle Komponenten~$\alpha_X$ invertierbar.
Dann können wir versuchen, eine inverse natürliche Transformation $\beta: G
\Rightarrow F$ über die Setzung
\[ \beta_X := (\alpha_X)^{-1} : G(X) \to F(X) \]
zu definieren. Sicher gilt dann $\alpha \circ \beta = \id_F$ und $\beta \circ \alpha
= \id_G$, aber es bleibt noch zu zeigen, dass $\beta$ auch wirklich eine natürliche
Transformation ist. Dazu rechnen wir für jeden Morphismus~$f: X \to Y$ die
Natürlichkeitsbedingung nach:
\begin{align*}
  & G(f) \circ \alpha_X = \alpha_Y \circ F(f) \\
  \Longrightarrow{}& (\alpha_Y)^{-1} \circ G(f) \circ \alpha_X \circ (\alpha_X)^{-1} = (\alpha_Y)^{-1} \circ \alpha_Y \circ F(f) \circ (\alpha_X)^{-1} \\
  \Longrightarrow{}& (\alpha_Y)^{-1} \circ G(f) = F(f) \circ (\alpha_X)^{-1} \\
  \Longrightarrow{}& \beta_Y \circ G(f) = F(f) \circ \beta_X \qedhere
\end{align*}
\end{itemize}
\end{proof}

\begin{defn}
Invertierbare natürliche Transformationen heißen auch \emph{natürliche
Isomorphismen}, und Funktoren, zwischen denen ein natürlicher Isomorphismus
verläuft, heißen \emph{zueinander (natürlich) isomorph}.
\end{defn}


\subsection{Kategorienäquivalenzen}

\begin{defn}
Eine \emph{Kategorienäquivalenz} zwischen Kategorien~$\C$ und~$\D$ besteht aus
Funktoren
\[ \xymatrix{\C \ar@/^/[r]^F & \D, \ar@/^/[l]^G} \]
die zueinander \emph{quasi-invers} sind, d.\,h. dass die Kompositionen von~$F$
und~$G$ jeweils natürlich isomorph zu den entsprechenden Identitätsfunktoren
sind:
\[ G \circ F \cong \Id_\C, \qquad F \circ G \cong \Id_\D. \]
Die Kategorien~$\C$ und~$\D$ heißen dann \emph{zueinander äquivalent}:
$\C \simeq \D.$
\end{defn}

Es gibt auch das Konzept der \emph{Isomorphie von Kategorien}. Da fordert man,
dass es Funktoren
\scalebox{0.6}{$\xymatrix{\C \ar@/^/[r]^F & \D
\ar@/^/[l]^G}$} gibt, die
zueinander nicht nur quasi-invers, sondern tatsächlich invers sind, d.\,h. die
Beziehungen
\[ G \circ F = \Id_\C, \qquad F \circ G = \Id_\D \]
erfüllen. Das ist aber in den meisten Fällen kein gutes Konzept:
Denn wie schon in Bemerkung~\ref{gleichheitfunktoren} festgehalten, ist die
Gleichheit von Funktoren eine böse Bedingung. In der Tat sind die meisten in
der Natur vorkommenden Kategorienäquivalenzen auch "`nur"' Äquivalenzen, keine
Isomorphismen. Kommt doch mal ein Isomorphismus von Kategorien vor, so ist das
meist ein überhaupt nicht tiefsinniger "`technischer Zufall"', der nur an
geeigneten Wahlen bestimmter Definitionen liegt.

Wieso man Äquivalenzen von Kategorien untersucht, liegt in folgendem
Motto begründet:
\begin{motto}\label{mottoeqv}%
Sei~$\varphi$ eine mathematische Aussage über Kategorien, die
sich nur unter Verwendung der Konzepte \emph{Objekt}, \emph{Morphismus},
\emph{Verkettung von Morphismen} und \emph{Gleichheit von Morphismen}
formulieren lässt. Sind dann~$\C$ und~$\D$ zueinander äquivalente Kategorien,
so gilt~$\varphi$ genau dann in~$\C$, wenn~$\varphi$ in~$\D$ gilt.
\end{motto}
Beispiele für Aussagen dieser Art sind etwa:
\begin{itemize}
\item Die Kategorie besitzt ein initiales Objekt.
\item Je zwei parallele Morphismen sind gleich.
\item Jeder Morphismus in ein initiales Objekt ist sogar schon ein
Isomorphismus.
\end{itemize}
Beispiele für Aussagen, die über die Reichweite des Mottos hinausgehen, sind:
\begin{itemize}
\item Die Kategorie besitzt genau ein Objekt.
\item Die Kategorie besitzt genau ein initiales Objekt.
\item Sind zwei Objekte zueinander isomorph, so sind sie schon gleich.
\item Je zwei Morphismen (egal zwischen welchen Objekten) sind gleich.
\end{itemize}
Man erachtet es nicht als schlimm, dass diese Aussagen nicht unter Äquivalenz
erhalten bleiben. Denn wegen der vorkommenden Vergleiche von Objekten auf Gleichheit
handelt es sich sowieso um böse Aussagen.

\begin{bem}Mit Techniken aus der formalen Logik kann man Motto~\ref{mottoeqv}
auch rigoros beweisen. Das ist nicht besonders schwer, die Hauptschwierigkeit
liegt darin, den Begriff \emph{Aussage} präzise zu definieren.\end{bem}

\begin{bsp}Die duale Kategorie~$\Set^\op$ kann nicht zu~$\Set$ äquivalent sein.
Denn in~$\Set$ ist jeder Morphismus in ein initiales Objekt schon ein
Isomorphismus, in~$\Set^\op$ aber nicht.\end{bsp}

\begin{bsp}
Die Kategorie $\Eins$, die nur aus einem Objekt~$\star$ sowie dessen
Iden\-ti\-täts\-mor\-phis\-mus besteht, ist zu jeder bewohnten
\emph{indiskreten Kategorie} $\C$ (d.\,h. einer solchen, die mindestens ein
Objekt besitzt und in der zwischen je zwei Objekten genau ein Morphismus
verläuft) äquivalent. 
\end{bsp}
\begin{proof}
Da~$\C$ bewohnt ist, gibt es ein Objekt $K \in \Ob \C$. Dann können wir die Funktoren
\[ \begin{array}{@{}rrcl@{}}
  F: & \Eins &\longrightarrow& \C \\
  & \star &\longmapsto& K \\
  & \id_\star &\longmapsto& \id_K \\\\
  G: & \C &\longrightarrow& \Eins \\
  & X &\longmapsto& \star \\
  & f &\longmapsto& \id_\star
\end{array} \]
definieren.
Da es nur einen einzigen Funktor von $\Eins$ nach $\Eins$ gibt (obacht! Es ist schlecht, das zu sagen), ist klar, dass $G \circ F \cong \Id_\Eins$ gilt. Noch zu zeigen ist also, dass auch $F \circ G \cong \Id_\C$ gilt.

Dazu definieren eine natürliche Transformation $\eta: F \circ G \Rightarrow
\Id_\C$, die sich als natürlicher Isomorphismus herausstellen wird. Dabei
verwenden wir für jedes $X \in \C$ als $\eta_X$ den \emph{eindeutigen}
Morphismus $(F\circ G) (X) \to \Id_\C(X)$. Da diese Definition gleichmäßig
in~$X$ ist, erwarten wir, dass die Natürlichkeitsbedingung erfüllt ist; und das
ist auch in der Tat der Fall:
\[ \xymatrixcolsep{5pc}\xymatrixrowsep{5pc}\xymatrix{
  K = (F \circ G)(X) \ar[r]^{(F \circ G)(f)=\id_K} \ar[d]_{\eta_X} & (F \circ G)(Y) = K \ar[d]^{\eta_Y} \\
  X = \Id_\C (X) \ar[r]_{\Id_\C(f)=f} & \Id_\C (Y) = Y
} \]
Da in einer indiskreten Kategorie alle Morphismen Isomorphismen sind, ist
insbesondere jede Komponente~$\eta_X$ ein Isomorphismus, und damit ist nach
Lemma \ref{natTransIsoLemma} auch $\eta$ selbst ein Isomorphismus.
\end{proof}

\begin{bem}Wer topologische Räume kennt, fühlt sich durch das Beispiel
vielleicht an folgende Beobachtung erinnert: Bewohnte topologische Räume, in
denen je zwei Punkte bis auf Homotopie durch genau einen Weg miteinander
verbunden werden können, sind zusammenziehbar. Die Ähnlichkeit ist nicht nur
formal: Jedem topologischen Raum kann man sein \emph{Fundamentalgruppoid}
zuordnen, das ist die Kategorie, deren Objekte durch die Punkte des Raums und
deren Morphismen durch die Homotopieklassen von Wegen gegeben sind. Der
Fundamentalgruppoid eines Raums, der obige Eigenschaft erfüllt, ist
indiskret.\end{bem}
% XXX: Da fehlt noch mehr!

% "konstante natürliche Trafo"
% weniger Abbildungen --> mehr Trafos!
% F ==> G ohne G ==> F
% eta_G Gruppenhomo? Andere Möglichkeiten?
% Bsp. für Äquivalenzen, die keine Isos sind
% Grund, wieso Isos schlecht (übel) sind


\section[Limiten und Kolimiten]{Limiten und Kolimiten \hfill \small
Kathrin Gimmi}

\emph{Werbung:} Wir verstehen, was allgemeine Limiten und Kolimiten von
Diagrammen sind. Dazu wird es viele Beispiele geben, unter anderem die uns
schon bekannten Produkte und Koprodukte. Speziell sind sog. filtrierte
Kolimiten wichtig, da diese in der täglichen Praxis oft vorkommen und besonders
schöne Eigenschaften haben. Abschließend werden wir die Frage diskutieren, wie
man Kategorien, denen es an Limiten oder Kolimiten mangelt, vervollständigen
kann.

In diesem Abschnitt wollen wir Funktoren~$\I \to \C$ auch als~($\I$-förmige)
Diagramme bezeichnen.

\begin{defn}Ein \emph{Kegel} eines Diagramms~$F : \I \to \C$ besteht aus
\begin{enumerate}
\item einem Objekt~$K \in \Ob \C$ (der sog. \emph{Kegelspitze}) zusammen mit
\item jeweils einem Morphismus~$\pi_i : K \to F(i)$ für jedes Objekt~$i \in \Ob\I$,
\end{enumerate}
sodass
für alle Morphismen~$f : i \to j$ in~$\I$ die Dreiecke
\[ \xymatrix{
  & K \ar[ld]_{\pi_i} \ar[rd]^{\pi_j} \\
  F(i) \ar[rr]_{F(f)} && F(j)
} \]
kommutieren.
\end{defn}
Die Notation etwas missbrauchend werden Kegel oft nur nach ihrer Kegelspitze
genannt, obwohl die Morphismen~$\pi_i$ mit zum Datum gehören.
Die Morphismen~$\pi_i$ werden manchmal als Projektionsmorphismen
bezeichnet, der Grund dafür wird beim ersten Beispiel klar werden.

\begin{defn}Ein \emph{Morphismus von Kegeln $K \to \widetilde K$} eines Diagramms~$F$ besteht aus
\begin{enumerate}
\item[] einem Morphismus der Kegelspitzen $\psi : K \to \widetilde K$ in~$\C$
\end{enumerate}
sodass
\begin{enumerate}
\item[]
für alle~$i \in \Ob\I$ die Dreiecke
\[ \xymatrix{
  K \ar[rr]^{\psi} \ar[rd]_{\pi_i} && \widetilde K \ar[ld]^{\widetilde\pi_i} \\
  & F(i)
} \]
kommutieren.
\end{enumerate}
\end{defn}
Abbildung~\ref{kegel} erklärt die Herkunft des Begriffs "`Kegel"'.

\begin{figure}
  \[
    \xymatrix{
      K \ar@[grey]@{-->}[rrrrr]
      \ar@[grey][dddr] \ar@[grey][dddrrr] \ar@[grey][dddrrrr] \ar@[grey][ddddrr] &&&&&
      \widetilde K \ar@[grey][dddl] \ar@[grey][dddll] \ar@[grey][dddllll]
      \ar@[grey][ddddlll]
      \\\\\\
      & F(i) \ar[rr] \ar[rd] && F(j) \ar[r] \ar[ld] & F(\ell) \\
      & & F(k)
    }
  \]
  \caption{\label{kegel}Zwei Kegel und ein Kegelmorphismus zwischen ihnen.}
\end{figure}

Kegelmorphismen kann man auf die offensichtliche Art und Weise miteinander
verketten (einfach die Morphismen der Kegelspitzen verketten). Daher ist es
sinnvoll, von der \emph{Kategorie der Kegel} zu einem festen
Diagramm~$F:\I\to\C$ zu sprechen. Terminale Objekte dieser Kategorie haben
einen besonderen Namen:
\begin{defn}Ein \emph{Limes} eines Diagramms~$F:\I\to\C$ ist ein terminales
Objekt in der Kategorie der Kegel zu~$F$.\end{defn}
Da allgemein terminale Objekte einer Kategorie bis auf eindeutige Isomorphie
eindeutig sind (siehe Aufgabe~2 von Übungsblatt~2), folgt sofort folgende
Beobachtung:
\begin{prop}Limiten sind bis auf eindeutige Isomorphie eindeutig. Die
Kegelspitzen von Limiten sind zumindest bis auf Isomorphie eindeutig.\end{prop}

Für ein anschauliches Verständnis von Limiten sind zwei Mottos wichtig:
\begin{motto}Ein Limes eines Diagramms ist ein bestes
(größtmöglichstes) Objekt, welches das Diagramm zu einem Kegel ergänzt.
\end{motto}
\emph{Größtmöglich} ist dabei nicht im wörtlichen Sinn, wie er etwa in der Kategorie
der Mengen vorstellbar ist, zu interpretieren, sondern nur so zu verstehen,
als dass jeder andere Kegel
(Möchtegern-Limes) einen Morphismus in den Limes hinein besitzt.

\begin{motto}\label{limessubsumiert}
Ein Limes subsumiert das gesamte Diagramm zu einem einzelnen Objekt (der
Kegelspitze) -- zumindest, was Morphismen in das Diagramm hinein
angeht.\end{motto}
Das ist so verstehen: Immer, wenn man einen Morphismus aus einem
Objekt~$\widetilde K$ "`in das Diagramm hinein"' gegeben hat (d.\,h. einen Kegel des Diagramms
gegeben hat), induziert die universelle Eigenschaft einen Morphismus
aus~$\widetilde K$ in den Limes.  Umgekehrt kann man aus jedem solchen
Morphismus durch Nachschaltung der Projektionen einen Kegel erhalten.
Dieses Motto werden wir sogar formal beweisen können: siehe
Proposition~\ref{homstetig}.


\subsection{Beispiele für Limiten}

\subsubsection*{Produkte}

Sei speziell~$\I = \mathbf{2}$ die Kategorie mit genau zwei Objekten und nur
den Iden\-ti\-täts\-mor\-phis\-men:
\[ \xymatrix{\bullet \ar@(ul,ur) & \bullet \ar@(ul,ur)} \]
Dann sind Diagramme~$\I \to \C$ einfach durch die Angabe zweier Objekte
von~$\C$ gegeben. Kegel solcher Diagramme haben wir früher schon untersucht:
unter dem Namen \emph{Möchtegern-Produkte}. Entsprechend sind Limiten
solcher Diagramme schlichtweg Produkte.

\subsubsection*{Faserprodukte (Pullbacks)}

Sei speziell~$\I$ die Kategorie
\[ \xymatrix{
  & \bullet \ar[d] \ar@(ur,ul) \\
  \bullet \ar[r] \ar@(ul,dl) & \bullet. \ar@(dr,ur)
} \]
Limiten von~$\I$-förmigen Diagrammen werden auch \emph{Faserprodukte} genannt
und konventionsmäßig gerne als sog. \emph{Faserprodukt-} oder
\emph{Pullbackdiagramm} skizziert:
\[ \xymatrix{
  \ar @{} [dr] |{\begin{array}{l}\lrcorner\ \ \ \ \ \ \ \ \ \\\\\end{array}}
  X \times_Z Y \ar[r] \ar[d] & Y \ar[d]^g \\
  X \ar[r]_f & Z
} \]
Dabei steht die Kegelspitze des Limes oben links. Der dritte
Projektionsmorphismus (auf~$Z$) ist nicht eingezeichnet, da er sowieso gleich
der Komposition des Wegs über~$X$ (oder über~$Y$) sein muss.

In der Kategorie der Mengen kann das Faserprodukt durch die Konstruktion
\[ X \times_Z Y := \{ (x,y) \in X \times Y \,|\, f(x) = g(y) \} \subseteq X \times Y \]
gegeben werden.

Man hat zwei vorschiedene Vorstellungen des Faserprodukts, die unterschiedliche
Aspekte betonen: Zum einen bilden die Objekte~$X$ und~$Y$ die Ausgangsbasis. Dann
stellt man als Faserprodukt das Objekt~$X \times_Z Y$ vor und sieht es als eine
Art "`verallgemeinertes Produkt"' an.

Zum anderen kann man sich aber auch den Morphismus~$g$ als Ausgangspunkt
vorstellen. Als Ergebnis betont man dann nicht das Objekt~$X \times_Z Y$
alleine, sondern den Morphismus~$X \times_Z Y \to X$. Diesen bezeichnet man
dann auch als \emph{Rückzug (Pullback)} von~$g$ längs~$f$ oder
\emph{Basiswechsel} von~$g$ nach~$X$.

\begin{bsp}Sei~$g:Y \to Z$ eine Abbildung von Mengen. Sei~$U \subseteq Y$ eine
Teilmenge. Dann passt das Urbild~$g^{-1}[U]$ in ein Pullbackdiagramm:
\[ \xymatrix{
  \ar @{} [dr] |{\begin{array}{l}\lrcorner\ \ \ \ \ \ \ \ \ \\\\\end{array}}
  g^{-1}[U] \ar@{^{(}->}[r] \ar[d] & Y \ar[d]^f \\
  U \ar@{^{(}->}[r] & Z
} \]
\end{bsp}
Dieser Standpunkt wird unter anderem in der algebraischen Geometrie verwendet. Da
sind dann \emph{Stabilitätsaussagen} wichtig: Hat ein Morphismus eine bestimmte
Eigenschaft, so hat sein Rückzug längs Morphismen einer bestimmten Klasse
dieselbe Eigenschaft.

\begin{bsp}Sei ein Pullbackdiagramm der Form
\[ \xymatrix{
  \ar @{} [dr] |{\begin{array}{l}\lrcorner\ \ \ \ \ \ \ \ \ \\\\\end{array}}
  X \times_Z Y \ar[r]^{f'} \ar[d]_{g'} & Y \ar[d]^g \\
  X \ar[r]_f & Z
} \]
in einer beliebigen Kategorie gegeben. Wenn~$g$ ein Monomorphismus ist, dann auch~$g'$.
Man sagt: \emph{Monomorphismen sind unter Rückzug stabil.}
\end{bsp}
\begin{bem}Es ist etwas besonderes, wenn auch Epimorphismen unter Rückzug
stabil sind. Das ist etwa in der Kategorie der Mengen und allen abelschen
Kategorien der Fall.\end{bem}


\subsection{Limiten in Funktorkategorien}

In diesem Abschnitt wollen wir untersuchen, wie Limiten in
Funktorkategorien~$\Funct(\C,\D)$ aussehen. Sei dazu ein Diagramm
\[ F : \I \longrightarrow \Funct(\C,\D) \]
gegeben. Durch "`Nachschaltung der Evaluierungsfunktoren"' erhält man aus
diesem Diagramm für jedes Objekt~$X \in \Ob\C$ jeweils ein Diagramm in~$\D$:
\[ \begin{array}{@{}rrcl@{}}
  F_X : & \I &\longrightarrow& \D \\
  & i &\longmapsto& F(i)(X)
\end{array} \]
Wenn wir voraussetzen, dass all diese Diagramme jeweils einen Limes~$\lim F_X$ in~$\D$
besitzen, können wir (wie in Aufgabe~4 von Übungsblatt~3) einen Funktor
\[ \begin{array}{@{}rrcl@{}}
  L : & \C &\longrightarrow& \D \\
  & X &\longmapsto& \lim F_X
\end{array} \]
basteln. Dann hat man folgende Beobachtung:
\begin{prop}
Der so konstruierte Funktor~$L$ wird (mit welchen Projektionen?) ein Limes des Diagramms~$F$.
\end{prop}


\subsection{Kofinale Unterdiagramme}

\begin{defn}
Wir nennen einen Funktor $H : \D_0 \to \D$ genau dann \emph{kofinal}, wenn
für alle~$d \in \Ob\D$\ldots
\begin{enumerate}
\item[1.] ein Objekt $d_0 \in \Ob\D_0$ und ein Morphismus $d \to
H(d_0)$ in~$\D$ existiert und
\item[2.] für je zwei solcher Morphismen ein Objekt~$\widetilde d_0 \in \Ob\D_0$ und
Morphismen $d_0 \to \widetilde d_0$, $d_0' \to \widetilde d_0$ existieren, deren Bilder
unter~$H$ das Diagramm
\[ \xymatrix{
  d_0 \ar[r] \ar[d] & H(d_0) \ar@{-->}[d] \\
  H(d_0') \ar@{-->}[r] & H(\widetilde d_0)
} \]
kommutieren lassen.
\end{enumerate}
\end{defn}

Etwa ist der Inklusionsfunktor~$B (2\NN) \to B(\NN)$ kofinal, wenn~$\NN$ die
Menge der natürlichen Zahlen mit ihrer gewöhnlichen Ordnung und~$2\NN$ die
Teilordnung der geraden Zahlen bezeichnet.

\section[Das Yoneda-Lemma]{Das Yoneda-Lemma \hfill \small Justin Gassner}

\emph{Werbung:} Wir werden das fundamentale Yoneda-Lemma und seine Korollare
verstehen. Dazu werden wir zunächst eine hilfreiche Intuition von sog. Prägarben auf
Kategorien entwickeln und verstehen, welche Signifikanz die Darstellbarkeit von
Prägarben hat. Dann können wir die Yoneda-Einbettung kennenlernen, ihre
Eigenschaften studieren und sehen, wozu sie nützlich ist. Das fundamentale
Motto der Kategorientheorie wird damit zu einem formalen Theorem.

Sei in diesem Abschnitt~$\C$ eine lokal kleine Kategorie.

\begin{defn}Funktoren~$\C^\op \to \Set$ heißen auch \emph{Prägarben
auf~$\C$}. Die Kategorie der Prägarben auf~$\C$ (mit natürlichen
Transformationen als Morphismen) ist~$\widehat\C := \Funct(\C^\op,\Set)$.
\end{defn}

\begin{defn}Eine Prägarbe~$F:\C^\op\to\Set$ auf~$\C$ heißt genau dann
\emph{darstellbar}, wenn es ein Objekt~$X \in \Ob\C$ mit $F \cong
\Hom_\C(\freist,X)$ gibt.\end{defn}

\begin{motto}Eine beliebige (nicht unbedingt darstellbare) Prägarbe~$F$
auf~$\C$ beschreibt die Beziehungen aller Objekte~$A$ von~$\C$ mit einem
eingebildeten, fiktiven, ideellen Objekt~$\heartsuit$: Wir stellen uns die
Menge~$F(A)$ als Menge der Morphismen~$A \to \heartsuit$ vor.\end{motto}

Die Prägarbenkategorie~$\hat\C$ enthält stets mindestens eine
nicht-darstellbare Prägarbe, und zwar die initiale Prägarbe
\[ \begin{array}{@{}rrcl@{}}
  0 : & \C^\op &\longrightarrow& \Set \\
  & A &\longmapsto& \emptyset \\
  & f &\longmapsto& \id_\emptyset.
\end{array} \]
\begin{prop}Die initiale Prägarbe ist nicht darstellbar.\end{prop}
\begin{proof}Sei~$0 \cong \Hom_\C(\freist,X)$ für ein Objekt~$X\in\Ob\C$.
Dann folgt~$\Hom_\C(X,X) \cong 0(X) = \emptyset$ im Widerspruch zu
$\id_X \in \Hom_\C(X,X)$.\end{proof}

Der volle Hom-Funktor geht von~$\C^\op \times \C$ zu~$\Set$. Aus diesem kann
man durch Curryfizierung einen Funktor~$\C \to \Funct(\C^\op,\Set)$ basteln:
\[ \begin{array}{@{}rrcl@{}}
  Y : & \C &\longrightarrow& \widehat\C \\
  & X &\longmapsto& \widehat X := \Hom_\C(\freist,X) \\
  & f &\longmapsto& \widehat f
\end{array} \]
Die Komponenten der natürlichen Transformation~$\widehat f$ sind dabei
für~$f: X \to Y$ durch
\[ \begin{array}{@{}rrcl@{}}
  (\widehat f)_A : & \Hom_\C(A,X) &\longrightarrow& \Hom_\C(A,Y) \\
  & g &\longmapsto& f \circ g
\end{array} \]
gegeben.

\begin{defn}Der Funktor~$Y:\C\to\widehat\C$ heißt \emph{Yoneda-Einbettung}
von~$\C$.\end{defn}

Die Yoneda-Einbettung hat folgende wichtige Eigenschaften:
\begin{prop}\begin{enumerate}
\item $Y$ ist treu und voll.
\item $Y$ erhält Limiten (aber kaum Kolimiten).
\item $Y$ ist dicht (siehe Aufgabe~2 von Übungsblatt~6).
\end{enumerate}\end{prop}
Vermöge~$Y$ können wir~$\C$ also als volle Unterkategorie der Kategorie der
fiktiven, ideellen Objekte ansehen.


\section[Adjungierte Funktoren]{Adjungierte Funktoren \hfill \small Peter
Uebele}

% XXX: adjungierte Funktoren in der Logik!


%Erinnerung: Äquivalenz von Kategorien $\cC, \cD$:\\
%$\exists F:\cD\rightarrow \cC$, $G:\cC\rightarrow \cD$ inverse Funktoren, d.h.
%$F\circ G\cong id_\cC$ und $G\circ G\cong id_\cD$

%\paragraph{Idee:} Hin- und herschieben von Morphismen zwischen $\cC, \cD$.\\
%$\rightsquigarrow$ Verallgemeinerung, s.d. $\cC,\cD$ nicht mehr äquivalent sein
%müssen.
\begin{defn}
Seien $F:\D\to\C$, $G:\C\to\D$ Funktoren. Genau dann heißt
\begin{itemize}
\item 
$F$ \emph{links-adjungiert zu} $G$ bzw.
\item 
$G$ \emph{rechts-adjungiert zu} $F$,
\end{itemize}
in Zeichen: $F \dashv G$, wenn es eine in~$X \in \C$ und~$Y \in \D$ natürliche
Isomorphie
\[ \Hom_\C(FY,X) \cong \Hom_\D(Y,GX) \]
gibt.
\end{defn}
Dabei ist \emph{natürlich} gemäß Bemerkung~\ref{interpretnat} zu verstehen:
Linke und rechte Seiten der Isomorphie sind als Funktoren
\[ \begin{array}{@{}rrcl@{}}
  \Hom_\C(F\freist,\freist) :&
    \D^\op \times \C &\longrightarrow& \Set \\
  & (Y,X) &\longmapsto& \Hom_\C(FY, X) \\\\
  \Hom_\D(\freist,G\freist) :&
    \D^\op \times \C &\longrightarrow& \Set \\
  & (Y,X) &\longmapsto& \Hom_\D(Y, GX)
\end{array} \]
zu verstehen. Die Natürlichkeitsbedingung bedeutet dann, dass
für alle Morphismen $f:X\rightarrow X'$ in~$\C$ und
$g:Y'\rightarrow Y$ in~$\D$ das Diagramm
\[ \xymatrix{
  \Hom_\C(FY,X) \ar[d]_\cong \ar[r] & \Hom_\C(FY',X') \ar[d]^\cong \\
  \Hom_\D(Y,GX) \ar[r] & \Hom_\D(Y',GX')
} \]
kommutiert.


\nocite{*}
\printbibliography

\end{document}
\begin{tikzpicture} [scale=3.3, descr/.style={fill=white,inner sep=2.5pt} ]
\matrix (m) [
  matrix of math nodes
  , row sep=3em
  , column sep=3em
  %, text height=3em
  %, text depth=0.25em
]
{
  \Hom_\cC(FY,X)   & \cong & \Hom_\cC(Y,GX)   & \phi\\
  \Hom_\cC(FY',X') & \cong & \Hom_\cC(Y',GX') & Gf\circ\phi\circ g\\
};
\path[->,font=\scriptsize,>=angle 90]
(m-1-1) edge node[right]{$\Hom(Fg,f)$} (m-2-1)
(m-1-3) edge node[left]{$\Hom(g,Gf)$} (m-2-3)
;

\path[>=stealth,|->]
(m-1-4) edge (m-2-4)
;
\end{tikzpicture}
\end{center}
kommutiert.

\end{document}
\begin{exmp}
$k$ Körper, $Vect_k \overset{U}{\rightarrow}Set$ Vergissfunktor\\
$Set\underset{F}{\rightarrow Vekt_k}$ \emph{freier Funktor}
$\underset{\mbox{Menge}}{X}\mapsto FX=\{\sum_{n=1}^N\lambda_ix_i\}$ endli.
Linearkombination von Elementen aus $X$. $S$ ist Basis von $FX$.\\
\[
(X\overset{f}{\rightarrow}X')\mapsto Ff
\]
wobei $Ff$ eine lineare Abbildung ist, die durch $x_i\mapsto f(x_i)$ gegeben
ist.
\[Ff(\sum^{N}_{i=1}{\lambda_ix_i})=\sum^{N}_{i=1}{\lambda_i}f(x_i)\]
\end{exmp}
\paragraph{Beh:} $F\dashv U$
\begin{proof}
\[
\Hom_\cC(FY,V)  \overset{?}{\cong} \Hom_\cC(Y,UV)
\]
wobei 
\begin{itemize}
\item $\Hom_\cC(FY,V)\bydef\{\mbox{lin. Abb }FY\rightarrow V\}$
\item $\Hom_\cC(Y,UV)\bydef\{\mbox{bel. Abb. }Y\rightarrow V\}\ni \phi$
\end{itemize}
\[
\phi \mapsto \big( \sum^{N}_{i=1}{\lambda_iy_i}\mapsto
\sum^{N}_{i=1}{\lambda_i\phi(y_i)} \big)
\]
bijektiv, da jede lineare Abbildung eindeutig durch die Basis festgelegt ist.

Natürlich in $Y$ und $V$: $f:V\mapsto V'$ und $g:Y\mapsto Y'$
\begin{center}
\begin{tikzpicture} [scale=3.3, descr/.style={fill=white,inner sep=2.5pt} ]
\matrix (m) [
  matrix of math nodes
  , row sep=3em
  , column sep=2em
  %, text height=3em
  %, text depth=0.25em
]
{
  \big(\lambda_iy_i\mapsto\sum\lambda_i\phi(y_i)\big) & & & & \phi\\
  & \Hom_\cC(FY,V)   & \cong & \Hom_\cC(Y,UV)   & \\
  & \Hom_\cC(FY',V') & \cong & \Hom_\cC(Y',UV') & \\
  \big( \sum\lambda_iy_i'\mapsto\sum\lambda_if(\phi(g(y_i')))
    \big)& & & & Uf\circ\phi\circ g\\
};
\path[->,font=\scriptsize,>=angle 90]
(m-2-2) edge node[right]{$\Hom(Fg,f)$} (m-3-2)
(m-2-4) edge node[left]{$\Hom(g,Gf)$} (m-3-4)
;

\path[>=stealth,|->]
(m-1-5) edge (m-1-1) 
(m-1-5) edge (m-4-5)
(m-1-1) edge (m-4-1)
(m-4-5) edge (m-4-1) 
;
\end{tikzpicture}
\end{center}
\end{proof}

\begin{exmp}
\[
U:Cat \rightarrow Set
\]
\begin{align*}
L: & Set \rightarrow Kat\\
  & X \mapsto \mbox{diskrete Kategorie auf $X$ (d.h. nur
Identitätsmorphismen)}\\
R: & Set \rightarrow Cat\\
  & X \mapsto \mbox{indiskrete Kategorie auf $X$(d.h. zwischen je zwei
Objekten genau ein Morphismus)}
\end{align*}
\paragraph{Beh:}
\begin{enumerate}
\item $L\dashv U$
\item $U\dashv R$
\end{enumerate}
\paragraph{zu 1.} $C=Cat$, $D=Set$
$\Hom_{Cat}(LX,\cC)\cong\Hom_{Set}(X,U\cC)$
wobei
\begin{itemize}
\item 
$\Hom_{Cat}(LX,\cC)\bydef\{\mbox{Funktoren }LX\rightarrow \cC \}$
\item 
$\Hom_{Set}(X,U\cC)\bydef\{\mbox{Abbildungen } X\rightarrow Obj(\cC)\}$
\end{itemize}
\paragraph{zu 2.}
$\Hom_{Set}(U\cC,X)\cong\Hom_{Cat}(\cC,RX)$
wobei
\begin{itemize}
\item 
$\Hom_{Set}(U\cC,X)\bydef\{\mbox{Abbildungen } Obj(\cC)\rightarrow X\}$
\item 
$\Hom_{Cat}(\cC,RX)\bydef\{\mbox{Funktoren }\cC\rightarrow RX \}$
\end{itemize}
bijektiv, da Morphismen in $RX$ durch Quelle und Ziel eindeutig bestimmt
\end{exmp}
\begin{thm}
$F:\cC\rightarrow\cD$ links-adjungiert zu $G:\cD\rightarrow\cD$, dann gilt:
\begin{itemize}
\item $F$ erhält Kolimiten von $D$
\item $G$ erhält Limiten von $C$
\end{itemize}
\end{thm}
\paragraph{Folgerung:}
\begin{itemize}
\item $U:Vect_k\rightarrow Set$ erhält Limiten, aber nicht Kolimiten und hat
somit keine rechts-adjungierte Funktoren
\item $U:Cat \rightarrow Set$ erhält Limiten und Kolimiten
\end{itemize}
\begin{proof}
Sei $I\overset{D}{\rightarrow}\cD$ ein Diagramm, mit Limes
$\underset{I}{\underleftarrow\lim}D$.
\begin{align*}
\Hom_\cD(Y,G\underset{I}{\underleftarrow\lim}D_i)
  &\cong \Hom_\cC(FY,\underset{I}{\underleftarrow\lim}D_i)\\
  &\cong \underset{I}{\underleftarrow\lim}\Hom_\cC(FY,D_i)\\
  &\cong \underset{I}{\underleftarrow\lim}\Hom_\cD(Y,GD_i)\\
  &\cong \Hom_\cD(Y,\underset{I}{\underleftarrow\lim}GD_i)\\
\end{align*}
$\rightsquigarrow$ Natürlich in $Y$
$\overset{\mbox{Yoneda Lemma}}{\rightsquigarrow}$
$G\underset{I}{\underleftarrow\lim}D_i\cong
\underset{I}{\underleftarrow\lim}GD_i$.
\end{proof}
\begin{exmp}
$\cC=Grp$, $\cD=Grp^2=Grp\times Grp$
\begin{align*}
F: & Grp^2\rightarrow Grp &\mbox{Produktfunktor}\\
  & (G_1,G_2) \mapsto G_1\times G_2\\
G: & Grp \rightarrow Grp^2 &\mbox{Diagonalfunktor}\\
  & G \mapsto (G,G)
\end{align*}
\paragraph{Beh:} $F\vdash G$
\[
\Hom_{Grp^2}((G,G),(H_1,H_2))\cong \Hom_{Grp}(G,H_1\times H_2)
\]
wobei
\begin{itemize}
\item $\Hom_{Grp^2}((G,G),(H_1,H_2))=\{\mbox{Gruppen-Homomorphismen}\}$
\item $ \Hom_{Grp}(G,H_1\times H_2) =\{G\rightarrow H_1\times H_2 \mbox{
Gruppen-Homomorphismen}\}$
\end{itemize}
\end{exmp}

%%%%%%%%%%%%%%%%%%%%%%%%%%%%%%%%%%%%%%%%%%%%%%%%%%%%%%%%%%%%%%%%%%%%%%%%%%%%%%%
\chapter{ Kombinatorische Spezies }

Was wollen Kombinatoriker?
$\rightsquigarrow$ Strukturen Zählen\\
\begin{exmp}
Auf wieviele Arten kann man $n$ bunte Steine auf einer Schildkröte
verteilen?
\end{exmp}
\paragraph{Modellierung}
\begin{itemize}
\item Wir haben eine \emph{Struktur}
\item die von \emph{Grundelementen} abhängt
\end{itemize}
\paragraph{ad 2)}: Es kommt nicht darauf an, ob steine, Zahlen, etc. entscheidend ist nur
ihre Anzahl.
\begin{defn}
Kategorie $\cB$:
\begin{itemize}
\item $Obj \cB=$ endliche Mengen
\item $Morph \cB=$ Bijektive Abbildungen
\end{itemize}
\end{defn}
\paragraph{ad 1):} Kategorie von \emph{Strukturen} (nicht so wichtig, muss
reihhaltig genug sein)\\
zB.: \underline{FinSet}, $\cB$
\paragraph{Abhängigkeit der Struktur von $\cB$} $\rightsquigarrow$ Funktor
\begin{defn}
Eine \emph{kombinatorische Spezies} ist ein Funktor $\cF:\cB\rightarrow\FinSet$
\end{defn}
\begin{exmp}
\begin{itemize}
\item Spezies der zyklischen Ordnungen
\begin{align*}
\mathcal{C}yc: &A \mapsto \{ a\rightarrow b\rightarrow\dots\rightarrow a\mid A=\{a,b,\dots\}\}
\end{align*}
\item Spezies $\mathcal{S}eq_k$ mit $k\in\N$
\begin{align*}
\mathcal{S}eq_k: A\mapsto \{\underset{k}{\underbrace{ \underbracket{a}\,\,\,
\underbracket{b,c}\,\,\,\dots }}\mid A=\{a,b,c,\dots\} \}
\end{align*}
\end{itemize}
\end{exmp}
Kombinatoriker wollen die Anzahl von Strukturen in der Abhängigkeit von der
Anzhal der Elemente in $A$ zählen
\begin{itemize}
\item Zu einer Spezies $\cF$ definiere $f_n:=|\cF(\{1,\dots,b\})|$ und
(exponentiell) \emph{erzeugende Funktionen}
\[
F=\sum_{n\geq0}f_n\cdot\frac{x^n}{n!}=f_0+f_1x+\frac{1}{2}f_2x^2
  +\frac{1}{6}f_3x^3+\dots
\]
\end{itemize}
\begin{exmp}
\begin{itemize}
\item zu $\mathcal{C}yc$
\begin{align*}
cyc_n=\frac{n!}{n}=(n-1)! \,, n\geq 1& & cyc_0=0
\end{align*}
Erzeugende Funktionen 
\[
Cyc=\sum_{n\geq
1}\frac{(n-1)!}{n!}x^n)=x+\frac{x^2}{2}+\frac{x^3}{3}+\dots=-ln(1-x)
\]
\item zu $\mathcal{S}eq_k$
\begin{align*}
2bsp_n=k^n\\
2Bsp=\sum_{n\geq0}\frac{k^n}{n!}x^n=\exp(kx)
\end{align*}
\end{itemize}
\end{exmp}
\begin{exmp}
\begin{itemize}
\item $\mathcal{X}$: Spezies der Einelementigen Menge
\[
\mathcal{X}: A\mapsto \begin{cases}
\{\star\} & ,\mbox{falls }|A|=1\\
\emptyset & ,\mbox{sonst}
\end{cases}
\]
$X=x$
\item $\textbf 1$: Spezies der $0$-elementigen Menge
\[
\textbf 1:A\mapsto \begin{cases}
\{\star\} & ,\mbox{falls }A=\emptyset\\
\emptyset & ,\mbox{sonst}
\end{cases}
\]
$1=1$
\item $\sE$: Spezies der Menge
\[
\sE:A\mapsto A
\]
$e_0=1$, $e_1=1$, $e_2=1$ $\dots$
\[
E=1\frac{x^0}{0!}+1\frac{x}{1!}+1\frac{x^2}{2!}+\dots=1+x+\frac{x^2}{2}+\dots
  =\exp(x)
\]
\item $\mathcal{P}erm=$ Spezies der Permmutationen\\
$perm_0=1$, $perm_1=1$, $perm_n=n!$\\
\[
Perm=\sum_{n\geq0}\frac{n!}{n!}x^n=\frac{1}{1-x}
\]
\end{itemize}
\end{exmp}
\begin{bem}
Erzeugende Funktionen kann man addieren, malnehmen, ineinander einsetzen, \dots
\end{bem}
$\rightsquigarrow$ was bedeutet das für Spezies?
\paragraph{zu $+$}
\begin{align*}
+: &\mbox{Disjukte Vereinigung von Spezies}\\
\cF+\cG: &A\mapsto \cF(A)\amalg \cG(A)
\end{align*}
$\cE^+$: Spezies der nichtleeren Mengen
\[
\cE = 1+ \sE^+
\]
%TODO: hier fehlt erklärung
% ist entweder leer oder nicht leer
und deshalb: $E^+=\exp(x)-1$
\paragraph{zu $\cdot$}
\begin{align*}
\cdot: &\mbox{Paarbildung}\\
\end{align*}
erzeugende Funktionen: 
\begin{align*}
\sum_{n\geq 0}f_n\frac{x^n}{n!}\sum_{n>0}g_n\frac{x^n}{n!} 
  &=\sum_{n\geq 0}\big( \sum_{k+l=n}\frac{n!}{k!l!}f_kg_l \big) \frac{x^n}{n!}
& (Cauchy-Produkt)
\end{align*}
\[
\cF\cdot\cG:A\mapsto \coprod_{B\dot\amalg C}\cF(B)\times \cG(C)
\]
lies als \emph{und}
\begin{exmp}
Injektionen von $\{1,\dots,4\}$ nach $A$:\\
Injektion = Bijektiong aufs Bild \underline{und} Rest
\[
\mathcal{I}nj_4=\mathcal{P}erm_4\cdot\sE
\]
\[
\,^4Inj=\frac{4!x^4}{4!}\cdot\exp(x)
\]
\end{exmp}
\begin{exmp}
Fixpunktfreie Permutationen $\mathcal{D}er$
\[
\mathcal{P}erm = \sE \cdot \mathcal{D}er
\]
$\frac{1}{1-x}=exp(x)Der$ $\Rightarrow$ $Der=\frac{exp(-x)}{1-x}$
\end{exmp}
\paragraph{zu Spezies einsetzen}
$\cF(\cG)$ lies als \emph{von}
\begin{exmp}
Partitionen (=Zerlegungen) einer Menge\\
eine Zerlegung ist eine Menge von (Uner-) nicht leeren Mengen
\[
\mathcal{P}art=\sE(\sE^+)
\]
und die Erzeugende Funktion
\[
Part=\exp(\exp(x)-1)
\]
\end{exmp}
\begin{exmp}[Binäre Bäume]
\begin{center}
\begin{tikzpicture} [scale=3.3, descr/.style={fill=white,inner sep=2.5pt} ]
\matrix (m) [
  matrix of math nodes
  , row sep=3em
  , column sep=2em
  %, text height=3em
  %, text depth=0.25em
]
{
   &  &         &         & \textcolor{red}{\bullet}\\
   &  &         & \textcolor{blue}{\bullet} &            &
\textcolor{green}{\bullet} \\
   &  & \textcolor{orange}{\bullet} &         & \textcolor{purple}{\bullet} \\
   &  &         & \textcolor{brown}{\bullet} &            &
\textcolor{pink}{\bullet} \\
};
\path
(m-1-5) edge (m-2-6)
(m-1-5) edge (m-2-4)
(m-2-4) edge (m-3-5)
(m-2-4) edge (m-3-3)
(m-3-5) edge (m-4-6)
(m-3-5) edge (m-4-4)
;
\end{tikzpicture}
\end{center}
\[
\mathcal{B}in\mathcal{T}ree= X + X\cdot \mathcal{B}in\mathcal{T}ree \cdot
\mathcal{B}in\mathcal{T}ree
\]
erzeugende Funktionen
$BT=X+X\cdot BT^2$ und damit
\begin{align*}
X\cdot BT^2-BT+X&= 0\\
BT &= \frac{1}{2x}(1-\sqrt(1^2-4x^2))
\end{align*}

\end{exmp}
Weiterführende Literatur:
\begin{itemize}
\item Sedgewick / Flajolet: Analytic Combinatorics
\end{itemize}
\paragraph{Was macht man, wenn man die Farben nicht unterscheidet?}
\begin{align*}
\cB': &Obj=\mbox{endl. Mengen}\\
  &Morph: \Hom(A,B)= \begin{cases}
\{\star\} & ,\mbox{falls }|A|=|B|\\
\emptyset & ,\mbox{sonst}
\end{cases}
\end{align*}
Erzeugende Funktionen; $F'=\sum_{n\geq0}f_n\frac{x^n}{1}$ \emph{gewöhnliche
erzeugende Funktionen}\\
\begin{itemize}
\item $+, \cdot$: interpretation wie bisher
\item einsetzen: geht nicht gut
\end{itemize}
\begin{exmp}
Spezies der Karnickel (nach Fibonacci)\\
Wollen die Anzahl der Möglichkeiten für Karnickel nach $n$ Jahren, wenn am
Anfang $1$ vorhanden ist
\[
\mathcal{K}=1+K\cdot \overbox{X}{1 Jahr warten} + K\cdot \overbox{X^2}{2 Jahre}
\]
$K=1+xK+x^2K$\\
$\Rightarrow$ $K=\frac{1}{1-x-x^2}$
\end{exmp}

\section[Kombinatorische Spezies]{Kombinatorische Spezies \hfill \small
Simon Kapfer}

\emph{Werbung:}
Auf wie viele Möglichkeiten kann man~$n$ gefärbte Kugeln in so und so viele
Urnen unter Beachtung dieser oder jener Zusatzbedingungen verpacken? Wir
werden eine einfache konzeptuelle Methode kennenlernen, um kombinatorische
Zählprobleme dieser Art zu lösen. Dabei werden uns sog. erzeugende
Funktionen wundersame Dienste leisten. Um die formale Formulierung sauber
hinzubekommen, werden wir uns ein wenig Kategorientheorie zunutze machen.


\section[Topologische Quantenfeldtheorien]{Topologische Quantenfeldtheorien \hfill \small
Sven Prüfer}

\nocite{*}
\printbibliography

\end{document}
