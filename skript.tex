\documentclass[a4paper,ngerman]{scrartcl}

%\usepackage{ucs}
\usepackage[utf8]{inputenc}

\usepackage[ngerman]{babel}

\usepackage{amsmath,amsthm,amssymb,amscd,color,graphicx}

%\usepackage[small,nohug]{diagrams}
%\diagramstyle[labelstyle=\scriptstyle]

\usepackage[protrusion=true,expansion=true]{microtype}

\usepackage{lmodern}
\usepackage{tabto}

\usepackage[all]{xy}

\usepackage{hyperref}

\setlength\parskip{\medskipamount}
\setlength\parindent{0pt}

\theoremstyle{definition}
\newtheorem{defn}{Definition}
\newtheorem{bsp}[defn]{Beispiel}

\theoremstyle{plain}

\newtheorem{prop}[defn]{Proposition}
\newtheorem{motto}[defn]{Motto}
\newtheorem{ueberlegung}[defn]{Überlegung}
\newtheorem{lemma}[defn]{Lemma}
\newtheorem{kor}[defn]{Korollar}
\newtheorem{hilfsaussage}[defn]{Hilfsaussage}
\newtheorem{satz}[defn]{Satz}

\theoremstyle{remark}
\newtheorem{bem}[defn]{Bemerkung}

\clubpenalty=10000
\widowpenalty=10000
\displaywidowpenalty=10000

\newcommand{\lra}{\longrightarrow}
\newcommand{\lhra}{\ensuremath{\lhook\joinrel\relbar\joinrel\rightarrow}}
\newcommand{\thlra}{\relbar\joinrel\twoheadrightarrow}

\newcommand{\ZZ}{\mathbb{ZZ}}
\newcommand{\QQ}{\mathbb{QQ}}
\newcommand{\RR}{\mathbb{RR}}
\newcommand{\C}{\mathcal{C}}
\newcommand{\Hom}{\mathrm{Hom}}
\newcommand{\id}{\mathrm{id}}
\newcommand{\freist}{\underline{\ \ }}
\DeclareMathOperator{\Ob}{Ob}
\DeclareMathOperator{\ggT}{ggT}
\newcommand{\op}{\mathrm{op}}

\newcommand{\XXX}[1]{\textcolor{red}{#1}}

\renewcommand*\theenumi{\alph{enumi}}
\renewcommand{\labelenumi}{\theenumi)}

%\newarrow{Equals}=====

%\usepackage{geometry}
%\geometry{tmargin=2cm,bmargin=4cm,lmargin=3cm,rmargin=3cm}

\begin{document}

\vspace*{2em}%
\begin{center}%
  \vskip 1em
  {\LARGE Pizzaseminar zur Kategorientheorie}
  \vskip 1.5em%
  {\large
   \lineskip .5em%
    \begin{tabular}[t]{c}%
      \today
    \end{tabular}\par}%
    \vskip 1em%
\end{center}\par
\par\vskip 1.5em

\begin{center}\emph{Warnung:} Die vielen Beispiele, Erklärungen und Hintergründe
fehlen bislang.\end{center}

\tableofcontents

\section[Was sollen Kategorien?]{Was sollen Kategorien? \hfill \small Ingo
Blechschmidt}

\begin{defn}
Eine \emph{Kategorie}~$\C$ besteht aus
\begin{enumerate}
  \item einer Klasse von \emph{Objekten} $\Ob \C$,
  \item zu je zwei Objekten $X,Y \in \Ob \C$ einer Klasse $\Hom_\C(X,Y)$ von
  \emph{Morphismen} zwischen ihnen und
  \item einer Kompositionsvorschrift:
  \begin{align*}
    \text{zu }\ & f \in \Hom_\C(X,Y) &
    \text{zu }\ & f : X \to Y \\
    \text{und }\ & g\in\Hom_\C(Y,Z) &
    \text{und }\ & g : Y \to Z \\
    \text{habe }\ & g\circ f\in\Hom_\C(X,Z), &
    \text{habe }\ & g\circ f : X \to Z,
  \end{align*}
\end{enumerate}
sodass
\begin{enumerate}
  \item die Komposition $\circ$ assoziativ ist und
  \item es zu jedem $X \in \Ob\C$ einen \emph{Identitätsmorphismus} $\id_X
  \in \Hom_\C(X,X)$ mit
  \[ f \circ \id_X = f, \quad \id_X \circ g = g \]
  für alle Morphismen $f,g$ gibt.
\end{enumerate}
\end{defn}

\begin{motto}[fundamental]Kategorientheorie stellt \emph{Beziehungen zwischen
Objekten} statt etwaiger innerer Struktur in den Vordergrund.\end{motto}

\begin{defn}
Ein Objekt~$X$ einer Kategorie~$\C$ heißt genau dann
\begin{itemize}
  \item \emph{initial}, wenn
    \[ \forall Y \in \Ob \C{:}\ \exists! f : X \to Y. \]
  \item \emph{terminal}, wenn
    \[ \forall Y \in \Ob \C{:}\ \exists! f : Y \to X. \]
\end{itemize}
\end{defn}

\begin{bsp}\begin{enumerate}
\item In der Kategorie der Mengen ist genau die leere Menge initial und
genau jede einelementige Menge terminal.
\item In der Kategorie der $K$-Vektorräume ist der Nullvektorraum $K^0$ initial
und terminal.
\end{enumerate}\end{bsp}

\begin{defn}
Ein Morphismus $f:X \to Y$ einer Kategorie~$\C$ heißt genau dann
\begin{itemize}
  \item \emph{Monomorphismus}, \tabto{3.35cm}wenn für alle Objekte~$A \in \Ob \C$ \\
  \tabto{3.35cm}und $p,q:A \to X$ gilt:
  \[ f \circ p = f \circ q \quad\Longrightarrow\quad p = q. \]
  \item \emph{Epimorphismus}, \tabto{3.35cm}wenn für alle Objekte~$A \in \Ob \C$ \\
  \tabto{3.35cm}und $p,q:Y \to A$ gilt:
  \[ p \circ f = q \circ f \quad\Longrightarrow\quad p = q. \]
\end{itemize}
\end{defn}

\begin{bsp}\begin{enumerate}
\item In den Kategorien der Mengen, Gruppen und $K$-Vektorräumen sind die
Monomorphismen genau die injektiven und die Epimorphismen genau die
surjektiven Abbildungen. Das ist jeweils eine interessante Erkenntnis über die
Struktur dieser Kategorien und nicht ganz leicht zu zeigen.
\item In der Kategorie der metrischen Räume mit stetigen Abbildungen gibt es
Epimorphismen, die nicht surjektiv sind: nämlich alle stetigen Abbildungen mit
dichtem Bild.
\end{enumerate}\end{bsp}

\begin{defn}
Ein \emph{Isomorphismus} $f:X \to Y$ in einer Kategorie ist ein
Morphismus, zu dem es einen Morphismus $g:Y \to X$ mit
\[ g \circ f = \id_X, \quad f \circ g = \id_Y \]
gibt. Statt "`$g$"' schreibt man auch "`$f^{-1}$"'. Existiert zwischen
Objekten~$X$ und~$Y$ ein Isomorphismus, so heißen die Objekte \emph{zueinander
isomorph}:
$X \cong Y$.
\end{defn}

\begin{defn}
Die zu einer Kategorie~$\C$ zugehörige \emph{duale Kategorie} $\C^\op$ ist
folgende:
\begin{align*}
  \Ob \C^\op &:= \Ob \C \\
  \Hom_{\C^\op}(X,Y) &:= \Hom_\C(Y,X)
\end{align*}
\end{defn}

\begin{bsp}\begin{enumerate}
\item Ein initiales Objekt in~$\C^\op$ ist ein terminales Objekt in~$\C$ und
umgekehrt.
\item Ein Epimorphismus in~$\C^\op$ ist ein Monomorphismus in~$\C$ und
umgekehrt.
\item Zwei Objekte sind genau dann in~$\C^\op$ zueinander isomorph, wenn sie es
in~$\C$ sind.
\end{enumerate}\end{bsp}


\section[Produkte und Koprodukte]{Produkte und Koprodukte \hfill \small
Matthias Hutzler}

\begin{defn}Seien~$X$, $Y$ Objekte einer Kategorie~$\C$. Dann besteht ein
\emph{Produkt} von~$X$ und~$Y$ aus
\begin{enumerate}
\item einem Objekt~$P \in \Ob \C$ und
\item Morphismen $\pi_X : P \to X$, $\pi_Y : P \to Y$,
\end{enumerate}
sodass für jedes andere \emph{Möchtegern-Produkt}, also
\begin{enumerate}
\item jedem Objekt~$\widetilde P \in \Ob \C$ zusammen mit
\item Morphismen $\widetilde \pi_X : \widetilde P \to X$, $\widetilde\pi_Y :
\widetilde P \to Y$
\end{enumerate}
genau ein Morphismus $\psi : \widetilde P \to P$ existiert, der das Diagramm
\[ \xymatrix{
    & \widetilde P \ar[ld]_{\widetilde \pi_X} \ar@{-->}[dd]_\psi \ar[rd]^{\widetilde \pi_Y} \\
  X & & Y \\
    & P \ar[lu]^{\pi_X} \ar[ru]_{\pi_Y}
  } \]
kommutieren lässt, also die Gleichungen
\begin{align*}
  \pi_X \circ \psi &= \widetilde \pi_X \\
  \pi_Y \circ \psi &= \widetilde \pi_Y
\end{align*}
erfüllt.
\end{defn}

\begin{motto}Ein Produkt ist ein bestes Möchtegern-Produkt.\end{motto}

Analog definiert man das Produkt von~$n$ Objekten, $n \geq 0$; und dual
definiert man das Koprodukt.

\begin{bsp}\begin{enumerate}
\item Das Produkt in der Kategorie der Mengen ist durch das kartesische Produkt
gegeben.
\item Das Produkt in der Kategorie der Gruppen ist durch das direkte Produkt
mit der komponentenweisen Verknüpfung gegeben.
\item Das Produkt in der von einer Quasiordnung induzierten Kategorie ist durch
das Infimum gegeben, siehe Aufgabe~3 von Übungsblatt~2.
\end{enumerate}\end{bsp}

\begin{prop}Die Objektteile je zweier Produkte von Objekten~$X$, $Y$ sind
zueinander isomorph.\end{prop}

\begin{bem}Es gilt sogar noch mehr, siehe Aufgabe~2 von Übungsblatt~2.\end{bem}

\begin{prop}Die Angabe eines Produkts von~$X$ und~$Y$ ist gleichwertig mit der
Angabe eines Produkts von~$Y$ und~$X$.\end{prop}

\begin{prop}Die Angabe eines Produkts von null vielen Objekten ist gleichwertig
mit der Angabe eines terminalen Objekts.\end{prop}

\end{document}
