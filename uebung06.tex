\documentclass{pizzablatt}

\begin{document}

\maketitle{6}{3. April 2013}

\begin{aufgabe}{Das Yoneda-Lemma}
Sei~$\C$ eine lokal kleine Kategorie und~$\widehat \C := \Funct(\C^\op,\Set)$
ihre Prägarbenkategorie. Wir wollen in mehreren Schritten das \emph{Yoneda-Lemma}
beweisen, demnach wir eine in~$X \in \Ob \C$ und~$F \in \widehat \C$
natürliche Bijektion
\begin{equation}\label{bij}\Hom_{\widehat\C}(\Hom_\C(\freist, X), F) \cong F(X)
\end{equation}
haben. Mit~$\Hom_\C(\freist,X)$ ist der kontravariante Hom-Funktor zu~$X$
bezeichnet, den wir auch~$\widehat X$ geschrieben haben.
\begin{enumerate}
\item Zeige, dass eine natürliche Transformation $\eta : \Hom(\freist, X)
\Rightarrow F$ durch ihren Wert~$s := \eta_X(\id_X) \in F(X)$ bereits eindeutig
festgelegt ist, und zwar über die Formel
\begin{equation}\label{eq} \eta_Y(f) = F(f)(s) \end{equation}
für alle Objekte~$Y$ und Morphismen~$f \in \Hom_\C(Y,X)$.
\item Zeige, dass umgekehrt für beliebiges~$s \in F(X)$ die Formel~\eqref{eq} eine
natürliche Transformation $\eta : \Hom(\freist, X) \Rightarrow F$ definiert.
\item Zeige mit a) und b), dass zumindest für festes~$X \in \Ob\C$ und~$F \in
\widehat\C$ eine Bijektion~\eqref{bij} existiert.
\item Linke und rechte Seite von~\eqref{bij} können als Auswertungen der
Funktoren
\[ \renewcommand{\arraystretch}{1.3}\begin{array}{@{}rl@{}}
  L : \C \times \widehat\C \longrightarrow \Set, &
  (X,F) \longmapsto \Hom_{\widehat\C}(\Hom_\C(\freist,X), F)
  \\
  R : \C \times \widehat\C \longrightarrow \Set, &
  (X,F) \longmapsto F(X)
\end{array} \]
an der Stelle~$(X,F)$ angesehen werden. Überlege, wie diese beiden Funktoren auf
Morphismen wirken, und zeige, dass sie zueinander isomorph sind.
\item Du hast soeben das Yoneda-Lemma bewiesen. Herzlichen Glückwunsch!
\end{enumerate}
\end{aufgabe}

\end{document}
