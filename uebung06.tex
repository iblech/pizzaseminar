\documentclass{pizzablatt}

\begin{document}

\maketitle{6}{3. April 2013}

\begin{aufgabe}{Das Yoneda-Lemma}
Sei~$\C$ eine lokal kleine Kategorie und~$\widehat \C := \Funct(\C^\op,\Set)$
ihre Prägarbenkategorie. Wir wollen in mehreren Schritten das \emph{Yoneda-Lemma}
beweisen, demnach wir eine in~$X \in \Ob \C$ und~$F \in \Ob \widehat \C$
natürliche Bijektion
\begin{equation}\label{bij}\Hom_{\widehat\C}(\Hom_\C(\freist, X), F) \cong F(X)
\end{equation}
haben. Mit~$\Hom_\C(\freist,X)$ ist der kontravariante Hom-Funktor zu~$X$
bezeichnet, den wir auch~$\widehat X$ geschrieben haben.
\begin{enumerate}
\item Zeige, dass eine natürliche Transformation $\eta : \Hom(\freist, X)
\Rightarrow F$ durch ihren Wert~$s := \eta_X(\id_X) \in F(X)$ bereits eindeutig
festgelegt ist, und zwar über die Formel
\begin{equation}\label{eq} \eta_Y(f) = F(f)(s) \end{equation}
für alle Objekte~$Y$ und Morphismen~$f \in \Hom_\C(Y,X)$.
\item Zeige, dass umgekehrt für beliebiges~$s \in F(X)$ die Formel~\eqref{eq} eine
natürliche Transformation $\eta : \Hom(\freist, X) \Rightarrow F$ definiert.
\item Zeige mit a) und b), dass zumindest für festes~$X \in \Ob\C$ und~$F \in
\widehat\C$ eine Bijektion~\eqref{bij} existiert.
\item Linke und rechte Seite von~\eqref{bij} können als Auswertungen der
Funktoren
\[ \renewcommand{\arraystretch}{1.3}\begin{array}{@{}rl@{}}
  L : \C^\op \times \widehat\C \longrightarrow \Set, &
  (X,F) \longmapsto \Hom_{\widehat\C}(\Hom_\C(\freist,X), F)
  \\
  R : \C^\op \times \widehat\C \longrightarrow \Set, &
  (X,F) \longmapsto F(X)
\end{array} \]
an der Stelle~$(X,F)$ angesehen werden. Überlege, wie diese beiden Funktoren auf
Morphismen wirken, und zeige, dass sie zueinander isomorph sind.
\item Du hast soeben das Yoneda-Lemma bewiesen. Herzlichen Glückwunsch!
\end{enumerate}
\end{aufgabe}

\begin{aufgabe}{Cayley-Einbettung}
Der Satz von Cayley aus der Gruppentheorie besagt, dass sich jede Gruppe~$G$ in
eine symmetrische Gruppe (der Gruppe der Bijektionen einer bestimmten Menge)
einbetten lässt. Genauer gibt es stets folgenden injektiven
Gruppenhomomorphismus:
\[ \begin{array}{@{}rcl@{}}
  G &\longrightarrow& \operatorname{Sym}(G) := \{ \varphi : G \to G \,|\,
  \text{$\varphi$ bijektiv} \} \\
  g &\longmapsto& g \circ \freist
\end{array} \]

Vergleiche diese Einbettung mit der Yoneda-Einbettung für~$\C := BG$.
Erinnere dich dazu daran, wodurch Funktoren~$(BG)^\op \to \Set$ schon gegeben
sind.
\end{aufgabe}

\emph{Bitte wenden!}

\newpage

\begin{aufgabe}{Dichtheit der Yoneda-Einbettung}
Sei~$H:\C\to\D$ ein Funktor. Dann wird jedes Objekt~$Y \in \Ob\D$ auf
kanonische Art und Weise (welche?) zu einem Kokegel des Diagramms
\[ H/Y \stackrel{U}{\longrightarrow} \C \stackrel{H}{\longrightarrow} \D. \]
Der Funktor~$H$ heißt genau dann \emph{dicht}, wenn diese Kokegel sogar
Kolimiten sind. Die Kategorie~$H/Y$ ist dabei die sog. \emph{Kommakategorie}
\begin{align*}
  \text{Objekte: } & \text{alle Morphismen $H(X) \xra{p} Y$ in~$\D$, $X \in \Ob\C$} \\
  \text{Morphismen: } &
    \Hom(H(X) \xra{p} Y, H(\widetilde X) \xra{\tilde p} Y) := \\ &
      \quad\left\{ X \xra{f} \widetilde X \,\middle|\, \text{$
        \vcenter{\xymatrix{
          H(X) \ar[rr]^{H(f)} \ar[rd]_p && H(\widetilde X) \ar[dl]^{\tilde p} \\
          & Y
        }}$ kommutiert} \right\}\!,
\end{align*}
und der Funktor~$U : H/Y \to \C$ der Vergissfunktor
\[ \begin{array}{@{}rrcl@{}}
  U : & H/Y &\longrightarrow& \C \\
  & (H(X) \xra{p} Y) &\longmapsto& X \\
  & (X \xra{f} \widetilde X) &\longmapsto& f.
\end{array} \]
Zeige: Die Yoneda-Einbettung~$\C \to \widehat\C$ ist dicht.

\emph{Bemerkung:} Um das zum Ausdruck zu bringen, schreibt man auch für
Prägarben~$F$ auf~$\C$
\[ F \cong \int^{X \in \C} \Hom_\C(\freist,X) \otimes F(X). \]
\end{aufgabe}

\end{document}
