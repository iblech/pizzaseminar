\documentclass{pizzablatt}

\begin{document}

\maketitle{5}{3. April 2013}

\textbf{XXX: Hier fehlt noch eine spannende \emph{konkrete} Aufgabe!}

\begin{aufgabe}{Spitze schon im Diagramm}
Sei~$F:\D \to \C$ ein Diagramm in einer Kategorie~$\C$. Besitze~$\D$ ein
terminales Objekt~$T$.
Zeige per Hand oder mit dem Kriterium aus Aufgabe~3, dass dann
schon~$F(T)$ selbst zu einem Kolimes von~$F$ wird.
\end{aufgabe}

\begin{aufgabe}{Monomorphe natürliche Transformationen}
\begin{enumerate}
\item Sei~$f:X \to Y$ ein Morphismus einer Kategorie. Zeige, dass~$f$ genau
dann ein Monomorphismus ist, wenn das Diagramm
\[ \xymatrix{
  X \ar[r]^\id \ar[d]_\id & X \ar[d]^f \\
  X \ar[r]_f & Y
} \]
ein Faserproduktdiagramm ist.
\item Sei~$\eta : F \Rightarrow G$ eine natürliche Transformation zwischen
Funktoren $F, G : \C \to \D$. Besitze~$\D$ alle Faserprodukte. Zeige: $\eta$
ist genau dann ein Monomorphismus in~$\Funct(\C,\D)$, wenn alle
Komponenten~$\eta_X$ Monomorphismen in~$\D$ sind.
\end{enumerate}
\end{aufgabe}

\begin{aufgabe}{Kofinale Unterdiagramme}\small
In der Analysis gibt es folgende Mottos: \emph{Das Weglassen endlich vieler Folgeglieder
ändert nicht das Konvergenzverhalten. Teilfolgen konvergenter Folgen
konvergieren ebenfalls, und zwar gegen denselben Grenzwert.} Diese Mottos
wollen wir auf (Ko-)Limiten in der Kategorientheorie übertragen.

Dazu nennen wir einen Funktor $H : \D_0 \to \D$ genau dann \emph{kofinal}, wenn
für alle~$d \in \Ob\D$\ldots
\begin{enumerate}
\item[1.] ein Objekt $d_0 \in \Ob\D_0$ und ein Morphismus $d \to
H(d_0)$ in~$\D$ existiert und
\item[2.] für je zwei solcher Morphismen ein Objekt~$\widetilde d_0 \in \Ob\D_0$ und
Morphismen $d_0 \to \widetilde d_0$, $d_0' \to \widetilde d_0$ existieren, deren Bilder
unter~$H$ das Diagramm
\[ \xymatrix{
  d_0 \ar[r] \ar[d] & H(d_0) \ar@{-->}[d] \\
  H(d_0') \ar@{-->}[r] & H(\widetilde d_0)
} \]
kommutieren lassen.
\end{enumerate}
Etwa ist der Inklusionsfunktor~$B (2\NN) \to B(\NN)$ kofinal, wenn~$\NN$ die
Menge der natürlichen Zahlen mit ihrer gewöhnlichen Ordnung und~$2\NN$ die
Teilordnung der geraden Zahlen bezeichnet.

Sei nun~$F : \D \to \C$ ein~$\D$-förmiges Diagramm in einer Kategorie~$\C$.
\begin{enumerate}
\item
Zeige: Die Kategorie der Kokegel von~$F$ ist äquivalent zur Kategorie der
Kokegel von~$F \circ H$.
\item Was folgt daher über das Verhältnis der Kolimiten von~$F$ und~$F \circ
H$?
\end{enumerate}
\end{aufgabe}

\end{document}
