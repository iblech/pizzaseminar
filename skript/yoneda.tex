\section[Das Yoneda-Lemma]{Das Yoneda-Lemma \hfill \small Justin Gassner}

\emph{Werbung:} Wir werden das fundamentale Yoneda-Lemma und seine Korollare
verstehen. Dazu werden wir zunächst eine hilfreiche Intuition von sog. Prägarben auf
Kategorien entwickeln und verstehen, welche Signifikanz die Darstellbarkeit von
Prägarben hat. Dann können wir die Yoneda-Einbettung kennenlernen, ihre
Eigenschaften studieren und sehen, wozu sie nützlich ist. Das fundamentale
Motto der Kategorientheorie wird damit zu einem formalen Theorem.


\subsection{Prägarben als ideelle Objekte}

Sei in diesem Abschnitt~$\C$ eine lokal kleine Kategorie.

\begin{defn}Funktoren~$\C^\op \to \Set$ heißen auch \emph{Prägarben
auf~$\C$}. Die Kategorie der Prägarben auf~$\C$ (mit natürlichen
Transformationen als Morphismen) ist~$\widehat\C := \Funct(\C^\op,\Set)$.
\end{defn}

\begin{defn}Eine Prägarbe~$F:\C^\op\to\Set$ auf~$\C$ heißt genau dann
\emph{darstellbar}, wenn es ein Objekt~$X \in \Ob\C$ mit $F \cong
\Hom_\C(\freist,X)$ gibt.\end{defn}

\begin{motto}Eine beliebige (nicht unbedingt darstellbare) Prägarbe~$F$
auf~$\C$ beschreibt die Beziehungen aller Objekte~$A$ von~$\C$ mit einem
eingebildeten, fiktiven, ideellen Objekt~$\heartsuit$: Wir stellen uns die
Menge~$F(A)$ als Menge der Morphismen~$A \to \heartsuit$ vor.\end{motto}

Die Prägarbenkategorie~$\widehat\C$ enthält stets mindestens eine
nicht-darstellbare Prägarbe, und zwar die initiale Prägarbe
\[ \begin{array}{@{}rrcl@{}}
  0 : & \C^\op &\longrightarrow& \Set \\
  & A &\longmapsto& \emptyset \\
  & f &\longmapsto& \id_\emptyset.
\end{array} \]
\begin{prop}Die initiale Prägarbe ist nicht darstellbar.\end{prop}
\begin{proof}Sei~$0 \cong \Hom_\C(\freist,X)$ für ein Objekt~$X\in\Ob\C$.
Dann folgt~$\Hom_\C(X,X) \cong 0(X) = \emptyset$ im Widerspruch zu
$\id_X \in \Hom_\C(X,X)$.\end{proof}


\subsection{Die Yoneda-Einbettung}

Der volle Hom-Funktor geht von~$\C^\op \times \C$ zu~$\Set$. Aus diesem kann
man durch Curryfizierung einen Funktor~$\C \to \Funct(\C^\op,\Set)$ basteln:
\[ \begin{array}{@{}rrcl@{}}
  Y : & \C &\longrightarrow& \widehat\C \\
  & X &\longmapsto& \widehat X := \Hom_\C(\freist,X) \\
  & f &\longmapsto& \widehat f
\end{array} \]
Die Komponenten der natürlichen Transformation~$\widehat f$ sind dabei
für~$f: X \to Y$ durch
\[ \begin{array}{@{}rrcl@{}}
  (\widehat f)_A : & \Hom_\C(A,X) &\longrightarrow& \Hom_\C(A,Y) \\
  & g &\longmapsto& f \circ g
\end{array} \]
gegeben.

\begin{defn}Der Funktor~$Y:\C\to\widehat\C$ heißt \emph{Yoneda-Einbettung}
von~$\C$.\end{defn}

Die Yoneda-Einbettung hat folgende wichtige Eigenschaften:
\begin{prop}\begin{enumerate}
\item $Y$ ist treu und voll.
\item $Y$ erhält Limiten (aber kaum Kolimiten).
\item $Y$ ist dicht (siehe Aufgabe~2 von Übungsblatt~6).
\end{enumerate}\end{prop}
Vermöge~$Y$ können wir~$\C$ also als volle Unterkategorie der Kategorie der
fiktiven, ideellen Objekte ansehen.


\subsection{Das Yoneda-Lemma}
