\section[Das Yoneda-Lemma]{Das Yoneda-Lemma \hfill \small Justin Gassner}

\emph{Werbung:} Wir werden das fundamentale Yoneda-Lemma und seine Korollare
verstehen. Dazu werden wir zunächst eine hilfreiche Intuition von sog. Prägarben auf
Kategorien entwickeln und verstehen, welche Signifikanz die Darstellbarkeit von
Prägarben hat. Dann können wir die Yoneda-Einbettung kennenlernen, ihre
Eigenschaften studieren und sehen, wozu sie nützlich ist. Das fundamentale
Motto der Kategorientheorie wird damit zu einem formalen Theorem.


\subsection{Prägarben als ideelle Objekte}

Sei in diesem Abschnitt~$\C$ eine lokal kleine Kategorie.

\begin{defn}Funktoren~$\C^\op \to \Set$ heißen auch \emph{Prägarben
auf~$\C$}. Die Kategorie der Prägarben auf~$\C$ (mit natürlichen
Transformationen als Morphismen) ist~$\widehat\C := \Funct(\C^\op,\Set)$.
\end{defn}

\begin{defn}Eine Prägarbe~$F:\C^\op\to\Set$ auf~$\C$ heißt genau dann
\emph{darstellbar}, wenn es ein Objekt~$X \in \Ob\C$ mit $F \cong
\Hom_\C(\freist,X)$ gibt.\end{defn}

\begin{motto}\label{praegarbeideell}%
Eine beliebige (nicht unbedingt darstellbare) Prägarbe~$F$
auf~$\C$ beschreibt die Beziehungen aller Objekte~$A$ von~$\C$ mit einem
eingebildeten, fiktiven, ideellen Objekt~$\heartsuit$: Wir stellen uns die
Menge~$F(A)$ als Menge der Morphismen~$A \to \heartsuit$ vor.\end{motto}

Die Prägarbenkategorie~$\widehat\C$ enthält stets mindestens eine
nicht-darstellbare Prägarbe, und zwar die initiale Prägarbe
\[ \begin{array}{@{}rrcl@{}}
  0 : & \C^\op &\longrightarrow& \Set \\
  & A &\longmapsto& \emptyset \\
  & f &\longmapsto& \id_\emptyset.
\end{array} \]
\begin{prop}Die initiale Prägarbe ist nicht darstellbar.\end{prop}
\begin{proof}Sei~$0 \cong \Hom_\C(\freist,X)$ für ein Objekt~$X\in\Ob\C$.
Dann folgt~$\Hom_\C(X,X) \cong 0(X) = \emptyset$ im Widerspruch zu
$\id_X \in \Hom_\C(X,X)$.\end{proof}


\subsection{Die Yoneda-Einbettung}

Der volle Hom-Funktor geht von~$\C^\op \times \C$ zu~$\Set$. Aus diesem kann
man durch Curryfizierung einen Funktor~$\C \to \Funct(\C^\op,\Set)$ basteln,
\[ \begin{array}{@{}rrcl@{}}
  Y : & \C &\longrightarrow& \widehat\C \\
  & X &\longmapsto& \widehat X := \Hom_\C(\freist,X) \\
  & f &\longmapsto& \widehat f,
\end{array} \]
wobei die Komponenten der natürlichen Transformation~$\widehat f$ durch
Nachkomponieren mit~$f$ wirken:
\[ \begin{array}{@{}rrcl@{}}
  (\widehat f)_A : & \Hom_\C(A,X) &\longrightarrow& \Hom_\C(A,Y) \\
  & g &\longmapsto& f \circ g
\end{array} \]

\begin{defn}Der Funktor~$Y:\C\to\widehat\C$ heißt \emph{Yoneda-Einbettung}
von~$\C$.\end{defn}

Die Yoneda-Einbettung hat viele gute Eigenschaften, etwa\ldots
\begin{enumerate}
\item ist sie treu und voll,
\item erhält Limiten (aber kaum Kolimiten) und
\item ist dicht.
\end{enumerate}
Vermöge der ersten Eigenschaft können wir daher~$\C$ als volle Unterkategorie
der Kategorie der ideellen Objekte ansehen. Im Bild der Vervollständigung der
rationalen Zahlen lautet die analoge Aussage, dass die Inklusion~$\QQ \to \RR$
tatsächlich injektiv ist. Der Beweis ist eine einfache
Anwendung des noch folgenden Yoneda-Lemmas.

Eigenschaft~b) drückt aus, dass Limesbildung verträglich mit dem Übergang von
tat\-säch\-li\-chen Objekten von~$\C$ zu ideellen Objekten ist. Analog erhält auch die
Inklusion~$\QQ \to \RR$ Limiten.

Die Yoneda-Einbettung ist im Allgemeinen weit entfernt davon, wesentlich
surjektiv zu sein. Konkret haben wir gesehen, dass die initiale Prägarbe
niemals isomorph zu einem Objekt der Form~$Y(X)$ ist. In Analogie ist die
Inklusion~$\QQ \to \RR$ weit entfernt davon, surjektiv zu sein. Es ist
allerdings jede reelle Zahl Limes rationaler Zahlen -- und so ist es hier auch:
Eigenschaft~c) besagt, dass jede Prägarbe auf kanonische Art und Weise Kolimes
darstellbarer Prägarben ist. Für eine genaue Formulierung siehe Aufgabe~3 von
Übungsblatt~6.


\subsection{Das Yoneda-Lemma}

\begin{lemma}[Yoneda-Lemma]Es gibt eine in~$X \in \Ob \C$ und~$F \in \Ob \widehat \C$
natürliche Bijektion
\[ \Hom_{\widehat\C}(\Hom_\C(\freist, X), F) \cong F(X). \]
\end{lemma}
\begin{proof}Siehe Aufgabe~1 von Übungsblatt~6.\end{proof}
Linke und rechte Seite der Isomorphie können als Auswertungen der Funktoren
\[ \renewcommand{\arraystretch}{1.3}\begin{array}{@{}rl@{}}
  L : \C^\op \times \widehat\C \longrightarrow \Set, &
  (X,F) \longmapsto \Hom_{\widehat\C}(\Hom_\C(\freist,X), F)
  \\
  R : \C^\op \times \widehat\C \longrightarrow \Set, &
  (X,F) \longmapsto F(X)
\end{array} \]
an der Stelle~$(X,F)$ angesehen werden. Bei~$\C^\op \times \widehat\C$ handelt
es sich um die Produktkategorie aus Definition~\ref{productcat}.
Das Prädikat \emph{natürlich} im Yoneda-Lemma bezieht
sich darauf, dass diese beide Funktoren zueinander natürlich isomorph sind.

\begin{bsp}Für eine Prägarbe~$F$ haben wir uns in
Motto~\ref{praegarbeideell} die Menge~$F(A)$ als Menge der Morphismen von~$A$
in ein ideelles Objekt~$\heartsuit$ vorgestellt. Natürlich können wir
nicht~"`$\Hom_\C(A,\heartsuit)$"' schreiben, da~$\heartsuit$ nicht wirklich ein
Objekt von~$\C$ ist; aber in der Prägarbenkategorie gibt es die Morphismenmenge
$\Hom_{\widehat\C}(\widehat A,F)$, und nach dem Yoneda-Lemma ist diese isomorph
zu~$F(A)$. In diesem Sinn gilt also~$\heartsuit = F$ und das Motto ist sogar formal
korrekt.\end{bsp}

% XXX: Weitere Bestärkung des Gleichmäßigkeitsmottos: Hom(__,X) ==> Hom(__,Y)
% schon durch X --> Y induziert.
