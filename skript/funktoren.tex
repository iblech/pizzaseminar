\section[Funktoren]{Funktoren \hfill \small Felicitas Hörmann}

So, wie es Gruppenhomomorphismen zwischen Gruppen gibt, gibt es Funktoren
zwischen Kategorien. Ihre beeindruckendste Anwendung liegt darin, dass sie
zwischen unterschiedlichen Teilgebieten der Mathematik vermitteln können -- das
ist ein Grundgedanke der algebraischen Topologie. Man verwendet sie aber auch,
um verschiedene Arten von Konstruktionen übersichtlich zu organisieren und
einen sinnvollen Rahmen für die Frage nach "`bestmöglichen"' Konstruktionen mit
vorgegebenem Ziel zu haben.

\begin{defn}Ein \emph{Funktor}~$F : \C \to \D$ zwischen Kategorien~$\C$, $\D$
besteht aus
\begin{enumerate}
\item einer Vorschrift, die jedem Objekt~$X \in \Ob \C$ ein Objekt~$F(X) \in \Ob \D$
zuordnet, und
\item einer Vorschrift, die jedem Morphismus~$f:X \to Y$ in~$\C$ einen
Morphismus~$F(f) : F(X) \to F(Y)$ zuordnet,
\end{enumerate}
sodass
\begin{enumerate}
\item $F(\id_X) = \id_{F(X)}$ für alle Objekte~$X \in \Ob \C$ und
\item $F(g \circ f) = F(g) \circ F(f)$ für alle komponierbaren Morphismen $g$, $f$
in~$\C$.
\end{enumerate}
\end{defn}
\begin{bem}\label{gleichheitfunktoren}%
Quelle und Ziel der abgebildeten Morphismen~$F(f)$ sind also durch
den Objektteil des Funktors schon vorgegeben. Es ist nicht sinnvoll, von der
Gleichheit von Funktoren~$F,G : \C \to \D$ zu sprechen -- denn das würde
naheliegenderweise ja die Aussage umfassen, dass für alle Objekte~$X \in \Ob \C$ die
Gleichheit
\[ F(X) = G(X) \]
von Objekten in~$\D$ gilt. Aber wie schon in Bemerkung~\ref{gleichheitobj}
festgehalten, ist das keine sinnvolle Aussage.
\end{bem}

\begin{prop}
Ein Funktor überführt kommutative Diagramme in kommutative Diagramme:
\[ \vcenter{ \xymatrix@=8ex{
  X \ar[r]^{f} \ar[rd]_{h} & Y \ar[d]^{g} \\
  & Z
} }
\qquad \overset{F}{\longmapsto} \qquad
\vcenter{ \xymatrix@=8ex{
  F(X) \ar[r]^{F(f)} \ar[rd]_{F(h)} & F(Y) \ar[d]^{F(g)} \\
  & F(Z)
} } \]
\end{prop}
\begin{proof}
Gilt $h = g \circ f$, so folgt~$F(h) = F(g \circ f) = F(g) \circ F(f)$.
\end{proof}


\subsection{Funktoren als Diagramme}

Es sei $\I$ die durch die folgende Skizze gegebene Kategorie und $\C$ eine beliebige Kategorie.

\[ \xymatrix{
  & \bullet_1 \ar[d] \ar@(ur,ul) \\
  \bullet_2 \ar[r] \ar@(ul,dl) & \bullet_3 \ar@(dr,ur)
} \]

Um einen Funktor $F : \I \to \C$ anzugeben, muss man
\begin{enumerate}
  \item Objekte~$X_1 = F(\bullet_1)$, $X_2 = F(\bullet_2)$ und~$X_3 =
  F(\bullet_3)$ in~$\C$ und
  \item Morphismen $f:X_1 \to X_3$ und $g:X_2 \to X_3$ in~$\C$
\end{enumerate}
spezifizieren. Ein solcher Funktor ist also durch ein Diagramm der Form

\[ \xymatrix{
  & X_1 \ar[d]^f \ar@(ur,ul) \\
  X_2 \ar[r]_g \ar@(ul,dl) & X_3 \ar@(dr,ur)
} \]
in~$\C$ gegeben. Da diese Überlegung analog mit anderen Kategorien~$\I$
funktioniert, sehen wir folgendes Motto:
\begin{motto}Funktoren~$\I \to \C$ sind~$\I$-förmige Diagramme
in~$\C$.\end{motto}

% XXX: gerichtete Graphen sind Funktoren (* ==> *) --> Set.


\subsection{Kontravariante Funktoren}

Wie kann man sich einen Funktor $F : \C^\op \to \D$ vorstellen?
\begin{enumerate}
  \item Objekte $X \in \Ob \C^\op = \Ob \C$ werden auf Objekte $F(X) \in \mathcal{D}$
  abgebildet.
  \item Morphismen $f : X \to Y$ in $\C^\op$ (d.\,h. $f : Y \to X$ in~$\C$)
  werden auf Morphismen $F(f) : F(X) \to F(Y)$ in $\mathcal{D}$ abgebildet.
\end{enumerate}
Das zweite Funktoraxiom lautet für Morphismen~$X \xra{f} Y \xra{g} Z$
in~$\C^\op$
\[ F(g \circ f) = F(f \bullet g) = F(f) \circ F(g), \] 
wobei wir zur Verdeutlichung "`$\circ$"' für die Komposition in $\C$ und
"`$\bullet$"' in $\C^\op$ schreiben. Die Zuordnung
\[ \begin{array}{@{}rcl@{}}
  \C &\longrightarrow& \D \\
  X  &\longmapsto& F(X) \\
  f  &\longmapsto& F(f)
\end{array} \]
ist also kein Funktor in unserem Sinne, da er Quelle und Ziel von Morphismen
vertauscht und das zweite Funktoraxiom dann nur in entsprechend umgekehrter
Kompositionsreihenfolge erfüllt. Solche Zuordnen sind trotzdem wichtig; sie
heißen \emph{kontravariante Funktoren}.


\subsection{Beispiele für Funktoren}

\subsubsection{Langweilige Funktoren}

\begin{enumerate}
  \item Für jede Kategorie $\C$ gibt es den \emph{Identitätsfunktor}
  \[ F : \C \to \C, \quad X \mapsto X, \quad f \mapsto f. \]
  \item Für ein festes Object $\heartsuit \in \C$ hat man den \emph{konstanten Funktor}
  \[ F : \I \to \C, \quad X \mapsto \heartsuit, \quad f \mapsto \id_\heartsuit. \]
\end{enumerate}

Diese Funktoren als solche sind langweilig. Interessant sind aber natürliche
Transformationen zwischen ihnen -- das werden wir im folgenden Vortrag sehen.


\subsubsection{Vergissfunktoren}

Die bekannten Strukturen in der Mathematik organisieren sich in einer
Hierarchie. Zwischen den Kategorien zu Strukturen verschiedener Stufen hat man sog.
Vergissfunktoren:

\begin{enumerate}
  \item Der Funktor
  \[ V : \Grp \to \Set, \quad (G,\circ) \mapsto G, \quad f \mapsto f. \]
  bildet Gruppen auf ihre zugrundeliegenden Mengen und Gruppenhomomorphismen
  auf ihre zugrundeliegenden Mengenabbildung ab. Er vergisst also die
  \emph{Struktur} der Gruppenverknüpfung.
  \item Der Funktor
  \[ V : \RR\text{-}\Vect \to \AbGrp, \quad (V,+,\cdot) \mapsto (V,+), \quad f \mapsto f. \]
  vergisst ebenfalls algebraische Struktur, nämlich die Skalarmultiplikation.
  \item Der Funktor
  \[ V : \Man \to \Top, \quad M \mapsto M, \]
  die einer Mannigfaltigkeit ihren zugrundeliegenden topologischen Raum
  zuordnet, vergisst (differentialgeometrische) Struktur.
  \item Der Funktor
  \[ V : \AbGrp \to \Grp, \quad (G,\circ) \mapsto (G,\circ), \quad f \mapsto f. \]
  vergisst die \emph{Eigenschaft} der Gruppenverknüpfung $\circ$, kommutativ zu sein.
  \item Schreibe $1$ für die Kategorie mit $\Ob = \lbrace \bullet \rbrace$ und $\Hom(\bullet,\bullet) = \lbrace \id_\bullet \rbrace$. Der Funktor
  \[ V : \Set \to 1, \quad M \mapsto \bullet, \quad f \mapsto \id_\bullet \]
  vergisst \emph{stuff}, also Zeug.
\end{enumerate}

Die Unterscheidung zwischen Eigenschaft, Struktur und Zeug stammt übrigens
von Teilnehmern eines Seminars über
Quantengravitation~\cite[Abschn.~2.4]{lectures-on-n-categories}, siehe
auch~\cite{ncatlab:stuff}.

Obwohl die Vergissfunktoren beinahe tautologisch definiert sind, sind sie aus
zwei Gründen wichtig: Zum einen ist es eine interessante Frage, inwieweit
man die Vergissfunktoren umkehren kann -- wie man etwa aus einer Menge eine
Gruppe machen kann. Wie diese Frage zu präzisieren und zu beantworten ist,
werden wir im Vortrag über adjungierte Funktoren lernen.

Zum anderen ist es wichtig zu wissen, ob ein Vergissfunktor Produkte (oder
allgemeinere Limiten) bewahrt. Etwa gilt für Vektorräume~$U, W$ und den
Vergissfunktor~$V:\RR\text{-}\Vect \to \Set$, dass
\[ V(U \times W) \cong V(U) \times V(W), \]
aber
\[ V(U \amalg W) \not\cong V(U) \amalg V(W). \]
Was das genau bedeutet, werden wir im Vortrag über Limiten sehen.


\subsubsection{Funktoren aus algebraischen Konstruktionen}

Zu jedem Ring~$R$ gibt es seinen Polynomring~$R[X]$ der formalen Polynome mit
Koeffizienten aus~$R$,
\[ R[X] = \Bigl\{ \sum_{i=0}^n a_i X^i \,\Big|\, a_0,\ldots,a_n \in R, n \geq 0
\Bigr\}. \]
Diese Konstruktion kann man zu einem Funktor erheben, den sog.
\emph{Polynomringfunktor} $F : \Ring \to \Ring$: Dieser ordnet einem Ring $R$
den Polynomring $R[X]$ und einem Ringhomomorphismus $f : R \to S$ folgenden
induzierten Ringhomomorphismus zu:
\[ F(f) : R[X] \to S[X], \quad \sum a_n X^n \mapsto \sum f(a_n) X^n. \]

\begin{bem}Algebraiker kann man daran erkennen, dass sie im Gegensatz zu
Analytikern die Polynomvariable groß schreiben.\end{bem}

Fast jede algebraische Konstruktion kann man auf diese Art und Weise behandeln.


\subsubsection{Funktoren und Mengen}

Zu jeder Menge $M$ gibt es die \emph{diskrete Kategorie} $DM$:
\begin{align*}
  \Ob DM &:= M \\
  \Hom_{DM}(m,\tilde{m}) &:=
  \left\{ \id_m \,\middle|\, m = \tilde m \right\}
\intertext{Die Angabe der Morphismenmengen ist etwas kryptisch geschrieben, ausführlich
kann man die Definition auch wie folgt angeben:}
  \Hom_{DM}(m,\tilde{m}) &:=
  \begin{cases}
    \lbrace \id_m \rbrace, & \text{falls $m = \tilde{m}$} \\
    \emptyset, & \text{sonst}
  \end{cases}
\end{align*}
Sind nun $M$ und $N$ zwei Mengen und $\varphi : M \to N$ eine Abbildung, so ist
\[ DM \to DN, \quad m \mapsto \varphi(m), \quad \id_m \mapsto \id_{\varphi(m)} \]
ein Funktor. [Hier fehlt eine Skizze.] Somit sehen wir folgendes Motto:
\begin{motto}Das Funktorkonzept verallgemeinert das Konzept der Abbildung
zwischen Mengen.\end{motto}

\subsubsubsection{Potenzmengenfunktoren}

Der \emph{kovariante Potenzmengenfunktor} $\mathcal{P} : \Set \to \Set$ ordnet einer Menge $M$ die Potenzmenge $\mathcal{P}(M)$ zu und einer Abbildung $f : M \to N$ die Abbildung
\[ \mathcal{P}(f) : \mathcal{P}(M) \to \mathcal{P}(N), \quad U \mapsto f[U], \]
wobei $f[U] := \left\{ f(u) : u \in U \right\}$ ist.

Definiert man $\mathcal{P}(f)$ stattdessen durch $U \mapsto f[U]^c$
(Komplement), so erhält man keinen Funktor.

Außerdem gibt es noch den \emph{kontravarianten Potenzmengenfunktor}
$\mathcal{P} : \Set^\op \to \Set$, der ebenfalls jeder Menge~$M$ ihre
Potenzmenge, aber jeder Abbildung~$f : M \to N$ die \emph{Urbild}abbildung
\[ \mathcal{P}(f) : \mathcal{P}(N) \to \mathcal{P}(M), \quad V \mapsto
f^{-1}[V] \]
zuordnet, wobei~$f^{-1}[V] := \left\{ x \in M \,|\, f(x) \in V \right\}$.
Dieser ist sehr bedeutsam, denn er zeigt die Äquivalenz der dualen
Kategorie~$\Set^\op$ mit der Kategorie vollständiger atomischer boolescher
Algebren, siehe~\cite[Thm.~2.4]{oosten}. Was \emph{Äquivalenz} bedeutet, werden
wir im folgenden Kapitel lernen.


\subsubsection{Funktoren und Gruppen}

Es sei ein Gruppenhomomorphismus $\varphi : G \to H$ gegeben. Dann ist
  \[ f : BG \to BH, \quad \bullet \mapsto \bullet, \quad g \mapsto \varphi(g) \]
ein Funktor. (Zur Konstruktion der Kategorien $BG$ und $BH$ siehe Übungsblatt~1, Aufgabe~5.)
Denn das erste Funktoraxiom ist erfüllt,
\[
  F(\id_\bullet) = F(e_G) = \varphi(e_G) = e_H = \id_\bullet,
\]
und das zweite ebenso: Für alle Morphismen~$g, \tilde g : \bullet \to \bullet$
(d.\,h. für alle Gruppenelemente~$g, \tilde g \in G$) gilt
\[
  F(\tilde{g} \circ g) = F(\tilde{g} \cdot g) = \varphi(\tilde{g} \cdot g) =
  \varphi(\tilde{g}) \cdot \varphi(g) = \varphi(\tilde{g}) \circ \varphi(g) =
  F(\tilde{g}) \circ F(g). \]
Damit sehen wir folgendes Motto:
\begin{motto}Das Funktorkonzept verallgemeinert das Konzept des
Gruppenhomomorphismus.\end{motto}

\subsubsubsection{Gruppenwirkungen}

Was muss man angeben, um einen Funktor $F : BG \to \Set$ zu spezifizieren? Eine
Menge $M := \varphi(\bullet)$ und zu jedem $g \in G$ eine Abbildung $\varphi_g : M \to M$, sodass
\[ \varphi_{\id_\bullet} = \id_M \quad \text{und} \quad \varphi_{g \circ h} = \varphi_g \circ \varphi_h \]
für alle $g, h \in G$ gilt. Mit der Schreibweise $\varphi_g(x) =: g \cdot x$, $g \in G$, $x \in X$, wird dies zu
\[ e \cdot x = x \quad \text{und} \quad (g \circ h) \cdot x = g \cdot (h \cdot x).\]
Eine solche Struktur bestehend aus einer Menge~$M$ und einer
Multiplikationsabbildung~$G \times M \to M$, die diese Axiome erfüllt, ist eine
sog. \emph{Gruppenwirkung von~$G$}. Wir sehen also: Funktoren~$BG \to \Set$
sind "`dasselbe"' wie Gruppenwirkungen von~$G$.

Analog kann man Funktoren~$BG \to K\text{-}\Vect$ untersuchen. Solche haben
auch einen klassischen Namen: Das sind sog. \emph{Gruppendarstellungen}.


\subsubsection{Funktoren als Datenbanken}

Wir wollen an einem Beispiel zeigen, dass auch so konkrete Dinge wie
Datenbanken aus der Informatik kategoriell verstanden werden können. Etwa gibt
das zugrundeliegende Datenbankschema der 0815/Datenbank aus Tafel~\ref{db0815}
Anlass zu folgender Kategorie~$\C$:

\[ \xymatrixcolsep{5pc} \xymatrixrowsep{5pc} \xymatrix{
  \text{Angestellte}
    \ar@(ul,ur)^{\text{Vorg.}}
    \ar[r]^{\text{Abt.}}
    \ar@/^2pc/[d]^{\text{Nachname}}
    \ar@/_2pc/[d]_{\text{Vorname}}
  & \text{Abteilung}
    \ar@/^3pc/[dl]^{\text{Titel}} \\
  \text{String}
} \]

Die Tabelleninhalte kann man dann über einen Funktor~$\C \to \Set$ kodieren,
der jedes Objekt (also jeden Tabellennamen) auf die Menge der Primärschlüssel
ihrer Zeilen und jeden Morphismus (also jeden Spaltennamen) auf die
entsprechende Abbildung zwischen den Primärschlüsseln der beteiligten Tabellen
abbildet.

Gewisse einfache Integritätsbedingungen kann man über die Angabe eines
geeigneten Kompositionsgesetzes in~$\C$ kodieren. Wenn man etwa ausdrücken
möchte, dass der Sekretär einer Abteilung selbst in dieser sitzt, kann man
\[ \text{Abt.} \circ \text{Sekretär} = \id_{\text{Abteilung}} :
  \text{Abteilung} \to \text{Abteilung} \]
definieren. Diese Sichtweise auf Datenbanken ist unter Anderem für das
Verständnis von Datenmigrierung bei Schemaänderungen hilfreich. Details hat
David Spivak erforscht~\cite{spivak1,spivak2,spivak3}.

\begin{figure}
  \begin{center}
    \small
    \begin{tabular}{|l||l|l|l|l|}
      \hline
      \multicolumn{5}{|c|}{Angestellte} \\ \hline
      \textbf{Nr.} & \textbf{Vorname} & \textbf{Nachname} & \textbf{Vorg.} & \textbf{Abt.} \\ \hline
      101 & David & Hilbert & 103 & q10 \\
      102 & Bertrand & Russel & 102 & x02 \\
      103 & Alan & Turing & 103 & q10 \\
      \hline
    \end{tabular}
    \quad
    \begin{tabular}{|l||l|l|}
      \hline
      \multicolumn{3}{|c|}{Abteilung} \\ \hline
      \textbf{Nr.} & \textbf{Titel} & \textbf{Sekretär} \\ \hline
      q10 & Vertrieb & 101 \\
      x02 & Produktion & 102 \\
      \hline
    \end{tabular}
  \end{center}

  \caption{\label{db0815}Ein Standardbeispiel einer Datenbank.}
\end{figure}

\subsubsection{Hom-Funktoren}

\begin{defn}
Sei $\C$ eine lokal kleine Kategorie (sodass ihre Hom-Klassen sogar schon
Hom-Mengen sind) und $A \in \Ob \C$. Dann ist\ldots
\begin{enumerate}
  \item der \emph{kovariante Hom-Funktor zu $A$} der Funktor
    \[ \begin{array}{@{}rrcl@{}}
      \Hom_\C(A,\freist): & \C &\longrightarrow& \Set \\
      & X &\longmapsto& \Hom_\C(A,X) \\
      & (f:X \to Y) &\longmapsto& f_\star
    \end{array} \]
  \item und der \emph{kontravariante Hom-Funktor zu~$A$} der Funktor
    \[ \begin{array}{@{}rrcl@{}}
      \Hom_\C(\freist,A): & \C^\op &\longrightarrow& \Set \\
      & X &\longmapsto& \Hom_\C(X,A) \\
      & (f:X \xra{\C} Y) &\longmapsto& f^\star.
    \end{array} \]
\end{enumerate}
Dabei sind die Abbildungen~$f_\star, f^\star$ wie folgt definiert:
\[ \begin{array}{@{}rrcl@{}}
  f_\star: & \Hom_\C(A,X) &\longrightarrow& \Hom_\C(A,Y) \\
  & g &\longmapsto& f \circ g \\
  \\
  f^\star: & \Hom_\C(Y,A) &\longrightarrow& \Hom_\C(X,A) \\
  & g &\longmapsto& g \circ f
\end{array} \]
\end{defn}

Die Hom-Funktoren kodieren die Beziehungen von~$A$ mit den Objekten aus~$\C$.
Das zentrale \emph{Yoneda-Lemma} wird uns sagen, dass~$A$ durch Kenntnis des
ko- oder kontravarianten Hom-Funktors schon bis auf Isomorphie eindeutig
bestimmt ist.


\subsubsection{Weitere Beispiele}

\begin{enumerate}
\item Den Prozess des Differenzierens glatter Abbildungen zwischen
Mannigfaltigkeiten kann man als Funktor auffassen, der jeder Mannigfaltigkeit
ihr Tangentialbündel und jeder glatter Abbildung ihr Differential zuordnet:
\[ \begin{array}{@{}rcl@{}}
  \Man &\longrightarrow& \Man \\
  M &\longmapsto& TM \\
  f &\longmapsto& Df
\end{array} \]
Es gibt auch eine "`lokale Version"', wenn man die Kategorie der
\emph{punktierten glatten Mannigfaltigkeiten}~$\Man_\star$ betrachtet: Die
Objekte dieser Kategorie sind Tupel $(M,x)$ aus einer Mannigfaltigkeit und
einem ausgezeichneten Basispunkt~$x \in M$, Morphismen sind
basispunkterhaltende glatte Abbildungen. Dann hat man den Funktor
\[ \begin{array}{@{}rcl@{}}
  \Man_\star &\longrightarrow& \RR\text{-}\Vect \\
  (M,x) &\longmapsto& T_x M \\
  f &\longmapsto& d_x f.
\end{array} \]
In beiden Fällen ist das zweite Funktoraxiom gerade deswegen erfüllt, weil die
Kettenregel gilt!

\item Hier könnte dein Beispiel stehen.
\end{enumerate}


\subsection{Die Kategorie der Kategorien}

Nach dem fundamentalen Motto der Kategorientheorie sollen wir die Beziehungen
zwischen Untersuchungsgegenständen ernst nehmen und daher die von ihnen
gebildete \emph{Kategorie} betrachten. Als wir bisher Kategorientheorie
betrieben haben, haben wir dieses Motto bezogen auf Kategorien selbst aber
sträflich vernachlässigt! Diesen Missstand behebt folgende Definition.
\begin{defn}
Die Kategorie $\Cat$ der (kleinen) Kategorien besteht aus:
\begin{align*}
  \Ob \Cat &:= \text{Klasse aller (kleinen) Kategorien} \\
  \Hom_\Cat(\C,\mathcal{D}) &:= \text{Klasse der Funktoren zwischen $\C$ und $\mathcal{D}$}
\end{align*}
\end{defn}
Die Verkettung~$G \circ F : \C \to \E$ zweier Funktoren~$F : \C \to \D$ und~$G
: \D \to \E$ ist dabei als der Funktor
\[ \begin{array}{@{}rrcl@{}}
  G \circ F: & \C &\longrightarrow& \E \\
  & X &\longmapsto& G(F(X)) \\
  & f &\longmapsto& G(F(f))
\end{array} \]
definiert.

\begin{bem}Ironischerweise ist es keine gute Idee, die so definierte
Kategorie~$\Cat$ zu untersuchen: Denn in Kategorien muss es sinnvoll sein, von
der Gleichheit zweier Morphismen zu sprechen -- der Gleichheitsbegriff zwischen
Funktoren ist aber, wie eingangs schon bemerkt, nicht interessant. Tatsächlich ist
die Kategorie~$\Cat$ nur eine erste Approximation an eine sog. 2-Kategorie, in
der es nicht nur Morphismen (Funktoren) zwischen Objekten (Kategorien), sondern
auch "`höhere Morphismen"', sog. 2-Morphismen (hier natürliche
Transformationen), zwischen den gewöhnlichen (1-)Morphismen gibt.
\end{bem}

% TODO:
% * Was bedeutet Vorschrift in der Funktordefinition?
%   * Eindeutigkeit der Zuordnung
% * Rückfrage, dass keine Kommutativitätsbedingung vorhanden ist
% * I darf beim konstanten Funktor auch leer sein
% * Gegenbspfkt.: Komplement, V |-> Menge seiner Basen
% * Q/R-Struktur
% * BG: Basisraum
