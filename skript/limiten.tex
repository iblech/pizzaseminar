\section[Limiten und Kolimiten]{Limiten und Kolimiten \hfill \small
Kathrin Gimmi}

\emph{Werbung:} Wir verstehen, was allgemeine Limiten und Kolimiten von
Diagrammen sind. Dazu wird es viele Beispiele geben, unter anderem die uns
schon bekannten Produkte und Koprodukte. Speziell sind sog. filtrierte
Kolimiten wichtig, da diese in der täglichen Praxis oft vorkommen und besonders
schöne Eigenschaften haben. Abschließend werden wir die Frage diskutieren, wie
man Kategorien, denen es an Limiten oder Kolimiten mangelt, vervollständigen
kann.

In diesem Abschnitt wollen wir Funktoren~$\I \to \C$ auch als~($\I$-förmige)
Diagramme bezeichnen.

\begin{defn}Ein \emph{Kegel} eines Diagramms~$F : \I \to \C$ besteht aus
\begin{enumerate}
\item einem Objekt~$K \in \Ob \C$ (der sog. \emph{Kegelspitze}) zusammen mit
\item jeweils einem Morphismus~$\pi_i : K \to F(i)$ für jedes Objekt~$i \in \Ob\I$,
\end{enumerate}
sodass
für alle Morphismen~$f : i \to j$ in~$\I$ die Dreiecke
\[ \xymatrix{
  & K \ar[ld]_{\pi_i} \ar[rd]^{\pi_j} \\
  F(i) \ar[rr]_{F(f)} && F(j)
} \]
kommutieren.
\end{defn}
Die Notation etwas missbrauchend werden Kegel oft nur nach ihrer Kegelspitze
genannt, obwohl die Morphismen~$\pi_i$ mit zum Datum gehören.
Die Morphismen~$\pi_i$ werden manchmal als Projektionsmorphismen
bezeichnet, der Grund dafür wird beim ersten Beispiel klar werden.

\begin{defn}Ein \emph{Morphismus von Kegeln $K \to \widetilde K$} eines Diagramms~$F$ besteht aus
\begin{enumerate}
\item[] einem Morphismus der Kegelspitzen $\psi : K \to \widetilde K$ in~$\C$
\end{enumerate}
sodass
\begin{enumerate}
\item[]
für alle~$i \in \Ob\I$ die Dreiecke
\[ \xymatrix{
  K \ar[rr]^{\psi} \ar[rd]_{\pi_i} && \widetilde K \ar[ld]^{\widetilde\pi_i} \\
  & F(i)
} \]
kommutieren.
\end{enumerate}
\end{defn}
Abbildung~\ref{kegel} erklärt die Herkunft des Begriffs "`Kegel"'.

\begin{figure}
  \[
    \xymatrix{
      K \ar@[grey]@{-->}[rrrrr]
      \ar@[grey][dddr] \ar@[grey][dddrrr] \ar@[grey][dddrrrr] \ar@[grey][ddddrr] &&&&&
      \widetilde K \ar@[grey][dddl] \ar@[grey][dddll] \ar@[grey][dddllll]
      \ar@[grey][ddddlll]
      \\\\\\
      & F(i) \ar[rr] \ar[rd] && F(j) \ar[r] \ar[ld] & F(\ell) \\
      & & F(k)
    }
  \]
  \caption{\label{kegel}Zwei Kegel und ein Kegelmorphismus zwischen ihnen.}
\end{figure}

Kegelmorphismen kann man auf die offensichtliche Art und Weise miteinander
verketten (einfach die Morphismen der Kegelspitzen verketten). Daher ist es
sinnvoll, von der \emph{Kategorie der Kegel} zu einem festen
Diagramm~$F:\I\to\C$ zu sprechen. Terminale Objekte dieser Kategorie haben
einen besonderen Namen:
\begin{defn}Ein \emph{Limes} eines Diagramms~$F:\I\to\C$ ist ein terminales
Objekt in der Kategorie der Kegel zu~$F$.\end{defn}
Da allgemein terminale Objekte einer Kategorie bis auf eindeutige Isomorphie
eindeutig sind (siehe Aufgabe~2 von Übungsblatt~2), folgt sofort folgende
Beobachtung:
\begin{prop}Limiten sind bis auf eindeutige Isomorphie eindeutig. Die
Kegelspitzen von Limiten sind zumindest bis auf Isomorphie eindeutig.\end{prop}

Für ein anschauliches Verständnis von Limiten sind zwei Mottos wichtig:
\begin{motto}Ein Limes eines Diagramms ist ein bestes
(größtmögliches) Objekt, welches das Diagramm zu einem Kegel ergänzt.
\end{motto}
\emph{Größtmöglich} ist dabei nicht im wörtlichen Sinn, wie er etwa in der Kategorie
der Mengen vorstellbar ist, zu interpretieren, sondern nur so zu verstehen,
als dass jeder andere Kegel
(Möchtegern-Limes) einen Morphismus in den Limes hinein besitzt.

\begin{motto}\label{limessubsumiert}
Ein Limes subsumiert das gesamte Diagramm zu einem einzelnen Objekt (der
Kegelspitze) -- zumindest, was Morphismen in das Diagramm hinein
angeht.\end{motto}
Das ist so verstehen: Immer, wenn man einen Morphismus aus einem
Objekt~$\widetilde K$ "`in das Diagramm hinein"' gegeben hat (d.\,h. einen Kegel des Diagramms
gegeben hat), induziert die universelle Eigenschaft einen Morphismus
aus~$\widetilde K$ in den Limes.  Umgekehrt kann man aus jedem solchen
Morphismus durch Nachschaltung der Projektionen einen Kegel erhalten.
Dieses Motto werden wir sogar formal beweisen können: siehe
Proposition~\ref{homstetig}.


\subsection{Beispiele für Limiten}

\subsubsection*{Produkte}

Sei speziell~$\I = \mathbf{2}$ die Kategorie mit genau zwei Objekten und nur
den Iden\-ti\-täts\-mor\-phis\-men:
\[ \xymatrix{\bullet \ar@(ul,ur) & \bullet \ar@(ul,ur)} \]
Dann sind Diagramme~$\I \to \C$ einfach durch die Angabe zweier Objekte
von~$\C$ gegeben. Kegel solcher Diagramme haben wir früher schon untersucht:
unter dem Namen \emph{Möchtegern-Produkte}. Entsprechend sind Limiten
solcher Diagramme schlichtweg Produkte.

\subsubsection*{Faserprodukte (Pullbacks)}

Sei speziell~$\I$ die Kategorie
\[ \xymatrix{
  & \bullet \ar[d] \ar@(ur,ul) \\
  \bullet \ar[r] \ar@(ul,dl) & \bullet. \ar@(dr,ur)
} \]
Limiten von~$\I$-förmigen Diagrammen werden auch \emph{Faserprodukte} genannt
und konventionsmäßig gerne als sog. \emph{Faserprodukt-} oder
\emph{Pullbackdiagramm} skizziert:
\[ \xymatrix{
  \ar @{} [dr] |{\begin{array}{l}\lrcorner\ \ \ \ \ \ \ \ \ \\\\\end{array}}
  X \times_Z Y \ar[r] \ar[d] & Y \ar[d]^g \\
  X \ar[r]_f & Z
} \]
Dabei steht die Kegelspitze des Limes oben links. Der dritte
Projektionsmorphismus (auf~$Z$) ist nicht eingezeichnet, da er sowieso gleich
der Komposition des Wegs über~$X$ (oder über~$Y$) sein muss. Wenn eine
Kategorie jedes~$\I$-förmige Diagramm zu einem Faserproduktdiagramm ergänzt
werden kann, sagt man auch, dass die Kategorie "`alle Faserprodukte besitzt"'.

In der Kategorie der Mengen kann das Faserprodukt durch die Konstruktion
\[ X \times_Z Y := \{ (x,y) \in X \times Y \,|\, f(x) = g(y) \} \subseteq X \times Y \]
gegeben werden.

Man hat zwei vorschiedene Vorstellungen des Faserprodukts, die unterschiedliche
Aspekte betonen: Zum einen kann man die Objekte~$X$ und~$Y$ als Ausgangsbasis
ansehen. Dann stellt man sich als Faserprodukt das Objekt~$X \times_Z Y$ vor
und sieht es als eine Art "`verallgemeinertes Produkt"' an.

Zum anderen kann man sich aber auch den Morphismus~$g$ als Ausgangspunkt
vorstellen. Als Ergebnis betont man dann nicht das Objekt~$X \times_Z Y$
alleine, sondern den Morphismus~$X \times_Z Y \to X$. Diesen bezeichnet man
dann auch als \emph{Rückzug (Pullback)} von~$g$ längs~$f$ oder
\emph{Basiswechsel} von~$g$ nach~$X$.

\begin{bsp}Sei~$g:Y \to Z$ eine Abbildung von Mengen. Sei~$U \subseteq Z$ eine
Teilmenge. Dann passt das Urbild~$g^{-1}[U]$ in ein Pullbackdiagramm:
\[ \xymatrix{
  \ar @{} [dr] |{\begin{array}{l}\lrcorner\ \ \ \ \ \ \ \ \ \\\\\end{array}}
  g^{-1}[U] \ar@{^{(}->}[r] \ar[d] & Y \ar[d]^f \\
  U \ar@{^{(}->}[r] & Z
} \]
\end{bsp}
Dieser Standpunkt wird unter anderem in der algebraischen Geometrie verwendet. Da
sind dann \emph{Stabilitätsaussagen} wichtig: Hat ein Morphismus eine bestimmte
Eigenschaft, so hat sein Rückzug längs Morphismen einer bestimmten Klasse
dieselbe Eigenschaft.

\begin{bsp}Sei ein Pullbackdiagramm der Form
\[ \xymatrix{
  \ar @{} [dr] |{\begin{array}{l}\lrcorner\ \ \ \ \ \ \ \ \ \\\\\end{array}}
  X \times_Z Y \ar[r]^{f'} \ar[d]_{g'} & Y \ar[d]^g \\
  X \ar[r]_f & Z
} \]
in einer beliebigen Kategorie gegeben. Wenn~$g$ ein Monomorphismus ist, dann auch~$g'$.
Man sagt: \emph{Monomorphismen sind unter Rückzug stabil.}
\end{bsp}
\begin{bem}Es ist etwas besonderes, wenn auch Epimorphismen unter Rückzug
stabil sind. Das ist etwa in der Kategorie der Mengen und allen abelschen
Kategorien der Fall.\end{bem}


\subsubsection*{Terminale Objekte}

Sei speziell~$\I = \mathbf{0}$ die leere Kategorie und~$F:\I \to \C$ das
einzige~$\I$-förmige Diagramm in~$\C$. Kegel von~$F$ sind dann einfach Objekte
von~$\C$ und Morphismen solcher Kegel Morphismen zwischen diesen Objekten; die
Kategorie der Kegel von~$F$ ist also~$\C$ selbst.

Damit ist klar: Limiten von~$F$ sind dasselbe wie terminale Objekte von~$\C$.


\subsubsection*{Differenzkerne (Equalizer)}

Sei speziell~$\I$ die Kategorie mit zwei Objekten und zwei parallelen
Morphismen:
\[ \xymatrix{
  \reflectbox{$\bullet \rotatebox{90}{$\circlearrowright$}$}
  \ar@/^/[r] \ar@/_/[r] & \bullet
  \rotatebox{90}{$\circlearrowleft$}
} \]
Diagramme~$\I \to \C$ sind dann durch die Angabe zweier paralleler Morphismen
$f,g:X \to Y$ in~$\C$ gegeben. Ein Limes eines solchen Diagramms heißt dann
\emph{Differenzkern} (Equalizer) von~$f$ und~$g$.

In der Kategorie der Mengen kann der Differenzkern durch die Konstruktion
\[ \mathrm{Eq}(f,g) := \{ x \in X \,|\, f(x) = g(x) \} \subseteq X \]
gegeben werden. Genauso funktioniert es in der Kategorie der $K$-Vektorräume,
wenn man diese Menge mit der Untervektorraumstruktur versieht; dann kann man
auch kürzer
\[ \mathrm{Eq}(f,g) = \ker(f - g) \]
schreiben und so die Begriffsherkunft verstehen.


\subsection{Zusammenhang mit dem Limesbegriff in der Analysis}

\subsection{Existenz von Limiten}

\begin{defn}Eine Kategorie~$\C$ heißt genau dann \emph{(ko-)vollständig}, wenn
jedes kleine Diagramm in~$\C$ einen (Ko-)Limes besitzt. Dabei heißt ein
Diagramm~$\I \to \C$ genau dann \emph{klein}, wenn seine Indexkategorie~$\I$
klein ist (d.\,h. wenn die Objekt- und Morphismenklassen sogar schon Mengen
bilden).\end{defn}

\begin{prop}Die Kategorie der Mengen ist vollständig und kovollständig:
Ist~$F:\I\to\Set$ ein kleines Diagramm, so wird die Menge
\begin{align*}
  \lim F &:= \Bigl\{ (x_i)_{i\in\Ob\I} \in \prod_{i \in \Ob\I} F(i) \ \Big|\
  \text{$F(f)(x_i) = x_j$ für alle~$f:i \to j$ in~$\I$} \Bigr\}
\intertext{vermöge der kanonischen Projektionsabbildungen zu einem Limes
von~$F$ und}
  \colim F &:= \Biggl(\coprod_{i \in \Ob\I} F(i)\Biggr)\Big/{\sim},
\end{align*}
wobei~$({\sim})$ die feinste Äquivalenzrelation mit
\[ \text{für alle $f : i \to j$ in~$\I$, $x \in F(i)$:}\quad
  \langle i, x \rangle \sim \langle j, F(f)(x) \rangle \]
ist, zu einem Kolimes von~$F$.
\end{prop}
\begin{proof}
Die Äquivalenzrelation~$({\sim})$ kann definiert werden als Schnitt über alle
Äqui\-va\-lenz\-re\-la\-tio\-nen auf~$\coprod_{i \in \Ob\I} F(i)$, die die angegebene
Bedingung erfüllen. Dann kann man die Behauptungen nachrechnen.
\end{proof}

\textbf{XXX:} Hier fehlt der Satz über die Berechnung von Limiten aus Produkten
und Differenzkernen.


\subsection{Limiten in Funktorkategorien}

In diesem Abschnitt wollen wir untersuchen, wie Limiten in
Funktorkategorien~$\Funct(\C,\D)$ aussehen. Sei dazu ein Diagramm
$F : \I \longrightarrow \Funct(\C,\D)$
gegeben. Durch "`Nachschaltung der Evaluierungsfunktoren"' erhält man aus
diesem Diagramm für jedes Objekt~$X \in \Ob\C$ jeweils ein Diagramm in~$\D$:
\[ \begin{array}{@{}rrcl@{}}
  F_X : & \I &\longrightarrow& \D \\
  & i &\longmapsto& F(i)(X)
\end{array} \]
Wenn wir voraussetzen, dass all diese Diagramme jeweils einen Limes~$\lim F_X$ in~$\D$
besitzen, können wir (wie in Aufgabe~4 von Übungsblatt~3) einen Funktor
\[ \begin{array}{@{}rrcl@{}}
  L : & \C &\longrightarrow& \D \\
  & X &\longmapsto& \lim F_X
\end{array} \]
basteln. Dann gilt:
\begin{prop}
Der so konstruierte Funktor~$L$ wird (mit welchen Projektionen?) ein Limes des Diagramms~$F$.
\end{prop}
Etwas ungenau kann man diesen Zusammenhang auch über die Formel
\[ (\lim F)(X) = \lim F_X = \lim_i F(i)(X) \]
ausdrücken. Als Motto kann man daher festhalten:
\begin{motto}Besitzt~$\D$ $\I$-förmige Limiten, so werden~$\I$-förmige Limiten
in der Funktorkategorie~$\Funct(\C,\D)$ punktweise berechnet. In diesem Fall
gilt also:
Ein Kegel eines Diagramms~$F : \I \to \Funct(\C,\D)$ ist genau dann ein Limes,
wenn sein Bild unter allen Auswertungsfunktoren~$\ev_X : \Funct(\C,\D) \to \D$,
$X \in \Ob\C$, jeweils ein Limes ist.
\end{motto}


\subsection{Bewahrung von (Ko-)Limiten}


\subsection{Vertauschung von (Ko-)Limiten}


\subsection{Kofinale Unterdiagramme}

\begin{defn}
Wir nennen einen Funktor $H : \D_0 \to \D$ genau dann \emph{kofinal}, wenn
für alle~$d \in \Ob\D$\ldots
\begin{enumerate}
\item[1.] ein Objekt $d_0 \in \Ob\D_0$ und ein Morphismus $d \to
H(d_0)$ in~$\D$ existiert und
\item[2.] für je zwei solcher Morphismen ein Objekt~$\widetilde d_0 \in \Ob\D_0$ und
Morphismen $d_0 \to \widetilde d_0$, $d_0' \to \widetilde d_0$ existieren, deren Bilder
unter~$H$ das Diagramm
\[ \xymatrix{
  d_0 \ar[r] \ar[d] & H(d_0) \ar@{-->}[d] \\
  H(d_0') \ar@{-->}[r] & H(\widetilde d_0)
} \]
kommutieren lassen.
\end{enumerate}
\end{defn}

Etwa ist der Inklusionsfunktor~$B (2\NN) \to B(\NN)$ kofinal, wenn~$\NN$ die
Menge der natürlichen Zahlen mit ihrer gewöhnlichen Ordnung und~$2\NN$ die
Teilordnung der geraden Zahlen bezeichnet.

% Hom linksexakt

% Rechenregel Hom
