\section[Adjungierte Funktoren]{Adjungierte Funktoren \hfill \small Peter
Uebele}

% XXX: adjungierte Funktoren in der Logik!


%Erinnerung: Äquivalenz von Kategorien $\cC, \cD$:\\
%$\exists F:\cD\rightarrow \cC$, $G:\cC\rightarrow \cD$ inverse Funktoren, d.h.
%$F\circ G\cong id_\cC$ und $G\circ G\cong id_\cD$

%\paragraph{Idee:} Hin- und herschieben von Morphismen zwischen $\cC, \cD$.\\
%$\rightsquigarrow$ Verallgemeinerung, s.d. $\cC,\cD$ nicht mehr äquivalent sein
%müssen.
\begin{defn}
Seien $F:\D\to\C$, $G:\C\to\D$ Funktoren. Genau dann heißt
\begin{itemize}
\item 
$F$ \emph{links-adjungiert zu} $G$ bzw.
\item 
$G$ \emph{rechts-adjungiert zu} $F$,
\end{itemize}
in Zeichen: $F \dashv G$, wenn es eine in~$X \in \Ob\C$ und~$Y \in \Ob\D$ natürliche
Isomorphie gibt:
\[ \Hom_\C(FY,X) \cong \Hom_\D(Y,GX) \]
\end{defn}
Dabei ist \emph{natürlich} gemäß Bemerkung~\ref{interpretnat} zu verstehen:
Linke und rechte Seiten der Isomorphie sind als Auswertungen der Funktoren
\[ \begin{array}{@{}rrcl@{}}
  \Hom_\C(F\freist,\freist) :&
    \D^\op \times \C &\longrightarrow& \Set \\
  & (Y,X) &\longmapsto& \Hom_\C(FY, X) \\\\
  \Hom_\D(\freist,G\freist) :&
    \D^\op \times \C &\longrightarrow& \Set \\
  & (Y,X) &\longmapsto& \Hom_\D(Y, GX)
\end{array} \]
zu lesen. Die Natürlichkeitsbedingung bedeutet dann, dass
für alle Morphismen $f:X\rightarrow X'$ in~$\C$ und
$g:Y'\rightarrow Y$ in~$\D$ das Diagramm
\[ \xymatrix{
  \Hom_\C(FY,X) \ar[d]_\cong \ar[r] & \Hom_\C(FY',X') \ar[d]^\cong \\
  \Hom_\D(Y,GX) \ar[r] & \Hom_\D(Y',GX')
} \]
kommutiert.

\begin{bsp}Sei~$F:\D \to \C$ quasi-invers zu~$G:\C \to \D$ (im Sinne von
Definition~\ref{cateqv}). Dann ist~$F$ links- und rechts-adjungiert
zu~$G$.\end{bsp}
Das Konzept zueinander adjungierter Funktoren ist also eine Verallgemeinerung
des Konzepts zueinander quasi-inverser Funktoren: Auch, wenn ein Funktor kein
Quasi-Inverses besitzt, kann man dennoch fragen, inwieweit man ihn zumindest
\emph{so gut wie möglich} invertieren kann. Das folgende Beispiel~\cite{smith}
soll diesen Gedanken illustrieren.

\begin{bsp}Sei~$i:\ZZ \to \QQ$ die Inklusion der ganzen in die rationalen Zahlen,
aufgefasst als partiell geordnete Mengen. Diese Abbildung besitzt keine
monotone Umkehrabbildung, aber zwei Beinahe-Inverse, nämlich die Auf- und
Abrundungsfunktionen:
\[ \begin{array}{@{}rrcl@{}}
  \lceil \freist \rceil : & \QQ &\longrightarrow& \ZZ \\
  & x &\longmapsto& \lceil x \rceil = \text{(kleinste ganze Zahl $\geq x$)} \\\\
  \lfloor \freist \rfloor : & \QQ &\longrightarrow& \ZZ \\
  & x &\longmapsto& \lfloor x \rfloor = \text{(größte ganze Zahl $\leq x$)}
\end{array} \]
Die von diesen monotonen Abbildungen induzierten Funktoren erfüllen tatsächlich
die Adjunktionsbeziehungen
\[ B\lceil\freist\rceil \dashv Bi \dashv B\lfloor\freist\rfloor, \]
siehe Aufgabe~1 von Übungsblatt~7.\end{bsp}

\begin{bsp}Sei~$K$ ein Körper (oder Ring), $U : \Vect{K} \to \Set$ der
Vergissfunktor und~$F : \Set \to \Vect{K}$ der Funktor, der jeder Menge~$X$
den sog. \emph{freien Vektorraum über~$X$}
\[ F(X) := \Biggl\{ \sum_{i=1}^n \lambda_i x_i \,\Bigg|\,
  n \geq 0, \lambda_1, \ldots, \lambda_n \in K, x_1, \ldots, x_n \in X \Biggr\} \]
zuordnet. Dessen Elemente sind sog. formale (endliche) Linearkombinationen der
Elemente von~$X$; addiert wird also nicht wirklich, man notiert lediglich vor
jedes Element aus~$X$ einen Koeffizienten aus~$K$.\footnote{In konstruktiver
Mathematik realisiert man~$F(X)$ als Menge von Wörtern über~$K \times X$ modulo
einer geeigneten Äquivalenzrelation, wenn man nicht voraussetzen möchte,
dass~$X$ als Menge diskret ist.} Die Elemente von~$X$ bilden dann eine Basis
von~$F(X)$.
Ist~$f : X \to X'$ eine Abbildung, so ist die induzierte
Abbildung durch
\[ F(f) : F(X) \to F(X'),\ \sum_{i=1}^n \lambda_i x_i \mapsto
  \sum_{i=1}^n \lambda_i f(x_i) \]
gegeben und tatsächlich linear.
\end{bsp}

\endinput

$k$ Körper, $Vect_k \overset{U}{\rightarrow}Set$ Vergissfunktor\\
$Set\underset{F}{\rightarrow Vekt_k}$ \emph{freier Funktor}
$\underset{\mbox{Menge}}{X}\mapsto FX=\{\sum_{n=1}^N\lambda_ix_i\}$ endli.
Linearkombination von Elementen aus $X$. $S$ ist Basis von $FX$.\\
\[
(X\overset{f}{\rightarrow}X')\mapsto Ff
\]
wobei $Ff$ eine lineare Abbildung ist, die durch $x_i\mapsto f(x_i)$ gegeben
ist.
\[Ff(\sum^{N}_{i=1}{\lambda_ix_i})=\sum^{N}_{i=1}{\lambda_i}f(x_i)\]
\end{exmp}
\paragraph{Beh:} $F\dashv U$
\begin{proof}
\[
\Hom_\cC(FY,V)  \overset{?}{\cong} \Hom_\cC(Y,UV)
\]
wobei 
\begin{itemize}
\item $\Hom_\cC(FY,V)\bydef\{\mbox{lin. Abb }FY\rightarrow V\}$
\item $\Hom_\cC(Y,UV)\bydef\{\mbox{bel. Abb. }Y\rightarrow V\}\ni \phi$
\end{itemize}
\[
\phi \mapsto \big( \sum^{N}_{i=1}{\lambda_iy_i}\mapsto
\sum^{N}_{i=1}{\lambda_i\phi(y_i)} \big)
\]
bijektiv, da jede lineare Abbildung eindeutig durch die Basis festgelegt ist.

Natürlich in $Y$ und $V$: $f:V\mapsto V'$ und $g:Y\mapsto Y'$
\begin{center}
\begin{tikzpicture} [scale=3.3, descr/.style={fill=white,inner sep=2.5pt} ]
\matrix (m) [
  matrix of math nodes
  , row sep=3em
  , column sep=2em
  %, text height=3em
  %, text depth=0.25em
]
{
  \big(\lambda_iy_i\mapsto\sum\lambda_i\phi(y_i)\big) & & & & \phi\\
  & \Hom_\cC(FY,V)   & \cong & \Hom_\cC(Y,UV)   & \\
  & \Hom_\cC(FY',V') & \cong & \Hom_\cC(Y',UV') & \\
  \big( \sum\lambda_iy_i'\mapsto\sum\lambda_if(\phi(g(y_i')))
    \big)& & & & Uf\circ\phi\circ g\\
};
\path[->,font=\scriptsize,>=angle 90]
(m-2-2) edge node[right]{$\Hom(Fg,f)$} (m-3-2)
(m-2-4) edge node[left]{$\Hom(g,Gf)$} (m-3-4)
;

\path[>=stealth,|->]
(m-1-5) edge (m-1-1) 
(m-1-5) edge (m-4-5)
(m-1-1) edge (m-4-1)
(m-4-5) edge (m-4-1) 
;
\end{tikzpicture}
\end{center}
\end{proof}

\begin{exmp}
\[
U:Cat \rightarrow Set
\]
\begin{align*}
L: & Set \rightarrow Kat\\
  & X \mapsto \mbox{diskrete Kategorie auf $X$ (d.h. nur
Identitätsmorphismen)}\\
R: & Set \rightarrow Cat\\
  & X \mapsto \mbox{indiskrete Kategorie auf $X$(d.h. zwischen je zwei
Objekten genau ein Morphismus)}
\end{align*}
\paragraph{Beh:}
\begin{enumerate}
\item $L\dashv U$
\item $U\dashv R$
\end{enumerate}
\paragraph{zu 1.} $C=Cat$, $D=Set$
$\Hom_{Cat}(LX,\cC)\cong\Hom_{Set}(X,U\cC)$
wobei
\begin{itemize}
\item 
$\Hom_{Cat}(LX,\cC)\bydef\{\mbox{Funktoren }LX\rightarrow \cC \}$
\item 
$\Hom_{Set}(X,U\cC)\bydef\{\mbox{Abbildungen } X\rightarrow Obj(\cC)\}$
\end{itemize}
\paragraph{zu 2.}
$\Hom_{Set}(U\cC,X)\cong\Hom_{Cat}(\cC,RX)$
wobei
\begin{itemize}
\item 
$\Hom_{Set}(U\cC,X)\bydef\{\mbox{Abbildungen } Obj(\cC)\rightarrow X\}$
\item 
$\Hom_{Cat}(\cC,RX)\bydef\{\mbox{Funktoren }\cC\rightarrow RX \}$
\end{itemize}
bijektiv, da Morphismen in $RX$ durch Quelle und Ziel eindeutig bestimmt
\end{exmp}
