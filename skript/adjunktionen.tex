\section[Adjungierte Funktoren]{Adjungierte Funktoren \hfill \small Peter
Uebele}

% XXX: adjungierte Funktoren in der Logik!


%Erinnerung: Äquivalenz von Kategorien $\cC, \cD$:\\
%$\exists F:\cD\rightarrow \cC$, $G:\cC\rightarrow \cD$ inverse Funktoren, d.h.
%$F\circ G\cong id_\cC$ und $G\circ G\cong id_\cD$

%\paragraph{Idee:} Hin- und herschieben von Morphismen zwischen $\cC, \cD$.\\
%$\rightsquigarrow$ Verallgemeinerung, s.d. $\cC,\cD$ nicht mehr äquivalent sein
%müssen.
\begin{defn}
Seien $F:\D\to\C$, $G:\C\to\D$ Funktoren. Genau dann heißt
\begin{itemize}
\item 
$F$ \emph{links-adjungiert zu} $G$ bzw.
\item 
$G$ \emph{rechts-adjungiert zu} $F$,
\end{itemize}
in Zeichen: $F \dashv G$, wenn es eine in~$X \in \C$ und~$Y \in \D$ natürliche
Isomorphie
\[ \Hom_\C(FY,X) \cong \Hom_\D(Y,GX) \]
gibt.
\end{defn}
Dabei ist \emph{natürlich} gemäß Bemerkung~\ref{interpretnat} zu verstehen:
Linke und rechte Seiten der Isomorphie sind als Funktoren
\[ \begin{array}{@{}rrcl@{}}
  \Hom_\C(F\freist,\freist) :&
    \D^\op \times \C &\longrightarrow& \Set \\
  & (Y,X) &\longmapsto& \Hom_\C(FY, X) \\\\
  \Hom_\D(\freist,G\freist) :&
    \D^\op \times \C &\longrightarrow& \Set \\
  & (Y,X) &\longmapsto& \Hom_\D(Y, GX)
\end{array} \]
zu verstehen. Die Natürlichkeitsbedingung bedeutet dann, dass
für alle Morphismen $f:X\rightarrow X'$ in~$\C$ und
$g:Y'\rightarrow Y$ in~$\D$ das Diagramm
\[ \xymatrix{
  \Hom_\C(FY,X) \ar[d]_\cong \ar[r] & \Hom_\C(FY',X') \ar[d]^\cong \\
  \Hom_\D(Y,GX) \ar[r] & \Hom_\D(Y',GX')
} \]
kommutiert.
