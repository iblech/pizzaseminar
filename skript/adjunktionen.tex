\section[Adjungierte Funktoren]{Adjungierte Funktoren \hfill \small Peter
Uebele}

%\emph{Werbung:} Adjungierte Funktoren treten überall in der Mathematik auf. In
%diesem Abschnitt wollen und es gibt viele
%Möglichkeiten...

% XXX: adjungierte Funktoren in der Logik!


%Erinnerung: Äquivalenz von Kategorien $\cC, \cD$:\\
%$\exists F:\cD\rightarrow \cC$, $G:\cC\rightarrow \cD$ inverse Funktoren, d.h.
%$F\circ G\cong id_\cC$ und $G\circ G\cong id_\cD$

%\paragraph{Idee:} Hin- und herschieben von Morphismen zwischen $\cC, \cD$.\\
%$\rightsquigarrow$ Verallgemeinerung, s.d. $\cC,\cD$ nicht mehr äquivalent sein
%müssen.
\begin{defn}
Seien $F:\D\to\C$, $G:\C\to\D$ Funktoren. Genau dann heißt
\begin{itemize}
\item 
$F$ \emph{linksadjungiert zu} $G$ bzw.
\item 
$G$ \emph{rechtsadjungiert zu} $F$,
\end{itemize}
in Zeichen: $F \dashv G$, wenn es eine in~$X \in \Ob\C$ und~$Y \in \Ob\D$ natürliche
Isomorphie gibt:
\[ \Hom_\C(FY,X) \cong \Hom_\D(Y,GX) \]
\end{defn}
Dabei ist \emph{natürlich} gemäß Bemerkung~\ref{interpretnat} zu verstehen:
Linke und rechte Seiten der Isomorphie sind als Auswertungen der Funktoren
\[ \begin{array}{@{}rrcl@{}}
  \Hom_\C(F\freist,\freist) :&
    \D^\op \times \C &\longrightarrow& \Set \\
  & (Y,X) &\longmapsto& \Hom_\C(FY, X) \\\\
  \Hom_\D(\freist,G\freist) :&
    \D^\op \times \C &\longrightarrow& \Set \\
  & (Y,X) &\longmapsto& \Hom_\D(Y, GX)
\end{array} \]
zu lesen. Die Natürlichkeitsbedingung bedeutet dann, dass
für alle Morphismen $f:X\rightarrow X'$ in~$\C$ und
$g:Y'\rightarrow Y$ in~$\D$ das Diagramm
\[ \xymatrix{
  \Hom_\C(FY,X) \ar[d]_\cong \ar[r] & \Hom_\C(FY',X') \ar[d]^\cong \\
  \Hom_\D(Y,GX) \ar[r] & \Hom_\D(Y',GX')
} \]
kommutiert.

\begin{bsp}Sei~$F:\D \to \C$ quasi-invers zu~$G:\C \to \D$ (im Sinne von
Definition~\ref{cateqv}). Dann ist~$F$ links- und rechtsadjungiert
zu~$G$.\end{bsp}
Das Konzept zueinander adjungierter Funktoren ist also eine Verallgemeinerung
des Konzepts zueinander quasi-inverser Funktoren: Auch, wenn ein Funktor kein
Quasi-Inverses besitzt, kann man dennoch fragen, inwieweit man ihn zumindest
\emph{so gut wie möglich} invertieren kann. Das folgende Beispiel~\cite{smith}
soll diesen Gedanken illustrieren.

\begin{bsp}Sei~$i:\ZZ \to \QQ$ die Inklusion der ganzen in die rationalen Zahlen,
aufgefasst als partiell geordnete Mengen. Diese Abbildung besitzt keine
monotone Umkehrabbildung, aber zwei Beinahe-Inverse, nämlich die Auf- und
Abrundungsfunktionen:
\[ \begin{array}{@{}rrcl@{}}
  \lceil \freist \rceil : & \QQ &\longrightarrow& \ZZ \\
  & x &\longmapsto& \lceil x \rceil = \text{(kleinste ganze Zahl $\geq x$)} \\\\
  \lfloor \freist \rfloor : & \QQ &\longrightarrow& \ZZ \\
  & x &\longmapsto& \lfloor x \rfloor = \text{(größte ganze Zahl $\leq x$)}
\end{array} \]
Die von diesen monotonen Abbildungen induzierten Funktoren erfüllen tatsächlich
die Adjunktionsbeziehungen
\[ B\lceil\freist\rceil \dashv Bi \dashv B\lfloor\freist\rfloor, \]
siehe Aufgabe~1 von Übungsblatt~7.\end{bsp}

\begin{bsp}Sei~$K$ ein Körper (oder Ring), $U : \Vect{K} \to \Set$ der
Vergissfunktor und~$F : \Set \to \Vect{K}$ der Funktor, der jeder Menge~$X$
den sog. \emph{freien Vektorraum über~$X$}
\[ F(X) := \Biggl\{ \sum_{i=1}^n \lambda_i x_i \,\Bigg|\,
  n \geq 0, \lambda_1, \ldots, \lambda_n \in K, x_1, \ldots, x_n \in X \Biggr\} \]
zuordnet. Dessen Elemente sind sog. formale (endliche) Linearkombinationen der
Elemente von~$X$; addiert wird also nicht wirklich, man notiert lediglich vor
jedes Element aus~$X$ einen Koeffizienten aus~$K$.\footnote{In konstruktiver
Mathematik realisiert man~$F(X)$ als Menge von Wörtern über~$K \times X$ modulo
einer geeigneten Äquivalenzrelation, wenn man nicht voraussetzen möchte,
dass~$X$ als Menge diskret ist.} Die Elemente von~$X$ bilden dann eine Basis
von~$F(X)$.
Ist~$f : X \to X'$ eine Abbildung, so ist die induzierte
Abbildung durch
\[ F(f) : F(X) \to F(X'),\ \sum_{i=1}^n \lambda_i x_i \mapsto
  \sum_{i=1}^n \lambda_i f(x_i) \]
gegeben und tatsächlich linear. Dann gilt:
\end{bsp}
\begin{prop}Der so definierte Funktor~$F$ ist
linksadjungiert zum Vergissfunktor~$U$.
\end{prop}
\begin{proof}
Wir geben den in~$Y \in \Ob\Set$ und~$V \in \Ob\Vect{K}$ natürlichen
Isomorphismus explizit an:
\[ \begin{array}{@{}rcl@{}}
  \Hom_{\Vect{K}}(F(Y), V) &\longrightarrow& \Hom_\Set(Y, U(V)) \\
  \varphi &\longmapsto& \varphi|_Y
\end{array} \]
Die Bijektivität dieser Zuordnung drückt gerade aus, dass die Werte auf einer
Basis genügen, um eine lineare Abbildung eindeutig festzulegen. Die
Natürlichkeitsbedingung besagt, dass für alle~$f:V \to V'$ in~$\Vect{K}$ und~$g:Y' \to Y$
das Diagramm
\[ \xymatrix{
  \Hom_{\Vect{K}}(F(Y),V) \ar[r] \ar[d]_\cong & \Hom_{\Vect{K}}(FY',U(V')) \ar[d]^\cong \\
  \Hom_\Set(Y,U(V)) \ar[r] & \Hom_\Set(Y',U(V'))
} \]
kommutiert, d.\,h. dass für alle~$\varphi \in \Hom_{\Vect{K}}(F(Y),V)$ die Gleichung
\[
  (f \circ \varphi \circ F(g))|_{Y'} = f \circ \varphi|_Y \circ g
\]
erfüllt ist. Das ist offensichtlich der Fall.
\end{proof}

\begin{bsp}Der Vergissfunktor~$U : \Cat \to \Set$, der einer kleinen Kategorie
ihre Menge von Objekten zuordnet, besitzt sowohl einen Links-, als auch einen
Rechtsadjungierten, nämlich
\[ \begin{array}{@{}rrclrcl@{}}
  L : & \Set &\longrightarrow& \Cat, & X &\longmapsto& \text{diskrete Kategorie auf~$X$,} \\
  R : & \Set &\longrightarrow& \Cat, & X &\longmapsto& \text{indiskrete Kategorie auf~$X$.}
\end{array} \]
Denn man hat natürliche Isomorphismen
\[ \Hom_\Cat(L(X), \E) \cong \Hom_\Set(X, \Ob\E)
  \quad\text{und}\quad
  \Hom_\Set(\Ob\E, Y) \cong \Hom_\Cat(\E, R(Y)). \]
\end{bsp}

\begin{bsp}Analog hat der Vergissfunktor~$U : \Top \to \Set$, der jedem
topologischen Raum seine zugrundeliegende Menge von Punkten zuordnet, einen
Links- und Rechtsadjungierten: die Konstruktion des diskreten bzw. indiskreten
topologischen Raums auf einer Menge.\end{bsp}

\begin{prop}
Sei $F:\C\to\D$ linksadjungiert zu $G:\D\to\C$. Dann gilt:
\begin{enumerate}
\item $F$ erhält Kolimiten von $\D$.
\item $G$ erhält Limiten von $\C$.
\end{enumerate}
\end{prop}

\begin{kor}\label{vergissstetig}
\begin{enumerate}
\item Der Vergissfunktor $U:\Vect{K} \to \Set$ erhält Limiten.
Da er aber nicht Kolimiten erhält (Beispiel~\ref{vectvergiss}), kann er keinen
Rechtsadjungierten besitzen.
\item \label{vergisscat}Der Vergissfunktor $U:\Cat \to \Set$ erhält Limiten und Kolimiten.
\item Der Vergissfunktor $U:\Top \to \Set$ erhält Limiten und Kolimiten.
\end{enumerate}
\end{kor}

\begin{proof}[Beweis der Proposition]
Der übliche Beweis geht so: Sei $D:\I\to\D$ ein Diagramm, mit Limes~$\lim_i D(i)$.
Dann folgt aus der in~$Y \in \Ob\D$ natürlichen Isomorphiekette
\begin{align*}
  \Hom_\D(Y,G(\lim_i D(i))) &\cong
  \Hom_\C(F(Y),\lim_i D(i)) \cong
  \lim_i \Hom_\C(F(Y), D(i)) \\
  & \cong
  \lim_i \Hom_\D(Y, G(D(i))) \cong
  \Hom_\D(Y, \lim_i G(D(i)))
\end{align*}
mit dem Yoneda-Lemma die Behauptung: $G(\lim_i D(i)) \cong \lim_i G(D(i))$.

Dieser Beweis hat aber noch zwei Lücken: Zum einen wurde die Existenz des
Limes~$\lim_i G(D(i))$ ohne Beweis verwendet, zum anderen wurde nur die
Isomorphie der Kegelspitzen nachgewiesen; das ist aber eine schwächere Aussage
als die eigentliche Behauptung der Limesbewahrung. Es gibt verschiedene
Möglichkeiten, die Lücken zu schließen, siehe etwa~\cite{gaillard,lin}.
%Sei~$\lim_i D(i)$ Limes eines Diagramms~$D:\I\to\D$, zusammen mit den
%Projektionsmorphismen~$\pi_i : \lim_i D(i) \to D(i)$. Wir müssen nachweisen, dass
%der induzierte Kegel bestehend aus den Morphismen
%\[ G(\pi_i) : G(\lim_i D(i)) \longrightarrow G(D(i)) \]
%ein Limes ist. Sei dazu ein beliebiger Kegel~$K$ bestehend aus Morphismen
%\[ \mu_i : K \longrightarrow G(D(i)) \]
%gegeben. Dann induziert für jedes Objekt~$Y \in \D$ der Hom-Funktor einen Kegel
%\[ (\mu_i)_\star : \Hom_\D(Y,K) \longrightarrow \Hom_D(Y,G(D(i))). \]
%Das obige Argument zeigt zumindest, dass~$\Hom_\D(A,G(\lim_i D(i)))$ zusammen
%mit den induzierten Projektionsmorphismen ein Limes (in~$\Set$) ist.
\end{proof}

\endinput
\begin{exmp}
$\cC=Grp$, $\cD=Grp^2=Grp\times Grp$
\begin{align*}
F: & Grp^2\rightarrow Grp &\mbox{Produktfunktor}\\
  & (G_1,G_2) \mapsto G_1\times G_2\\
G: & Grp \rightarrow Grp^2 &\mbox{Diagonalfunktor}\\
  & G \mapsto (G,G)
\end{align*}
\paragraph{Beh:} $F\vdash G$
\[
\Hom_{Grp^2}((G,G),(H_1,H_2))\cong \Hom_{Grp}(G,H_1\times H_2)
\]
wobei
\begin{itemize}
\item $\Hom_{Grp^2}((G,G),(H_1,H_2))=\{\mbox{Gruppen-Homomorphismen}\}$
\item $ \Hom_{Grp}(G,H_1\times H_2) =\{G\rightarrow H_1\times H_2 \mbox{
Gruppen-Homomorphismen}\}$
\end{itemize}
\end{exmp}
\endinput
