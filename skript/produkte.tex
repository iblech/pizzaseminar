\section[Produkte und Koprodukte]{Produkte und Koprodukte \hfill \small
Matthias Hutzler}

\begin{defn}Seien~$X$, $Y$ Objekte einer Kategorie~$\C$. Dann besteht ein
\emph{Produkt} von~$X$ und~$Y$ aus
\begin{enumerate}
\item einem Objekt~$P \in \Ob \C$ und
\item Morphismen $\pi_X : P \to X$, $\pi_Y : P \to Y$,
\end{enumerate}
sodass für jedes andere \emph{Möchtegern-Produkt}, also
\begin{enumerate}
\item jedem Objekt~$\widetilde P \in \Ob \C$ zusammen mit
\item Morphismen $\widetilde \pi_X : \widetilde P \to X$, $\widetilde\pi_Y :
\widetilde P \to Y$
\end{enumerate}
genau ein Morphismus $\psi : \widetilde P \to P$ existiert, der das Diagramm
\[ \xymatrix{
    & P \ar[ld]_{\pi_X} \ar[rd]^{\pi_Y} \\
  X & & Y \\
    & \widetilde P \ar[lu]^{\widetilde \pi_X} \ar@{-->}[uu]_\psi \ar[ru]_{\widetilde \pi_Y}
  } \]
kommutieren lässt, also die Gleichungen
\begin{align*}
  \pi_X \circ \psi &= \widetilde \pi_X \\
  \pi_Y \circ \psi &= \widetilde \pi_Y
\end{align*}
erfüllt.
\end{defn}

\begin{motto}Ein Produkt ist ein bestes Möchtegern-Produkt.\end{motto}

Statt~"`$P$"' schreibt man gerne~"`$X \times Y$"'; es muss aus dem Kontext klar
werden, ob das Kreuzzeichen speziell das kartesische Produkt von Mengen
oder das allgemeine kategorielle Produkt bezeichnen soll.
Analog definiert man das Produkt von~$n$ Objekten, $n \geq 0$, und dual
definiert man das Koprodukt.


\subsection{Beispiele}

\begin{bsp}\begin{enumerate}
\item Das Produkt in der Kategorie der Mengen ist durch das kartesische Produkt
gegeben, das Koprodukt durch die disjunkt-gemachte Vereinigung.
\item Das Produkt in der Kategorie der Gruppen ist durch das direkte Produkt
mit der komponentenweisen Verknüpfung gegeben, das Koprodukt durch das sog.
freie Produkt von Gruppen.
\item Produkt und Koprodukt endlich vieler Objekte in der Kategorie
der~$K$-Vektorräume sind durch die äußere direkte Summe gegeben. Produkte und
Koprodukte von unendlich vielen Objekten unterscheiden sich allerdings.
\item Das Produkt in der von einer Quasiordnung induzierten Kategorie ist durch
das Infimum gegeben. Dual ist das Koprodukt
durchs Supremum gegeben.
\end{enumerate}\end{bsp}
\begin{proof}
\begin{enumerate}
\item Wir zeigen die Aussage über das kartesische Produkt. Seien also~$X$
und~$Y$ Mengen. Dann wird das kartesische Produkt~$X \times Y$ vermöge der
kanonischen Projektionsabbildungen
\begin{align*}
  \pi_X : X \times Y \to X,\ (x,y) \mapsto x \\
  \pi_Y : X \times Y \to Y,\ (x,y) \mapsto y
\end{align*}
zu einem Möchtegern-Produkt von~$X$ und~$Y$:
\[ \xymatrix{
  & X \times Y \ar[ld]_{\pi_X} \ar[rd]^{\pi_Y} \\
  X & & Y
} \]
Um zu zeigen, dass dieses Möchtegern-Produkt ein tatsächliches Produkt von~$X$
und~$Y$ ist, müssen wir noch die universelle Eigenschaft nachweisen. Sei also
ein Möchtegern-Produkt~$(X \leftarrow \widetilde P \to Y)$ gegeben. Dann müssen
wir nachweisen, dass es genau einen Morphismus~$\psi:\widetilde P \to X \times
Y$ gibt, der die beiden Dreiecke im Diagramm
\[ \xymatrix{
    & X \times Y \ar[ld]_{\pi_X} \ar[rd]^{\pi_Y} \\
  X & & Y \\
    & \widetilde P \ar[lu]^{\widetilde \pi_X} \ar@{-->}[uu]_\psi \ar[ru]_{\widetilde \pi_Y}
  } \]
kommutieren lässt. Ausgeschreiben besagen die Kommutativitätsbedingungen, dass
für alle~$p \in \widetilde P$ die Gleichungen
\begin{align*}
  \text{(erste Komponente von $\psi(p)$)} &= \widetilde \pi_X(p) \\
  \text{(zweite Komponente von $\psi(p)$)} &= \widetilde \pi_Y(p)
\end{align*}
gelten sollen.
Es ist klar, dass diese beiden Bedingung genau durch eine Abbildung~$\psi$
erfüllt werden, nämlich durch
\[ \psi : \widetilde P \to X \times Y,\ p \mapsto (\widetilde \pi_X(p),
\widetilde \pi_Y(p)). \]
\item Der Produkt-Fall geht analog: Zusätzlich kann man jetzt voraussetzen,
dass~$\widetilde \pi_X$ und~$\widetilde \pi_Y$ Gruppenhomomorphismen sind; im
Gegenzug muss man aber nachweisen, dass die konstruierte Abbildung~$\psi$ ein
Gruppenhomomorphismus wird.
\item Übungsaufgabe.
\item Siehe Übungsblatt 2, Aufgabe 3. \qedhere
\end{enumerate}
\end{proof}


\subsection{Erste Eigenschaften}

\begin{prop}Die Objektteile je zweier Produkte von Objekten~$X$, $Y$ sind
zueinander isomorph.\end{prop}

\begin{bem}Es gilt sogar noch mehr, siehe Aufgabe~2 von Übungsblatt~2.\end{bem}

\begin{prop}\label{prodkomm}Die Angabe eines Produkts von~$X$ und~$Y$ ist gleichwertig mit der
Angabe eines Produkts von~$Y$ und~$X$.\end{prop}
Man sagt auch: Das kategorielle Produkt ist \emph{kommutativ bis auf
Isomorphie}.

\begin{prop}Die Angabe eines Produkts von null vielen Objekten ist gleichwertig
mit der Angabe eines terminalen Objekts.\end{prop}

% Gegenbeispiel: Dreier-Produkte ohne Zweier-Produkte
