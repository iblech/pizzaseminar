\documentclass[a4paper,ngerman]{scrartcl}

%\usepackage{ucs}
\usepackage[utf8]{inputenc}

\usepackage[ngerman]{babel}

\usepackage{amsmath,amsthm,amssymb,amscd,color,graphicx}

%\usepackage[small,nohug]{diagrams}
%\diagramstyle[labelstyle=\scriptstyle]

\usepackage[protrusion=true,expansion=true]{microtype}

\usepackage{lmodern}
\usepackage{tabto}
\usepackage{mathabx}
\usepackage[retainorgcmds]{IEEEtrantools}

\usepackage[natbib=true,style=numeric]{biblatex}
\usepackage[babel]{csquotes}
\bibliography{lit}

\usepackage[all]{xy}

\usepackage{pifont}
\newcommand{\cmark}{\ding{51}}
\newcommand{\xmark}{\ding{55}}

%\usepackage{hyperref}

\setlength\parskip{\medskipamount}
\setlength\parindent{0pt}

\theoremstyle{definition}
\newtheorem{defn}{Definition}

\theoremstyle{plain}

\newtheorem*{prop}{Proposition}
\newtheorem{lemma}[defn]{Lemma}
\newtheorem{kor}[defn]{Korollar}
\newtheorem{hilfsaussage}[defn]{Hilfsaussage}
\newtheorem{ueberlegung}[defn]{Überlegung}

\theoremstyle{remark}
\newtheorem*{satz}{Satz}
\newtheorem*{bsp}{Beispiel}
\newtheorem{bem}[defn]{Bemerkung}

\clubpenalty=10000
\widowpenalty=10000
\displaywidowpenalty=10000

\newcommand{\lra}{\longrightarrow}
\newcommand{\lhra}{\ensuremath{\lhook\joinrel\relbar\joinrel\rightarrow}}
\newcommand{\thlra}{\relbar\joinrel\twoheadrightarrow}

\newcommand{\A}{\mathcal{A}}
\newcommand{\Z}{\mathbb{Z}}
\newcommand{\Q}{\mathbb{Q}}
\newcommand{\R}{\mathbb{R}}
\newcommand{\C}{\mathcal{C}}
\newcommand{\RP}{\mathbb{R}\mathrm{P}}
\newcommand{\Hom}{\mathrm{Hom}}
\newcommand{\Set}{\mathrm{Set}}
\newcommand{\Vect}[1]{#1\text{-}\mathrm{Vect}}
\newcommand{\MP}{\mathrm{MP}}
\newcommand{\Spur}[1]{\operatorname{Spur}#1}
\newcommand{\rank}[1]{\operatorname{rank}#1}
\newcommand{\sgn}[1]{\operatorname{sgn}#1}
\newcommand{\id}{\mathrm{id}}
\newcommand{\Id}{\mathrm{Id}}
\newcommand{\Aut}[1]{\operatorname{Aut}(#1)}
\newcommand{\GL}[1]{\operatorname{GL}(#1)}
\newcommand{\ORTH}[1]{\operatorname{O}(#1)}
\newcommand{\freist}{\underline{\ \ }}
\newcommand{\op}{\mathrm{op}}
\DeclareMathOperator{\rk}{rk}
\DeclareMathOperator{\Spec}{Spec}
\DeclareMathOperator{\Bild}{im}
\DeclareMathOperator{\Kern}{ker}
\DeclareMathOperator{\Int}{int}
\DeclareMathOperator{\Ob}{Ob}
\newcommand{\Zzwei}{\Z_2}

\newcommand{\D}{\mathcal{D}}
\newcommand{\Grp}{\mathrm{Grp}}
\newcommand{\AbGrp}{\mathrm{AbGrp}}
\newcommand{\Pset}{\mathcal{P}}

\newcommand{\XXX}[1]{\textcolor{red}{#1}}

\renewcommand*\theenumi{\alph{enumi}}
\renewcommand{\labelenumi}{\theenumi)}

\usepackage{enumerate}

%\newarrow{Equals}=====

\usepackage{geometry}
\geometry{tmargin=3cm,bmargin=3cm,lmargin=3cm,rmargin=3cm}

\begin{document}

\vspace*{-4em}
\begin{flushright}Universität Augsburg \\ 11. April 2013 \\ Tim Baumann\end{flushright}

\begin{center}\Large \textbf{Pizzaseminar zur Kategorientheorie} \\
Lösung zum 4. Übungsblatt
\end{center}
\vspace{2em}

%\newbox{\mybox}
%\setbox\mybox=\hbox{\textbf{Aufgabe 1:}}

\begin{list}{}{\labelwidth0em \leftmargin0em \itemindent0.5em \itemsep 1.3em}
\item[\textbf{Aufgabe 1:}]\mbox{}

\begin{enumerate}
\item Sei $M$ eine Menge, $m \in M$ beliebig und $\eta:\Id_{\Set} \Rightarrow \Id_{\Set}$ eine natürliche Transformation. Wir wollen beweisen, dass $\eta_{M}(m) = m$ ist. Wir definieren dazu $1 := \{ \heartsuit \}$ und $f : 1 \to M, \heartsuit \mapsto m$. Die Aussage folgt nun durch eine Diagrammjagd im Natürlichkeitsdiagramm von $\eta$:
\[ \xymatrixcolsep{2pc}\xymatrix{
  \heartsuit \ar@{|->}[dddd] \ar@{|->}[rrrr] &&&& f(\heartsuit) = m \ar@{|->}[ddd] \\
  & 1 \ar[dd]_{\eta_{1}} \ar[rr]^{f} && M \ar[dd]^{\eta_{M}} \\
  \\
  & 1 \ar[rr]_{f} && M & \eta_{M}(m) \ar@{=}@/^1pc/[dl] \\
  \eta_{1}(\heartsuit) = \heartsuit \ar@{|->}[rrr] &&& f(\heartsuit) = m
} \]

\item Sei $M$ eine Menge, $m \in M$ beliebig und $\omega : \Id_{\Set} \Rightarrow K$ eine natürliche Transformation. Wir wollen wieder den gleichen Trick wie in Teilaufgabe \emph{a)} anwenden. Dazu definieren wir wie oben $f : 1 \to M,\ \heartsuit \mapsto m$ und führen dann eine Diagrammjagd durch:
\[ \xymatrixcolsep{2pc}\xymatrix{
  \heartsuit \ar@{|->}[dddd] \ar@{|->}[rrrr] &&&& f(\heartsuit) = m \ar@{|->}[ddd] \\
  & 1 \ar[dd]_{\omega_{1}} \ar[rr]^{f} && M \ar[dd]^{\omega_{M}} \\
  \\
  & 1 \times 1 \ar[rr]_{f \times f} && M \times M & \omega_{M}(m) \ar@{=}@/^1pc/[dl] \\
  \omega_{1}(\heartsuit) = (\heartsuit, \heartsuit) \ar@{|->}[rrr] &&& (m, m)
} \]

\item Angenommen, es gäbe eine natürliche Transformation $\epsilon : P \Rightarrow \Id_{\Set}$. Dann würde die Komponente $\epsilon_{\emptyset}$ von $\Pset(\emptyset) = \{ \emptyset \}$ nach $\emptyset$ verlaufen, also hätte die leere Menge ein Element $f(\emptyset)$. Widerspruch.

In die andere Richtung gibt es eine natürliche Transformation $\eta : \Id_{\Set} \Rightarrow P$ mit
\[ \eta_X : X \to \mathcal{P}(X),\ x \mapsto \{x\}. \]
Wir müssen noch die Natürlichkeit überprüfen. Seien dazu $X, Y$ Mengen und $f : X \to Y$ eine Abbildung. Wir machen eine Diagrammjagd, dieses Mal aber um die Kommutativität des Diagramms zu beweisen:
\[ \xymatrixcolsep{2pc}\xymatrix{
  x  \ar@{|->}[dddd] \ar@{|->}[rrrr] &&&& f(x) \ar@{|->}[ddd] \\
  & X \ar[dd]_{\eta_{X}} \ar[rr]^{f} && Y \ar[dd]^{\eta_{Y}} \\
  \\
  & \Pset(X) \ar[rr]_{f[\cdot]} && \Pset(Y) & \{f(x)\} \ar@{=}@/^1pc/[dl] \\
  \{ x \} \ar@{|->}[rrr] &&& f[\{ x \}]
} \]

\emph{Bemerkung:} Es gibt noch andere natürliche Transformationen~$\Id_\Set
\Rightarrow P$, etwa die, jedes Element einer jeder Menge auf die leere
Teilmenge schickt. In klassischer Logik gibt es dann keine weiteren natürlichen
Transformationen.

\item Betrachte die Menge $M := \{1, 2\}$. Sei $f : 1 \to M$ die Funktion, die
$\heartsuit$ auf das Element aus $M$ schickt, das nicht das ausgewählte Element $a_M$ ist.
Wenn $\tau$ eine natürliche Transformation wäre, müsste folgendes Diagramm kommutieren:
\[ \xymatrixcolsep{2pc}\xymatrix{
  1 \ar[dd]_{\tau_{1}} \ar[rr]^{f} && M \ar[dd]^{\tau_{M}} \\
  \\
  1 \ar[rr]_{f} && M
} \]

Dieses Diagramm kommutiert aber gerade nicht, da $\tau_{M}$ die Funktion ist, die
alles konstant auf $a_M$ schickt und wir $f$ geschickterweise so gewählt haben, dass der Wert von $f$ eben nicht $a_M$ ist.

%\[ f(\heartsuit) = \begin{cases}
%  2, & \text{wenn } a_M = 1\\
%  1, & \text{wenn } a_M = 2
%\end{cases}
%\]

\item In der Kategorie der reellen Vektorräume gibt es für jedes $\lambda \in \R$ die
natürliche Transformation $\mu$ gegeben durch
\[ \mu_{V} : V \to V,\ v \mapsto \lambda v, \]
wie man leicht nachrechnet, wenn man sich an die Eigenschaften von linearen Abbildungen
erinnert.

Sind das schon alle natürliche Transformationen von $\Id_{\Vect{\R}}$ nach $\Id_{\Vect{\R}}$? Angenommen, wir haben eine solche natürliche Transformation $\eta$ gegeben. Sei $V$ ein reeller Vektorraum und $v \in V$ beliebig. Wir definieren die lineare Abbildung $f : \R \to V,\ r \mapsto rv$ und betrachten das Natürlichkeitsdiagramm von $\eta$:
\[ \xymatrixcolsep{2pc}\xymatrix{
  1  \ar@{|->}[dddd] \ar@{|->}[rrrr] &&&& f(1) = v \ar@{|->}[ddd] \\
  & \R \ar[dd]_{\eta_{\R}} \ar[rr]^{f} && V \ar[dd]^{\eta_{V}} \\
  \\
  & \R \ar[rr]_{f} && V & \eta_{V}(v) \ar@{=}@/^1pc/[dl] \\
  \eta_{\R}(1) = \lambda, \lambda \in \R \ar@{|->}[rrr] &&& f(\lambda) = \lambda v
} \]
Dadurch sehen wir, dass $\eta$ tatsächlich die Form $\eta_{V}(v) = \lambda v$
für ein festes $\lambda\,\in\,\R$ besitzen muss.

\emph{Bemerkung:} Die Kategorie~$\Vect{\R}$ ist eine sogenannte abelsche
Kategorie. In jeder abelschen Kategorie wird der
Monoid~$\operatorname{End}(\Id)$ der Endomorphismen der
Identitätstransformation auf kanonische Art und Weise zu einem Ring. Die
Argumentation zeigt dann (fast):
\[ \operatorname{End}(\Id_{\Vect{\R}}) \cong \R. \]
Allgemeiner gilt das für beliebige (kommutative) Ringe~$R$. Das ist eine
Möglichkeit, folgendes Motto der Ringtheorie zu verstehen:
\begin{center}Studiere einen Ring dadurch, indem du seine Kategorie von Modul
untersuchst!\end{center}

\end{enumerate}

\item[\textbf{Aufgabe 2:}]\mbox{}

\begin{enumerate}
\item

Mit dem Lemma aus dem Skript, dass Kategorienäquivalenzen volltreu und wesentlich surjektiv sind, lassen sich diese und viele weitere ähnliche Aussagen elegant beweisen:

Sei $0$ initiales Objekt in $\C$ und $X \in \Ob D$ beliebig. Wir wollen zeigen, dass $FX$ initial in $\D$ ist, es also genau einen Morphismus von $F0$ nach $X$ gibt.
Da $X$ isomorph zu $FGX$ und der Funktor $F$ volltreu ist, haben wir eine Bijektion
\[ \Hom(F0, X) \cong \Hom(F0, FGX) \cong \Hom(0, GX). \]
Weil $0$ initial ist, enthält die rechte Hom-Menge und somit auch die linke Hom-Menge genau einen Morphismus.
\item Wir bezeichnen die Kategorie der Möchtegern-Produkte von $X$ und $Y$ mit $\MP_{X,Y}$, die der Möchtegernprodukte von $Y$ und $X$ mit $\MP_{Y,X}$. Da Möchtegern-Produkte von $X$ und $Y$ aus Symmetriegründen auch Möchtegernprodukte von $Y$ und $X$ sind (genauer: als solche angesehen werden können), können wir den Funktor $F : \MP_{X,Y} \to \MP_{Y,X}$ definieren:
\[ \begin{array}{@{}rcl@{}}
  \left(\vcenter{\xymatrix{X & \ar[l]_{\pi_X} Q \ar[r]^{\pi_Y} & Y}}\right) & \longmapsto & \left(\vcenter{\xymatrix{Y & \ar[l]_{\pi_Y} Q \ar[r]^{\pi_X} & X}}\right) \\\\
  \left(\vcenter{\xymatrix{
    & Q \ar[ld] \ar[dd]^{f} \ar[rd] \\
  X & & Y \\
    & R \ar[lu] \ar[ru]
  }}\right)
  & \longmapsto &
  \left(\vcenter{\xymatrix{
    & Q \ar[ld] \ar[dd]^{f} \ar[rd] \\
  Y & & X \\
    & R \ar[lu] \ar[ru]
  }}\right)
\end{array} \]

Den zu $F$ quasi-inversen Funktor $G:\MP_{Y,X} \to \MP_{X,Y}$ definieren wir genau spiegelverkehrt zu $F$. Wie man leicht nachprüft, ergeben $F$ und $G$ eine Äquivalenz von $\MP_{X,Y}$ und $\MP_{Y,X}$, wobei die natürlichen Transformationen zwischen $F$ und $G$ nur aus den Identitätsmorphismen bestehen.

Ein initiales Objekt in $\MP_{X,Y}$ ist ein Produkt von $X$ und $Y$, ein initiales Objekt in $\MP_{Y,X}$ ein Produkt von $Y$ und $X$. Mit Teilaufgabe a) folgt, dass ein Produkt von $X$ und $Y$ auch ein Produkt von $Y$ und $X$ ist und umgekehrt.

\end{enumerate}

\item[\textbf{Aufgabe 3:}]\mbox{}

Wähle für jeden endlich-dimensionalen Vektorraum $V$ eine feste Basis $(b_1, \ldots, b_{\dim V})$ und definiere das Koordinatensystem $\eta_V$ bezüglich dieser Basis durch die Setzung
\[ \eta_V:\R^{\dim V} \to V,\ e_i \mapsto b_i. \]
Zwischen der Numerikerkategorie $\C$ und der $\Vect{\R}_{\mathrm{fd}}$ verlaufen die Funktoren
\[ \begin{array}{@{}rcrcl@{}}
  F &:& \C & \longrightarrow & \Vect{\R}_{\mathrm{fd}}\\
  && \R^{n} & \longmapsto & \R^{n}\\
  && M \in \R^{m \times n} & \longmapsto & \text{die von der Matrix $M$ dargestellte lineare Abbildung}\\
  &&&& \text{zwischen $\R^n$ und $\R^m$ bezüglich der kanonischen Basen}\\\\
  G &:& \Vect{\R}_{\mathrm{fd}} & \longrightarrow & \C\\
  && V & \longmapsto & \R^{\dim V}\\
  && (f:V \to W) & \longmapsto & \eta_W^{-1} \circ f \circ \eta_V \text{ (bzw. die Matrix dieser Abbildung).}
\end{array} \]
Diese Funktoren bilden eine Äquivalenz zwischen den beiden Kategorien, da folgende Natürlichkeitsdiagramme für alle $(V \xrightarrow{f} W) \in \Vect{\R}_{\mathrm{fd}}$ bzw. $(\R^n \xrightarrow{M} \R^m) \in \C$ offensichtlicherweise kommutieren:
\[ \xymatrixcolsep{7pc}\xymatrixrowsep{3pc}
\xymatrix{
  GFV \ar[d]_{\eta_V} \ar[r]^{GFM = \eta_W^{-1} \circ M \circ \eta_V} & GFW \ar[d]^{\eta_W}\\
  \R^{n} \ar[r]_{M} & \R^{m}
}
\quad
\xymatrix{
  FGV \ar[d]_{\eta_V} \ar[r]^{FGf = \eta_W^{-1} \circ f \circ \eta_V} & FGW \ar[d]^{\eta_W}\\
  V \ar[r]_{f} & W
} \]

\newpage
\item[\textbf{Projektaufgabe:}]\mbox{}

Wir haben einen Morphismus $\varphi:A \to B$ gegeben und wollen eine natürliche Transformation $\eta:\Hom_{\C}(\freist, A) \Rightarrow \Hom_{\C}(\freist, B)$ finden, d.\,h. es muss für alle $f:Y \to X$ aus $\C$ das Natürlichkeitsdiagramm kommutieren:
\[ \xymatrixcolsep{5pc}\xymatrixrowsep{3pc}\xymatrix{
  \Hom_{\C}(X, A) \ar[d]_{\eta_X} \ar[r]^{g\ \mapsto\ g \circ f} & \Hom_{\C}(Y, A) \ar[d]^{\eta_Y} \\
  \Hom_{\C}(X, B) \ar[r]_{g\ \mapsto\ g \circ f} & \Hom_{\C}(Y, B)
} \]
Wir setzen $\eta_{Z} := (g \mapsto \phi \circ g)$. Wenn wir nun einen Morphismus $p:X \to A$ im Diagramm von oben links nach unten rechts verfolgen, erhalten wir einerseits $((\varphi \circ p) \circ f)$ und andererseits
$(\varphi \circ (p \circ f))$. Aufgrund der Assoziativität der Verknüpfung von Morphismen sind diese Ergebnisse gleich und das Diagramm kommutiert wie gewünscht.

\end{list}

\end{document}
