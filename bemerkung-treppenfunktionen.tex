\documentclass[a4paper,ngerman]{scrartcl}

\usepackage[utf8]{inputenc}
\usepackage[ngerman]{babel}
\usepackage{amsmath,amssymb}
\usepackage[all]{xy}
\usepackage[protrusion=true,expansion=true]{microtype}
\usepackage{lmodern}
\usepackage{hyperref}

\newcommand{\RR}{\mathbb{R}}

\setlength\parskip{\medskipamount}
\setlength\parindent{0pt}

\begin{document}

\section*{Bemerkung zu Treppenfunktionen in der Analysis~II}

Es gibt eine sehr elegante Art und Weise, um den Integralbegriff kategoriell zu
definieren. Das macht diese Notiz nicht. Wir können aber den konventionellen
Zugang, wie er gerade in der Analysis-II-Vorlesung gelehrt wird, kategoriell
verstehen.

Dazu beobachten wir, dass wir auf der Menge der Zerlegungen eines festen
Intervalls~$[a,b]$ sogar eine Partialordnung definieren können:
\[ Z \preceq Z' \quad:\Longleftrightarrow\quad
  \text{$Z'$ ist feiner als~$Z$}
  \quad:\Longleftrightarrow\quad
  Z \subseteq Z'. \]
Somit können wir die Zerlegungen von~$[a,b]$ sogar zu einer \emph{Kategorie}
organisieren. (Diese ist sogar \emph{filtriert}, da es wenigstens eine
Zerlegung gibt und je zwei zu einer gemeinsamen verfeinert werden können.)

Zu jeder Zerlegung~$Z$ haben wir den Vektorraum
\[ T_Z := \{ f : [a,b] \to \RR \,|\, \text{$f$ konstant auf
den inneren Teilstücken von~$Z$} \} \]
der Treppenfunktionen zu~$Z$. Ist~$Z'$ eine feinere Zerlegung, haben wir die
kanonische Inklusionsabbildung
$T_Z \longrightarrow T_{Z'}$.
Somit liegt hier tatsächlich ein \emph{Diagramm} (Funktor) vor:
\[ \begin{array}{@{}rcl@{}}
  \text{Kategorie der Zerlegungen von $[a,b]$}
    &\longrightarrow& \RR\text{-}\mathrm{Vect} \\
  Z &\longmapsto& T_Z.
\end{array} \]
Der \emph{Kolimes} dieses Diagramms ist gerade der Vektorraum aller
Treppenfunktionen auf~$[a,b]$.
Wie kommt nun Integration ins Spiel? Jeder der Vektorräume~$T_Z$ erlaubt die
kanonische Integrationsabbildung
\[ \begin{array}{@{}rcl@{}}
  T_Z &\longrightarrow& \RR \\
  f &\longmapsto&
    \sum_{j=1}^n \text{(Wert von~$f$ auf~$(x_{j-1},x_j)$)} \cdot (x_j - x_{j-1}),
\end{array} \]
wobei~$Z = \{ x_0 < \cdots < x_n \}$. Diese Abbildungen sind miteinander in dem
Sinn verträglich, als dass für~$Z \preceq Z'$ das Diagramm
\[ \xymatrix{
  T_Z \ar[rr] \ar[dr] && T_{Z'} \ar[dl] \\
  & \RR
} \]
kommutiert. Diese Abbildungen definieren also tatsächlich einen \emph{Kokegel}!
Da~$T_{[a,b]}$ der initiale Kokegel ist, wird somit eine Abbildung
\[ T_{[a,b]} \longrightarrow \RR \]
induziert. Das ist die Integrationsabbildung für Treppenfunktionen aus der
Analysis~II.

\end{document}
