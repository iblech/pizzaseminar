\documentclass[12pt,utf8,notheorems,compress]{beamer}

\usepackage[english]{babel}

\usepackage{amsmath,amssymb}
%\usepackage[framed,amsmath,thmmarks,hyperref]{ntheorem}

%\usepackage[small,nohug]{diagrams}
%\diagramstyle[labelstyle=\scriptstyle]

%\usepackage[protrusion=true,expansion=false]{microtype}

%\usepackage{lmodern}
\usepackage{tabto}
\usepackage{tikz}
\usepackage{array}
\usepackage[all]{xy}

%\usepackage[natbib=true,style=numeric]{biblatex}
%\usepackage[babel]{csquotes}
%\bibliography{lit}

%\usepackage{hyperref}

\setlength\parskip{\medskipamount}
\setlength\parindent{0pt}

%\theoremseparator{:}
\theoremstyle{plain}  %nonumberplain
%\newtheorem{beh}{Behauptung}
\newtheorem{proposition}{Proposition}
\newtheorem{corollary}{Korollar}
\newtheorem{theorem}{Satz}
\theoremstyle{definition}
\newtheorem{definition}{Definition}
%\newtheorem{kor}{Korollar}
%\newtheorem{satz}{Satz}
%\newtheorem{lemma}{Lemma}
%\newtheorem{hilfsaussage}{Hilfsaussage}
%\theorembodyfont{\normalfont}
\newtheorem{axiom}{Axiom}
%\newtheorem{defnprop}{Definition/Proposition}
%\newtheorem{bem}{Bemerkung}
%\newtheorem{bsp}{Beispiel}
%\theoremsymbol{\ensuremath{\openbox}}
%\newtheorem{proof}{Beweis}
%\newtheorem{defn}{Definition}

\newcommand{\lra}{\longrightarrow}
\newcommand{\lhra}{\ensuremath{\lhook\joinrel\relbar\joinrel\rightarrow}}
\newcommand{\thlra}{\relbar\joinrel\twoheadrightarrow}

\newcommand{\Z}{\mathbb{Z}}
\renewcommand{\C}{\mathcal{C}}
\newcommand{\N}{\mathbb{N}}
\newcommand{\R}{\mathbb{R}}
\newcommand{\Hom}{\mathrm{Hom}}
\newcommand{\id}{\mathrm{id}}
\newcommand{\Aut}[1]{\operatorname{Aut}(#1)}
\newcommand{\GL}[1]{\operatorname{GL}(#1)}
\newcommand{\freist}{\_{}\_{}}
\newcommand{\Set}{\mathrm{Set}}
\newcommand{\Grp}{\mathrm{Grp}}
\newcommand{\Vect}{\mathrm{Vect}}

\def\longleadsto{\mathrel{-}\joinrel\leadsto}
\DeclareMathOperator{\ggT}{ggT}
\DeclareMathOperator{\Ob}{Ob}
\newcommand{\op}{\mathrm{op}}

\title{Markov chains and MCMC methods}
\author[Kleine Bayessche AG]{%
  %\includegraphics[scale=0.4]{relationen.png}
  What are Markov chains? How can they be used as models? How can they be used
  to draw samples from distributions? How can they be used to evaluate
  integrals over high-dimensional domains? How can they be used to build IRC
  chat bots?
  }
%Ingo Blechschmidt \\ mit Illustrationen von Carina Willbold}
%\institute{Pizzaseminar in Mathematik}
\date{7. November 2014}

%\usetheme{Warsaw}  %Warsaw, Berkeley?
\usetheme{Warsaw}
\useoutertheme{split}
\usecolortheme{seahorse}
\usefonttheme{serif}
\usepackage{kurier}
\useinnertheme{rectangles}
%\usepackage{bookman}
%\setbeamercovered{transparent}

\setbeamertemplate{navigation symbols}{}
%\setbeamertemplate{footline}{}
%\setbeamertemplate{headline}{}

%\beamertemplateboldcenterframetitle
%\setbeamerfont{frametitle}{size={\Large}}

\newcommand*\oldmacro{}%
\let\oldmacro\insertshorttitle%
\renewcommand*\insertshorttitle{%
  \oldmacro\hfill\insertframenumber\,/\,\inserttotalframenumber\hfill}

\newenvironment{changemargin}[2]{%
  \begin{list}{}{%
    \setlength{\topsep}{0pt}%
    \setlength{\leftmargin}{#1}%
    \setlength{\rightmargin}{#2}%
    \setlength{\listparindent}{\parindent}%
    \setlength{\itemindent}{\parindent}%
    \setlength{\parsep}{\parskip}%
  }%
  \item[]}{\end{list}}

\newcommand{\hil}[1]{{\usebeamercolor[fg]{item}{#1}}}

\begin{document}

\setbeameroption{show notes}
\setbeamertemplate{note page}[plain]

\frame{\titlepage}
%\frame[t]{\frametitle{Gliederung}\begin{minipage}{\textwidth}\begin{small}\tableofcontents\end{small}\end{minipage}}
\frame[t]{\frametitle{Outline}\tableofcontents}

\section{Live demo: an IRC chat bot}

\frame{
  \begin{center}
    \huge
    Live demo: an IRC chat bot
  \end{center}
}


\section{What is a Markov chain?}

\frame[t]{\frametitle{What is a Markov chain?}
  \begin{itemize}
    \item A Markov chain is a system which undergoes transitions from one state
    to another according to probabilities~$P(X' = j \,|\, X = i) =: p_{ij}$.
    \item More abstractly, a Markov chain on a state space~$S$ is a map~$S \to
    D(S)$, where~$D(S)$ is the set of probability measures on~$S$.
    \item Categorically, a Markov chain is a coalgebra for the functor~$D :
    \Set \to \Set$.
  \end{itemize}
}

\note{
  The following systems can be modeled by Markov chains:
  \begin{itemize}
  \item the position of a peg in the game of snakes and ladders
  \item the position of a random walk
  \item the weather, if we oversimplify a lot
  \end{itemize}

  The following cannot:

  \begin{itemize}
  \item the state of a game of blackjack
  \end{itemize}
}

\frame[t]{\frametitle{Basic theory on Markov chains}
  \begin{itemize}
    \item Transition matrix is a stochastic matrix
    \item Main theorem
    \item XXX
  \end{itemize}
}

\end{document}
