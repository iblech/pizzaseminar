\section[Topologische Quantenfeldtheorien]{Topologische Quantenfeldtheorien \hfill \small
Sven Prüfer}

\emph{Werbung:}
Topologische Quantenfeldtheorien haben neben ihrer physikalischen
Bedeutung vielseitige Anwendungen in der Mathematik. Ihre kategorielle
Formulierung erlaubt es zum Beispiel, Berechnungen von topologischen oder
geometrischen Invarianten so zu organisieren, dass diese durch Zerlegen
des geometrischen Objekts in kleinere Bestandteile erheblich einfacher
werden. Konkrete Beispiele sind etwa die Spin--Hurwitz-Zahlen oder
Chern--Simons-Theorie.

In diesem Vortrag werden monoidale Kategorien sowie Funktoren eingeführt,
alle Begriffe erklärt, die zum Definieren der Kategorien für topologische
Quantenfeldtheorien notwendig sind und am Ende das einfachste
niedrig-dimensionale Beispiel durchgerechnet. Insbesondere wird kein
physikalisches oder tieferes topologisches bzw. geometrisches Grundwissen
vorausgesetzt.
