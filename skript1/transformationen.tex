\section[Natürliche Transformationen]{Natürliche Transformationen \hfill \small
Tim Baumann}

\emph{Werbung:} Wir werden verstehen, was natürliche Transformationen sind,
weshalb ihre Definition ganz einfach ist und wozu man sie benötigt. Ihre
Bedeutung werden wir aus verschiedenen Blickwinkeln beleuchten. Mit natürlichen
Transformationen können wir dann auch Funktorkategorien definieren, die für das
Yoneda-Lemma später sehr wichtig sind. Außerdem können wir definieren, wann
zwei Kategorien zueinander äquivalent sind.

\textbf{XXX:} Hier fehlt noch Motivation für das Konzept.

\begin{defn}Eine \emph{natürliche Transformation} $\eta : F \Rightarrow G$
zwischen Funktoren $F, G : \C \to \D$ besteht aus
\begin{enumerate}
\item[] einem Morphismus~$\eta_X : F(X) \to G(X)$ für jedes Objekt~$X \in \Ob \C$
\end{enumerate}
sodass
\begin{enumerate}
\item[]
für alle Morphismen $f : X \to Y$ in~$\C$ das Diagramm
\[ \xymatrix{
  F(X) \ar[r]^{F(f)} \ar[d]_{\eta_X} & F(Y) \ar[d]^{\eta_Y} \\
  G(X) \ar[r]_{G(f)} & G(Y)
} \]
kommutiert.
\end{enumerate}
\end{defn}

\begin{motto}\label{nattrafoglm}
Die Komponenten einer natürlichen Transformation sind \emph{gleichmäßig} über
alle Objekte $X \in \Ob\C$ definiert.
\end{motto}


\subsection{Beispiele für natürliche Transformationen}


\subsubsection*{Erste Beispiele mit Mengen}

Seien $\Id_\Set, K: \Set \to \Set$ die Funktoren mit
\[ \begin{array}{@{}rrcl@{}}
  \Id_\Set: & X &\longmapsto& X \\
  & f &\longmapsto& f \\\\
  K: & X &\longmapsto& X \times X \\
  & f:X\to Y &\longmapsto& (X \times X \to Y \times Y,\ (x_1,x_2) \mapsto (f(x_1),f(x_2)).
\end{array} \]

Dann kann man folgende Beobachtungen treffen:

\begin{enumerate}
\item Natürlich gibt es für jede konkrete Menge~$X$ im Allgemeinen viele
Abbildungen
\[ X \longrightarrow X. \]
Aber es gibt nur eine natürliche Transformation~$\eta : \Id_\Set \Rightarrow
\Id_\Set$, nämlich die mit
\[ \eta_X : X \longrightarrow X,\ x \longmapsto x. \]
Wir sehen das Motto in diesem Beispiel bestätigt: Denn der
Funktionsterm von~$\eta_X$ ist in der Tat gleichmäßig definiert, es kommt keine
Fallunterscheidung über~$X$ vor.
\item Analog gibt es für jede konkrete Menge~$X$ im Allgemeinen viele
Abbildungen~$X \to X \times X$ (also~$\Id_\Set(X) \to K(X)$). Aber es gibt nur
eine einzige natürliche Transformation $\eta : \Id_\Set \Rightarrow K$, nämlich
die mit
\[ \eta_X: X \to X \times X,\ x \mapsto (x,x) \]
für alle Mengen~$X$. Auch hier ist das Motto bestätigt.
\item Für konkrete Mengen~$X$ gibt es im Allgemeinen viele Abbildungen
\[ \P(X) \longrightarrow X, \]
aber es gibt keine natürliche Transformation $\P \Rightarrow \Id_\Set$.
Auch dieser Sachverhalt illustriert das Motto: Denn uns fällt kein
Abbildungsterm ein, der ohne Fallunterscheidung über~$X$ Funktionen des
Typs~$\P(X) \to X$ definieren könnte.
\end{enumerate}


\subsubsection*{Entgegengesetzte Gruppe}

In der Gruppentheorie trifft man folgende Beobachtung: \emph{Jede
Gruppe~$(G,\circ)$ ist natürlich isomorph zu ihrer entgegengesetzten Gruppe
$(G^\op,\bullet)$.}
Dabei hat~$G^\op$ dieselben Elemente wie~$G$, die
Gruppenverknüpfung~$\bullet$ ist aber
genau anders herum definiert,
\[ g \bullet h := h \circ g. \]
In der Tat ist die Abbildung
\[ \begin{array}{@{}rrcl@{}}
  \eta_G : & G &\longrightarrow& G^\op \\
  & g &\longmapsto& g^{-1}
\end{array} \]
bijektiv und auch wirklich ein Gruppenhomomorphismus, da für alle~$g,h \in G$
die Rechnung
\[ \eta_G(g \circ h) = (g \circ h)^{-1} = h^{-1} \circ g^{-1} =
  g^{-1} \bullet h^{-1} = \eta_G(g) \bullet \eta_G(h) \]
gilt. Ohne den Begriff der natürlichen Transformation kann man aber nicht
verstehen, wieso dieser Isomorphismus das Prädikat \emph{natürlich} verdient
hat: Man kann sich nur mit der Aussage begnügen, der Isomorphismus sei
\emph{kanonisch} definiert; das ist jedoch ein informaler Begriff.

Kategoriell verstehen wir: Die Isomorphismen~$\eta_G$ sind Komponenten einer
natürlichen Transformation, und zwar einer vom Identitätsfunktor auf~$\Grp$ in
den "`entgegengesetzte Gruppe"'-Funktor~$F$:
\[ \begin{array}{@{}rrcl@{}}
  F : & \Grp &\longrightarrow& \Grp \\
  & G &\longmapsto& G^\op \\
  & f &\longmapsto& f^\op
\end{array} \]
Der Gruppenhomomorphismus~$f^\op$ ist als Abbildung derselbe wie~$f$; durch das
doppelte Bilden der entgegengesetzten Gruppen ist er auch wirklich ein
Gruppenhomomorphismus. Das Natürlichkeitsdiagramm
\[ \xymatrixcolsep{4pc}\xymatrixrowsep{4pc}\xymatrix{
  \Id_\Grp(G)=G \ar[rr]^{f} \ar[d]_{\eta_G} && H=\Id_\Grp(H) \ar[d]_{\eta_H} \\
  F(G)=G^\op \ar[rr]_{f^\op} && H^\op=F(H)
} \]
kommutiert tatsächlich, wie eine Diagrammjagd zeigt:
\[ \xymatrixcolsep{4pc}\xymatrixrowsep{4pc}\xymatrix{
  g \ar@{|->}^f[rr] \ar@{|->}^{\eta_G}[d] & & f(g) \ar@{|->}^{\eta_H}[d] \\
  g^{-1} \ar@{|->}^f[r]  & f(g^{-1}) \ar@{=}_{\text{$f$ Homo}}[r] & (f(g))^{-1}
} \]


\begin{bem}\label{interpretnat}%
Manchmal findet man Aussagen der Art "`es gibt eine natürliche Abbildung
von $\ldots$ nach $\ldots$"' in der Literatur. Damit ist dann oft gemeint, dass
man Quelle und Ziel als Auswertungen zweier Funktoren verstehen kann und dass
zwischen diesen Funktoren eine natürliche Transformation verläuft.\end{bem}

\textbf{XXX:} Es fehlt noch ein weiteres Beispiel:
"`Doppeldualraum"'


\subsubsection*{Determinante}

In der linearen Algebra lernt man die Determinante von Matrizen kennen. Diese
ist bezüglich des Grundrings (oder Grundkörpers) gleichmäßig definiert -- das
erkennt man entweder an der laplaceschen Entwicklungsformel oder an der
Charakterisierung als eindeutige multilineare Abbildung mit gewissen
Eigenschaften.

Unserem Motto zufolge sollte die Determinantenabbildung daher
auch als natürliche Transformation verstanden werden können. Das ist in der Tat
der Fall: Für festes~$n \geq 0$ haben wir zwei Funktoren von der Kategorie der
Ringe in die Kategorie der Monoide, nämlich
\begin{enumerate}
\item den Funktor, der jedem Ring~$R$ den multiplikativen Monoid der~$(n \times
n)$-Matrizen über~$R$ zuordnet, und
\item den Funktor, der jedem Ring~$R$ seinen zugrundeliegenden multiplikativen
Monoid zuordnet.
\end{enumerate}
Die Determinante ist eine natürliche Transformation zwischen diesen beiden
Funktoren; das Natürlichkeitsdiagramm
\[ \xymatrix{
  R^{n \times n} \ar[r] \ar[d]_\det & S^{n \times n} \ar[d]^\det \\
  R \ar[r] & S
} \]
besagt für Ringhomomorphismen~$f : R \to S$, dass es keine Rolle spielt, ob man
eine Matrix über~$R$ als Matrix mit Koeffizienten in~$S$ interpretiert und dann
die Determinante nimmt, oder ob man umgekehrt zuerst die Determinante als
Element von~$R$ berechnet und dann in den Ring~$S$ transportiert.

% XXX: Ist Punktefunktor eines Schemas!


\subsection{Funktorkategorien}

\begin{defn}
Seien Funktoren $F,G,H: \C \to \D$ und natürliche Transformationen $\alpha: F
\Rightarrow G$ und $\beta: G \Rightarrow H$ gegeben:
\[ \xymatrix{\C \ar@/^1.7em/[rr]^F_{\Downarrow \alpha} \ar[rr]^G \ar@/_1.7em/[rr]^{\Downarrow \beta}_H & & \D} \]
Dann heißt $\beta \circ \alpha:F \Rightarrow G$ die \emph{(vertikale)
Verkettung} von~$\alpha$ und~$\beta$ und ist komponentenweise durch
\[ (\beta \circ \alpha)_X := \beta_X \circ \alpha_X : F(X) \to H(X) \]
gegeben.
\end{defn}

\begin{prop}Die so definierte Zuordnung~$\beta \circ \alpha$ ist in der Tat
eine natürliche Transformation.\end{prop}
\begin{proof}Da für alle~$f:X \to Y$ in~$\C$ die beiden Teilquadrate
im Diagramm
\[ \xymatrix{
  F(X) \ar[r]^{F(f)} \ar[d]_{\alpha_X} \ar@/_ 1cm/[dd]_{(\beta \circ \alpha)_X := \beta_X \circ \alpha_X} & F(Y) \ar[d]_{\alpha_Y} \ar@/^ 1cm/[dd]^{\beta_Y \circ \alpha_Y =: (\beta \circ \alpha)_Y} \\
  G(X) \ar[r]^{G(f)} \ar[d]_{\beta_X} & G(Y) \ar[d]_{\beta_Y} \\
  H(X) \ar[r]^{H(f)} & H(Y)
} \]
kommutieren, kommutiert auch das äußere Rechteck. Das ist gerade das
Natür\-lich\-keits\-dia\-gramm für~$\beta\circ\alpha$.
\end{proof}

Außerdem gibt es für jeden Funktor~$F:\C\to\D$ eine natürliche
Identitätstransformation $\id_F : F \Rightarrow F$ mit $\Id_X := id_{F(X)}$.
Damit wird folgende Definition möglich:
\begin{defn}
Die \emph{Funktorkategorie} $\Funct(\C,\D)$ zu zwei Kategorien $\C$ und $\D$
ist die Kategorie mit Funktoren $F: \C \to \D$ als Objekten und natürlichen
Transformationen als Morphismen.
\end{defn}

\textbf{XXX:} Es fehlt noch eine Bemerkung über die horizontale Verkettung von
natürlichen Transformationen.

\begin{lemma}\label{natTransIsoLemma}
Seien $\C,\D$ Kategorien, $F,G: \C \to \D$ Funktoren und $\alpha: F \Rightarrow
G$ eine natürliche Transformation. Dann ist $\alpha$ genau dann ein
Isomorphismus in der Funktorkategorie~$\Funct(\C,\D)$, wenn alle Komponenten
$\alpha_X$, $X \in \Ob\C$, jeweils Isomorphismen in~$\D$ sind.
\end{lemma}

\begin{proof}\begin{itemize}
\item[\glqq$\Rightarrow$\grqq] Sei $\alpha$ ein Isomorphismus, dann existiert
also eine natürliche Transformation $\alpha^{-1}$ mit $\alpha \circ \alpha^{-1}
= \id_F$, $\alpha^{-1} \circ \alpha = \id_G$. Das bedeutet, dass für jedes $X
\in \Ob\C$ die Gleichheiten
\begin{align*}
	\id_{F(X)} = (\id_F)_X = (\alpha \circ \alpha^{-1})_X = \alpha_X \circ \alpha^{-1}_X \\	
	\id_{G(X)} = (\id_G)_X = (\alpha^{-1} \circ \alpha)_X = \alpha^{-1}_X \circ \alpha_X
\end{align*}
gelten. Also sind die Komponenten $\alpha_X$ jeweils Isomorphismen in $\D$.
\item[\glqq$\Leftarrow$\grqq] Seien alle Komponenten~$\alpha_X$ invertierbar.
Dann können wir versuchen, eine inverse natürliche Transformation $\beta: G
\Rightarrow F$ über die Setzung
\[ \beta_X := (\alpha_X)^{-1} : G(X) \to F(X) \]
zu definieren. Sicher gilt dann $\alpha \circ \beta = \id_F$ und $\beta \circ \alpha
= \id_G$, aber es bleibt noch zu zeigen, dass $\beta$ auch wirklich eine natürliche
Transformation ist. Dazu rechnen wir für jeden Morphismus~$f: X \to Y$ die
Natürlichkeitsbedingung nach:
\begin{align*}
  & G(f) \circ \alpha_X = \alpha_Y \circ F(f) \\
  \Longrightarrow{}& (\alpha_Y)^{-1} \circ G(f) \circ \alpha_X \circ (\alpha_X)^{-1} = (\alpha_Y)^{-1} \circ \alpha_Y \circ F(f) \circ (\alpha_X)^{-1} \\
  \Longrightarrow{}& (\alpha_Y)^{-1} \circ G(f) = F(f) \circ (\alpha_X)^{-1} \\
  \Longrightarrow{}& \beta_Y \circ G(f) = F(f) \circ \beta_X \qedhere
\end{align*}
\end{itemize}
\end{proof}

\begin{defn}
Invertierbare natürliche Transformationen heißen auch \emph{natürliche
Isomorphismen}, und Funktoren, zwischen denen ein natürlicher Isomorphismus
verläuft, heißen \emph{zueinander (natürlich) isomorph}.
\end{defn}


\subsection{Kategorienäquivalenzen}

\begin{defn}\label{cateqv}
Eine \emph{Kategorienäquivalenz} zwischen Kategorien~$\C$ und~$\D$ besteht aus
Funktoren
\[ \xymatrix{\C \ar@/^/[r]^F & \D, \ar@/^/[l]^G} \]
die zueinander \emph{quasi-invers} sind, d.\,h. dass die Kompositionen von~$F$
und~$G$ jeweils natürlich isomorph zu den entsprechenden Identitätsfunktoren
sind:
\[ G \circ F \cong \Id_\C, \qquad F \circ G \cong \Id_\D. \]
Die Kategorien~$\C$ und~$\D$ heißen dann \emph{zueinander äquivalent}:
$\C \simeq \D.$
\end{defn}

Es gibt auch das Konzept der \emph{Isomorphie von Kategorien}. Da fordert man,
dass es Funktoren
\scalebox{0.6}{$\xymatrix{\C \ar@/^/[r]^F & \D
\ar@/^/[l]^G}$} gibt, die
zueinander nicht nur quasi-invers, sondern tatsächlich invers sind, d.\,h. die
Beziehungen
\[ G \circ F = \Id_\C, \qquad F \circ G = \Id_\D \]
erfüllen. Das ist aber in den meisten Fällen kein gutes Konzept:
Denn wie schon in Bemerkung~\ref{gleichheitfunktoren} festgehalten, ist die
Gleichheit von Funktoren eine böse Bedingung. In der Tat sind die meisten in
der Natur vorkommenden Kategorienäquivalenzen auch "`nur"' Äquivalenzen, keine
Isomorphismen. Kommt doch mal ein Isomorphismus von Kategorien vor, so ist das
meist ein überhaupt nicht tiefsinniger "`technischer Zufall"', der nur an
geeigneten Wahlen bestimmter Definitionen liegt.

Wieso man Äquivalenzen von Kategorien untersucht, liegt in folgendem
Motto begründet:
\begin{motto}\label{mottoeqv}%
Sei~$\varphi$ eine mathematische Aussage über Kategorien, die
sich nur unter Verwendung der Konzepte \emph{Objekt}, \emph{Morphismus},
\emph{Verkettung von Morphismen} und \emph{Gleichheit von Morphismen}
formulieren lässt. Sind dann~$\C$ und~$\D$ zueinander äquivalente Kategorien,
so gilt~$\varphi$ genau dann in~$\C$, wenn~$\varphi$ in~$\D$ gilt.
\end{motto}
Beispiele für Aussagen dieser Art sind etwa:
\begin{itemize}
\item Die Kategorie besitzt ein initiales Objekt.
\item Je zwei parallele Morphismen sind gleich.
\item Jeder Morphismus in ein initiales Objekt ist sogar schon ein
Isomorphismus.
\end{itemize}
Beispiele für Aussagen, die über die Reichweite des Mottos hinausgehen, sind:
\begin{itemize}
\item Die Kategorie besitzt genau ein Objekt.
\item Die Kategorie besitzt genau ein initiales Objekt.
\item Sind zwei Objekte zueinander isomorph, so sind sie schon gleich.
\item Je zwei Morphismen (egal zwischen welchen Objekten) sind gleich.
\end{itemize}
Man erachtet es nicht als schlimm, dass diese Aussagen nicht unter Äquivalenz
erhalten bleiben. Denn wegen der vorkommenden Vergleiche von Objekten auf Gleichheit
handelt es sich sowieso um böse Aussagen.

\begin{bem}Mit Techniken aus der formalen Logik kann man Motto~\ref{mottoeqv}
auch rigoros beweisen. Das ist nicht besonders schwer, die Hauptschwierigkeit
liegt darin, den Begriff \emph{Aussage} präzise zu definieren.\end{bem}


\subsubsection*{Beispiele für Kategorienäquivalenzen}

\begin{bsp}\label{setopset}%
Die duale Kategorie~$\Set^\op$ kann nicht zu~$\Set$ äquivalent sein.
Denn in~$\Set$ ist jeder Morphismus in ein initiales Objekt schon ein
Isomorphismus, in~$\Set^\op$ aber nicht.\end{bsp}

\begin{bsp}
Die Kategorie $\Eins$, die nur aus einem Objekt~$\star$ sowie dessen
Iden\-ti\-täts\-mor\-phis\-mus besteht, ist zu jeder bewohnten
\emph{indiskreten Kategorie} $\C$ (d.\,h. einer solchen, die mindestens ein
Objekt besitzt und in der zwischen je zwei Objekten genau ein Morphismus
verläuft) äquivalent. 
\end{bsp}
\begin{proof}
Da~$\C$ bewohnt ist, gibt es ein Objekt $K \in \Ob \C$. Dann können wir die Funktoren
\[ \begin{array}{@{}rrcl@{}}
  F: & \Eins &\longrightarrow& \C \\
  & \star &\longmapsto& K \\
  & \id_\star &\longmapsto& \id_K \\\\
  G: & \C &\longrightarrow& \Eins \\
  & X &\longmapsto& \star \\
  & f &\longmapsto& \id_\star
\end{array} \]
definieren.
Da es nur einen einzigen Funktor von $\Eins$ nach $\Eins$ gibt (obacht! Es ist schlecht, das zu sagen), ist klar, dass $G \circ F \cong \Id_\Eins$ gilt. Noch zu zeigen ist also, dass auch $F \circ G \cong \Id_\C$ gilt.

Dazu definieren eine natürliche Transformation $\eta: F \circ G \Rightarrow
\Id_\C$, die sich als natürlicher Isomorphismus herausstellen wird. Dabei
verwenden wir für jedes $X \in \C$ als $\eta_X$ den \emph{eindeutigen}
Morphismus $(F\circ G) (X) \to \Id_\C(X)$. Da diese Definition gleichmäßig
in~$X$ ist, erwarten wir, dass die Natürlichkeitsbedingung erfüllt ist; und das
ist auch in der Tat der Fall:
\[ \xymatrixcolsep{5pc}\xymatrixrowsep{5pc}\xymatrix{
  K = (F \circ G)(X) \ar[r]^{(F \circ G)(f)=\id_K} \ar[d]_{\eta_X} & (F \circ G)(Y) = K \ar[d]^{\eta_Y} \\
  X = \Id_\C (X) \ar[r]_{\Id_\C(f)=f} & \Id_\C (Y) = Y
} \]
Da in einer indiskreten Kategorie alle Morphismen Isomorphismen sind, ist
insbesondere jede Komponente~$\eta_X$ ein Isomorphismus, und damit ist nach
Lemma \ref{natTransIsoLemma} auch $\eta$ selbst ein Isomorphismus.
\end{proof}

\begin{bem}Wer topologische Räume kennt, fühlt sich durch das Beispiel
vielleicht an folgende Beobachtung erinnert: Bewohnte topologische Räume, in
denen je zwei Punkte bis auf Homotopie durch genau einen Weg miteinander
verbunden werden können, sind zusammenziehbar. Die Ähnlichkeit ist nicht nur
formal: Jedem topologischen Raum kann man seinen \emph{Fundamentalgruppoid}
zuordnen, das ist die Kategorie, deren Objekte durch die Punkte des Raums und
deren Morphismen durch die Homotopieklassen von Wegen gegeben sind. Der
Fundamentalgruppoid eines Raums, der obige Eigenschaft erfüllt, ist
indiskret.\end{bem}
% XXX: Da fehlt noch mehr!

\begin{bsp}Die Kategorie der endlich-dimensionalen~$K$-Vektorräume ist
äquivalent zu folgender sehr konkreten Kategorie:
\begin{align*}
  \text{Objekte: } & n \in \mathbb{N} \\
  \text{Morphismen: } & \Hom(n,m) := K^{m \times n} \\
  \text{Verkettung: } & \text{gegeben durch Matrixmultiplikation}
\end{align*}
\end{bsp}

\begin{aufg}Zeige, dass folgende
Kategorien jeweils nicht zueinander äquivalent sind:
\begin{enumerate}
\item $\Grp$ und $\Set$
\item $\AbGrp$ und $\Grp$
\item Die Kategorie der Ringe und die der Körper
\end{enumerate}
In diesen Fällen würde man natürlich auch keine Äquivalenz erwarten. Es ist
aber ähnlich wie in Beispiel~\ref{setopset} interessant, welche kategoriellen
Eigenschaften die Unterscheidung möglich machen.
\end{aufg}


\subsubsection*{Charakterisierung von Kategorienäquivalenzen}

Funktoren, die Bestandteil einer Kategorienäquivalenz sind, kann man auch ohne
Bezug auf ihr Quasi-Inverses charakterisieren:
\begin{lemma}\label{equivVolltreuWesSurj} Sei $F : \C \to \D$ eine Äquivalenz von
Kategorien mit Quasi-Inversem $G : \D \to \C$. Dann sind $F$ und $G$ volltreu und
wesentlich surjektiv. Die Umkehrung gilt auch, falls der umgebende logische
Rahmen erlaubt, zu jedem Objekt aus~$\C$ bzw.~$\D$ ein Urbild (bis auf
Isomorphie) in~$\D$ bzw.~$\C$ zu wählen.
\end{lemma}
\begin{proof}Wir beweisen nur die Hinrichtung.
Seien $\eta:G \circ F \Rightarrow \Id_{\C}$ und $\mu:F \circ G \Rightarrow
\Id_{\D}$ die natürlichen Isomorphismen der Kategorienäquivalenz. Dann ist
jedes Objekt $A \in \Ob \C$ isomorph zu $G(F(A))$ mit dem Isomorphismus
$\eta_{A}:G(F(A)) \to A$ und der Funktor $G$ damit wesentlich surjektiv.

Es kommutiert für alle $(A \xra{f} B) \in \C$ das erweiterte Natürlichkeitsdiagramm von~$\eta$:
\[ \xymatrixcolsep{4pc}\xymatrixrowsep{4pc}\xymatrix{
GFA \ar@<1ex>[d]^{\eta_{A}} \ar[r]^{G(F(f))} & GFB \ar@<1ex>[d]^{\eta_{B}} \\
A \ar@<1ex>[u]^{\eta_{A}^{-1}} \ar[r]_{f} & B \ar@<1ex>[u]^{\eta_{B}^{-1}}
} \]

Hieraus kann man direkt ablesen, dass $(G \circ F):\Hom(A, B) \to \Hom(GFA, GFB)$ eine Bijektion mit Umkehrabbildung
  \[ g : \Hom(GFA, GFB) \to \Hom(A, B),\ m \mapsto \eta_{B} \circ m \circ \eta_{A}^{-1} \]
ist. Insbesondere ist für alle $A, B \in \Ob \C$ die Abbildung
\[ F:\Hom(A, B) \to \Hom(FA, FB) \]
injektiv und
\[ G : \Hom(FA, FB) \to \Hom(GFA, GFB) \]
surjektiv. Wir können sogar zeigen, dass $G$ surjektiv auf $\Hom(E, P)$ für alle $E, P \in \Ob \D$ ist. Sei dazu $f \in \Hom(GE, GP)$ beliebig. Dann ist
\begin{align*}
f &= G(\mu_{P}) \circ \underbrace{(G(\mu_{P}^{-1}) \circ f \circ G(\mu_{E}))}_{\in\,\Hom(GFGE, GFGP)} \circ G(\mu_{E}^{-1})\\
&= G(\mu_{P}) \circ G(h) \circ G(\mu_{E}^{-1})\\
&= G(\underbrace{\mu_{P} \circ h \circ \mu_{E}^{-1}}_{\in\,Hom(E, P)})
\end{align*}
Um das passende Urbild~$h$ im zweiten Schritt zu erhalten, haben wir die schon
bewiesene Surjektivität der Abbildung
  \[ G:\Hom(FGE, FGP) \to \Hom(GFGE, GFGP) \]
ausgenutzt.

Da die Vorraussetzungen symmetrisch in $F$ und $G$ sind, ist $F$ auch surjektiv und $G$ injektiv auf Hom-Mengen und $F$ wesentlich surjektiv. Nach Definition sind $F$ und $G$ damit volltreu.
\end{proof}


% "konstante natürliche Trafo"
% weniger Abbildungen --> mehr Trafos!
% F ==> G ohne G ==> F
% eta_G Gruppenhomo? Andere Möglichkeiten?
% Bsp. für Äquivalenzen, die keine Isos sind
% Grund, wieso Isos schlecht (übel) sind

% Mottos:
% - quer über alle X gleichmäßig definiert
% - insbesondere: keine willkürlichen Wahlen, "basisfrei"
%   - genauer: Basiswahl okay, solange sie keine Rolle spielt
% - die Def. der eta_X verwenden nichts bestimmtes von X, sondern nur Dinge,
%   die für alle Objekte der Quellkategorie gleichermaßen verfügbar sind
%
% Theorems for free:
% - id
% - maybeToList
% - reverse
% - take
% - sort!
%
% Kegel

% Homotopietheoretische Sichtweise!
