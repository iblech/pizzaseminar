\section[Was sollen Kategorien?]{Was sollen Kategorien? \hfill \small Ingo
Blechschmidt}

\subsection{Beispiele für kategorielles Verständnis}

\subsubsection*{Beispiel: Produkte}

Von manchen Konstruktionen in verschiedenen Teilgebieten der Mathematik wird
man das Gefühl nicht los, dass sie einem gemeinsamen Ursprung entstammen: Etwa
kennt man\ldots
\begin{itemize}
  \item das kartesische Produkt von Mengen: $X \times Y$,
  \item das kartesische Produkt von Vektorräumen: $V \times W$,
  \item das kartesische Produkt von Gruppen: $G \times H$,
  \item das kartesische Produkt von Garben: $\mathcal{F} \times \mathcal{G}$,
  \item das kartesische Produkt von Vektorbündeln: $\mathcal{E} \times \mathcal{F}$,
  \item das Minimum von Zahlen: $\min\{n,m\}$,
  \item den größten gemeinsamen Teiler von Zahlen: $\ggT(n,m)$,
  \item den Paartyp in Programmiersprachen: \texttt{(a,b)},
  \item den Produktautomat zweier endlicher Automaten,
  \item den Mutterknoten zweier Knoten in einem Graph.
\end{itemize}
Die Ähnlichkeit untereinander ist mal mehr, mal weniger deutlich. Nur mit
Kategorientheorie versteht man: All dies sind Spezialfälle des allgemeinen
\emph{kategoriellen Produkts}. Ferner erfüllen all diese Konstruktionen sehr
ähnliche Gesetze, etwa gilt
\begin{align*}
  X \times (Y \times Z) &\cong (X \times Y) \times Z, \\
  U \times (V \times W) &\cong (U \times V) \times W, \\
  \min\{m,\min\{n,p\}\} &= \min\{\min\{m,n\},p\}, \\
  \ggT(m,\ggT(n,p)) &= \ggT(\ggT(m,n),p),
\end{align*}
wobei in der ersten Zeile~$X$, $Y$ und~$Z$ Mengen sein und das
Isomorphiezeichen für Gleichmächtigkeit stehen soll und in der zweiten
Zeile~$U$, $V$ und~$W$ Vektorräume sein und das Isomorphiezeichen für
Vektorraumisomorphie stehen soll. Mit Kategorientheorie versteht man:
All dies sind Spezialfälle der allgemeinen Assoziativität des kategoriellen
Produkts.

\subsubsection*{Beispiel: Isomorphie}

Ferner fällt auf, dass in vielen Teilgebieten der Mathematik jeweils ein
speziell zugeschnittener Isomorphiebegriff vorkommt: Etwa können\ldots
\begin{itemize}
  \item zwei Mengen $X,Y$ \tabto{4.63cm} gleichmächtig sein,
  \item zwei Vektorräume $V,W$ \tabto{4.63cm} isomorph sein,
  \item zwei Gruppen $G,H$ \tabto{4.63cm} isomorph sein,
  \item zwei top. Räume $X,Y$ \tabto{4.63cm} homöomorph sein,
  \item zwei Zahlen $n,m$ \tabto{4.63cm} gleich sein,
  \item zwei endliche Automaten \tabto{4.63cm} isomorph sein,
  \item zwei Typen \texttt{a}, \texttt{b} \tabto{4.63cm} sich verlustfrei ineinander umwandeln lassen.
\end{itemize}
All dies sind Spezialfälle des allgemeinen \emph{kategoriellen
Isomorphiekonzepts}.

\subsubsection*{Beispiel: Dualität}

Von folgenden Konzepten hat man im Gefühl, dass sie in einem gewissen Sinn
\emph{zueinander dual} sein sollten:
\begin{center}
  \setlength{\extrarowheight}{0.3em}
  \begin{tabular}{r|l}
    $f \circ g$ & $g \circ f$ \\
    $\leq$ & $\geq$ \\
    injektiv & surjektiv \\
    $\{\star\}$ & $\emptyset$ \\
    $\times$ & $\amalg$ \\
    ggT & kgV \\
    $\cap$ & $\cup$ \\
    Teilmenge & Faktormenge
  \end{tabular}
\end{center}
Mit Kategorientheorie versteht man: All dies sind Spezialfälle eines allgemeinen
\emph{kategoriellen Dualitätsprinzips} -- und diese Erkenntnis kann man nutzen,
um Ergebnisse für jeweils eines der Konzepte auf sein duales Gegenstück zu
übertragen.

\subsection{Grundlagen}

\begin{defn}
Eine \emph{Kategorie}~$\C$ besteht aus
\begin{enumerate}
  \item einer Klasse von \emph{Objekten} $\Ob \C$,
  \item zu je zwei Objekten $X,Y \in \Ob \C$ einer Klasse $\Hom_\C(X,Y)$ von
  \emph{Morphismen} zwischen ihnen und
  \item einer Kompositionsvorschrift:
  \begin{align*}
    \text{zu }\ & f \in \Hom_\C(X,Y) &
    \text{zu }\ & f : X \to Y \\
    \text{und }\ & g\in\Hom_\C(Y,Z) &
    \text{und }\ & g : Y \to Z \\
    \text{habe }\ & g\circ f\in\Hom_\C(X,Z), &
    \text{habe }\ & g\circ f : X \to Z,
  \end{align*}
\end{enumerate}
sodass
\begin{enumerate}
  \item die Komposition $\circ$ assoziativ ist und
  \item es zu jedem $X \in \Ob\C$ einen \emph{Identitätsmorphismus} $\id_X
  \in \Hom_\C(X,X)$ mit
  \[ f \circ \id_X = f, \quad \id_X \circ g = g \]
  für alle Morphismen $f,g$ gibt.
\end{enumerate}
\end{defn}

Die Morphismen von Kategorien müssen nicht unbedingt Abbildungen
sein, die Schreibweise "`$f:X \to Y$"' missbraucht also Notation. Die genaue
Bedeutung von \emph{Klassen} im Gegensatz zu \emph{Mengen} hängt von der
persönlich gewählten logischen Fundierung der Mathematik ab. Für uns genügt
folgende naive Sichtweise: Klassen können (im Gegensatz zu Mengen) beliebige
mathematische Objekte enthalten, sind aber selbst nicht mathematische Objekte.
Daher gibt es etwa widerspruchsfrei die Klasse aller Mengen, von einer Klasse
aller Klassen kann man aber nicht sprechen.

\begin{bsp}\begin{enumerate}
  \item Archetypisches Beispiel ist $\Set$, die Kategorie der Mengen und Abbildungen:
  \begin{align*}
    \Ob \Set &:= \{ M \,|\, \text{$M$ ist eine Menge} \} \\
    \Hom_\Set(X,Y) &:= \{ f:X \to Y \,|\, \text{$f$ ist eine Abbildung} \}
  \end{align*}
  \item Die meisten Teilgebiete der Mathematik studieren jeweils eine bestimmte
  Kategorie: Gruppentheoretiker beschäftigen sich etwa mit der Kategorie
  $\Grp$ der Gruppen und Gruppenhomomorphismen:
  \begin{align*}
    \Ob \Grp &:= \text{Klasse aller Gruppen} \\
    \Hom_\Grp(G,H) &:= \{ f:G \to H \,|\, \text{$f$ ist ein Gruppenhomo} \}
  \end{align*}
  \item Es gibt aber auch wesentlich kleinere Kategorien. Etwa kann man aus
  jeder Quasiordnung~$(P,\preceq)$ eine Kategorie~$\C$ basteln:
  \begin{align*}
    \Ob \C &:= P \\
    \Hom_\C(x,y) &:= \begin{cases}
      \text{einelementige Menge}, & \text{falls $x \preceq y$,} \\
      \text{leere Menge}, & \text{sonst}
    \end{cases}
  \end{align*}
  \item Auch sind gewisse endliche Kategorien bedeutsam, etwa die durch
  folgende Skizze gegebene:

  \[ \xymatrix{
    & \bullet \ar[d] \ar@(ur,ul) \\
    \bullet \ar[r] \ar@(ul,dl) & \bullet \ar@(dr,ur)
  } \]
\end{enumerate}\end{bsp}

\begin{motto}[fundamental]Kategorientheorie stellt \emph{Beziehungen zwischen
Objekten} statt etwaiger innerer Struktur in den Vordergrund.\end{motto}

\begin{defn}Eine Kategorie~$\C$ heißt \emph{lokal klein}, wenn ihre Hom-Klassen
jeweils schon Mengen (statt echte Klassen) sind. Eine Kategorie~$\C$ heißt
\emph{klein}, wenn zusätzlich auch ihre Klasse von Objekten schon eine Menge
bildet.\end{defn}


\subsubsection*{Initiale und terminale Objekte}

In Kategorien sind folgende zwei Arten von Objekten aufgrund ihrer
ausgezeichneten Beziehungen zu allen (anderen) Objekten besonders wichtig:
\begin{defn}
Ein Objekt~$X$ einer Kategorie~$\C$ heißt genau dann
\begin{itemize}
  \item \emph{initial}, wenn
    \[ \forall Y \in \Ob \C{:}\ \exists! f : X \to Y. \]
  \item \emph{terminal}, wenn
    \[ \forall Y \in \Ob \C{:}\ \exists! f : Y \to X. \]
\end{itemize}
\end{defn}
Diese Definitionen geben ein erstes Beispiel für sog. \emph{universellen
Eigenschaften}.

\begin{bsp}\begin{enumerate}
\item In der Kategorie der Mengen ist genau die leere Menge initial und
genau jede einelementige Menge terminal. Diese Erkenntnis ist ein erstes
Beispiel dafür, wie das fundamentale Motto gemeint ist: Eine Definition der
leeren bzw. einer einelementigen Menge über eine Aufzählung ihrer Elemente
betont ihre innere Struktur, während eine Definition als initiales bzw.
terminales Objekt die besonderen Beziehungen zu allen Mengen hervorhebt.
\item In der Kategorie der $K$-Vektorräume ist der Nullvektorraum $K^0$ initial
und terminal.
\end{enumerate}\end{bsp}
Viele kategorielle Konstruktionen realisiert man als initiales oder
terminales Objekt in einer geeigneten Kategorie von Möchtegern-Konstruktionen.
Ein erstes Beispiel dazu werden wir im folgenden Kapitel über Produkte finden.


\subsubsection*{Mono-, Epi- und Isomorphismen}

\begin{defn}
Ein Morphismus $f:X \to Y$ einer Kategorie~$\C$ heißt genau dann
\begin{itemize}
  \item \emph{Monomorphismus}, \tabto{3.35cm}wenn für alle Objekte~$A \in \Ob \C$
  und $p,q:A \to X$ gilt:
  \[ f \circ p = f \circ q \quad\Longrightarrow\quad p = q. \]
  \item \emph{Epimorphismus}, \tabto{3.35cm}wenn für alle Objekte~$A \in \Ob \C$
  und $p,q:Y \to A$ gilt:
  \[ p \circ f = q \circ f \quad\Longrightarrow\quad p = q. \]
\end{itemize}
\end{defn}

\begin{bsp}\begin{enumerate}
\item In den Kategorien der Mengen, Gruppen und $K$-Vektorräumen sind die
Monomorphismen genau die injektiven und die Epimorphismen genau die
surjektiven Abbildungen. Das ist jeweils eine interessante Erkenntnis über die
Struktur dieser Kategorien und nicht ganz leicht zu zeigen.
\item In der Kategorie der metrischen Räume mit stetigen Abbildungen gibt es
Epimorphismen, die nicht surjektiv sind: nämlich alle stetigen Abbildungen mit
dichtem Bild.
\end{enumerate}\end{bsp}

\begin{defn}
Ein \emph{Isomorphismus} $f:X \to Y$ in einer Kategorie ist ein
Morphismus, zu dem es einen Morphismus $g:Y \to X$ mit
\[ g \circ f = \id_X, \quad f \circ g = \id_Y \]
gibt. Statt "`$g$"' schreibt man auch "`$f^{-1}$"'. Existiert zwischen
Objekten~$X$ und~$Y$ ein Isomorphismus, so heißen die Objekte \emph{zueinander
isomorph}:
$X \cong Y$.
\end{defn}

\begin{bem}\label{gleichheitobj}In den meisten Kategorien ist die Frage, ob
Objekte~$X,Y$ tatsächlich gleich (statt nur isomorph) sind, keine interessante
Frage: Denn für alle praktischen Belange sind schon zueinander isomorphe
Objekte "`gleich gut"'. Diesen Gedanken werden wir noch manche Male aufgreifen
und weiter entwickeln.\end{bem}

\begin{aufg}Zeige, dass in jeder Kategorie~$\C$ die Klasse der Isomorphismen die
\emph{2-aus-3-Eigenschaft} hat: Sind~$f:X \to Y$ und~$g:Y \to Z$ komponierbare
Morphismen in~$\C$ und sind zwei der drei Morphismen~$f$, $g$ und~$g \circ f$
Isomorphismen, so auch der dritte.\end{aufg}


\subsubsection*{Die duale Kategorie}

Aus jeder Kategorie~$\C$ kann man durch "`Umdrehen aller Pfeile"' eine weitere
Kategorie erhalten, die sogenannte duale Kategorie von~$\C$:

\begin{defn}
Die zu einer Kategorie~$\C$ zugehörige \emph{duale Kategorie} $\C^\op$ ist
folgende:
\begin{align*}
  \Ob \C^\op &:= \Ob \C \\
  \Hom_{\C^\op}(X,Y) &:= \Hom_\C(Y,X)
\end{align*}
\end{defn}

Das ist ein rein formaler Prozess, der mit dem Invertieren bijektiver
Abbildungen nichts zu tun hat. Die duale Kategorie ist nützlich, um sich der
Dualität mancher kategorieller Konzepte gewahr zu werden:

\begin{bsp}\begin{enumerate}
\item Ein initiales Objekt in~$\C^\op$ ist ein terminales Objekt in~$\C$ und
umgekehrt.
\item Ein Epimorphismus in~$\C^\op$ ist ein Monomorphismus in~$\C$ und
umgekehrt.
\item Zwei Objekte sind genau dann in~$\C^\op$ zueinander isomorph, wenn sie es
in~$\C$ sind. Isomorphie ist also ein \emph{selbstduales} Konzept.
\end{enumerate}\end{bsp}

Spannend ist es, wenn duale Kategorien durch andere, natürlich
auftretende Kategorien beschrieben werden können.


% XXX: verallgemeinertes "the"
