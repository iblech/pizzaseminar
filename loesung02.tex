\documentclass[a4paper,ngerman]{scrartcl}

%\usepackage{ucs}
\usepackage[utf8]{inputenc}

\usepackage[ngerman]{babel}

\usepackage{amsmath,amsthm,amssymb,amscd,color,graphicx}

%\usepackage[small,nohug]{diagrams}
%\diagramstyle[labelstyle=\scriptstyle]

\usepackage[protrusion=true,expansion=true]{microtype}

\usepackage{lmodern}
\usepackage{tabto}

\usepackage[natbib=true,style=numeric]{biblatex}
\usepackage[babel]{csquotes}
\bibliography{lit}

\usepackage[all]{xy}

%\usepackage{hyperref}

\setlength\parskip{\medskipamount}
\setlength\parindent{0pt}

\theoremstyle{definition}
\newtheorem{defn}{Definition}
\newtheorem{bsp}[defn]{Beispiel}

\theoremstyle{plain}

\newtheorem{prop}[defn]{Proposition}
\newtheorem{ueberlegung}[defn]{Überlegung}
\newtheorem{lemma}[defn]{Lemma}
\newtheorem{kor}[defn]{Korollar}
\newtheorem{hilfsaussage}[defn]{Hilfsaussage}
\newtheorem{satz}[defn]{Satz}

\theoremstyle{remark}
\newtheorem{bem}[defn]{Bemerkung}

\clubpenalty=10000
\widowpenalty=10000
\displaywidowpenalty=10000

\newcommand{\lra}{\longrightarrow}
\newcommand{\lhra}{\ensuremath{\lhook\joinrel\relbar\joinrel\rightarrow}}
\newcommand{\thlra}{\relbar\joinrel\twoheadrightarrow}

\newcommand{\A}{\mathcal{A}}
\newcommand{\Z}{\mathbb{Z}}
\newcommand{\Q}{\mathbb{Q}}
\newcommand{\R}{\mathbb{R}}
\newcommand{\C}{\mathcal{C}}
\newcommand{\RP}{\mathbb{R}\mathrm{P}}
\newcommand{\Hom}{\mathrm{Hom}}
\newcommand{\Set}{\mathrm{Set}}
\newcommand{\Spur}[1]{\operatorname{Spur}#1}
\newcommand{\rank}[1]{\operatorname{rank}#1}
\newcommand{\sgn}[1]{\operatorname{sgn}#1}
\newcommand{\id}{\mathrm{id}}
\newcommand{\Aut}[1]{\operatorname{Aut}(#1)}
\newcommand{\GL}[1]{\operatorname{GL}(#1)}
\newcommand{\ORTH}[1]{\operatorname{O}(#1)}
\newcommand{\freist}{\underline{\ \ }}
\newcommand{\op}{\mathrm{op}}
\DeclareMathOperator{\rk}{rk}
\DeclareMathOperator{\Spec}{Spec}
\DeclareMathOperator{\Bild}{im}
\DeclareMathOperator{\Kern}{ker}
\DeclareMathOperator{\Int}{int}
\DeclareMathOperator{\Ob}{Ob}
\newcommand{\Zzwei}{\Z_2}

\newcommand{\XXX}[1]{\textcolor{red}{#1}}

\newenvironment{indentblock}{%
  \list{}{\leftmargin\leftmargin}%
  \item\relax
}{%
  \endlist
}

\renewcommand*\theenumi{\alph{enumi}}
\renewcommand{\labelenumi}{\theenumi)}

\usepackage{enumerate}

%\newarrow{Equals}=====

\usepackage{geometry}
\geometry{tmargin=3cm,bmargin=3cm,lmargin=3cm,rmargin=3cm}

\begin{document}

\vspace*{-4em}
\begin{flushright}Universität Augsburg \\ 19. März 2013 \\ Matthias Hutzler\end{flushright}

\begin{center}\Large \textbf{Pizzaseminar zur Kategorientheorie} \\
Lösung zum 2. Übungsblatt
\end{center}
\vspace{2em}

\newbox{\mybox}
\setbox\mybox=\hbox{\textbf{Aufgabe 1:}}

%\begin{list}{}{\labelwidth\wd\mybox \leftmargin\wd\mybox \itemsep 1.3em}
\begin{list}{}{\labelwidth0em \leftmargin0em \itemindent0.5em \itemsep 1.3em}
\item[\textbf{Aufgabe 1:}]\mbox{}
\begin{enumerate}
\item 
Zu zeigen: Es gibt Morphismen $X\to X$ und $X\to 1$, mit denen $X$ ein Produkt von $X$ und $1$ ist.

Wähle als Morphismen $\id_X:X\to X$ und den (da $1$ terminales Objekt ist) eindeutig bestimmten Morphismus~$!:X\to 1$. Nun muss für jedes Möchtegern-Produkt~$P$ mit den Morphismen $f:P\to X$ und $g:P\to 1
$ genau ein Morphismus~$h:P\to X$ existieren, sodass folgendes Diagramm kommutiert:
\begin{center}
$\xymatrix{
 & X \ar[dl]_{\id_X} \ar[dr]^{\mathrm{!}} &\\
X & & 1\\
 & P \ar[ul]^f \ar[ur]_g \ar@{-->}[uu]^h &
}$
\end{center}

Existenz von~$h$:\\
Setze $h:=f$. Es gilt also $\id_X\circ h=h=f$. (Das linke Dreieck kommutiert.) Und da $!\circ h$ und $
g$ zwei Morphismen~$P\to 1$ sind (und $1$ terminal ist), gilt auch $!\circ h=g$. (Das rechte Dreieck kommutiert.)

Eindeutigkeit von~$h$:\\
Für jedes~$h:P\to X$, das das Diagramm kommutieren lässt, gilt: $\id_X\circ h=f$. Es folgt also sofort
 $h=f$.


\item
Die duale Aussage lautet:\\
Besitzt $\C$ ein initiales Objekt~$0$, so gilt
\[X \amalg 0 \cong X.\]
($X$ kann mit den Morphismen $\id_X:X\to X$ und $!:0\to X$ als Koprodukt von $X$ und $0$ dienen.)
\end{enumerate}

\item[\textbf{Aufgabe 2:}]\mbox{}
\begin{enumerate}
\item
Seien $T_1$ und $T_2$ zwei terminale Objekte einer Kategorie~$\C$.\\
Zu zeigen: Es gibt genau einen Isomorphismus~$T_1\to T_2$.

Sei $f$ der eindeutig bestimmte Morphismus~$T_1\to T_2$ und $g$ der eindeutig bestimmte Morphismus~$T_2\to T_1$:
\[ \xymatrix{
T_1\ar@<0.2em>[r]^f & T_2\ar@<0.2em>[l]^g
} \]
Sowohl $g\circ f$ als auch $\id_{T_1}$ sind Morphismen~$T_1\to T_1$, also gilt $g\circ f=\id_{T_1}$ (denn $T_1$ ist terminal). Analog gilt $f\circ g=\id_{T_2}$ (da $T_2$ terminal ist). Wir dürfen also schreiben $g=f^{-1}$ und $f$ ist ein Isomorphismus.

Die Eindeutigkeit des Isomorphismus $f$ folgt direkt daraus, dass $T_2$ ein terminales Objekt ist.

\item
Die Definition eines terminalen Objektes lautet angewandt auf die Kategorie der Möchtegern-Produkte von $X$ und $Y$:
\begin{indentblock}\emph{%
  Ein terminales Objekt ist ein Diagramm der Form \mbox{$X\leftarrow R\to Y$}, sodass für jedes Diagramm der Form \mbox{$X\leftarrow Q\to Y$} genau ein Morphismus \mbox{$(X\leftarrow Q\to Y)$} $\to$ \mbox{$(X\leftarrow R\to Y)$} existiert.
}\end{indentblock}

Ein Morphismus \mbox{$(X\leftarrow Q\to Y)$} $\to$ \mbox{$(X\leftarrow R\to Y)$} ist dabei ein kommutatives Diagramm folgender Form (wobei $h:Q\to R$ beliebig ist):
\[ \xymatrix{
 & Q \ar[ld]\ar[rd]\ar@{-->}[dd]_h & \\
X & & Y \\
 & R \ar[lu]\ar[ru] &
} \]
Da in diesem Diagramm aber bereits alle Objekte und Morphismen außer $h$ vorgegeben sind, lassen sich die Morphismen \mbox{$(X\leftarrow Q\to Y)$} $\to$ \mbox{$(X\leftarrow R\to Y)$} (in der Kategorie der Möchtegern-Produkte von $X$ und $Y$) mit denjenigen Morphismen $h:Q\to R$ (in der Kategorie $\C$) identifizieren, die das obige Diagramm kommutieren lassen.

Eine weitere äquivalente Definition (und zwar mit den Begriffen der Kategorie $\C$) eines terminalen Objektes der Kategorie der Möchtegern-Produkte von $X$ und $Y$ lautet also:
\begin{indentblock}\emph{%
  Ein Objekt $R$ zusammen mit zwei Morphismen $R\to X$ und $R\to Y$, sodass für jedes Objekt $Q$ zusammen mit zwei Morphismen $Q\to X$ und $Q\to Y$ genau ein Morphismus $h:Q\to R$ existiert, der obiges Diagramm kommutieren lässt.
}\end{indentblock}
(Das entspricht genau der Definition eines Produktes von $X$ und $Y$ in $\C$.)

Dies zeigt in Kombination mit Teilaufgabe a), dass das Produkt von zwei Objekten einer Kategorie eindeutig bis auf \emph{eindeutige} Isomorphie ist, wenn man Verträglichkeit mit den Projektionsmorphismen fordert.
\end{enumerate}


\item[\textbf{Aufgabe 3:}]\mbox{}
\begin{enumerate}
\item
Die Objekte der Kategorie~$\C$ seien gerade die Elemente von $X$:
\footnotetext[1]{Ohne eine Fallunterscheidung kann man die Definition auch etwas kryptisch einfach als
$\Hom(a,b) := \{ \text{"`$a \preceq b$"'} \,|\, a \preceq b \}$ formulieren. Dann funktioniert die Definition auch in einem konstruktiven Hintergrund, bei dem man ohne Zusatzforderung an~$X$ nicht weiß, dass $a \preceq b$ gilt oder nicht gilt.}
\begin{align*}
\Ob \C &:= X.
\intertext{%
  Von einem Objekt~$a$ zu einem Objekt~$b$ soll es genau dann genau einen Morphismus geben, wenn $a\preceq b$ gilt. (Diesen Morphismus nennen wir dann "`$a\preceq b$"'.) Ansonsten soll es keinen Morphismus~$a\to b$ geben:\footnotemark
}
\Hom(a,b) &:=
\begin{cases}
\{\text{"`$a\preceq b$"'}\} & \text{falls } a\preceq b, \\
\emptyset & \text{ansonsten.}
\end{cases}
\end{align*}
Da nun zwischen zwei Objekten (in einer Richtung) immer höchstens ein Morphismus existiert, gibt es für die Definition der Verknüpfungsvorschrift nur eine Möglichkeit: Wir definieren für beliebige Objekte $a,b,c\in \Ob \C$ und Morphismen $\text{"`$a\preceq b$"'}:a\to b$ und $\text{"`$b\preceq c$"'}:b\to c$:
\begin{align*}
\text{"`$b\preceq c$"'}\circ \text{"`$a\preceq b$"'} := \text{"`$a\preceq c$"'}
\end{align*}
(Dass dieser Morphismus existiert, folgt direkt aus der Transitivität der Relation~$\preceq$.)

Zu jedem Objekt $a$ lässt sich ein Identitätsmorphismus finden, nämlich "`$a\preceq a$"', denn $\preceq$ ist auch reflexiv. Dass jeder Morphismus bei Verknüpfung mit "`$a\preceq a$"' unverändert bleibt, ist klar, da hier Morphismen durch Quelle und Ziel bereits eindeutig bestimmt sind. Aus demselben Grund ist die Verknüpfung von Morphismen assoziativ. Damit sind alle Kategorienaxiome erfüllt.

Ein kleiner Ausschitt der Kategorie, die auf diese Weise aus der Menge der ganzen Zahlen mit der Teilbarkeitsrelation entsteht, sieht wie folgt aus (Identitätsmorphismen weggelassen):
\[ \xymatrix@=8ex{
 & & 1\ar@<0.25ex>[r] \ar[dddll]\ar[dddl] \ar[dd]\ar[ddr] \ar[ddrrr]\ar[ddrrrr] \ar[dddrr]\ar[dddrrr] & -1\ar@<0.25ex>[l] \ar[dddlll]\ar[dddll] \ar[ddl]\ar[dd] \ar[ddrr]\ar[ddrrr] \ar[dddr]\ar[dddrr] \\ \\
 & & 2\ar@<0.25ex>[r] \ar[dll]\ar[dl] \ar[drr]\ar[drrr] & -2\ar@<0.25ex>[l] \ar[dlll]\ar[dll] \ar[dr]\ar[drr] & & 3\ar@<0.25ex>[r] \ar[dl]\ar[d] & -3\ar@<0.25ex>[l] \ar[dll]\ar[dl]\\
4\ar@<0.25ex>[r] & -4\ar@<0.25ex>[l] & & & 6\ar@<0.25ex>[r] & -6\ar@<0.25ex>[l]
} \]


\item
Zwei Objekte $a$, $b$ einer solchen Kategorie sind genau dann isomorph, wenn sowohl $a\preceq b$ als auch $b\preceq a$ gilt:
\begin{align*}
a\cong b \quad \Longleftrightarrow \quad a\preceq b \ \wedge\ b \preceq a
\end{align*}
\begin{itemize}
\item["`$\Rightarrow$"':] Wenn $a$ und $b$ isomorph sind, muss es Morphismen $a\to b$ und $b\to a$ geben. Also gilt $a\preceq b$ und $b\preceq a$.
\item["`$\Leftarrow$"':] Es existieren die Morphismen "`$a\preceq b$"' und "`$b\preceq a$"'. Diese sind offensichtlich Isomorphismen, denn: $\text{"`$b\preceq a$"'}\circ\text{"`$a\preceq b$"'}$ ist ein Morphismus $a\to a$, also identisch mit $\id_a$ und $\text{"`$a\preceq b$"'}\circ\text{"`$b\preceq a$"'}$ ist ein Morphismus $b\to b$, also identisch mit $\id_b$.
\end{itemize}


\item
Definition:
\begin{align*}
p\in X\text{ Infimum von }a,b\in X :\Longleftrightarrow\\
\forall x\in X{:}\quad x\preceq a \wedge x\preceq b \Longleftrightarrow x\preceq p
\end{align*}
Zu zeigen:
\begin{center}
$p$ ist Infimum von $a$ und $b$. $\Longleftrightarrow$\\
Es gibt Morphismen, mit denen $p$ ein Produkt $a\times b$ in $\C$ ist.
\end{center}
\begin{itemize}
\item["`$\Rightarrow$"':] Da $\preceq$ reflexiv ist, also $p\preceq p$ gilt, folgt aus der Definition des Infimums (von rechts nach links), dass die Morphismen "`$p\preceq a$"' und "`$p\preceq b$"' existieren. Mit diesen Morphismen ist $p$ tatsächlich ein Produkt von $a$ und $b$, denn:\\
Sei $q\in X$ mit "`$q\preceq a$"' und "`$q\preceq b$"' beliebiges Möchtegern-Produkt von $a$ und~$b$. Dann gilt nach Definition des Infimums (von links nach rechts) $q\preceq p$ und die Gleichungen $\text{"`$p\preceq a$"'}\circ\text{"`$q\preceq p$"'}=\text{"`$q\preceq a$"'}$ und $\text{"`$p\preceq b$"'}\circ\text{"`$q\preceq p$"'}=\text{"`$q\preceq b$"'}$ sind offensichtlich erfüllt.
\item["`$\Leftarrow$"':] Sei $x\in X$ beliebig.\\
Falls $x\preceq a$ und $x\preceq b$ gilt, so ist $x$ mit "`$x\preceq a$"' und "`$x\preceq b$"' Möchtegern-Produkt von $a$ und $b$, es existiert also der Morphismus "`$x\preceq p$"'.\\
Außerdem müssen die Morphismen "`$p\preceq a$"' und "`$p\preceq b$"' existieren (mit denen $p$ Produkt von $a$ und $b$ ist), sodass aus $x\preceq p$ auch $x\preceq a$ und $x\preceq b$ folgt.\\
Damit ist $p$ Infimum von $a$ und $b$.
\end{itemize}
\end{enumerate}
\end{list}

\end{document}
