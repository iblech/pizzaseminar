\documentclass[a4paper,ngerman]{scrartcl}

%\usepackage{ucs}
\usepackage[utf8]{inputenc}

\usepackage[ngerman]{babel}

\usepackage{amsmath,amsthm,amssymb,amscd,color,graphicx}

%\usepackage[small,nohug]{diagrams}
%\diagramstyle[labelstyle=\scriptstyle]

\usepackage[protrusion=true,expansion=true]{microtype}

\usepackage{lmodern}
\usepackage{tabto}

\usepackage[natbib=true,style=numeric]{biblatex}
\usepackage[babel]{csquotes}
\bibliography{lit}

\usepackage[all]{xy}

%\usepackage{hyperref}

\setlength\parskip{\medskipamount}
\setlength\parindent{0pt}

\theoremstyle{definition}
\newtheorem{defn}{Definition}
\newtheorem{bsp}[defn]{Beispiel}

\theoremstyle{plain}

\newtheorem{prop}[defn]{Proposition}
\newtheorem{ueberlegung}[defn]{Überlegung}
\newtheorem{lemma}[defn]{Lemma}
\newtheorem{kor}[defn]{Korollar}
\newtheorem{hilfsaussage}[defn]{Hilfsaussage}
\newtheorem{satz}[defn]{Satz}

\theoremstyle{remark}
\newtheorem{bem}[defn]{Bemerkung}

\clubpenalty=10000
\widowpenalty=10000
\displaywidowpenalty=10000

\newcommand{\lra}{\longrightarrow}
\newcommand{\lhra}{\ensuremath{\lhook\joinrel\relbar\joinrel\rightarrow}}
\newcommand{\thlra}{\relbar\joinrel\twoheadrightarrow}

\newcommand{\A}{\mathcal{A}}
\newcommand{\Z}{\mathbb{Z}}
\newcommand{\Q}{\mathbb{Q}}
\newcommand{\R}{\mathbb{R}}
\newcommand{\C}{\mathcal{C}}
\newcommand{\RP}{\mathbb{R}\mathrm{P}}
\newcommand{\Hom}{\mathrm{Hom}}
\newcommand{\Set}{\mathrm{Set}}
\newcommand{\Spur}[1]{\operatorname{Spur}#1}
\newcommand{\rank}[1]{\operatorname{rank}#1}
\newcommand{\sgn}[1]{\operatorname{sgn}#1}
\newcommand{\id}{\mathrm{id}}
\newcommand{\Aut}[1]{\operatorname{Aut}(#1)}
\newcommand{\GL}[1]{\operatorname{GL}(#1)}
\newcommand{\ORTH}[1]{\operatorname{O}(#1)}
\newcommand{\freist}{\underline{\ \ }}
\newcommand{\op}{\mathrm{op}}
\DeclareMathOperator{\rk}{rk}
\DeclareMathOperator{\Spec}{Spec}
\DeclareMathOperator{\Bild}{im}
\DeclareMathOperator{\Kern}{ker}
\DeclareMathOperator{\Int}{int}
\DeclareMathOperator{\Ob}{Ob}
\newcommand{\Zzwei}{\Z_2}

\newcommand{\XXX}[1]{\textcolor{red}{#1}}

\renewcommand*\theenumi{\alph{enumi}}
\renewcommand{\labelenumi}{\theenumi)}

\usepackage{enumerate}

%\newarrow{Equals}=====

\usepackage{geometry}
\geometry{tmargin=3cm,bmargin=3cm,lmargin=3cm,rmargin=3cm}

\begin{document}

\vspace*{-4em}
\begin{flushright}Universität Augsburg \\ 16. März 2013 \\ Matthias Hutzler\end{flushright}

\begin{center}\Large \textbf{Pizzaseminar zur Kategorientheorie} \\
Lösung zum 2. Übungsblatt
\end{center}
\vspace{2em}

\newbox{\mybox}
\setbox\mybox=\hbox{\textbf{Aufgabe 1:}}

%\begin{list}{}{\labelwidth\wd\mybox \leftmargin\wd\mybox \itemsep 1.3em}
\begin{list}{}{\labelwidth0em \leftmargin0em \itemindent0.5em \itemsep 1.3em}
\item[\textbf{Aufgabe 1:}]\mbox{}
\begin{enumerate}
\item 
Zu zeigen: Es gibt Morphismen $X\to X$ und $X\to 1$, mit denen $X$ ein Produkt von $X$ und $1$ ist.

Wähle als Morphismen $\id_X:X\to X$ und den (da $1$ terminales Objekt ist) eindeutig bestimmten Morphismus~$!:X\to 1$. Nun muss für jedes Möchtegern-Produkt~$P$ mit den Morphismen $f:P\to X$ und $g:P\to 1
$ genau ein Morphismus~$h:P\to X$ existieren, sodass folgendes Diagramm kommutiert:
\begin{center}
$\xymatrix{
 & X \ar[dl]_{\id_X} \ar[dr]^{\mathrm{!}} &\\
X & & 1\\
 & P \ar[ul]^f \ar[ur]_g \ar@{-->}[uu]^h &
}$
\end{center}

Existenz von~$h$:\\
Setze $h:=f$. Es gilt also $\id_X\circ h=h=f$. (Das linke Dreieck kommutiert.) Und da $!\circ h$ und $
g$ zwei Morphismen~$P\to 1$ sind (und $1$ terminal ist), gilt auch $!\circ h=g$. (Das rechte Dreieck kommutiert.)

Eindeutigkeit von~$h$:\\
Für jedes~$h:P\to X$, das das Diagramm kommutieren lässt, gilt: $\id_X\circ h=f$. Es folgt also sofort
 $h=f$.


\item
Die duale Aussage lautet:\\
Besitzt $C$ ein initiales Objekt~$0$, so gilt
\[X \amalg 0 \cong X.\] % Bin nicht sicher, ob hier \amalg als Symbol für des Koprodukt nicht besser wäre
($X$ kann mit den Morphismen $\id_X:X\to X$ und $!:0\to X$ als Koprodukt von $X$ und $0$ dienen.)
\end{enumerate}
\end{list}

\end{document}
