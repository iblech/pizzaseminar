\documentclass[a4paper,ngerman]{scrartcl}

%\usepackage{ucs}
\usepackage[utf8]{inputenc}

\usepackage[ngerman]{babel}

\usepackage{amsmath,amsthm,amssymb,amscd,color,graphicx}

%\usepackage[small,nohug]{diagrams}
%\diagramstyle[labelstyle=\scriptstyle]

\usepackage[protrusion=true,expansion=true]{microtype}

\usepackage{lmodern}
\usepackage{tabto}

\usepackage[natbib=true,style=numeric]{biblatex}
\usepackage[babel]{csquotes}
\bibliography{lit}

\usepackage[all]{xy}

%\usepackage{hyperref}

\setlength\parskip{\medskipamount}
\setlength\parindent{0pt}

\theoremstyle{definition}
\newtheorem{defn}{Definition}
\newtheorem{bsp}[defn]{Beispiel}

\theoremstyle{plain}

\newtheorem{prop}[defn]{Proposition}
\newtheorem{ueberlegung}[defn]{Überlegung}
\newtheorem{lemma}[defn]{Lemma}
\newtheorem{kor}[defn]{Korollar}
\newtheorem{hilfsaussage}[defn]{Hilfsaussage}
\newtheorem{satz}[defn]{Satz}

\theoremstyle{remark}
\newtheorem{bem}[defn]{Bemerkung}

\clubpenalty=10000
\widowpenalty=10000
\displaywidowpenalty=10000

\newcommand{\lra}{\longrightarrow}
\newcommand{\lhra}{\ensuremath{\lhook\joinrel\relbar\joinrel\rightarrow}}
\newcommand{\thlra}{\relbar\joinrel\twoheadrightarrow}

\newcommand{\A}{\mathcal{A}}
\newcommand{\Z}{\mathbb{Z}}
\newcommand{\Q}{\mathbb{Q}}
\newcommand{\R}{\mathbb{R}}
\newcommand{\C}{\mathcal{C}}
\newcommand{\RP}{\mathbb{R}\mathrm{P}}
\newcommand{\Hom}{\mathrm{Hom}}
\newcommand{\Set}{\mathrm{Set}}
\newcommand{\Spur}[1]{\operatorname{Spur}#1}
\newcommand{\rank}[1]{\operatorname{rank}#1}
\newcommand{\sgn}[1]{\operatorname{sgn}#1}
\newcommand{\id}{\mathrm{id}}
\newcommand{\Aut}[1]{\operatorname{Aut}(#1)}
\newcommand{\GL}[1]{\operatorname{GL}(#1)}
\newcommand{\ORTH}[1]{\operatorname{O}(#1)}
\newcommand{\freist}{\underline{\ \ }}
\newcommand{\op}{\mathrm{op}}
\DeclareMathOperator{\rk}{rk}
\DeclareMathOperator{\Spec}{Spec}
\DeclareMathOperator{\Bild}{im}
\DeclareMathOperator{\Kern}{ker}
\DeclareMathOperator{\Int}{int}
\DeclareMathOperator{\Ob}{Ob}
\newcommand{\Zzwei}{\Z_2}

\newcommand{\XXX}[1]{\textcolor{red}{#1}}

\renewcommand*\theenumi{\alph{enumi}}
\renewcommand{\labelenumi}{\theenumi)}

\pagestyle{empty}

%\newarrow{Equals}=====

\usepackage{geometry}
\geometry{tmargin=2cm,bmargin=3cm,lmargin=3cm,rmargin=3cm}

\begin{document}

\vspace*{-4em}
\begin{flushright}Universität Augsburg \\ 19. Februar 2014\end{flushright}

\begin{center}\Large \textbf{Pizzaseminar zu erzeugenden Funktionen} \\
1. Übungsblatt
\end{center}
\vspace{2em}

\newbox{\mybox}
\setbox\mybox=\hbox{\textbf{Aufgabe 1:}}

\begin{list}{}{\labelwidth\wd\mybox \leftmargin\wd\mybox \itemsep 1.3em}
\item[\textbf{Aufgabe 0:}] Spiele mit erzeugenden Funktionen auf dem Rechner! Erstelle dir zum Beispiel auf \url{sage.math.uni-augsburg.de} einen Account und probiere dann den folgenden Code aus:
\begin{verbatim}
X.<x> = PowerSeriesRing(QQ)
print 1/(1-x-x^2)
print  
XY.<y> = PowerSeriesRing(X)
print 1/(1-y-x*y)
\end{verbatim} 

\item[\textbf{Aufgabe 1:}] Sei $f_n$ die Anzahl der Teilmengen der Menge $\{1,\ldots,n\}$, welche keine zwei benachbarten Zahlen enthalten. Bestimme eine Rekursionsgleichung für die Folge $(f_n)$ und die (gewöhnliche) erzeugende Funktion.

Verfeinere anschließend die Folge zur Doppelfolge durch Einführung eines neuen Parameters $k$, der die Mächtigkeit zählt: $b_{n,\,k}$ sei die Anzahl der $k$-Teilmengen der Menge $\{1,\ldots,n\}$, welche keine zwei benachbarten Zahlen enthalten. Suche auch hier Rekursionsgleichung und (doppelt gewöhnliche) erzeugende Funktion.
\item[\textbf{Aufgabe 2:}] Sei $(f_n)_{n\geq 0}$ eine Folge, die durch die Rekursionsgleichung 
\begin{align*}
f_n &= a_1f_{n-1} + a_2f_{n-2} + a_3f_{n-3} + c, & n &\geq 3 \\
f_n &= b_n , & n &= 0,1,2
\end{align*}
mit festen $a_1, a_2, a_3, b_0, b_1, b_2, c$ gegeben ist. Bestimme die gewöhnliche und exponentiell erzeugende Funktion der Folge.

\item[\textbf{Aufgabe 3:}] Sei $s_{n,\,p} := \sum_{k=0}^n k^p$ die Verallgemeinerung des kleinen Gauß. Leite eine Rekursionsformel für die Doppelfolge $(s_{n,\,p})$ her (benutze binomische Formel) und bestimme die erzeugende Funktion 
$$ S(x,t) = \sum_{n,p \geq 0} s_{n,\,p}\;x^n\frac{t^p}{p!}
$$
\end{list}

\end{document}
