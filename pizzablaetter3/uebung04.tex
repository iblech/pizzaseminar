\documentclass[a4paper,ngerman]{scrartcl}

%\usepackage{ucs}
\usepackage[utf8]{inputenc}

\usepackage[ngerman]{babel}

\usepackage{amsmath,amsthm,amssymb,amscd,color,graphicx}

%\usepackage[small,nohug]{diagrams}
%\diagramstyle[labelstyle=\scriptstyle]

\usepackage[protrusion=true,expansion=true]{microtype}

\usepackage{lmodern}
\usepackage{tabto}

\usepackage[natbib=true,style=numeric]{biblatex}
\usepackage[babel]{csquotes}
\bibliography{lit}

\usepackage[all]{xy}

%\usepackage{hyperref}

\setlength\parskip{\medskipamount}
\setlength\parindent{0pt}

\theoremstyle{definition}
\newtheorem{defn}{Definition}
\newtheorem{bsp}[defn]{Beispiel}

\theoremstyle{plain}

\newtheorem{prop}[defn]{Proposition}
\newtheorem{ueberlegung}[defn]{Überlegung}
\newtheorem{lemma}[defn]{Lemma}
\newtheorem{kor}[defn]{Korollar}
\newtheorem{hilfsaussage}[defn]{Hilfsaussage}
\newtheorem{satz}[defn]{Satz}

\theoremstyle{remark}
\newtheorem{bem}[defn]{Bemerkung}

\clubpenalty=10000
\widowpenalty=10000
\displaywidowpenalty=10000



\newcommand{\XXX}[1]{\textcolor{red}{#1}}

\renewcommand*\theenumi{\alph{enumi}}
\renewcommand{\labelenumi}{\theenumi)}
\newcommand{\diff}[2]{\frac{d#2 }{d #2 #1 }}
\pagestyle{empty}

%\newarrow{Equals}=====

\usepackage{geometry}
\geometry{tmargin=2cm,bmargin=2cm,lmargin=3cm,rmargin=3cm}

\begin{document}

\vspace*{-4em}
\begin{flushright}Universität Augsburg \\ 26. März 2014\end{flushright}

\begin{center}\Large \textbf{Pizzaseminar zu erzeugenden Funktionen} \\
4. Übungsblatt
\end{center}
\vspace{1.5em}

\newbox{\mybox}
\setbox\mybox=\hbox{\textbf{Aufgabe 1:}}

\begin{list}{}{\labelwidth0em \leftmargin0em \itemindent0.5em \itemsep 1.3em}
\item[\textbf{Aufgabe 1:}] \emph{Die negative Binomialverteilung}

Berechne die wahrscheinlichkeitserzeugende Funktion der negativen
Binomialverteilung.

\item[\textbf{Aufgabe 2:}] \emph{Aussterbewahrscheinlichkeit bei geometrischer
Verteilung}

Sei bei einem Galton--Watson-Prozess die Zufallsgröße~$C$ der Anzahl Nachkommen
geometrisch verteilt mit Trefferwahrscheinlichkeit~$p$. Berechne die
Aussterbewahrscheinlichkeit.

\emph{Hinweis:} Die wahrscheinlichkeitserzeugende Funktion von~$C$
ist~$G_C(s) = \frac{ps}{1-qs}$.

\item[\textbf{Aufgabe 3:}] \emph{Simulation von Galton--Watson-Prozessen}

Entwerfe in der Programmiersprache deiner Wahl ein Programm zur Simulation von
Galton--Watson-Prozessen. Verwende als Verteilung der
Nachkommenszahl~$C$ deine Lieblingsverteilung.

\item[\textbf{Aufgabe 4:}] \emph{Allgemeine Theorie zu Galton--Watson-Prozessen}

Beweise die im Vortrag nur noch angegebene Proposition über
Galton--Watson-Prozesse: Ist~$P(C = 0) > 0$, so ist die
Aussterbewahrscheinlichkeit genau dann~$1$, wenn der Erwartungswert von~$C$
kleiner oder gleich~$1$ ist.

\emph{Tipp:} Unterscheide die Fälle~$G_C'(1) > 1$ und~$G_C'(1) \leq 1$.
Skizziere in beiden Fällen die Funktion~$G_C$ auf~$[0,1]$. Verwende den Satz
von Rolle, um anschauliche Intuition rigoros machen zu können.
\end{list}

\end{document}
