\documentclass[a4paper,ngerman]{scrartcl}

%\usepackage{ucs}
\usepackage[utf8]{inputenc}

\usepackage[ngerman]{babel}

\usepackage{amsmath,amsthm,amssymb,amscd,color,graphicx}

%\usepackage[small,nohug]{diagrams}
%\diagramstyle[labelstyle=\scriptstyle]

\usepackage[protrusion=true,expansion=true]{microtype}

\usepackage{lmodern}
\usepackage{tabto}

\usepackage[natbib=true,style=numeric]{biblatex}
\usepackage[babel]{csquotes}
\bibliography{lit}

\usepackage[all]{xy}

%\usepackage{hyperref}

\setlength\parskip{\medskipamount}
\setlength\parindent{0pt}

\theoremstyle{definition}
\newtheorem{defn}{Definition}
\newtheorem{bsp}[defn]{Beispiel}

\theoremstyle{plain}

\newtheorem{prop}[defn]{Proposition}
\newtheorem{ueberlegung}[defn]{Überlegung}
\newtheorem{lemma}[defn]{Lemma}
\newtheorem{kor}[defn]{Korollar}
\newtheorem{hilfsaussage}[defn]{Hilfsaussage}
\newtheorem{satz}[defn]{Satz}

\theoremstyle{remark}
\newtheorem{bem}[defn]{Bemerkung}

\clubpenalty=10000
\widowpenalty=10000
\displaywidowpenalty=10000



\newcommand{\XXX}[1]{\textcolor{red}{#1}}

\renewcommand*\theenumi{\alph{enumi}}
\renewcommand{\labelenumi}{\theenumi)}
\newcommand{\diff}[2]{\frac{d#2 }{d #2 #1 }}
\pagestyle{empty}

%\newarrow{Equals}=====

\usepackage{geometry}
\geometry{tmargin=2cm,bmargin=2cm,lmargin=3cm,rmargin=3cm}

\begin{document}

\vspace*{-4em}
\begin{flushright}Universität Augsburg \\ 12. März 2014\end{flushright}

\begin{center}\Large \textbf{Pizzaseminar zu erzeugenden Funktionen} \\
3. Übungsblatt
\end{center}
\vspace{1.5em}

\newbox{\mybox}
\setbox\mybox=\hbox{\textbf{Aufgabe 1:}}

\begin{list}{}{\labelwidth0em \leftmargin0em \itemindent0.5em \itemsep 1.3em}
\item[\textbf{Aufgabe 1:}] \emph{Symmetrische multivariate erzeugende Funktionen}
\begin{enumerate} 
\item \emph{Verallgemeinerung der Wegezählaufgabe von Jessi.} \newline
Sei $w(n_1,\ldots,n_d)$ die Anzahl der Wege vom Nullpunkt zum Punkt $(n_1,\ldots,n_d)\in\mathbb{Z}^d$. Ein Weg ist dabei eine Folge von zulässigen Schritten, wobei ein zulässiger Schritt jede Koordinate entweder um 0 oder um 1 erhöht. Finde die erzeugende Funktion
$$W(x_1,\ldots,x_d) = \sum_{n_1,\ldots,n_d} w(n_1,\ldots,n_d)\,x^{n_1}_1\ldots x^{n_d}_d .$$

\item \emph{Erzeugende Funktion des Minimums.} Verifiziere:
$$\sum_{n_1,\ldots,n_k\geq 1} \min\{n_1,n_2,\ldots,n_k\} t_1^{n_1}t_2^{n_2}\ldots t_k^{n_k} = 
\frac{t_1t_2\ldots t_k}{(1-t_1)(1-t_2)\ldots(1-t_k)(1-t_1t_2\ldots t_k)} $$
\end{enumerate}
\item[\textbf{Aufgabe 2:}] \emph{Umrechnungen von Basen der symmetrischen Polynome.} Bezeichne $\mathbf{e}_n$ und $\mathbf{m}_n$ die elementar-- und die monomial symmetrischen Funktionen. Zeige, daß in der Basisdarstellung durch $\mathbf{m}_\lambda$ der Koeffizient von $\mathbf{m}_{(n,n)}$ in $(\mathbf{e}_1+\mathbf{e}_2)^k$ gleich $\binom{k}{n}\binom{k-n}{n}$ ist.

\item[\textbf{Aufgabe 3:}] \emph{Formel von Molien für die äußere Algebra. } Sei $G$ eine endliche Gruppe, die auf einem $\mathbb{C}$--Vektorraum $V$ wirkt und damit eine Wirkung auf $\Lambda^mV$ induziert. Zeige:
$$ \sum_{m\geq 0} \dim\left(\left(\Lambda^mV\right)^G\right)\,t^m \ = \ \frac{1}{|G|}\sum_{g\in G} \det(I+gt) $$
\item[\textbf{Aufgabe 4:}] \emph{Symmetrisches Produkt der affinen Ebene.} Auf dem Polynomring in vier Variablen $\mathbb{R}[x_1,x_2,y_1,y_2]$ wirkt die symmetrische Gruppe in zwei Elementen durch Vertauschen der Indizes. Berechne die Hilbert--Poincaré--Reihe mithilfe der Formel von Molien.
\item[\textbf{Aufgabe 5:}] \emph{Anwendungen des Pólyatheorems.} \begin{enumerate}
\item Gegeben seien ein im Raum bewegliches Quadrat und ein Farbkasten mit $k$ Farben. Wie viele Möglichkeiten gibt es (bis auf Isomorphie), die Ecken des Quadrats zu färben?
\item Wie viele Möglichkeiten sind es, wenn man das Quadrat durch ein regelmäßiges Tetraeder ersetzt?
\item Wie viele Möglichkeiten sind es, wenn man verlangt, daß die Farbe Schwarz genau ein Mal verwendet werden muß?
\end{enumerate}
\end{list}

\end{document}
