\documentclass[a4paper,ngerman]{scrartcl}

%\usepackage{ucs}
\usepackage[utf8]{inputenc}

\usepackage[ngerman]{babel}

\usepackage{amsmath,amsthm,amssymb,amscd,color,graphicx}

%\usepackage[small,nohug]{diagrams}
%\diagramstyle[labelstyle=\scriptstyle]

\usepackage[protrusion=true,expansion=true]{microtype}

\usepackage{lmodern}
\usepackage{tabto}

\usepackage[natbib=true,style=numeric]{biblatex}
\usepackage[babel]{csquotes}
\bibliography{lit}

\usepackage[all]{xy}

%\usepackage{hyperref}

\setlength\parskip{\medskipamount}
\setlength\parindent{0pt}

\theoremstyle{definition}
\newtheorem{defn}{Definition}
\newtheorem{bsp}[defn]{Beispiel}

\theoremstyle{plain}

\newtheorem{prop}[defn]{Proposition}
\newtheorem{ueberlegung}[defn]{Überlegung}
\newtheorem{lemma}[defn]{Lemma}
\newtheorem{kor}[defn]{Korollar}
\newtheorem{hilfsaussage}[defn]{Hilfsaussage}
\newtheorem{satz}[defn]{Satz}

\theoremstyle{remark}
\newtheorem{bem}[defn]{Bemerkung}

\clubpenalty=10000
\widowpenalty=10000
\displaywidowpenalty=10000



\newcommand{\XXX}[1]{\textcolor{red}{#1}}

\renewcommand*\theenumi{\alph{enumi}}
\renewcommand{\labelenumi}{\theenumi)}
\newcommand{\diff}[2]{\frac{d#2 }{d #1 #2 }}
\pagestyle{empty}

%\newarrow{Equals}=====

\usepackage{geometry}
\geometry{tmargin=2cm,bmargin=2cm,lmargin=3cm,rmargin=3cm}

\begin{document}

\vspace*{-4em}
\begin{flushright}Universität Augsburg \\ 26. Februar 2014\end{flushright}

\begin{center}\Large \textbf{Pizzaseminar zu erzeugenden Funktionen} \\
2. Übungsblatt
\end{center}
\vspace{1.5em}

\newbox{\mybox}
\setbox\mybox=\hbox{\textbf{Aufgabe 1:}}

\begin{list}{}{\labelwidth0em \leftmargin0em \itemindent0.5em \itemsep 1.3em}
\item[\textbf{Aufgabe 1:}]
\emph{Nochmal kleiner Gauß / Faulhabersche Formel. } Zeige:
$$\sum_{k=0}^{n-1} k^p\ =\ \sum_{j=0}^{p+1} a_{pj}\, n^j,$$
wobei die Matrix der Koeffizienten $A = (a_{ij})_{ij}$ sich schreiben läßt als: $A = UZV$. Dabei sind $ U=\left(\genfrac\{\}{0pt}{}{i}{j}\right)_{ij} $ und $V=\left(\genfrac[]{0pt}{}{i}{j}\right)_{ij} $ Matrizen von Stirlingzahlen. Was ist $Z$ für eine Matrix?

\item[\textbf{Aufgabe 2:}] \emph{Stirlingzahlen und Differentialoperatoren}.
\begin{enumerate}
\item Stelle den Operator $\left(x\diff{x}{}\right)^n$ als Polynom in den Operatoren $\left\{x^k\diff{x}{^k},\ k\geq0\right\}$ dar.
\item Berechne den Kommutator $\left[x^n\diff{x}{^n},\,x^m\diff{x}{^m}\right] :=x^m\diff{x}{^m} x^n\diff{x}{^n} - x^n \diff{x}{^n}x^m\diff{x}{^m} $.
\item Sei $x=e^w$ und $f$ eine Potenzreihe. Zeige: $ \left(x\diff{x}{}\right)^n f(x)=
%\sum_{k=1}^n S(n,k) \,x^k\diff{x}{^k}\,f(x)
\diff{w}{^n}f(x).
$
\item Allgemeiner sei $\lambda(x)$ ein Polynom und $x = x(w)$ eine Lösung der Gleichung $\diff{w}{}x = \lambda(x)$.  Zeige:
$\left(\lambda(x)\diff{x}{}\right)^n f(x)= \diff{w}{^n}\,f(x). $
\end{enumerate} 

\item[\textbf{Aufgabe 3: }] Jeder der Zahlen $1,\ldots,n$ soll eine der folgenden Farben zugeordnet werden: Rot, Blau, Weiß, Grün. Dabei müssen ungerade viele Zahlen rot gefärbt und gerade viele Zahlen grün gefärbt werden. Abhängig von $n$, wieviele Möglichkeiten gibt es? Benutze exponentiell erzeugende Funktionen!

\item[\textbf{Aufgabe 4: }] Gegeben sind $n$ verschiedene Zahlen. Diese sollen in Dreiecke, Vierecke, Fünfecke usw.~aufgeteilt werden. Zwei Vielecke gelten dabei als gleich, wenn die (zyklische) Anordnung ihrer Zahlen übereinstimmt. Finde die exponentiell erzeugende Funktion der Anzahl solcher möglichen Aufteilungen.

\item[\textbf{Aufgabe 5: }] Sei $(p_n)_{n\geq 0}$ eine Folge von Polynomen, gegeben durch: $p_n(x) = \sum_k\genfrac\{\}{0pt}{}{n}{k}x^k$. Zeige:
\begin{enumerate}
\item Es gibt eine eindeutige Potenzreihe $f$, welche
$ \sum_n p_n(x)\frac{u^n}{n!} \; =\; \exp(xf(u))
$ erfüllt.
\item Es gibt einen linearen Operator $D$, welcher $Dp_n(x) = np_{n-1}(x)$ erfüllt und außerdem mit der Operation $x\mapsto x+a$ verträglich ist.
\item Es gibt eine eindeutige Potenzreihe $q$, die $q(\frac{d}{dx}) = D$ erfüllt. Es ist $q$ die Umkehrfunktion zu $f$.
\end{enumerate}
\end{list}

\end{document}
