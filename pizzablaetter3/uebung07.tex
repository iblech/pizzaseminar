\documentclass[a4paper,ngerman]{scrartcl}

%\usepackage{ucs}
\usepackage[utf8]{inputenc}

\usepackage[ngerman]{babel}

\usepackage{amsmath,amsthm,amssymb,amscd,color,graphicx}

%\usepackage[small,nohug]{diagrams}
%\diagramstyle[labelstyle=\scriptstyle]

\usepackage[protrusion=true,expansion=true]{microtype}

\usepackage{lmodern}
\usepackage{tabto}

\usepackage[natbib=true,style=numeric]{biblatex}
\usepackage[babel]{csquotes}
\bibliography{lit}

\usepackage[all]{xy}

%\usepackage{hyperref}

\setlength\parskip{\medskipamount}
\setlength\parindent{0pt}

\theoremstyle{definition}
\newtheorem{defn}{Definition}
\newtheorem{bsp}[defn]{Beispiel}

\theoremstyle{plain}

\newtheorem{prop}[defn]{Proposition}
\newtheorem{ueberlegung}[defn]{Überlegung}
\newtheorem{lemma}[defn]{Lemma}
\newtheorem{kor}[defn]{Korollar}
\newtheorem{hilfsaussage}[defn]{Hilfsaussage}
\newtheorem{satz}[defn]{Satz}

\theoremstyle{remark}
\newtheorem{bem}[defn]{Bemerkung}

\clubpenalty=10000
\widowpenalty=10000
\displaywidowpenalty=10000



\newcommand{\XXX}[1]{\textcolor{red}{#1}}

\renewcommand*\theenumi{\alph{enumi}}
\renewcommand{\labelenumi}{\theenumi)}
\newcommand{\diff}[2]{\frac{d#2 }{d #2 #1 }}
\pagestyle{empty}

%\newarrow{Equals}=====

\usepackage{geometry}
\geometry{tmargin=2cm,bmargin=2cm,lmargin=3cm,rmargin=3cm}

\begin{document}

\vspace*{-4em}
\begin{flushright}Universität Augsburg \\ Tim Baumann \\ 2. April 2014\end{flushright}

\begin{center}\Large \textbf{Pizzaseminar zum Vier-Farben-Satz} \\
7. Übungsblatt
\end{center}
\vspace{1.5em}

Der Vier-Farben-Satz lässt sich auf viele verschiedene äquivalente Art und
Weisen formulieren. Eine völlig verblüffende Variante ist folgende:
\begin{quote}
Betrachte zwei beliebige Klammerungen des Kreuzprodukts einer
endlichen Familie von Vektoren im~$\mathbb{R}^3$, wie zum Beispiel
\[ L = a \times (b \times ((c \times d) \times e) \quad\text{und}\quad
R = ((a \times b) \times c) \times (d \times e). \]
Dann kann man den Variablen~$a,b,c,\ldots$ auf eine solche Art und Weise
Werte aus~$\{ e_1, e_2, e_3 \}$ zuweisen, sodass~$L = R$ gilt und beide Seiten
ungleich Null sind.
\end{quote}

Zeige, dass aus dem Vier-Farben-Satz diese Aussage über die Assoziativität des
Kreuzprodukts folgt.

\emph{Tipp:} Wie wir im Vortrag gesehen haben, ist der Vier-Farben-Satz
äquivalent zu Aussage, dass jeder brückenlose, planare und kubische
Graph eine 3-Kantenfärbung besitzt.

\end{document}
